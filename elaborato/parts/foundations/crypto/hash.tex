\section{Secure Hash functions}\label{sec:hash_functions}
Secure Hash functions are a fundamental tool of cryptography, as they can be used to produce 
\(\BigO{1}\) or \(\BigO{\call{\log}{n}}\) commitments for any message of length \(\abs{n}\).
\begin{definition}[Hash function]
  Given \(n \in \mathbb{N}\), an \emph{\(n\)-bit hash function} is a function 
  \(H\colon \Set{0, 1}^{*} \to \Set{0, 1}^{n}\).
\end{definition}
The input of an hash function is called the \emph{message}, while its output is called the 
\emph{digest}.
From the definition, it is immediate to see that there are an infinite number of messages which map
to the same digest.
Now, a very simple hash function might be truncation (i.e.\ take the first \(n\) bits and discard 
anything coming afterwards), but it is not of much interest for cryptography, as we require 
additional \emph{security guarantees}.
\begin{remark}
  When we say that it is \emph{easy} (resp.\  \emph{hard}) to compute a function, we mean that 
  there is (resp.\ there is not) a probabilistic Turing machine which can compute such function in
  at most polynomial time.
  In particular, a \emph{one-way function} is an easily computable function \(f\) whose inverse is 
  hard to compute, and a \emph{trapdoor function} is a one-way function whose inverse becomes easy 
  to compute given some additional knowledge, called the \emph{key}.
\end{remark}

\begin{remark}\label{rem:xor_vs_add}
	The \(\oplus \) symbol used in this chapter can have different meanings: in the literature, 
	it is mostly used to denote bitwise \textsc{xor}. Since a message/string over \(\Set{0, 1}^{*}\) 
	can be considered as a vector over \(\mathbb{Z}_{2}^n\), bitwise \textsc{xor} is the same as 
	vector addition.
	However, there are instances, particularly in ZK systems, where messages are interpreted as 
	elements of some prime field \(\mathbb{Z}_p\): in such cases, \(\oplus \) will denote addition
	modulo \(p\), as defined in \Cref{chap:math}.
	However, such cases will be explicitly specified, so we decided to keep the ambiguity to be 
	coherent with the literature.
\end{remark}

\begin{definition}[Cryptographic hash function~\cite{AlKuwariDB2011}]
	Given \(n \in \mathbb{N}\), an \emph{\(n\)-bit cryptographic hash function (CHF)} is a \(n\)-bit 
  hash function which satisfies the following properties:
	\begin{itemize}
		\item \textbf{Collision resistance}: It is hard to find two messages \(m_1, m_2\) such
		      that \(\call{H}{m_1} = \call{H}{m_2}\).
		\item \textbf{Preimage resistance}: Given some digest \(d\), it is hard to find a
		      message \(m\) such that \(\call{H}{m} = d\) (\(H\) is a one-way function).
		\item \textbf{Second preimage resistance}: Given some message \(m_1\), it is hard to
		      find a message \(m_2\) such that \(\call{H}{m_1} = \call{H}{m_2}\).
	\end{itemize}
\end{definition}

Note that second preimage resistance implies first preimage resistance (if given any \(m_1\) 
we can find a colliding \(m_2\), we can do it also without being given \(m_1\)), and in turn 
preimage resistance implies second preimage resistance (if given any \(d\) we can find a colliding 
\(m\), then given any \(m'\) we can compute \(d\) and find the collision).

A perfect \(n\)-bit CHF provides \(n/2\) bits of security for collision resistance 
(birthday paradox), meaning that every possible adversary is expected to need at least 
\(\BigO{2^{n/2}}\) time to find a collision, 
while for preimage resistance it provides \(n\) bits of security.
A CHF can be built by applying some known secure construction to functions which are
simpler to devise.
\begin{definition}[Padding function]
	An \(n\)-bit \emph{padding function} is a function 
	\(\Pad\colon \Set{0, 1}^{*} \to \Parens{\Set{0, 1}^n}^{*}\).
\end{definition}
\begin{definition}[Pseudorandom permutation]
	Given \(l \in \mathbb{N}\), an \(l\)-bit \emph{pseudorandom permutation} is a permutation 
	\(P\colon \Set{0, 1}^l \to \Set{0, 1}^l\) which is indistinguishable from an uniform random 
	distribution.
\end{definition}
\begin{definition}[Keyed permutation]
	Given \(l, n \in \mathbb{N}\), an \emph{\(l/n\)-bit keyed permutation} is
	a function \(F\colon \Set{0, 1}^l \times \Set{0, 1}^n \to \Set{0, 1}^l\) which is a permutation 
  on its first parameter.
\end{definition}
\begin{definition}[Block cipher]
  A \emph{block cipher} is a trapdoor pseudorandom keyed permutation.
\end{definition}
\begin{definition}[One-way compression function]
	Given \(l_1, n, l_2 \in \mathbb{N}\) such that \(l_1 + n > l_2\), an 
	\emph{\(l_1/n/l_2\)-bit one-way compression function (OWCF)} is a one-way function 
	\(F\colon \Set{0, 1}^{l_1} \times \Set{0, 1}^{n} \to \Set{0, 1}^{l_2}\).
\end{definition}
Since many compression functions exploit their second component (the key), rather than
the first component to compress a message, we will call \(n\)-to-\(m\)(-bit) compression function 
any compression function which, in any way, reduces an input of length \(n\) to an output of 
length \(m\).

A \emph{pseudorandom keyed permutation} (PKP) is typically built by iterating a 
``somewhat pseudorandom'' keyed permutation \(F\) for an adequate number of \emph{rounds}, until 
inverting the function becomes hard.
A block cipher can be built from a PKP by following some secure construction scheme, 
like the Feistel-Luby-Rackoff construction~\cite{MenezesOV2001}.
A OWCF can also be derived from a PKP or a block cipher, by applying a secure construction scheme, 
like the Davies-Meyer construction~\cite{Preneel2005}.
Finally, CHF can again be derived either by a OWCF, for exaple through the Merkle-Damg\r{a}rd 
construction~\cite{Merkle1979}, or directly from a pseudorandom permutation, like in the sponge 
construction~\cite{BertoniDPA2007, Tiwari2017}.
\begin{proposition}[Feistel-Luby-Rackoff construction~\cite{MenezesOV2001}]
	Given an \(l/n\)-bit pseudorandom keyed permutation \(P\), some message
  \(m = \Tuple{m_1, m_2}\) such that \(m_1, m_2 \in \Set{0, 1}^l\), a number of rounds 
  \(r > 3\), and a set of keys \(k_1, \dots, k_{r} \in \Set{0, 1}^n\), then the function \(E_{r}\) 
	is a \(2l/n\) block cipher, where:
  \[
		 \call{E_{i}}{m, k_i} = \Tuple{x_i, y_i} = 
		\begin{cases}
			\Tuple{m_1, m_2}                                       & i = 0 \\
			\Tuple{y_{i-1}, x_{i-1} \oplus \call{P}{y_{i-1}, k_i}} & 1 \le i \le r
		\end{cases}
	\]
\end{proposition}

\begin{proposition}[Davies-Meyer construction~\cite{Preneel2005}]
	Given an \(l/n\)-bit pseudorandom keyed permutation \(P\), some number of blocks \(b \in \mathbb{N}\), 
  some \emph{initial value} \(v \in \Set{0, 1}^l\), and some message 
	\(m = \Tuple{m_1, \dots, m_b}\) such that each \(m_i \in \Set{0, 1}^{n}\), then the function 
	\(F_{b}\) is an \(l/{bn}/l\) one-way compression function, where:
  \[
		\call{F_{i}}{v, m} =
		\begin{cases}
			v                                													& i = 0 \\
			\call{P}{\call{F_{i-1}}{v, m}, m_i} \oplus \call{F_{i-1}}{v, m} & 1 \le i \le b \\
		\end{cases}
	\]
\end{proposition}

\begin{proposition}[Merkle-Damg\r{a}rd construction~\cite{Merkle1979}]
	Given an \(l/n/l\)-bit one way compression function \(F\), some initial value 
	\(v \in \Set{0, 1}^{l}\), some message \(m \in \Set{0, 1}^*\), a padding extension length \(k\) 
	and an \(n\)-bit padding function \(\Pad \) such that, 
	\(\forall x, y \in \Set{0, 1}^{*}\):
	\begin{align*}
    & \abs{\call{\Pad}{x}} = \abs{x} + \Parens*{-\abs{x} \bmod n} + kn \\
		& \abs{x} = \abs{y} \implies \abs{\call{\Pad}{x}} = \abs{\call{\Pad}{y}} \\
    & \abs{x} \neq \abs{y} \implies 
			\call{\Pad}{x}_{\abs{\call{\Pad}{x}}} \neq \call{\Pad}{y}_{\abs{\call{\Pad}{y}}}
  \end{align*}
	then the function \(H_{\abs{\call{\Pad}{m}}}\) is an \(l\)-bit cryptographic hash function, 
	where:
	\[
		\call{H_{i}}{m} =
		\begin{cases}
			v                                               & i = 0   \\
			\call{F}{\call{H_{i-1}}{m}, \call{\Pad}{m}_{i}} & 1 \le i \le \abs{\call{\Pad}{m}}
		\end{cases}
	\]
\end{proposition}

\begin{proposition}[Sponge construction~\cite{BertoniDPA2007}]
	Given an \(l\)-bit pseudorandom permutation \(P\), a message \(m \in \Set{0, 1}^{*}\), a 
	\emph{rate} \(r\) and a \emph{capacity} \(c\) such that \(r + c = l\), an initial value 
	\(v \in \Set{0, 1}^b\) and an \(r\)-bit padding function \(\Pad \), the 
	function \(H_{\abs{\call{\Pad}{m}}}\) is an \(r-bit\) cryptographic hash function, where:
	\begin{align*}
		& \call{S_{i}}{m} = 
		\begin{cases}
			v & i = 0 \\
			\call{P}{\call{S_{i-1}}{m} \oplus \call{\Pad}{m}_i} & 1 \le i \le \abs{\call{\Pad}{m}}
		\end{cases} \\
		& \call{H_{i}}{m} = \call{S_{i}}{m}_{1,\dots,r}
	\end{align*}
\end{proposition}

The sponge construction is particularly interesting as it is very flexible: given a pseudorandom 
permutation, it can be used to build pseudorandom keyed permutations, cryptographic hash functions, 
random number generators and authenticated encryptions~\cite{BertoniDPA2012}.
