\section{Tree-like modes of hashing}\label{sec:tree_hash}
Consider an \(n\)-bit CHF \(H\), and suppose that a prover claims to know some message \(m\): 
the digest \(d = \call{H}{m}\) can be considered as a \emph{short binding commitment} for \(m\): 
By asking the prover to share the digest, whose size \(\abs{d} = n\) is typically considered to be 
\(\BigO{1}\) (or \(\BigO{\call{\log}{\abs{m}}}\) in some cases), a verifier is convinced that the 
prover does know \(m\) with probability \(\approx 1 - {1}/{2^n}\).
A modern standard CHF like SHA-256~\cite{Dang2015} produces digests of length at least \(256\) bits,
making the \(1 - {1}/{2^n}\) bound really hard to bruteforce through.
Note that the verifier needs not to know \(m\) in advance: the commitment \(d\) is (temporarily) 
appended to a public \emph{blockchain} and, at any point in the future, when the verifier becomes 
aware of some \(m'\) provided by the prover, if \(\call{H}{m'} = d\), the commitment can be 
approved or rejected.

Now, suppose that the prover wants to commit to a list of \(k\) messages: the simplest solution 
would be to publish the hash of every message, which would require to append \(\BigO{k}\) 
elements on the blockchain.
Another way would be for the prover to share \(\call{H}{\Tuple{m_1, \dots, m_k}}\): the 
communication cost would only be \(\BigO{1}\) but, in general, not all the messages belong to the 
same prover, so this method would not work, and we need a better solution.

\subsection{Merkle tree}
\begin{definition}[Binary Merkle tree~\cite{Merkle1988}]
	A \emph{binary Merkle tree (MT)} of height \(h \in \mathbb{N}\) over a \(2n\)-to-\(n\) compression 
	function \(C\), is the complete binary tree of height \(h\) such that, given a sequence of input 
	messages \(\Tuple{m_1, \dots, m_{2^{h-1}}}\) over \(\Set{0, 1}^{2n}\), produces an
	output digest \(d \in \Set{0, 1}^{n}\) in the following way:
	\begin{enumerate}
		\item The leaf nodes \(\nu_1, \dots, \nu_{2^{h-1}}\) contain 
					\(\call{C}{m_1}, \dots, \call{C}{m_{2^{h-1}}}\).
		\item Every other node \(\nu \) contains \(\call{C}{\nu_l, \nu_r}\), where \(\nu_l\) is
		      the left child of \(\nu \) and \(\nu_r\) is the right child of \(\nu \).
		\item The output digest \(d\) is the content of the root node. 
	\end{enumerate}
\end{definition}

The set of the sibling nodes visited in the path from a leaf of the tree to the root, including the 
leaf itself, is the \emph{authentication path} of the leaf.
By using Merkle trees, the prover only needs to send to the verifier, as a commitment for
some message \(m_i\) among \(n = 2^h\) messages, the contents of the co-path from the leaf 
containing \(m_i\) to the root, in addition plus the hash of \(m_i\): this requires
just \(\BigO{\call{\log}{n}}\) cost to validate the commitment.
Merkle trees bottom-up construction is very easy to parallelize, and they can be used in the 
multiple-provers scenario: each prover only needs to commit to the path from its own leaf to the 
root of the tree.
It is immediate to generalize the notion of binary Merkle tree to arbitrary arity.
\begin{proposition}[Security of Merkle tree mode of hash~\cite{Merkle1988}]
	Given a one-way \(tn\)-to-\(n\) compression function \(C\), the \(t\)-ary Merkle tree over 
	\(C\) is a cryptographic hash function.
\end{proposition}

\begin{example}\label{ex:merkle_tree}
	Consider the sequence of pre-hashed messages \(S = \Tuple{3, 4, 7, 7}\) and the 
	compression function 
	\(\call{C}{x, y}: \Tuple{x, y} \mapsto \Parens*{xy \bmod 13} + 1\) (for ease of exposition, 
	we work over integers instead of bit strings, but the two can be readily converted into one 
	another).
	\Cref{fig:merkle_tree} shows the contents of the associated Merkle Tree.
	Note that the real message is not stored in the Merkle Tree, but only the `first level' of hashes.
	The authentication path of the leaf labelled with \(3\) consists of the tuple \(\Tuple{3, 4, 11}\):
	by computing \(\call{H}{3, 4} = 13\) and then \(\call{H}{13, 11} = 1\) we can verify that the 
	commitment is respected.
\end{example}
\begin{figure}
	\centering
	\begin{tikzpicture}[node distance={32pt}, node/.style = {draw, circle},on grid=true]
		\node[node] (x1) {\(3\)};
		\node[node,draw=none] (n1) [right of=x1] {};
		\node[node] (x2) [right of=n1] {\(4\)};
		\node[node,shape=rectangle] (c1) [above of=n1] {\(C\)};
		\node[node,draw=none] (n2) [right of=x2] {};
		\node[node] (x3) [right of=n2] {\(7\)};
		\node[node,draw=none] (n3) [right of=x3] {};
		\node[node] (x4) [right of=n3] {\(7\)};
		\node[node,shape=rectangle] (c2) [above of=n3] {\(C\)};
		\node[node] (x5) [above of=c1] {\(13\)};
		\node[node,draw=none] (n3) [above of=n2] {};
		\node[node,draw=none] (n4) [above of=n3] {};
		\node[node] (x6) [above of=c2] {\(11\)};
		\node[node,shape=rectangle] (c3) [above of=n4] {\(C\)};
		\node[node] (x7) [above of=c3] {\(1\)};
		\draw[->] (x1) to (c1);
		\draw[->] (x2) to (c1);
		\draw[->] (x3) to (c2);
		\draw[->] (x4) to (c2);
		\draw[->] (c1) to (x5);
		\draw[->] (c2) to (x6);
		\draw[->] (x5) to (c3);
		\draw[->] (x6) to (c3);
		\draw[->] (c3) to (x7);
	\end{tikzpicture}
	\caption{Merkle tree of \Cref{ex:merkle_tree}.}\label{fig:merkle_tree}
\end{figure}

\subsection{Augmented Binary Tree}
The Merkle tree is the de-facto standard for blockchain applications, and basically for any 
scenario for which a `linear' hash function cannot be used.
In~\cite{Stam2008}, it was given a lower bound on the amount of queries necessary to obtain a 
collision for a \(\Parens*{m+s}\)-to-\(s\)-bit CHF \(H\) (the \(m\) is variable) built from a 
\(\Parens*{n+c}\)-to-\(n\)-bit OWCF \(C\): if \(H\) makes \(r\) queries to \(C\), it is possible 
to find a collision by making \(2^{\frac{nr + cr - m}{r + 1}}\) queries to \(H\).
By combining this result with the \(2^{s/2}\) upper bound of the birthday paradox, one can 
immediately obtain a tight bound \(m = \frac{2nr + 2cr -sr - s}{2}\) for the variable length \(m\) 
of the message.

\begin{definition}[Compactness~\cite{AndreevaBR2021}]
	The \emph{compactness} of an \(\Parens*{m+s}\)-to-\(s\)-bit hash function making \(r\) queries to 
	an underlying \(\Parens*{n+c}\)-to-\(n\)-bit one-way compression function is the value
	\(\alpha = \frac{2m}{2nr + 2cr -sr - s}\).
\end{definition}

\begin{example}\label{ex:mtree_compactness}
	Consider a \(2n\)-to-\(n\) bit OWCF and a Merkle Tree of height \(h\): the computation 
	of the tree is a \(\Parens*{2^{h-1}n}\)-to-\(n\)-bit hash function, and makes exactly 
	\(r = 2^{h-1} - 1\) queries to \(C\).
	We have \(s = c = n\) and \(m = 2^{h-1}n - n = nr\), therefore the compactness of the Merkle 
	Tree construction is:
	\[
		\alpha = \frac{2m}{2nr + 2cr -sr - s} = 
		\frac{2nr}{2nr + 2nr - nr - n} =
		\frac{2r}{3r - 1}
	\]
	Which tends to \(2/3\) when \(r\) tends to infinity.
\end{example}

\begin{definition}[Augmented Binary tRee~\cite{AndreevaBR2021}]
	An \emph{Augmented Binary tRee (ABR)} of height \(h \in \mathbb{N}\) over a 
	\(2n\)-to\(n\) compression function \(C\) is a complete binary tree of height \(h\) 
	augmented with \emph{middle} nodes such that, given a sequence of input messages
	\(S = \Tuple{m_1, \dots, m_{2^{h-1} + 2^{h-2}-1} \mid \forall i\colon m_i \in \Set{0, 1}^{*}}\), 
	it produces an output digest \(d \in \Set{0, 1}^n\) in the following way:
	\begin{enumerate}
		\item The leaf nodes \(\nu_{1}, \dots, \nu_{2^{h-1}}\) contain \(\call{C}{m_1}, \dots,
		      \call{C}{m_{2^{h-1}}}\).
		\item There are no middle nodes in the leaf layer.
		\item The middle nodes \(\nu_{2^{h-1}+1}, \dots, \nu_{\abs{S}}\) contain
		      \(\call{C}{m_{2^{h-1}+1}}, \dots, \call{C}{m_{\abs{S}}}\).
		\item Every other node \(\nu \) contains \(\call{C}{\nu_l \oplus \nu_m, \nu_r \oplus
		      \nu_m} \oplus \nu_r \), where \(\nu_l\) is the left child of \(\nu \), \(\nu_r\)
		      is the right child of \(\nu \), and \(\nu_m\) is the middle child of \(\nu \), or \(0\)
		      if \(\nu \) doesen't have a middle child.
	\end{enumerate}
\end{definition}

The authentication path of the ABR is similar to the one of the Merkle tree, but also includes 
the middle nodes encountered during the traversal.

\begin{proposition}[Security of ABR mode of hash~\cite{AndreevaBR2021}]
	Given a one-way \(2n\)-to-\(n\) compression function \(C\), the ABR over \(C\) is a cryptographic 
	hash function.
\end{proposition}

An ABR of height \(h\) can process 50\% more messages than a Merkle Tree of the same height, 
while performing the same number of queries to the underlying compression function, with the 
additional cost introduced by the intermediate \(\oplus \) operations being negligible in most 
scenarios.

\begin{example}
	Consider a \(2n\)-to-\(n\) bit OWCF and an ABR of height \(h\): the computation 
	of the tree is a \(\Parens*{2^{h-1} + 2^{h-2}-1}n\)-to-\(n\)-bit hash function, 
	and makes exactly \(r = 2^{h-1} - 1\) queries to \(C\).
	Like in \Cref{ex:mtree_compactness}, we have \(s = c = n\), but this time 
	\(m = \Parens*{2^{h-1} + 2^{h-2}-1}n - n = nr + {nr}/2 - n\), so the compactness of the ABR 
	construction is:
	\[
		\alpha = \frac{2m}{2nr + 2cr - sr - s} = 
		\frac{2nr + nr - 2n}{2nr + 2nr - nr - n} =
		\frac{3r - 2}{3r - 1}
	\]
	Which approaches \(1\) as \(r\) approaches infinity, meaning that the ABR construction achieves
	optimal compactness.
\end{example}

It is worth of notice that, while the ABR hash mode achieves collision resistance, it does not 
achieve \emph{indifferentiability} (a weaker notion of indistinguishability between Turing 
machines~\cite{MaurerRH2003}), hence a modified construction, called ABR+, was also proposed, 
although it does not achieve perfect compactness.
\begin{figure}
	\centering
	\begin{tikzpicture}[node distance={32pt}, node/.style = {draw, circle},on grid=true]
		\node[node] (x1) {\(3\)};
		\node[node,draw=none] (n1) [right of=x1] {};
		\node[node] (x2) [right of=n1] {\(4\)};
		\node[node,shape=rectangle] (c1) [above of=n1] {\(C\)};
		\node[node,draw=none] (n2) [right of=x2] {};
		\node[node] (x3) [right of=n2] {\(7\)};
		\node[node,draw=none] (n3) [right of=x3] {};
		\node[node] (x4) [right of=n3] {\(7\)};
		\node[node,shape=rectangle] (c2) [above of=n3] {\(C\)};
		\node[node,draw=none] (n4) [above of=n2] {};
		\node[node] (x8) [above of=n4] {\(10\)};
		\node[node] (x5) [above of=c1] {\(13\)};
		\node[node] (x6) [above of=c2] {\(11\)};
		\node[node,draw=none] (n5) [above of=x8] {};
		\node[node,shape=rectangle] (c3) [above of=n5] {\(C\)};
		\node[node,shape=rectangle] (a1) [below left of=c3] {\(\oplus \)};
		\node[node,shape=rectangle] (a2) [below right of=c3] {\(\oplus \)};
		\node[node,shape=rectangle] (a3) [above of=c3] {\(\oplus \)};
		\node[node] (x7) [above of=a3] {\(10\)};
		\draw[->] (x1) to (c1);
		\draw[->] (x2) to (c1);
		\draw[->] (x3) to (c2);
		\draw[->] (x4) to (c2);
		\draw[->] (c1) to (x5);
		\draw[->] (c2) to (x6);
		\draw[->] (x5) to (a1);
		\draw[->] (x6) to (a2);
		\draw[->] (x8) to (a1);
		\draw[->] (x8) to (a2);
		\draw[->] (a1) to (c3);
		\draw[->] (a2) to (c3);
		\draw[->] (c3) to (a3);
		\draw[->] (x5) [bend left] to (a3);
		\draw[->] (a3) to (x7);
	\end{tikzpicture}
	\caption{ABR of \Cref{ex:abr}.}\label{fig:abr}
\end{figure}
\begin{example}\label{ex:abr}
	Consider the same compression function \(C\) of \Cref{ex:merkle_tree}, and consider the 
	sequence of pre-hashed messages \(S' = \Tuple{3, 4, 7, 7, 10}\), in this case we interpret 
	\(x \oplus y \equiv \Parens*{x + y \bmod 13} + 1\).
	\Cref{fig:abr} shows the resulting ABR\@.
	The authentication path of the node labelled with \(3\) consists of the tuple 
	\(\Tuple{3, 4, 10, 11}\): by computing \(\call{H}{3, 4} = 13\) and then 
	\(\call{H}{13 \oplus 10, 11 \oplus 10} \oplus 13 = 10\) we can verify that the commitment is 
	respected.
\end{example}
