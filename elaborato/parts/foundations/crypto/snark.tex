\section{ZK-SNARK systems}
As we saw in \Cref{subsec:nizk}, researchers were able to construct ZK-NARK systems whose 
verification complexity was linear in the size of the problem instance, which is provided as a 
boolean circuit.
Furthermore, in the CRS model, by using a block cipher, it is also possible to have 
\emph{publicly verifiable} constructions~\cite{LapidotS1991}, meaning that \emph{any} verifier, 
not just the one that engages the protocol, is able to check the proof, which is encrypted with a 
\emph{proving key}, by using a public \emph{verification key}.

\begin{proposition}[Fiat-Shamir heuristic~\cite{FiatS1987}]
  Suppose a probabilistic I/O TM \(\mathcal{P}\) with access to a CHF \(H\) wants to prove its 
  knowledge of the discrete logarithm \(x = \call{\log}{y}\) for some value 
  \(y \in \mathbb{Z}_p\), where \(p\) is a large prime number.
  Then \(\mathcal{P}\) can sample a random value \(v\) from \(\Taperand \), compute the digest 
  \(d = \call{H}{p, y, p^v}\), the result \(r = {v - dx} \bmod \Parens*{p - 1}\), and finally 
  output the quadruple \(\Tuple{p, y, p^v, r}\).
  Any \textnormal{\textsc{PTIME}} TM \(\mathcal{V}\) with access to \(\Tuple{p, y, p^v, r}\) 
  and \(H\) can recompute \(d\) and check whether \(p^v = p^{r}y^{d}\)
  (If \(\mathcal{P}\) is not cheating, then \(p^{r}y^{d} = p^{v - dx}\Parens*{p^{x}}^d = 
  p^{v - dx}p^{dx} = p^{v - dx + dx} = p^v\)).
  Assuming that the discrete logarithm is hard and that true CHF exist, if equality holds 
  \(\mathcal{V}\) is convinced that \(\mathcal{P}\) knows \(x\) but is not able to retrieve it 
  except with negligible probability.
\end{proposition}

\begin{definition}[Succint proof]
  A \emph{succint proof} for a statement \(\sigma \) over a language \(L \subseteq \Set{0, 1}^*\) 
  is a proof \(\pi \) such that \(\abs{\pi} = \BigO{\call{\log}{\abs{\sigma}}}\).
\end{definition}

Similarly, one can define the notion of succint argument of knowledge, and in particular, a 
succint ZK-NARK system is called a ZK-SNARK system.
\begin{definition}[Probabilistically checkable proof system~\cite{BabaiFLS1991,FeigeGLSS1991}]
  A \emph{probabilistically checkable proof system} (PCP system) is an interactive proof system 
  \(\Tuple{\mathcal{P, V}}\) such that for any proof \(\pi \) provided by \(\mathcal{P}\):
  \(\exists k\colon \call{\Time}{\mathcal{V}} = \BigO{\call{\log^k}{\abs{\pi}}}\).
\end{definition}

In a PCP system, the prover \(\mathcal{P}\) constructs a proof \(\pi \) of size polynomial in the 
length of the original statement \(\sigma \); since the verifier \(\mathcal{V}\) is 
polylogarithmically bound to the size of the proof, it can only query a small portion of it, 
however, it is enough to get statistical completeness and soundness.

In~\cite{Kilian1992}, the author uses Merkle trees to have the prover commit to a proof \(\pi \), 
(the bits of \(\pi \) are the leaves and the root, whcih has constant size, is sent to the verifier). 
The verifier then queries a certain number of authentication paths, which have length
\(\BigO{\call{\log}{\abs{\pi}}}\), and decides whether to accept or reject.
In this sense, the protocol is therefore succint.
In~\cite{Micali2000}, the construction was extended and, by applying the Fiat-Shamir heuristic, it 
is possible to make the protocol non-interactive.

One of the first \emph{succint} ZK-NARK (ZK-SNARK) systems that didn't make explicit use of 
PCPs was devised in~\cite{Groth2010}, but had one important drawback: while the size of the proof 
is constant, the size of the CRS, and the computation that the prover has to perform is 
\emph{quadratic} in the size of the input circuit
(this bound was slightly improved in~\cite{Lipmaa2011}).

However, by first transforming the circuit into \emph{quadratic span programs} (QSPs), 
the boolean equivalent of QAPs (\Cref{subsec:qap}), it was shown that it is possible to reduce both 
the size of the CRS and the prover's computational complexity to linear, while still having succint 
proofs~\cite{GennaroGPR2012}.
Since all these constructions make use of encryption based on the hardness of finding the discrete
logarithm of a number over a big finite field, dealing with boolean circuits and QSPs is not 
very efficient; although both polynomially sized boolean and arithmetic circuits are equivalent to 
\textsc{PTIME} Turing machines~\cite{PippengerF1979}, working over arithmetic programs, and hence 
using QAPs over QSPs, can greatly reduce the constant factors involved 
in such constructions, although this depends on the kind of input problem (numeric problems 
can exploit arithmetic circuits much better than, say, propositional problems).

\subsection{Pinocchio}
An important application of ZK-SNARK systems is in \emph{verifiable computation}.
Consider a client (say, a mobile phone) that wants to delegate to a server (say, a cloud provider) 
some computation, for which several inputs are required: some are provided by the client, 
and some by the server:
\begin{itemize}
  \item The client does not trust the server, so we would need a proof system, but since the server 
        is not computationally unbounded, an \emph{argument of knowledge} system will suffice.
  \item The server might have to interact with many clients or, similarly, many different clients
        might require the same computation, the system must be \emph{non-interactive}.
  \item Verifying the computation must be cheaper than performing it, otherwise the client wouldn't 
        have to ask the server in the first place, the system must provide \emph{succint} proofs.
  \item The server has too interests in to the client that the computation was correct, say to 
        avoid legal liability, but it is not willing to share its own inputs, so our system must
        be \emph{zero-knowledge}.
\end{itemize}
Clearly, among the various systems we saw up to now, ZK-SNARKs are the only one that can reasonably 
fulfill all of the requirements above.
However, all the constructions we saw, due to the high overheads involved in generating the CRS, 
building the QSP/QAP, generating the proof, and even verifying it, were (much) slower than native 
execution by the client.

The first construction that was efficient enough to be usable in practice, and that in many cases
broke the `native execution' barrier for verification time, was 
\emph{Pinocchio}~\cite{ParnoGHR2013}.
\begin{definition}[Bilinear map~\cite{BonehF2001}]
  A \emph{bilinear map} (w.r.t.\ exponentiation) over two groups \(\mathbb{G}\) and \(\mathbb{G}'\) 
  is a map \(B\colon \mathbb{G} \times \mathbb{G} \to \mathbb{G}_T\) such that
  \(\forall a, b \in \mathbb{Z}\colon \call{B}{x^a, y^b} = \call{B}{x, y}^{ab}\).
\end{definition}

\begin{example}\label{ex:bilinear_map}
  Recall \Cref{ex:cyclic_group}.
  Consider a group \(\mathbb{G} = \gengroup{g}\), and define 
  \(B\colon \mathbb{G} \times \mathbb{G} \to \mathbb{Z}_{\abs{\mathbb{G}}}\) such that 
  \(\call{B}{x, y} = g^{\call{\log}{x}\call{\log}{y}}\). The map \(B\) is bilinear 
  w.r.t.\ exponentiation, since:
  \[
    \forall a,b \in \mathbb{Z}\colon \call{B}{x^a, y^b} = g^{\call{\log}{x^a}\call{\log}{y^b}} = 
    g^{a\call{\log}{x}b\call{\log}{y}} = \Parens*{g^{\call{\log}{x}\call{\log}{y}}}^{ab}
    = \call{B}{x, y}^{ab}
  \]
\end{example}

The two main ingredients of Pinocchio are QAPs and bilinear maps.
Given in input an arithmetic formula \(\varphi \) over some field \(\mathbb{F} \cong \gengroup{g}\) 
equipped with a bilinear map \(B\colon \gengroup{g} \times \gengroup{g} \to \mathbb{F}\) defined as 
in \Cref{ex:bilinear_map}, the Pinocchio protocol is organized in three phases: the 
\emph{setup phase}, the \emph{prover phase}, and the \emph{verifier phase}.

\subsubsection*{Setup phase}
In the setup phase, any of the parties derives the explicit formula \(\explicit{\varphi}\), 
the circuit \(\mathcal{G}\) of \(\explicit{\varphi}\), the R1CS \(\mathcal{C}\) of \(\mathcal{G}\) 
and finally the \(m/n\) QAP \(\mathcal{Q} = \Tuple{t, \bm{v}, \bm{w}, \bm{y}}\) of \(\mathcal{C}\).
Recall that \(m = \abs{\bm{v}} = \abs{\bm{w}} = \abs{\bm{y}} = \abs{\mathcal{G}_{\otimes}}\)\footnote{
  As we will see, it is possible to reduce this number in some special cases by exploiting the 
  structure of R1CS constraints.
  Also note that the original protocol as in~\cite{ParnoGHR2013} does not make explicit use of 
  R1CS, but we want to stress its importance as it is the standard way to represent arithmetic 
  circuits in \texttt{libsnark}~\cite{SassonCTV2013}.
  }, 
that \(n = \call{\deg}{t} = m + 1 + \abs{\mathcal{G}_{in}}\), and that 
\(\forall i\colon \call{\deg}{\bm{v}_i} = \call{\deg}{\bm{w}_i} = \call{\deg}{\bm{y}_i} = n - 1\).
The QAP (and eventually all the other componenets) are made public.

Either via a trusted third party or an ensamble of authorities~\cite{GrothO2006}, we generate 
a CRS \(\overbar{\sigma}\), which has length 
\(\abs{\overbar{\sigma}} = 8\Ceil*{\call{\log}{\abs{\mathbb{F}}}}\) and interpret every 
\(\Ceil*{\call{\log}{\abs{\mathbb{F}}}}\)-bit subsequence as elements of \(\mathbb{F}\): 
\[\overbar{\sigma} = \Tuple{r_v, r_w, s, \alpha_v, \alpha_w, \alpha_y, \beta, \gamma}\]

Finally, after fixing \(r_y = r_{v}r_{w}\), we build the \emph{proving key} and the 
\emph{verification key}:
\begin{align*}
  & \ProverKey = 
  \begin{pmatrix*}
    \Set{g^{r_{v}\call{\bm{v}_j}{s}}}, &
    \Set{g^{r_{w}\call{\bm{w}_j}{s}}}, &
    \Set{g^{r_{y}\call{\bm{y}_j}{s}}}, \\
    \Set{g^{r_{v}\alpha_v\call{\bm{v}_j}{s}}}, &
    \Set{g^{r_{w}\alpha_w\call{\bm{w}_j}{s}}}, &
    \Set{g^{r_{y}\alpha_y\call{\bm{y}_j}{s}}}, \\
    \Set{g^{s^k}}, &
    \Set{g^{r_{v}{\beta}\call{\bm{v}_j}{s}}g^{r_{w}{\beta}\call{\bm{w}_j}{s}}g^{r_{y}{\beta}\call{\bm{y}_j}{s}}}
  \end{pmatrix*} \\
  & \VerifierKey = \Tuple{
    g, 
    g^{\alpha_v}, 
    g^{\alpha_w}, 
    g^{\alpha_y}, 
    g^{\gamma}, 
    g^{\beta\gamma}, 
    g^{r_{y}\call{t}{s}},
    \Set{g^{r_{v}\call{\bm{v}_i}{s}}}, 
    \Set{g^{r_{w}\call{\bm{w}_i}{s}}}, 
    \Set{g^{r_{y}\call{\bm{y}_i}{s}}}
  }
\end{align*}
where \(i\) ranges over \(\Iinterval*{0}{\abs{G_{in}}}\), \(j\) ranges over 
\(\Iinterval*{\abs{\mathcal{G}_{in}}}{n-1}\), and \(k\) ranges over \(\Iinterval*{1}{n}\).

It is important to remark that \(\overbar{\sigma}\) must be deleted immediately after generating 
the two keys, as it can be exploited to tamper the protocol 
(in jargon, we say that it is \emph{toxic waste}).

\subsubsection*{Prover phase}
After fixing the input vector \(\bm{x}\), the prover creates a circuit input for \(\mathcal{G}\), 
computes the induced evaluation and extracts the assignment \(\mathcal{A}\).
From \(\mathcal{A}\), the prover derives the associated solution \(\bm{c}\) of \(\mathcal{C}\),
and computes the polynomials 
\(p = \Parens*{\bm{vc}}\Parens*{\bm{wc}} - \Parens*{\bm{yc}}\) and \(h = p/t\).
Now, to build a short proof \(\pi \), the prover should sum the contributions of all the 
left input, right input, and output polynomials:
\begin{align*}
  & v_{\otimes} = \sum_{i=\abs{\mathcal{G}_{in}}}^{n-1}{\bm{c}_{i}\call{\bm{v}_{i}}{s}} &&
  w_{\otimes} = \sum_{i=\abs{\mathcal{G}_{in}}}^{n-1}{\bm{c}_{i}\call{\bm{w}_{i}}{s}} &&
  y_{\otimes} = \sum_{i=\abs{\mathcal{G}_{in}}}^{n-1}{\bm{c}_{i}\call{\bm{y}_{i}}{s}}
\end{align*}
and obtain the proof:
\[
  \pi = \Tuple{
    g^{r_{v}v_{\otimes}}, 
    g^{r_{w}w_{\otimes}}, 
    g^{r_{y}y_{\otimes}}, 
    g^{\call{h}{s}},
    g^{r_{v}\alpha_{v}v_{\otimes}}, 
    g^{r_{w}\alpha_{w}w_{\otimes}}, 
    g^{r_{y}\alpha_{y}y_{\otimes}}, 
    g^{r_{v}{\beta}v_{\otimes}}g^{r_{w}{\beta}w_{\otimes}}g^{r_{y}{\beta}y_{\otimes}}
    }
\]
Although \(r_v\), \(r_w\) or \(r_y\) have been deleted, and finding the discrete logarithm is hard,
by using \(\ProverKey \), it is possible to compute:
\begin{align}\label{eq:compute_proof}
  & g^{r_{v}v_{\otimes}} = 
  \prod_{i=\abs{\mathcal{G}_{in}}}^{n - 1}{\Parens*{g^{r_{v}\call{v_{i}}{s}}}^{\bm{c}_{i}}} &&
  & g^{r_{w}w_{\otimes}} = 
  \prod_{i=\abs{\mathcal{G}_{in}}}^{n - 1}{\Parens*{g^{r_{w}\call{w_{i}}{s}}}^{\bm{c}_{i}}} &&
  & g^{r_{y}y_{\otimes}} = 
  \prod_{i=\abs{\mathcal{G}_{in}}}^{n - 1}{\Parens*{g^{r_{y}\call{y_{i}}{s}}}^{\bm{c}_{i}}}
\end{align}
and similarly for the other values (except for \(g^{\call{h}{s}}\) which can be computed directly).
Finally, the prover publishes the pair \(\Tuple{\call{\phi}{\bm{x}}, \pi}\).

\subsubsection*{Verifier phase}
At this point, any potential verifier with access to \(\VerifierKey \) and to the bilinear map 
\(B\) can check the alleged proof.
First, the verifier should check whether \(p = ht\), which, as we noted in \Cref{subsec:qap}, it is 
statistically equivalent to checking whether \(\call{p}{s} = \call{h}{s}\call{t}{s}\).
In the spirit of \Cref{eq:compute_proof}, using \(\VerifierKey \), the verifier can compute the 
input/output contributions \(g^{r_{v}v_{I/O}}\), \(g^{r_{w}w_{I/O}}\) and \(g^{r_{y}y_{I/O}}\).
Clearly, for any arbitrary \(k \in \mathbb{F} \setminus \Set{0}\), we have that 
\(\call{p}{s} = \call{h}{s}\call{t}{s} \iff k^{\call{p}{s}} = k^{\call{h}{s}\call{t}{s}}\) (we 
``work in the exponent'').
If we fix \(k = \call{B}{g^{r_v}, g^{r_w}}\), with some algebraic effort the previous 
equation can be transformed into the equivalent one:
\[
  \call{B}{g^{r_{v}v_{I/O}}g^{r_{v}v_{\otimes}}, g^{r_{w}w_{I/O}}g^{r_{w}w_{\otimes}}} =
  \call{B}{g^{r_{y}\call{t}{s}}, g^{\call{h}{s}}}\call{B}{g^{r_{y}y_{I/O}}g^{r_{y}y_{\otimes}}, g}
\]

Even if the divisibility check passes, it might still be the case that the alleged 
proof was not built using the polynomials in \(\mathcal{Q}\).
To address this eventuality, the verifier checks whether:
\begin{align*}
  &  \call{B}{g^{r_{v}\alpha_{v}v_{\otimes}}, g} = \call{B}{g^{r_{v}v_{\otimes}}, g^{\alpha_v}}
  && \call{B}{g^{r_{w}\alpha_{w}w_{\otimes}}, g} = \call{B}{g^{r_{w}w_{\otimes}}, g^{\alpha_w}}
  && \call{B}{g^{r_{y}\alpha_{y}y_{\otimes}}, g} = \call{B}{g^{r_{y}y_{\otimes}}, g^{\alpha_y}}
\end{align*}

Finally, the prover might have used the correct polynomials, but didn't use the same 
coefficients \(\bm{c}_i\) when building \(v_{\otimes}\), \(w_{\otimes}\) and \(y_{\otimes}\).
This last concern is resolved by checking whether:
\[
  \call{B}{g^{r_{v}{\beta}v_{\otimes}}g^{r_{w}{\beta}w_{\otimes}}g^{r_{y}{\beta}y_{\otimes}}, g^{\gamma}} =
  \call{B}{g^{r_{v}v_{\otimes}}g^{r_{w}w_{\otimes}}g^{r_{y}y_{\otimes}}, g^{\beta\gamma}}
\]

The protocol as described is not zero-knowledge, but it is quite simple to make it so:
the verification key is extended to include \(g^{r_{v}\alpha_{v}\call{t}{s}}\), 
\(g^{r_{v}\beta{v}\call{t}{s}}\), and then the prover generates a random value \(\delta_v\) and 
replaces each polynomial \(\bm{v}_i\) with \(\bm{v}_i + \delta_{v}t\). 
The same thing is done for \(\bm{w}\) and \(\bm{y}\): the validity of the checks performed by the 
verifier is not affected by this change, but it can be shown that the scheme is now 
statistically zero-knowledge.

It is immediate to see that the proving key \(\ProverKey \) contains 
\(8\abs{\mathcal{G}_{\otimes}} + \abs{\mathcal{G}_{in}} + 1\) field elements, the verification 
key \(\VerifierKey \) contains \(3\abs{\mathcal{G}_{in}} + 7\) field elements (\(6\) more in the 
zero-knowledge setting), and the proof \(\pi \) contains \(8\) field elements, meaning that its size
is independent from the input circuit.

\subsection{\texttt{libsnark}}
In the last ten years, new improvements were put forward to reduce the size of the messages and 
the complexity of the computation, especially on the prover's side~\cite{Lipmaa2013} (for example, 
in~\cite{Groth2016} the size of the proof was reduced to just \(3\) field elements).
Furthermore, much effort has been put into making working implementations of ZK-SNARK 
systems, such as \texttt{libsnark}\footnote{\url{https://github.com/scipr-lab/libsnark}, you can also find 
a nice empirical comparison of the various protocols.}, a
C\texttt{++}~\cite{Stroustrup2013} library which implements and refines several ZK-SNARK 
protocols~\cite{DanezisFGK2014,GrothM2017,BackesBFR2014}, although the core component is based 
mostly on~\cite{SassonCTV2013,SassonCGTV2013} and~\cite{Groth2016}, which are in spirit very 
similar to the Pinocchio protocol, improving on the QAP (and QSP) model 
of~\cite{GennaroGPR2012}.

In \texttt{libsnark}, if we want to implement some function \(f\) which can be expressed as an 
arithmetic expression \(\varphi \) (i.e.\ no variable-length loops or recursion), we must provide 
both the arithmetic circuit and the associated R1CS\@.
In simple cases, deriving the R1CS from the arithmetic circuit might seem trivial, but there are 
instances where having them separate allows for some quite nice optimizations\footnote{One could 
argue that a \emph{smart} compiler should be able to perform such optimizations automatically, 
however, the one that we are aware of only perform the trivial translations.}.
