\section{\texttt{libsnark}}\label{sec:libsnark}
In the last ten years, new improvements were put forward to reduce the size of the messages and 
the complexity of the computation, especially on the prover's side~\cite{Lipmaa2013} (for example, 
in~\cite{Groth2016} the size of the proof was reduced to just \(3\) field elements).
Furthermore, much effort has been put into making working implementations of ZK-SNARK 
systems, such as \texttt{libsnark}\footnote{\url{https://github.com/scipr-lab/libsnark}, you can 
also find a nice empirical comparison of the various protocols.}, 
a C\texttt{++}~\cite{Stroustrup2013} library which implements and refines several ZK-SNARK 
protocols~\cite{DanezisFGK2014,GrothM2017,BackesBFR2014,SassonCGTV2013}, although the core 
component (pre-processing ZK-SNARK, or PPZK-SNARK) is based mostly on the Pinocchio protocol and 
on~\cite{SassonCTV2013,Groth2016}, which are all extensions and improvements of the QAP (and QSP) 
model of~\cite{GennaroGPR2012}.

In order to implement some function \(f\) which can be expressed as an 
arithmetic expression \(\varphi \) (i.e.\ no variable-length loops or recursion) in 
\texttt{libsnark}, we must provide both the arithmetic circuit and the associated R1CS\@.
In many cases, deriving the R1CS from the arithmetic circuit is quite trivial, but there are 
instances where having them separate allows for some quite nice optimizations.
Although there has been work on \emph{compilers} that translate high-level code to R1CS 
constraints~\cite{EberhardtT2018,BellesBDM2022}, as it is often the case, this comes at a cost 
of flexibility.
Intuitively, we can divide the usage of \texttt{libsnark} in same three phases of the Pinocchio 
system (or really any other SNARK).

\subsubsection*{Preprocessing phase}
The first thing to do in \texttt{libsnark} is choosing, at compile time (or \emph{a priori}, in a 
theoretical interpretation), which bilinear group to use for the protocol. 
To obtain bilinear groups that are paired to an underlying prime field, the typical construction 
makes use of elliptic curves: for example, the curve/group BN254~\cite{BarretoN2005}, which is the 
standard curve used \texttt{libsnark}, or the curve
BLS12\footnote{\url{https://electriccoin.co/blog/new-snark-curve/}}~\cite{BonehLS2001}, which 
will also be of interest to us.

After choosing the group, we setup the \emph{protoboard} which, as the 
name suggests, it is the object where one places the components of the circuit:
\begin{lstlisting}[language=C++]
  libsnark::protoboard<Field> board;
  board.set_input_sizes(PUBLIC_N);
\end{lstlisting}
The template argument \texttt{Field} specifies the underlying field, and \texttt{PUBLIC\_N} 
specifies the number of output (i.e.\ public) variables in the circuit.
On the protoboard, we allocate \emph{variables} that will carry the input/intermediate/output
values of the circuit evaluation, together with an annotation (for debug).
However, it is usually much more convenient to only declare the input and output variables, and 
delegate the wiring to \emph{gadgets} which act as composable black-boxes:
\begin{lstlisting}[language=C++]
  libsnark::pb_variable<Field> output_var;
  libsnark::pb_variable<Field> input_var;

  output_var.allocate(board, "out");
  input_var.allocate(board, "in");
  FooGadget gadget{board, input_var, output_var};
\end{lstlisting}
Note that the first \texttt{PUBLIC\_N} variables allocated will be considered public, while the 
remaining ones will be private.
A typical gadget must provide two methods: one to generate the R1CS constraints, and one to 
compute the circuit evaluation given a circuit input, which will be used in the proving phase.
First, we generate the constraints:
\begin{lstlisting}[language=C++]
  gadget.generate_r1cs_constraints();
\end{lstlisting}
Internally, \texttt{gadget} allocates the required intermediate variables and constrains their 
linear combinations. 
A linear combination can be expressed quite naturally:
\begin{lstlisting}[language=C++] 
  libsnark::pb_linear_combination<Field> lc = a1*x1 + ... + an*xn;
\end{lstlisting}
An R1CS constraint of the type \(\bm{ab} = \bm{c}\) is expressed as:
\begin{lstlisting}[language=C++]
  libsnark::r1cs_constraint<Field> constraint{lc_a, lc_b, lc_c};
  board.add_r1cs_constraint(constraint);
\end{lstlisting}
Once all the constraints have been specified, the last thing to do is to convert the R1CS to a 
QAP and get the keys \(\ProverKey \) and \(\VerifierKey \).
The whole process is done transparently by the library:
\begin{lstlisting}[language=C++]
  auto keypair = libsnark::r1cs_ppzksnark_generator(board.get_constraint_system());
\end{lstlisting}
Now the constraint system and \(\VerifierKey \) can be made public, while \(\ProverKey \) is only 
known by the prover. 

\subsubsection*{Proving phase}
To generate a (valid) proof, the prover has to first provide a circuit input and compute the 
induced evaluation. 
At the highest level, this is done by setting the values of the input variables and letting the 
gadget generate the intermediate and the output values:
\begin{lstlisting}[language=C++]
  Field x; // some field element
  board.val(input_var) = x;
  gadget.generate_r1cs_witness(circuit_inputs);
\end{lstlisting}
Now, he can generate the proof, which is done transparently by the library by using the 
proving key and the primary (i.e.\ public) and auxiliary (i.e.\ private) inputs:
\begin{lstlisting}[language=C++]
  auto proof = libsnark::r1cs_ppzksnark_prover(keypair.pk, board.primary_input(), board.auxiliary_input());
\end{lstlisting}
Finally, the prover can make public the primary inputs of the protoboard together with the
proof.

\subsubsection*{Verification phase}
At this point, the verifier has to check the proof by using the public key, and there are two 
kinds of choices regarding how to perform such verification: 
\begin{itemize}
  \item \emph{weak} vs.\  \emph{strong} verification: in the former case, it is ok for the prover 
        to only provide some of the primary inputs, which will be zero-padded. 
        In the latter case, this is not considered acceptable.
  \item \emph{offline} vs.\  \emph{online} verification: in the former case, the verification key 
        is used `as is', in the latter, it is processed to get faster verification times when used 
        multiple times.
\end{itemize}
In any case, the whole thing is managed transparently by the library.
For example, if we choose strong, online verification:
\begin{lstlisting}[language=C++]
  auto proc_vk = libsnark::r1cs_ppzksnark_verifier_process_vk(keypair.vk);
  bool halt = libsnark::r1cs_ppzksnark_online_verifier_strong_IC(proc_vk, board.primary_input(), proof);
\end{lstlisting}
The verifier will accept the proof if \texttt{halt} is \texttt{true}, while it will reject if it 
is \texttt{false}.

\begin{example}
  Consider the R1CS in \Cref{ex:r1cs}.
  \Cref{lst:libsnark_example} shows the implementation of the corresponding gadget in 
  \texttt{libsnark}.
\end{example}

\begin{algorithm}
  \centering
  \begin{lstlisting}[caption={The \texttt{libsnark} gadget corresponding to the 
    R1CS in \Cref{ex:r1cs}.}\label{lst:libsnark_example},language=C++]
    template<typename FieldT>
    class FooGadget : public gadget<FieldT>
    {
        const pb_variable<FieldT> x1, x2, y;
        pb_variable<FieldT> t1, t2, t3;

        FooGadget(protoboard<FieldT> &board, 
                  const std::string &ann,
                  const pb_variable<FieldT> &x1, 
                  const pb_variable<FieldT> &x2, 
                  const pb_variable<FieldT> &y) : 
            gadget<FieldT>{board, ann}, x1{x1}, x2{x2}, y{y}
        {
            t1.allocate(board, "t1");
            t2.allocate(board, "t2");
            t3.allocate(board, "t3");
        }

        void generate_r1cs_constraints()
        {
            board.add_r1cs_constraint(r1cs_constraint<FieldT>{x1, x1, t1});
            board.add_r1cs_constraint(r1cs_constraint<FieldT>{t1, x1, 8 + 9*x2 + t2});
            board.add_r1cs_constraint(r1cs_constraint<FieldT>{x2, t2, y});
        }

        void generate_r1cs_witness()
        {
            board.val(t1) = board.val(x1) * board.val(x1);
            board.val(t2) = board.val(t1) * board.val(x1) + 4*board.val(x2) + 5;
            board.val(y)  = board.val(x2) * board.val(t2);
        }
    };
  \end{lstlisting}
\end{algorithm}

