\chapter{Cryptographical Background}\label{chap:crypto}
Cryptography, quite unsurprisingly, is the field in which zero-knowledge proof systems have 
got the most attention.
The possibility of two or more parties to cooperate and exchange information one with another in a 
zero-knowledge manner is the fundamental idea behind many branches of cryptography such as 
\emph{Multi Party Computation} (MPC)~\cite{Yao1982-2} and \emph{Fully Homomorphic Encryption} 
(FHE)~\cite{ArmknechtEtAl2015}.

An important application of zero-knowledge protocols lies in \emph{verifiable computation}:
a server (the prover) wants to convince some users (the verifiers) that some function has been 
executed properly, without revealing the inputs.

In \Cref{sec:hash_functions}, we review the fundamental notions concerning 
\emph{cryptographic hash functions}, and some of the standard ways to construct them.
In \Cref{sec:tree_hash}, we go in more depth over \emph{tree-like modes of hashing}, which 
constitute the basis of some of the most important use-cases of verifiable computation.
In \Cref{sec:zk-snark} we study the construction of zero-knowledge succint NARKs, with a focus on 
the \emph{Pinocchio} protocol~\cite{ParnoGHR2013}.
Finally, in \Cref{sec:libsnark} we describe \texttt{libsnark}, a C\texttt{++} library which 
implements (a variant of) the Pinocchio protocol. 

\section{Secure Hash functions}
Secure Hash functions are a fundamental tool of cryptography, as they can be used to produce 
\(\BigO{1}\) or \(\BigO{\call{\log}{n}}\) commitments for any message of length \(\abs{n}\).
\begin{definition}[Hash function]
  Given some \(n \in \mathbb{N}\), an \emph{\(n\)-bit hash function} is a function 
  \(H\colon \Set{0, 1}^{*} \to \Set{0, 1}^{n}\).
\end{definition}
The input of an hash function is called the \emph{message}, while its output is called the 
\emph{digest}.
From the definition, it is immediate to see that there are an infinite number of messages which map
to the same digest.
Now, a very simple hash function might be truncation (i.e.\ take the first \(n\) bits and discard 
anything coming afterwards), but it is not of much interest for cryptography, as we require 
additional \emph{security guarantees}.
\begin{remark}
  When we say that it is \emph{easy} (resp.\  \emph{hard}) to compute a function, we mean that 
  there is (resp.\ there is not) a probabilistic Turing machine which can compute such function in
  at most polynomial time.
  In particular, a \emph{one-way function} is an easily computable function \(f\) whose inverse is 
  hard to compute, and a \emph{trapdoor function} is a one-way function whose inverse becomes easy 
  to compute given some additional knowledge, called the \emph{key}.
\end{remark}

\begin{definition}[Cryptographic hash function~\cite{AlKuwariDB2011}]
	Given \(n \in \mathbb{N}\), an \emph{\(n\)-bit cryptographic hash function (CHF)} is a \(n\)-bit 
  hash function which satisfies the following properties:
	\begin{itemize}
		\item \textbf{Collision resistance}: It is hard to find two messages \(m_1, m_2\) such
		      that \(\call{H}{m_1} = \call{H}{m_2}\).
		\item \textbf{Preimage resistance}: Given some digest \(d\), it is hard to find a
		      message \(m\) such that \(\call{H}{m} = d\) (\(H\) is a one-way function).
		\item \textbf{Second preimage resistance}: Given some message \(m_1\), it is hard to
		      find a message \(m_2\) such that \(\call{H}{m_1} = \call{H}{m_2}\).
	\end{itemize}
\end{definition}

Note that second preimage resistance implies first preimage resistance (if given any \(m_1\) 
we can find a colliding \(m_2\), we can do it also without being given \(m_1\)), and in turn 
preimage resistance implies second preimage resistance (if given any \(d\) we can find a colliding 
\(m\), then given any \(m'\) we can compute \(d\) and find the collision).

With high probability, a perfect \(n\)-bit CHF provides \(n/2\) bits of security
for collision resistance, meaning that even an optimal adversary needs at least \(\BigO{2^{n/2}}\) 
time to find a collision, while for preimage resistance it provides \(n\) bits of security.
A CHF can be built by applying some known secure construction to functions which are
simpler to devise.
\begin{definition}[Padding function]
	An \(n\)-bit \emph{padding function} is a function 
	\(\Pad\colon \Set{0, 1}^{*} \to \Parens{\Set{0, 1}^n}^{*}\).
\end{definition}
\begin{definition}[Pseudorandom permutation]
	Given \(l \in \mathbb{N}\), an \(l\)-bit \emph{pseudorandom permutation} is a permutation 
	\(P\colon \Set{0, 1}^l \to \Set{0, 1}^l\) which is indistinguishable from an uniform random 
	distribution.
\end{definition}
\begin{definition}[Keyed permutation]
	Given \(l, n \in \mathbb{N}\), an \emph{\(l/n\)-bit keyed permutation} is
	a function \(F\colon \Set{0, 1}^l \times \Set{0, 1}^n \to \Set{0, 1}^l\) which is a permutation 
  on its first parameter.
\end{definition}
\begin{definition}[Block cipher]
  A \emph{block cipher} is a trapdoor pseudorandom keyed permutation.
\end{definition}
\begin{definition}[One-way compression function]
	Given \(l_1, n, l_2 \in \mathbb{N}\) such that \(l_1 + n > l_2\), an 
	\emph{\(l_1/n/l_2\)-bit one-way compression function (OWCF)} is a one-way function 
	\(F\colon \Set{0, 1}^{l_1} \times \Set{0, 1}^{n} \to \Set{0, 1}^{l_2}\).
\end{definition}

A \emph{pseudorandom keyed permutation} (PKP) is typically built by iterating a 
``somewhat pseudorandom'' keyed permutation \(F\) for an adequate number of \emph{rounds}, until 
inverting the function becomes hard.
A block cipher can be built from a PKP by following some secure construction scheme, 
like the Feistel-Luby-Rackoff construction~\cite{MenezesOV2001}.
A OWCF can also be derived from a PKP or a block cipher, by applying a secure construction scheme, 
like the Davies-Meyer construction~\cite{Preneel2005}.
Finally, CHF can again be derived either by a OWCF, for exaple through the Merkle-Damg\r{a}rd 
construction~\cite{Merkle1979}, or directly from a pseudorandom permutation, like in the sponge 
construction~\cite{BertoniDPA2007, Tiwari2017}.
\begin{proposition}[Feistel-Luby-Rackoff construction~\cite{MenezesOV2001}]
	Given an \(l/n\)-bit pseudorandom keyed permutation \(P\), some message
  \(m = \Tuple{m_1, m_2}\) such that \(m_1, m_2 \in \Set{0, 1}^l\), a number of rounds 
  \(r > 3\), and a set of keys \(k_1, \dots, k_{r} \in \Set{0, 1}^n\), then the function \(E_{r}\) 
	is a \(2l/n\) block cipher, where:
  \[
		 \call{E_{i}}{m, k_i} = \Tuple{x_i, y_i} = 
		\begin{cases}
			\Tuple{m_1, m_2}                                       & i = 0 \\
			\Tuple{y_{i-1}, x_{i-1} \oplus \call{P}{y_{i-1}, k_i}} & 1 \le i \le r
		\end{cases}
	\]
\end{proposition}

\begin{proposition}[Davies-Meyer construction~\cite{Preneel2005}]
	Given an \(l/n\)-bit pseudorandom keyed permutation \(P\), some number of blocks \(b \in \mathbb{N}\), 
  some \emph{initial value} \(v \in \Set{0, 1}^l\), and some message 
	\(m = \Tuple{m_1, \dots, m_b}\) such that each \(m_i \in \Set{0, 1}^{n}\), then the function 
	\(F_{b}\) is an \(l/{bn}/l\) one-way compression function, where:
  \[
		\call{F_{i}}{v, m} =
		\begin{cases}
			v                                													& i = 0 \\
			\call{P}{\call{F_{i-1}}{v, m}, m_i} \oplus \call{F_{i-1}}{v, m} & 1 \le i \le b \\
		\end{cases}
	\]
\end{proposition}

\begin{proposition}[Merkle-Damg\r{a}rd construction~\cite{Merkle1979}]
	Given an \(l/n/l\)-bit one way compression function \(F\), some initial value 
	\(v \in \Set{0, 1}^{l}\), some message \(m \in \Set{0, 1}^*\), a padding extension length \(k\) 
	and an \(n\)-bit padding function \(\Pad \) such that, 
	\(\forall x, y \in \Set{0, 1}^{*}\):
	\begin{align*}
    & \abs{\call{\Pad}{x}} = \abs{x} + \Parens*{-\abs{x} \bmod n} + kn \\
		& \abs{x} = \abs{y} \implies \abs{\call{\Pad}{x}} = \abs{\call{\Pad}{y}} \\
    & \abs{x} \neq \abs{y} \implies 
			\call{\Pad}{x}_{\abs{\call{\Pad}{x}}} \neq \call{\Pad}{y}_{\abs{\call{\Pad}{y}}}
  \end{align*}
	then the function \(H_{\abs{\call{\Pad}{m}}}\) is an \(l\)-bit cryptographic hash function, 
	where:
	\[
		\call{H_{i}}{m} =
		\begin{cases}
			v                                               & i = 0   \\
			\call{F}{\call{H_{i-1}}{m}, \call{\Pad}{m}_{i}} & 1 \le i \le \abs{\call{\Pad}{m}}
		\end{cases}
	\]
\end{proposition}

\begin{proposition}[Sponge construction~\cite{BertoniDPA2007}]
	Given an \(l\)-bit pseudorandom permutation \(P\), a message \(m \in \Set{0, 1}^{*}\), a 
	\emph{rate} \(r\) and a \emph{capacity} \(c\) such that \(r + c = l\), an initial value 
	\(v \in \Set{0, 1}^b\) and an \(r\)-bit padding function \(\Pad \), the 
	function \(H_{\abs{\call{\Pad}{m}}}\) is an \(r-bit\) cryptographic hash function, where:
	\begin{align*}
		& \call{S_{i}}{m} = 
		\begin{cases}
			v & i = 0 \\
			\call{P}{\call{S_{i-1}}{m} \oplus \call{\Pad}{m}_i} & 1 \le i \le \abs{\call{\Pad}{m}}
		\end{cases} \\
		& \call{H_{i}}{m} = \call{S_{i}}{m}_{1,\dots,r}
	\end{align*}
\end{proposition}

The sponge construction is particularly interesting as it is very flexible: given a pseudorandom 
permutation, it can be used to build pseudorandom keyed permutations, cryptographic hash functions, 
random number generators and authenticated encryptions~\cite{BertoniDPA2012}.

\section{Tree-like modes of hashing}\label{sec:tree_hash}
Consider an \(n\)-bit CHF \(H\), and suppose that a prover claims to know some message \(m\): 
the digest \(d = \call{H}{m}\) can be considered as a \emph{short binding commitment} for \(m\): 
By asking the prover to share the digest, whose size \(\abs{d} = n\) is typically considered to be 
\(\BigO{1}\) (or \(\BigO{\call{\log}{\abs{m}}}\) in some cases), a verifier is convinced that the 
prover does know \(m\) with probability \(\approx 1 - {1}/{2^n}\).
A modern standard CHF like SHA-256~\cite{Dang2015} produces digests of length at least \(256\) bits,
making the \(1 - {1}/{2^n}\) bound really hard to bruteforce through.
Note that the verifier needs not to know \(m\) in advance: the commitment \(d\) is (temporarily) 
appended to a public \emph{blockchain} and, at any point in the future, when the verifier becomes 
aware of some \(m'\) provided by the prover, if \(\call{H}{m'} = d\), the commitment can be 
approved or rejected.

Now, suppose that the prover wants to commit to a list of \(k\) messages: the simplest solution 
would be to publish the hash of every message, which would require to append \(\BigO{k}\) 
elements on the blockchain.
Another way would be for the prover to share \(\call{H}{\Tuple{m_1, \dots, m_k}}\): the 
communication cost would only be \(\BigO{1}\) but, in general, not all the messages belong to the 
same prover, so this method would not work, and we need a better solution.

\subsection{Merkle tree}
\begin{definition}[Binary Merkle tree~\cite{Merkle1988}]
	A \emph{binary Merkle tree (MT)} of height \(h \in \mathbb{N}\) over a \(2n\)-to-\(n\) compression 
	function \(C\), is the complete binary tree of height \(h\) such that, given a sequence of input 
	messages \(\Tuple{m_1, \dots, m_{2^{h-1}}}\) over \(\Set{0, 1}^{2n}\), produces an
	output digest \(d \in \Set{0, 1}^{n}\) in the following way:
	\begin{enumerate}
		\item The leaf nodes \(\nu_1, \dots, \nu_{2^{h-1}}\) contain 
					\(\call{C}{m_1}, \dots, \call{C}{m_{2^{h-1}}}\).
		\item Every other node \(\nu \) contains \(\call{C}{\nu_l, \nu_r}\), where \(\nu_l\) is
		      the left child of \(\nu \) and \(\nu_r\) is the right child of \(\nu \).
		\item The output digest \(d\) is the content of the root node. 
	\end{enumerate}
\end{definition}

The set of the sibling nodes visited in the path from a leaf of the tree to the root, including the 
leaf itself, is the \emph{authentication path} of the leaf.
By using Merkle trees, the prover only needs to send to the verifier, as a commitment for
some message \(m_i\) among \(n = 2^h\) messages, the contents of the co-path from the leaf 
containing \(m_i\) to the root, in addition plus the hash of \(m_i\): this requires
just \(\BigO{\call{\log}{n}}\) cost to validate the commitment.
Merkle trees bottom-up construction is very easy to parallelize, and they can be used in the 
multiple-provers scenario: each prover only needs to commit to the path from its own leaf to the 
root of the tree.
It is immediate to generalize the notion of binary Merkle tree to arbitrary arity.
\begin{proposition}[Security of Merkle tree mode of hash~\cite{Merkle1988}]
	Given a one-way \(tn\)-to-\(n\) compression function \(C\), the \(t\)-ary Merkle tree over 
	\(C\) is a cryptographic hash function.
\end{proposition}

\begin{example}\label{ex:merkle_tree}
	Consider the sequence of pre-hashed messages \(S = \Tuple{3, 4, 7, 7}\) and the 
	compression function 
	\(\call{C}{x, y}: \Tuple{x, y} \mapsto \Parens*{xy \bmod 13} + 1\) (for ease of exposition, 
	we work over integers instead of bit strings, but the two can be readily converted into one 
	another).
	\Cref{fig:merkle_tree} shows the contents of the associated Merkle Tree.
	Note that the real message is not stored in the Merkle Tree, but only the `first level' of hashes.
	The authentication path of the leaf labelled with \(3\) consists of the tuple \(\Tuple{3, 4, 11}\):
	by computing \(\call{H}{3, 4} = 13\) and then \(\call{H}{13, 11} = 1\) we can verify that the 
	commitment is respected.
\end{example}
\begin{figure}
	\centering
	\begin{tikzpicture}[node distance={32pt}, node/.style = {draw, circle},on grid=true]
		\node[node] (x1) {\(3\)};
		\node[node,draw=none] (n1) [right of=x1] {};
		\node[node] (x2) [right of=n1] {\(4\)};
		\node[node,shape=rectangle] (c1) [above of=n1] {\(C\)};
		\node[node,draw=none] (n2) [right of=x2] {};
		\node[node] (x3) [right of=n2] {\(7\)};
		\node[node,draw=none] (n3) [right of=x3] {};
		\node[node] (x4) [right of=n3] {\(7\)};
		\node[node,shape=rectangle] (c2) [above of=n3] {\(C\)};
		\node[node] (x5) [above of=c1] {\(13\)};
		\node[node,draw=none] (n3) [above of=n2] {};
		\node[node,draw=none] (n4) [above of=n3] {};
		\node[node] (x6) [above of=c2] {\(11\)};
		\node[node,shape=rectangle] (c3) [above of=n4] {\(C\)};
		\node[node] (x7) [above of=c3] {\(1\)};
		\draw[->] (x1) to (c1);
		\draw[->] (x2) to (c1);
		\draw[->] (x3) to (c2);
		\draw[->] (x4) to (c2);
		\draw[->] (c1) to (x5);
		\draw[->] (c2) to (x6);
		\draw[->] (x5) to (c3);
		\draw[->] (x6) to (c3);
		\draw[->] (c3) to (x7);
	\end{tikzpicture}
	\caption{Merkle tree of \Cref{ex:merkle_tree}.}\label{fig:merkle_tree}
\end{figure}

\subsection{Augmented Binary Tree}
The Merkle tree is the de-facto standard for blockchain applications, and basically for any 
scenario for which a `linear' hash function cannot be used.
In~\cite{Stam2008}, it was given a lower bound on the amount of queries necessary to obtain a 
collision for a \(\Parens*{m+s}\)-to-\(s\)-bit CHF \(H\) (the \(m\) is variable) built from a 
\(\Parens*{n+c}\)-to-\(n\)-bit OWCF \(C\): if \(H\) makes \(r\) queries to \(C\), it is possible 
to find a collision by making \(2^{\frac{nr + cr - m}{r + 1}}\) queries to \(H\).
By combining this result with the \(2^{s/2}\) upper bound of the birthday paradox, one can 
immediately obtain a tight bound \(m = \frac{2nr + 2cr -sr - s}{2}\) for the variable length \(m\) 
of the message.

\begin{definition}[Compactness~\cite{AndreevaBR2021}]
	The \emph{compactness} of an \(\Parens*{m+s}\)-to-\(s\)-bit hash function making \(r\) queries to 
	an underlying \(\Parens*{n+c}\)-to-\(n\)-bit one-way compression function is the value
	\(\alpha = \frac{2m}{2nr + 2cr -sr - s}\).
\end{definition}

\begin{example}\label{ex:mtree_compactness}
	Consider a \(2n\)-to-\(n\) bit OWCF and a Merkle Tree of height \(h\): the computation 
	of the tree is a \(\Parens*{2^{h-1}n}\)-to-\(n\)-bit hash function, and makes exactly 
	\(r = 2^{h-1} - 1\) queries to \(C\).
	We have \(s = c = n\) and \(m = 2^{h-1}n - n = nr\), therefore the compactness of the Merkle 
	Tree construction is:
	\[
		\alpha = \frac{2m}{2nr + 2cr -sr - s} = 
		\frac{2nr}{2nr + 2nr - nr - n} =
		\frac{2r}{3r - 1}
	\]
	Which tends to \(2/3\) when \(r\) tends to infinity.
\end{example}

\begin{definition}[Augmented Binary tRee~\cite{AndreevaBR2021}]
	An \emph{Augmented Binary tRee (ABR)} of height \(h \in \mathbb{N}\) over a 
	\(2n\)-to\(n\) compression function \(C\) is a complete binary tree of height \(h\) 
	augmented with \emph{middle} nodes such that, given a sequence of input messages
	\(S = \Tuple{m_1, \dots, m_{2^{h-1} + 2^{h-2}-1} \mid \forall i\colon m_i \in \Set{0, 1}^{*}}\), 
	it produces an output digest \(d \in \Set{0, 1}^n\) in the following way:
	\begin{enumerate}
		\item The leaf nodes \(\nu_{1}, \dots, \nu_{2^{h-1}}\) contain \(\call{C}{m_1}, \dots,
		      \call{C}{m_{2^{h-1}}}\).
		\item There are no middle nodes in the leaf layer.
		\item The middle nodes \(\nu_{2^{h-1}+1}, \dots, \nu_{\abs{S}}\) contain
		      \(\call{C}{m_{2^{h-1}+1}}, \dots, \call{C}{m_{\abs{S}}}\).
		\item Every other node \(\nu \) contains \(\call{C}{\nu_l \oplus \nu_m, \nu_r \oplus
		      \nu_m} \oplus \nu_r \), where \(\nu_l\) is the left child of \(\nu \), \(\nu_r\)
		      is the right child of \(\nu \), and \(\nu_m\) is the middle child of \(\nu \), or \(0\)
		      if \(\nu \) doesen't have a middle child.
	\end{enumerate}
\end{definition}

The authentication path of the ABR is similar to the one of the Merkle tree, but also includes 
the middle nodes encountered during the traversal.

\begin{proposition}[Security of ABR mode of hash~\cite{AndreevaBR2021}]
	Given a one-way \(2n\)-to-\(n\) compression function \(C\), the ABR over \(C\) is a cryptographic 
	hash function.
\end{proposition}

An ABR of height \(h\) can process 50\% more messages than a Merkle Tree of the same height, 
while performing the same number of queries to the underlying compression function, with the 
additional cost introduced by the intermediate \(\oplus \) operations being negligible in most 
scenarios.

\begin{example}
	Consider a \(2n\)-to-\(n\) bit OWCF and an ABR of height \(h\): the computation 
	of the tree is a \(\Parens*{2^{h-1} + 2^{h-2}-1}n\)-to-\(n\)-bit hash function, 
	and makes exactly \(r = 2^{h-1} - 1\) queries to \(C\).
	Like in \Cref{ex:mtree_compactness}, we have \(s = c = n\), but this time 
	\(m = \Parens*{2^{h-1} + 2^{h-2}-1}n - n = nr + {nr}/2 - n\), so the compactness of the ABR 
	construction is:
	\[
		\alpha = \frac{2m}{2nr + 2cr - sr - s} = 
		\frac{2nr + nr - 2n}{2nr + 2nr - nr - n} =
		\frac{3r - 2}{3r - 1}
	\]
	Which approaches \(1\) as \(r\) approaches infinity, meaning that the ABR construction achieves
	optimal compactness.
\end{example}

It is worth of notice that, while the ABR hash mode achieves collision resistance, it does not 
achieve \emph{indifferentiability} (a weaker notion of indistinguishability between Turing 
machines~\cite{MaurerRH2003}), hence a modified construction, called ABR+, was also proposed, 
although it does not achieve perfect compactness.
\begin{figure}
	\centering
	\begin{tikzpicture}[node distance={32pt}, node/.style = {draw, circle},on grid=true]
		\node[node] (x1) {\(3\)};
		\node[node,draw=none] (n1) [right of=x1] {};
		\node[node] (x2) [right of=n1] {\(4\)};
		\node[node,shape=rectangle] (c1) [above of=n1] {\(C\)};
		\node[node,draw=none] (n2) [right of=x2] {};
		\node[node] (x3) [right of=n2] {\(7\)};
		\node[node,draw=none] (n3) [right of=x3] {};
		\node[node] (x4) [right of=n3] {\(7\)};
		\node[node,shape=rectangle] (c2) [above of=n3] {\(C\)};
		\node[node,draw=none] (n4) [above of=n2] {};
		\node[node] (x8) [above of=n4] {\(10\)};
		\node[node] (x5) [above of=c1] {\(13\)};
		\node[node] (x6) [above of=c2] {\(11\)};
		\node[node,draw=none] (n5) [above of=x8] {};
		\node[node,shape=rectangle] (c3) [above of=n5] {\(C\)};
		\node[node,shape=rectangle] (a1) [below left of=c3] {\(\oplus \)};
		\node[node,shape=rectangle] (a2) [below right of=c3] {\(\oplus \)};
		\node[node,shape=rectangle] (a3) [above of=c3] {\(\oplus \)};
		\node[node] (x7) [above of=a3] {\(10\)};
		\draw[->] (x1) to (c1);
		\draw[->] (x2) to (c1);
		\draw[->] (x3) to (c2);
		\draw[->] (x4) to (c2);
		\draw[->] (c1) to (x5);
		\draw[->] (c2) to (x6);
		\draw[->] (x5) to (a1);
		\draw[->] (x6) to (a2);
		\draw[->] (x8) to (a1);
		\draw[->] (x8) to (a2);
		\draw[->] (a1) to (c3);
		\draw[->] (a2) to (c3);
		\draw[->] (c3) to (a3);
		\draw[->] (x5) [bend left] to (a3);
		\draw[->] (a3) to (x7);
	\end{tikzpicture}
	\caption{ABR of \Cref{ex:abr}.}\label{fig:abr}
\end{figure}
\begin{example}\label{ex:abr}
	Consider the same compression function \(C\) of \Cref{ex:merkle_tree}, and consider the 
	sequence of pre-hashed messages \(S' = \Tuple{3, 4, 7, 7, 10}\), in this case we interpret 
	\(x \oplus y \equiv \Parens*{x + y \bmod 13} + 1\).
	\Cref{fig:abr} shows the resulting ABR\@.
	The authentication path of the node labelled with \(3\) consists of the tuple 
	\(\Tuple{3, 4, 10, 11}\): by computing \(\call{H}{3, 4} = 13\) and then 
	\(\call{H}{13 \oplus 10, 11 \oplus 10} \oplus 13 = 10\) we can verify that the commitment is 
	respected.
\end{example}

\section{ZK-SNARK systems}\label{sec:zk-snark}
As we saw in \Cref{subsec:nizk}, researchers were able to construct ZK-NARK systems whose 
verification complexity was linear in the size of the problem instance, which is provided as a 
boolean circuit.
Furthermore, in the CRS model, by using a block cipher, it is also possible to have 
\emph{publicly verifiable} constructions~\cite{LapidotS1991}, meaning that \emph{any} verifier, 
not just the one that engages the protocol, is able to check the proof, which is encrypted with a 
\emph{proving key}, by using a public \emph{verification key}.

\begin{proposition}[Fiat-Shamir heuristic~\cite{FiatS1987}]
  Suppose a probabilistic I/O TM \(\mathcal{P}\) with access to a CHF \(H\) wants to prove its 
  knowledge of the discrete logarithm \(x = \call{\log}{y}\) for some value 
  \(y \in \mathbb{Z}_p\), where \(p\) is a large prime number.
  Then \(\mathcal{P}\) can sample a random value \(v\) from \(\Taperand \), compute the digest 
  \(d = \call{H}{p, y, p^v}\), the result \(r = {v - dx} \bmod \Parens*{p - 1}\), and finally 
  output the quadruple \(\Tuple{p, y, p^v, r}\).
  Any \textnormal{\textsc{PTIME}} TM \(\mathcal{V}\) with access to \(\Tuple{p, y, p^v, r}\) 
  and \(H\) can recompute \(d\) and check whether \(p^v = p^{r}y^{d}\)
  (If \(\mathcal{P}\) is not cheating, then \(p^{r}y^{d} = p^{v - dx}\Parens*{p^{x}}^d = 
  p^{v - dx}p^{dx} = p^{v - dx + dx} = p^v\)).
  Assuming that the discrete logarithm is hard and that true CHF exist, if equality holds 
  \(\mathcal{V}\) is convinced that \(\mathcal{P}\) knows \(x\) but is not able to retrieve it 
  except with negligible probability.
\end{proposition}

\begin{definition}[Succint proof]
  A \emph{succint proof} for a statement \(\sigma \) over a language \(L \subseteq \Set{0, 1}^*\) 
  is a proof \(\pi \) such that \(\abs{\pi} = \BigO{\call{\log}{\abs{\sigma}}}\).
\end{definition}

Similarly, one can define the notion of succint argument of knowledge, and in particular, a 
succint ZK-NARK system is called a ZK-SNARK system.
\begin{definition}[Probabilistically checkable proof system~\cite{BabaiFLS1991,FeigeGLSS1991}]
  A \emph{probabilistically checkable proof system} (PCP system) is an interactive proof system 
  \(\Tuple{\mathcal{P, V}}\) such that for any proof \(\pi \) provided by \(\mathcal{P}\):
  \(\exists k\colon \call{\Time}{\mathcal{V}} = \BigO{\call{\log^k}{\abs{\pi}}}\).
\end{definition}

In a PCP system, the prover \(\mathcal{P}\) constructs a proof \(\pi \) of size polynomial in the 
length of the original statement \(\sigma \); since the verifier \(\mathcal{V}\) is 
polylogarithmically bound to the size of the proof, it can only query a small portion of it, 
however, it is enough to get statistical completeness and soundness.

In~\cite{Kilian1992}, the author uses Merkle trees to have the prover commit to a proof \(\pi \), 
(the bits of \(\pi \) are the leaves and the root, whcih has constant size, is sent to the verifier). 
The verifier then queries a certain number of authentication paths, which have length
\(\BigO{\call{\log}{\abs{\pi}}}\), and decides whether to accept or reject.
In this sense, the protocol is therefore succint.
In~\cite{Micali2000}, the construction was extended and, by applying the Fiat-Shamir heuristic, it 
is possible to make the protocol non-interactive.

One of the first \emph{succint} ZK-NARK (ZK-SNARK) systems that didn't make explicit use of 
PCPs was devised in~\cite{Groth2010}, but had one important drawback: while the size of the proof 
is constant, the size of the CRS, and the computation that the prover has to perform is 
\emph{quadratic} in the size of the input circuit
(this bound was slightly improved in~\cite{Lipmaa2011}).

However, by first transforming the circuit into \emph{quadratic span programs} (QSPs), 
the boolean equivalent of QAPs (\Cref{subsec:qap}), it was shown that it is possible to reduce both 
the size of the CRS and the prover's computational complexity to linear, while still having succint 
proofs~\cite{GennaroGPR2012}.
Since all these constructions make use of encryption based on the hardness of finding the discrete
logarithm of a number over a big finite field, dealing with boolean circuits and QSPs is not 
very efficient; although both polynomially sized boolean and arithmetic circuits are equivalent to 
\textsc{PTIME} Turing machines~\cite{PippengerF1979}, working over arithmetic programs, and hence 
using QAPs over QSPs, can greatly reduce the constant factors involved 
in such constructions, although this depends on the kind of input problem (numeric problems 
can exploit arithmetic circuits much better than, say, propositional problems).

\subsection{Pinocchio}
An important application of ZK-SNARK systems is in \emph{verifiable computation}.
Consider a client (say, a mobile phone) that wants to delegate to a server (say, a cloud provider) 
some computation, for which several inputs are required: some are provided by the client, 
and some by the server:
\begin{itemize}
  \item The client does not trust the server, so we would need a proof system, but since the server 
        is not computationally unbounded, an \emph{argument of knowledge} system will suffice.
  \item The server might have to interact with many clients or, similarly, many different clients
        might require the same computation, the system must be \emph{non-interactive}.
  \item Verifying the computation must be cheaper than performing it, otherwise the client wouldn't 
        have to ask the server in the first place, the system must provide \emph{succint} proofs.
  \item The server has too interests in to the client that the computation was correct, say to 
        avoid legal liability, but it is not willing to share its own inputs, so our system must
        be \emph{zero-knowledge}.
\end{itemize}
Clearly, among the various systems we saw up to now, ZK-SNARKs are the only one that can reasonably 
fulfill all of the requirements above.
However, all the constructions we saw, due to the high overheads involved in generating the CRS, 
building the QSP/QAP, generating the proof, and even verifying it, were (much) slower than native 
execution by the client.

The first construction that was efficient enough to be usable in practice, and that in many cases
broke the `native execution' barrier for verification time, was 
\emph{Pinocchio}~\cite{ParnoGHR2013}.
\begin{definition}[Bilinear group~\cite{BonehF2001}]
  A \emph{bilinear group} is a group \(\mathbb{G}\) and \(\mathbb{G}'\) 
  such that there is bilinear map (w.r.t.\ exponentiation) 
  \(B\colon \mathbb{G} \times \mathbb{G} \to \mathbb{G}_T\) such that 
  \(\forall a, b \in \mathbb{Z}\colon \call{B}{x^a, y^b} = \call{B}{x, y}^{ab}\).
\end{definition}

The definition of maps linear w.r.t.\ exponentiation is a useful trick to avoid the need to 
introduce the notion of \emph{elliptic curve}~\cite{Silverman2009}, which is out of the scope 
of this thesis.

\begin{example}\label{ex:bilinear_map}
  Recall \Cref{ex:cyclic_group}.
  Consider a group \(\mathbb{G} = \gengroup{g}\), and define 
  \(B\colon \mathbb{G} \times \mathbb{G} \to \mathbb{Z}_{\abs{\mathbb{G}}}\) such that 
  \(\call{B}{x, y} = g^{\call{\log}{x}\call{\log}{y}}\). The map \(B\) is bilinear 
  w.r.t.\ exponentiation, since:
  \[
    \forall a,b \in \mathbb{Z}\colon \call{B}{x^a, y^b} = g^{\call{\log}{x^a}\call{\log}{y^b}} = 
    g^{a\call{\log}{x}b\call{\log}{y}} = \Parens*{g^{\call{\log}{x}\call{\log}{y}}}^{ab}
    = \call{B}{x, y}^{ab}
  \]
\end{example}

The two main ingredients of Pinocchio are QAPs and bilinear maps.
Given in input an arithmetic formula \(\varphi \) over some field \(\mathbb{F} \cong \gengroup{g}\) 
equipped with a bilinear map \(B\colon \gengroup{g} \times \gengroup{g} \to \mathbb{F}\) defined as 
in \Cref{ex:bilinear_map}, the Pinocchio protocol is organized in three phases: the 
\emph{setup phase}, the \emph{prover phase}, and the \emph{verifier phase}.

\subsubsection*{Setup phase}
In the setup phase, any of the parties derives the explicit formula \(\explicit{\varphi}\), 
the circuit \(\mathcal{G}\) of \(\explicit{\varphi}\), the R1CS \(\mathcal{C}\) of \(\mathcal{G}\) 
and finally the \(m/n\) QAP \(\mathcal{Q} = \Tuple{t, \bm{v}, \bm{w}, \bm{y}}\) of \(\mathcal{C}\).
Recall that \(m = \abs{\bm{v}} = \abs{\bm{w}} = \abs{\bm{y}} = \abs{\mathcal{G}_{\otimes}}\)\footnote{
  As we will see, it is possible to reduce this number in some special cases by exploiting the 
  structure of R1CS constraints.
  Also note that the original protocol as in~\cite{ParnoGHR2013} does not make explicit use of 
  R1CS, but we want to stress its importance as it is the standard way to represent arithmetic 
  circuits in \texttt{libsnark}~\cite{SassonCTV2013}.
  }, 
that \(n = \call{\deg}{t} = m + 1 + \abs{\mathcal{G}_{in}}\), and that 
\(\forall i\colon \call{\deg}{\bm{v}_i} = \call{\deg}{\bm{w}_i} = \call{\deg}{\bm{y}_i} = n - 1\).
The QAP (and eventually all the other componenets) are made public.

Either via a trusted third party or an ensamble of authorities~\cite{GrothO2006}, we generate 
a CRS \(\overbar{\sigma}\), which has length 
\(\abs{\overbar{\sigma}} = 8\Ceil*{\call{\log}{\abs{\mathbb{F}}}}\) and interpret every 
\(\Ceil*{\call{\log}{\abs{\mathbb{F}}}}\)-bit subsequence as elements of \(\mathbb{F}\): 
\[\overbar{\sigma} = \Tuple{r_v, r_w, s, \alpha_v, \alpha_w, \alpha_y, \beta, \gamma}\]

Finally, after fixing \(r_y = r_{v}r_{w}\), we build the \emph{proving key} and the 
\emph{verification key}:
\begin{align*}
  & \ProverKey = 
  \begin{pmatrix*}
    \Set{g^{r_{v}\call{\bm{v}_j}{s}}}, &
    \Set{g^{r_{w}\call{\bm{w}_j}{s}}}, &
    \Set{g^{r_{y}\call{\bm{y}_j}{s}}}, \\
    \Set{g^{r_{v}\alpha_v\call{\bm{v}_j}{s}}}, &
    \Set{g^{r_{w}\alpha_w\call{\bm{w}_j}{s}}}, &
    \Set{g^{r_{y}\alpha_y\call{\bm{y}_j}{s}}}, \\
    \Set{g^{s^k}}, &
    \Set{g^{r_{v}{\beta}\call{\bm{v}_j}{s}}g^{r_{w}{\beta}\call{\bm{w}_j}{s}}g^{r_{y}{\beta}\call{\bm{y}_j}{s}}}
  \end{pmatrix*} \\
  & \VerifierKey = \Tuple{
    g, 
    g^{\alpha_v}, 
    g^{\alpha_w}, 
    g^{\alpha_y}, 
    g^{\gamma}, 
    g^{\beta\gamma}, 
    g^{r_{y}\call{t}{s}},
    \Set{g^{r_{v}\call{\bm{v}_i}{s}}}, 
    \Set{g^{r_{w}\call{\bm{w}_i}{s}}}, 
    \Set{g^{r_{y}\call{\bm{y}_i}{s}}}
  }
\end{align*}
where \(i\) ranges over \(\Iinterval{0}{\abs{G_{in}}}\), \(j\) ranges over 
\(\Iinterval{\abs{\mathcal{G}_{in}}}{n-1}\), and \(k\) ranges over \(\Iinterval{1}{n}\).

It is important to remark that \(\overbar{\sigma}\) must be deleted immediately after generating 
the two keys, as it can be exploited to tamper the protocol 
(in jargon, we say that it is \emph{toxic waste}).

\subsubsection*{Prover phase}
After fixing the input vector \(\bm{x}\), the prover creates a circuit input for \(\mathcal{G}\), 
computes the induced evaluation and extracts the assignment \(\mathcal{A}\).
From \(\mathcal{A}\), the prover derives the associated solution \(\bm{c}\) of \(\mathcal{C}\),
and computes the polynomials 
\(p = \Parens*{\bm{vc}}\Parens*{\bm{wc}} - \Parens*{\bm{yc}}\) and \(h = p/t\).
Now, to build a short proof \(\pi \), the prover should sum the contributions of all the 
left input, right input, and output polynomials:
\begin{align*}
  & v_{\otimes} = \sum_{i=\abs{\mathcal{G}_{in}}}^{n-1}{\bm{c}_{i}\call{\bm{v}_{i}}{s}} &&
  w_{\otimes} = \sum_{i=\abs{\mathcal{G}_{in}}}^{n-1}{\bm{c}_{i}\call{\bm{w}_{i}}{s}} &&
  y_{\otimes} = \sum_{i=\abs{\mathcal{G}_{in}}}^{n-1}{\bm{c}_{i}\call{\bm{y}_{i}}{s}}
\end{align*}
and obtain the proof:
\[
  \pi = \Tuple{
    g^{r_{v}v_{\otimes}}, 
    g^{r_{w}w_{\otimes}}, 
    g^{r_{y}y_{\otimes}}, 
    g^{\call{h}{s}},
    g^{r_{v}\alpha_{v}v_{\otimes}}, 
    g^{r_{w}\alpha_{w}w_{\otimes}}, 
    g^{r_{y}\alpha_{y}y_{\otimes}}, 
    g^{r_{v}{\beta}v_{\otimes}}g^{r_{w}{\beta}w_{\otimes}}g^{r_{y}{\beta}y_{\otimes}}
    }
\]
Although \(r_v\), \(r_w\) or \(r_y\) have been deleted, and finding the discrete logarithm is hard,
by using \(\ProverKey \), it is possible to compute:
\begin{align}\label{eq:compute_proof}
  & g^{r_{v}v_{\otimes}} = 
  \prod_{i=\abs{\mathcal{G}_{in}}}^{n - 1}{\Parens*{g^{r_{v}\call{v_{i}}{s}}}^{\bm{c}_{i}}} &&
  & g^{r_{w}w_{\otimes}} = 
  \prod_{i=\abs{\mathcal{G}_{in}}}^{n - 1}{\Parens*{g^{r_{w}\call{w_{i}}{s}}}^{\bm{c}_{i}}} &&
  & g^{r_{y}y_{\otimes}} = 
  \prod_{i=\abs{\mathcal{G}_{in}}}^{n - 1}{\Parens*{g^{r_{y}\call{y_{i}}{s}}}^{\bm{c}_{i}}}
\end{align}
and similarly for the other values (except for \(g^{\call{h}{s}}\) which can be computed directly).
Finally, the prover publishes the pair \(\Tuple{\call{\phi}{\bm{x}}, \pi}\).

\subsubsection*{Verifier phase}
At this point, any potential verifier with access to \(\VerifierKey \) and to the bilinear map 
\(B\) can check the alleged proof.
First, the verifier should check whether \(p = ht\), which, as we noted in \Cref{subsec:qap}, it is 
statistically equivalent to checking whether \(\call{p}{s} = \call{h}{s}\call{t}{s}\).
In the spirit of \Cref{eq:compute_proof}, using \(\VerifierKey \), the verifier can compute the 
input/output contributions \(g^{r_{v}v_{I/O}}\), \(g^{r_{w}w_{I/O}}\) and \(g^{r_{y}y_{I/O}}\).
Clearly, for any arbitrary \(k \in \mathbb{F} \setminus \Set{0}\), we have that 
\(\call{p}{s} = \call{h}{s}\call{t}{s} \iff k^{\call{p}{s}} = k^{\call{h}{s}\call{t}{s}}\) (we 
``work in the exponent'').
If we fix \(k = \call{B}{g^{r_v}, g^{r_w}}\), with some algebraic effort the previous 
equation can be transformed into the equivalent one:
\[
  \call{B}{g^{r_{v}v_{I/O}}g^{r_{v}v_{\otimes}}, g^{r_{w}w_{I/O}}g^{r_{w}w_{\otimes}}} =
  \call{B}{g^{r_{y}\call{t}{s}}, g^{\call{h}{s}}}\call{B}{g^{r_{y}y_{I/O}}g^{r_{y}y_{\otimes}}, g}
\]

Even if the divisibility check passes, it might still be the case that the alleged 
proof was not built using the polynomials in \(\mathcal{Q}\).
To address this eventuality, the verifier checks whether:
\begin{align*}
  &  \call{B}{g^{r_{v}\alpha_{v}v_{\otimes}}, g} = \call{B}{g^{r_{v}v_{\otimes}}, g^{\alpha_v}}
  && \call{B}{g^{r_{w}\alpha_{w}w_{\otimes}}, g} = \call{B}{g^{r_{w}w_{\otimes}}, g^{\alpha_w}}
  && \call{B}{g^{r_{y}\alpha_{y}y_{\otimes}}, g} = \call{B}{g^{r_{y}y_{\otimes}}, g^{\alpha_y}}
\end{align*}

Finally, the prover might have used the correct polynomials, but didn't use the same 
coefficients \(\bm{c}_i\) when building \(v_{\otimes}\), \(w_{\otimes}\) and \(y_{\otimes}\).
This last concern is resolved by checking whether:
\[
  \call{B}{g^{r_{v}{\beta}v_{\otimes}}g^{r_{w}{\beta}w_{\otimes}}g^{r_{y}{\beta}y_{\otimes}}, g^{\gamma}} =
  \call{B}{g^{r_{v}v_{\otimes}}g^{r_{w}w_{\otimes}}g^{r_{y}y_{\otimes}}, g^{\beta\gamma}}
\]

The protocol as described is not zero-knowledge, but it is quite simple to make it so:
the verification key is extended to include \(g^{r_{v}\alpha_{v}\call{t}{s}}\), 
\(g^{r_{v}\beta{v}\call{t}{s}}\), and then the prover generates a random value \(\delta_v\) and 
replaces each polynomial \(\bm{v}_i\) with \(\bm{v}_i + \delta_{v}t\). 
The same thing is done for \(\bm{w}\) and \(\bm{y}\): the validity of the checks performed by the 
verifier is not affected by this change, but it can be shown that the scheme is now 
statistically zero-knowledge.

It is immediate to see that the proving key \(\ProverKey \) contains 
\(8\abs{\mathcal{G}_{\otimes}} + \abs{\mathcal{G}_{in}} + 1\) field elements, the verification 
key \(\VerifierKey \) contains \(3\abs{\mathcal{G}_{in}} + 7\) field elements (\(6\) more in the 
zero-knowledge setting), and the proof \(\pi \) contains \(8\) field elements, meaning that its size
is independent from the input circuit.

\section{\texttt{libsnark}}
In the last ten years, new improvements were put forward to reduce the size of the messages and 
the complexity of the computation, especially on the prover's side~\cite{Lipmaa2013} (for example, 
in~\cite{Groth2016} the size of the proof was reduced to just \(3\) field elements).
Furthermore, much effort has been put into making working implementations of ZK-SNARK 
systems, such as \texttt{libsnark}\footnote{\url{https://github.com/scipr-lab/libsnark}, you can 
also find a nice empirical comparison of the various protocols.}, 
a C\texttt{++}~\cite{Stroustrup2013} library which implements and refines several ZK-SNARK 
protocols~\cite{DanezisFGK2014,GrothM2017,BackesBFR2014,SassonCGTV2013}, although the core 
component (pre-processing ZK-SNARK, or PPZK-SNARK) is based mostly on the Pinocchio protocol and 
on~\cite{SassonCTV2013,Groth2016}, which are all extensions and improvements of the QAP (and QSP) 
model of~\cite{GennaroGPR2012}.

In order to implement some function \(f\) which can be expressed as an 
arithmetic expression \(\varphi \) (i.e.\ no variable-length loops or recursion) in 
\texttt{libsnark}, we must provide both the arithmetic circuit and the associated R1CS\@.
In many cases, deriving the R1CS from the arithmetic circuit is quite trivial, but there are 
instances where having them separate allows for some quite nice optimizations.
Although there has been work on \emph{compilers} that translate high-level code to R1CS 
constraints~\cite{EberhardtT2018,BellesBDM2022}, as it is often the case, this comes at a cost 
of flexibility.
Intuitively, we can divide the usage of \texttt{libsnark} in same three phases of the Pinocchio 
system (or really any other SNARK).

\subsubsection*{Preprocessing phase}
The first thing to do in \texttt{libsnark} is choosing, at compile time (or \emph{a priori}, in a 
theoretical interpretation), which bilinear group to use for the protocol. 
The standard choice is BN254~\cite{BarretoN2005}, but there are also other groups available, such 
as BLS12\footnote{\url{https://electriccoin.co/blog/new-snark-curve/}}~\cite{BonehLS2001}.
It is important to note that all these groups are paired to a prime field.
After choosing the group, we setup the \emph{protoboard} which, as the 
name suggests, it is the object where one places the components of the circuit:
\begin{lstlisting}[language=C++]
  libsnark::protoboard<Field> board;
  board.set_input_sizes(PUBLIC_N);
\end{lstlisting}
The template argument \texttt{Field} specifies the underlying field, and \texttt{PUBLIC\_N} 
specifies the number of output (i.e.\ public) variables in the circuit.
On the protoboard, we allocate \emph{variables} that will carry the input/intermediate/output
values of the circuit evaluation, together with an annotation (for debug).
However, it is usually much more convenient to only declare the input and output variables, and 
delegate the wiring to \emph{gadgets} which act as composable black-boxes:
\begin{lstlisting}[language=C++]
  libsnark::pb_variable<Field> output_var;
  libsnark::pb_variable<Field> input_var;

  output_var.allocate(board, "out");
  input_var.allocate(board, "in");
  FooGadget gadget{board, input_var, output_var};
\end{lstlisting}
Note that the first \texttt{PUBLIC\_N} variables allocated will be considered public, while the 
remaining ones will be private.
A typical gadget must provide two methods: one to generate the R1CS constraints, and one to 
compute the circuit evaluation given a circuit input, which will be used in the proving phase.
First, we generate the constraints:
\begin{lstlisting}[language=C++]
  gadget.generate_r1cs_constraints();
\end{lstlisting}
Internally, \texttt{gadget} allocates the required intermediate variables and constrains their 
linear combinations. 
A linear combination can be expressed quite naturally:
\begin{lstlisting}[language=C++] 
  libsnark::pb_linear_combination<Field> lc = a1*x1 + ... + an*xn;
\end{lstlisting}
An R1CS constraint of the type \(\bm{ab} = \bm{c}\) is expressed as:
\begin{lstlisting}[language=C++]
  libsnark::r1cs_constraint<Field> constraint{lc_a, lc_b, lc_c};
  board.add_r1cs_constraint(constraint);
\end{lstlisting}
Once all the constraints have been specified, the last thing to do is to convert the R1CS to a 
QAP and get the keys \(\ProverKey \) and \(\VerifierKey \).
The whole process is done transparently by the library:
\begin{lstlisting}[language=C++]
  auto keypair = libsnark::r1cs_ppzksnark_generator(board.get_constraint_system());
\end{lstlisting}
Now the constraint system and \(\VerifierKey \) can be made public, while \(\ProverKey \) is only 
known by the prover. 

\subsubsection*{Proving phase}
To generate a (valid) proof, the prover has to first provide a circuit input and compute the 
induced evaluation. 
At the highest level, this is done by setting the values of the input variables and letting the 
gadget generate the intermediate and the output values:
\begin{lstlisting}[language=C++]
  Field x; // some field element
  board.val(input_var) = x;
  gadget.generate_r1cs_witness(circuit_inputs);
\end{lstlisting}
Now, he can generate the proof, which is done transparently by the library by using the 
proving key and the primary (i.e.\ public) and auxiliary (i.e.\ private) inputs:
\begin{lstlisting}[language=C++]
  auto proof = libsnark::r1cs_ppzksnark_prover(keypair.pk, board.primary_input(), board.auxiliary_input());
\end{lstlisting}
Finally, the prover can make public the primary inputs of the protoboard together with the
proof.

\subsubsection*{Verification phase}
At this point, the verifier has to check the proof by using the public key, and there are two 
kinds of choices regarding how to perform such verification: 
\begin{itemize}
  \item \emph{weak} vs.\  \emph{strong} verification: in the former case, it is ok for the prover 
        to only provide some of the primary inputs, which will be zero-padded. 
        In the latter case, this is not considered acceptable.
  \item \emph{offline} vs.\  \emph{online} verification: in the former case, the verification key 
        is used `as is', in the latter, it is processed to get faster verification times when used 
        multiple times.
\end{itemize}
In any case, the whole thing is managed transparently by the library.
For example, if we choose strong, online verification:
\begin{lstlisting}[language=C++]
  auto proc_vk = libsnark::r1cs_ppzksnark_verifier_process_vk(keypair.vk);
  bool halt = libsnark::r1cs_ppzksnark_online_verifier_strong_IC(proc_vk, board.primary_input(), proof);
\end{lstlisting}
The verifier will accept the proof if \texttt{halt} is \texttt{true}, while it will reject if it 
is \texttt{false}.

\begin{example}
  Consider the R1CS in \Cref{ex:r1cs}.
  \Cref{lst:libsnark_example} shows the implementation of the corresponding gadget in 
  \texttt{libsnark}.
\end{example}

\begin{algorithm}
  \centering
  \begin{lstlisting}[caption={The \texttt{libsnark} gadget corresponding to the 
    arithmetic formula in \Cref{ex:arithmetic_formula}.}\label{lst:libsnark_example},language=C++]
    #include <libsnark/gadgetlib1/gadget.hpp>

    using namespace libsnark; // don't do this in real code please

    template<typename FieldT>
    class FooGadget : public gadget<FieldT>
    {
        using Var = pb_variable<FieldT>;

        const Var x1, x2, y;
        Var t1, t2, t3;

        FooGadget(protoboard<FieldT> &board, 
                  const std::string &ann,
                  const Var &x1, 
                  const Var &x2, 
                  const Var &y) : 
            gadget<FieldT>{board, ann}, x1{x1}, x2{x2}, y{y}
        {
            t1.allocate(board, "t1");
            t2.allocate(board, "t2");
            t3.allocate(board, "t3");
        }

        void generate_r1cs_constraints()
        {
            board.add_r1cs_constraint(r1cs_constraint<FieldT>{x1, x1, t1});
            board.add_r1cs_constraint(r1cs_constraint<FieldT>{t1, x1, 8 + 9*x2 + t2});
            board.add_r1cs_constraint(r1cs_constraint<FieldT>{x2, t2, y});
        }

        void generate_r1cs_witness()
        {
            board.val(t1) = board.val(x1) * board.val(x1);
            board.val(t2) = board.val(t1) * board.val(x1) + 4*board.val(x2) + 5;
            board.val(y)  = board.val(x2) * board.val(t2);
        }
    };
  \end{lstlisting}
\end{algorithm}


