\chapter{Hash Functions and Succint Proofs}\label{chap:crypto}
Cryptography, quite unsurprisingly, is the field in which zero-knowledge proof systems have 
got the most attention.
The possibility of two or more parties to cooperate and exchange information one with another in a 
zero-knowledge manner is the fundamental idea behind many branches of cryptography such as 
\emph{Multi Party Computation} (MPC)~\cite{Yao1982-2} and \emph{Fully Homomorphic Encryption} 
(FHE)~\cite{ArmknechtEtAl2015}.

An important application of zero-knowledge protocols lies in \emph{verifiable computation}:
a server (the prover) wants to convince some users (the verifiers) that some function has been 
executed properly, without revealing the inputs.

In \Cref{sec:hash_functions}, we review the fundamental notions concerning 
\emph{cryptographic hash functions}, and some of the standard ways to construct them.
In \Cref{sec:tree_hash}, we go in more depth over \emph{tree-like modes of hashing}, which 
constitute the basis of some of the most important use-cases of verifiable computation.
In \Cref{sec:zk-snark} we study the construction of zero-knowledge succint NARKs, with a focus on 
the \emph{Pinocchio} protocol~\cite{ParnoGHR2013}.
Finally, in \Cref{sec:libsnark} we describe \texttt{libsnark}, a C\texttt{++} library which 
implements (a variant of) the Pinocchio protocol. 

\chapter{Cryptographic Primitives from Generalized Triangular Dynamical Systems}\label{chap:arion}
One of the most important applications of zero-knowledge verifiable computation lies in digital 
currency transactions over the blockchain infrastructure.
An example of ZK-SNARK applied in the real world is the ZCash cryptocurrency~\cite{SassonCGGMTV2014}, 
which is inspired by the more famous Bitcoin~\cite{NarayananBFMG2016}, and was devised by the 
authors of \texttt{libsnark} (which frames the zero-knowledge backend of the currency).

As we discussed in \Cref{sec:tree_hash}, the fundamental component of a blockchain is the 
Merkle tree, which uses one-way compression functions in order to produce the binding 
commitment.
In a digital currency scenario, the leaves of the Merkle tree consist of the details of some 
transaction, typical information include the ID of the sender, the ID of the recipient, and the 
amount of currency to be transferred. 
Without a zero-knowledge framework in place, when one wants to verify whether a user did abide to 
their commitment, the only possible solution is to ask the user to disclose his transaction, 
together with the authentication path, and check that the tree commitment is respected. 
When using currencies like Bitcoin or Ethereum\footnote{\url{https://ethereum.org/}}, anyone 
can see the details of every single transaction being performed on the relative 
blockchain, meaning that there is no privacy whatsoever\footnote{For example, on 
\url{https://etherscan.io/} you can see the transactions on the Ethereum blockchain. %It is 
%curious how privacy has often been foisted as a feature of mainstream cryptocurrencies while, 
%on the contrary, any bank offers much more privacy!
}.
However, if we translate the Merkle tree computation in an equivalent circuit, it is possible to 
apply a zero-knowledge scheme that allows a verifier to be sure (with overwhelming probability) 
of the validity of a transaction without actually having to see it!
Since a Merkle tree applies over and over the underlying compression function, the problem of 
creating a circuit for the former immediately reduces to the problem of creating a circuit for the 
latter.

In \Cref{sec:sota} we will review the evolution of the state of the art concerning zero-knowledge 
friendly compression functions.
Then, in \Cref{sec:gtds}, we present a new algebraic framework to represent cryptographic 
primitives, the \emph{Generalized Triangular Dynamic System}, and apply it to construct the 
\Arion{} block cipher and the \Arionhash{} hash function.
Finally, in \Cref{sec:performance}, we compare our new construction to the state of the art using 
the \texttt{libsnark} library, showing extremely competitive results.
\section{State of the art}\label{sec:sota}
The standard compression function used in Merkle trees is usually one of the SHA-2 or SHA-3 
functions~\cite{Dang2015}: this is certainly the most sensible choice in a \emph{native} 
environment, as SHA is specifically designed to be fast in both software and 
hardwaare~\cite{DaddaMO2004,MichailAKTG2012} implementations, and is the most studied hash function 
from a security standpoint (e.g.\ for SHA-2 see~\cite{KhovratovichRS2012,GuoLRW2010,DobraunigEM2016}).

However, when working with arithmetic circuits over a prime field \(\mathbb{F}_p\), SHA has a lot 
of issues: the underlying operations being performed are bitwise XOR, bitwise AND, 
bit shifts/rotations and additions modulo \(2^{32}\).
While shifts and rotations come at no cost, as they basically consist in a renaming of the circuit 
wires/variables, bitwise operations and addition which is not modulo \(p\) have to be simulated 
bit-by-bit, and the overhead introduced in such a translation is huge.
For example, for SHA-256, over a bilinear group like BN254 for which 
\(\abs{\mathbb{F}_p} \approx 2^{256}\), we would need \(256\) input variables each holding a
\(256\)-bit integer to simulate the behaviour of every single bit during the SHA computation; 
clearly, this is decisely suboptimal.

\begin{example}
  Suppose we are given two strings \(a, b \in \Set{0, 1}^{n}\), and we want to compute 
  \(a \bitxor b\).
  By interpreting them as vectors \(\bm{v}, \bm{w} \in \mathbb{F}_{p}^{n}\), we can simulate 
  bitwise XOR by computing, \(\forall i \le n\):
  \[\bm{v}_{i} \bitxor \bm{w}_{i} = \bm{v}_{i} + \bm{w}_{i} - 2\bm{v}_{i}\bm{w}_{i}\]
  that is, every XOR operation requires one multiplication gate.
  Similarly, bitwise AND and non-native addition also require multiplications to be simulated.
  Furthermore, we must guarantee that the values \(\bm{v}_i\) and \(\bm{w}_i\) are boolean, as 
  in principle they could assume any value in \(\mathbb{F}_p\), so we must also add constraints of 
  the kind \(\bm{v}_{i}\Parens*{\bm{v}_i - 1} = 0\).
\end{example}

\subsection{MiMC}
In an effort to find secure cryptographic designs that could be efficient in zero-knowledge 
settings, called \emph{zk-friendly} designs, researchers began to study the properties of 
permutations that make use of a low number of multiplications 
(\emph{multiplicative complexity})~\cite{AlbrechtRSTZ2016}.

One of the first constructions over finite fields was the \emph{Minimal Multiplicative Complexity}
(MiMC) family of cryptographic permutations~\cite{AlbrechtGRRT2016}.
The idea of MiMC, reprising an older proposal~\cite{NybergK1995}, is to use a very simple 
polynomial permutation as its core component, and by repeating it for an adequate number of rounds,
obtain a secure construction.
\begin{definition}[MiMC keyed permutation]
  Given a finite field \(\mathbb{F}_p\), a number of rounds 
  \(r = \Ceil*{\frac{\call{\log}{p}}{\call{\log}{3}}}\), some constants 
  \(c_1, \dots, c_r \in \mathbb{F}_p\) and a set of functions 
  \(f_1, \dots, f_r\colon \mathbb{F}_p \times \mathbb{F}_p \to \mathbb{F}_p\) such that 
  \(\forall i \le r\colon \call{f_i}{x, k} = x^3 + k + c_i\), the \emph{MiMC keyed permutation}
  is defined as:
  \[
    \call{E_{MiMC}}{x, k}\colon \mathbb{F}_p \times \mathbb{F}_p \to \mathbb{F}_p = 
    \call{\Parens*{f_r \compose \dots \compose f_1}}{x, k} + k
  \]
\end{definition}

The MiMC keyed permutation is also called MiMC-\(n/n\). 
By applying the Feistel construction on the MiMC permutation, one obtains the Feistel MiMC function, 
or MiMC-\(2n/n\).
Finally, by applying the sponge construction, one can obtain the MiMC hash function.
In alternative, it is also possible to build an hash function using first the Davies-Meyer 
construction to obtain a one-way compression function, and then the Merkle-Damg\"{a}rd construction
to obtain an hash function.

There are some important observations to be made on the MiMC construction.
First, the round permutation uses a low degree polynomial, but it is repeated for a high number of 
rounds: for example, if the size of the underlying field is \(\approx 2^{256}\), the number of 
rounds will be \(r = 162\). 
Note that \(x^3\) might not actually induce a permutation over \(\mathbb{F}_p\), as in general 
\(3\) is not coprime with \(\call{\totient}{p}\) (in fact, in the underlying fields of both BN254 
and BLS12, \(3\) is a factor of \(p - 1\)).
In such cases, one should modify the definition to consider the smallest prime number \(d\) such 
that \(\call{\gcd}{d, \call{\totient}{p}} = 1\), and reduce the number of rounds to
\(r = \Ceil*{\frac{\call{\log}{p}}{\call{\log}{d}}}\).

A second observation is that \(r\) must be chosen to thwart many different types of cryptanalysis 
techniques: since the MiMC permutation corresponds to the 
polynomial \(p = \Parens*{x^3 + k + c_1}\dots\Parens*{x^3 + k + c_r}\) 
(which has degree \(\call{\deg}{p} = 3^r\)), in addition to the traditional \emph{brute-force}, 
\emph{meet-in-the-middle}~\cite{DiffieH1977}, \emph{differential}~\cite{BihamS1991} and 
\emph{linear}~\cite{Matsui1994} attacks, one must also consider \emph{algebraic attacks}, 
which exploit the inherent nature of this type of constructions.

In fact, traditional attacks don't tend to pose a major threat to these kinds of constructions:
brute force is clearly too expensive and meet-in-the-middle is also infeasible both due to the high 
number of rounds and to the huge degree of the inverse permutation (usually \(1/3 \gg 3\)).
The permutation \(x^3\) is not approximable by a linear function~~\cite{AbdelraheemABL2012}, 
hence linear attacks are not a threat, and since it can be easily shown that any arbitrary input 
difference \(\delta_{in} \) propagates to any arbitrary output difference \(\delta_{out} \) with a 
probability of at most \({2}/{2^n}\), differential attacks are also ineffective~\cite{Nyberg1994}.

On the side of algebraic cryptanalysis, one might attempt an \emph{interpolation attack}, which 
uses Lagrange interpolation to find a polynomial \(\tilde{p}\) which behaves like a keyless version 
of \(p\)~\cite{JakobsenK1997}.
This attack's complexity depends solely on \(\call{\deg}{p}\) (in fact, an interpolation can be 
computed in \(\BigO{n\call{\log}{n}}\), where \(n = \call{\deg}{p}\)~\cite{Stoss1985}), hence we 
must be sure that the degree of \(p\) also grows exponentially round by round (as it is the case).
Another kind of algebraic attack is the \emph{GCD attack}: by using two plaintext/ciphertext pairs,
once can compute their greatest common divisor which will allow to easily retrieve the secret key.
Again, computing the GCD depends almost linearly on the degree of the polynomial, hence one must 
again be sure that the degree grows exponentially.

\subsubsection*{MiMC vs.\ SHA-256}

\subsection{Poseidon}
\subsection{Griffin}
\subsection{Other designs}

\section{\Arion{}: A new ZK-friendly permutation}\label{sec:gtds}
The constructions that we saw in \Cref{sec:sota} are prominent examples of different 
\emph{generations} of \emph{Arithmetization Oriented} (AO) designs.
For example, \Mimc{} is an example of a Gen-I design: its main purpose was mostly to demonstrate 
that it was indeed possible to construct secure and efficient cryptographic primitives by 
stacking simple, low-degree round functions.
On the other hand, the \Hades{} framework, its derivative \Poseidon{} and other similar 
constructions like \Rescue{} are examples of Gen-II designs: by tweaking the SPN and 
Feistel constructions, their purpose was to massively improve the efficiency over Gen-I designs.
Finally, \Griffin{} is an example of a Gen-III design: the underlying \Horst{} scheme is neither 
a ``pure'' Feistel nor SPN design and, by deviating from such standard constructions, its authors 
were able to improve the efficiency even further. 
Furthermore, in third generation designs it was shown that one does not necessarily need to use 
round functions with a low degree, as long as the resulting constraint system is not affected 
negatively\footnote{As it is often the case in research, the separation line is a bit blurry, 
as \Rescue, which we said to be a Gen-II design, already used inverse exponentiations}.

Other from \Griffin, there is another very recent Gen-III construction, called 
\Anemoi~\cite{BouvierBCPSVW2022} and based on the \Flystel{} design, which has some common points 
with the \Horst{} construction.
A very interesting fact used in \Flystel{} was the notion of CCZ-equivalence~\cite{CarletCZ1998},
a concept which generalizes the intuition behind the idea that there is ``no difference'' 
between, say, using \(x^{d}\) and \(x^{\frac{1}{d}}\).
\begin{definition}[Affine function]
  An \emph{affine function} over an \(n\)-dimensional vector space \(\mathbb{F}^n\) is a function
  \(\call{f}{\bm{x}}\colon \mathbb{F}^n \to \mathbb{F}^n = \bm{Mx} + q\), where 
  \(\bm{M} \in \mathbb{F}^{n \times n}\) and \(q \in \mathbb{F}^n\).
\end{definition}
\begin{definition}[Function graph]
  Given a set \(S\), the \emph{function graph} of a function \(f\colon S \to S\) is the pair 
  \(\Gamma = \Tuple{S, E}\) where \(E = \Set{\Tuple{x, \call{f}{x}} \mid x \in S}\).
\end{definition}
\begin{definition}[Induced permutation]
  Given a \emph{function graph} \(\Gamma = \Tuple{S, E}\) and a permutation \(P\colon S^2 \to S^2\), 
  the \emph{induced permutation} of \(\Gamma \) by \(P\) is the function graph 
  \(\call{P}{\Gamma} = \Tuple{S, E'}\) where \(E' = \Set{\call{P}{e} \mid e \in E}\).
\end{definition}
\begin{definition}[CCZ equivalence~\cite{BouvierBCPSVW2022,CarletCZ1998}]
  Given a vector space \(\mathbb{V}\) and two functions 
  \(f,g\colon \mathbb{V} \to \mathbb{V}\), \(f\) and \(g\) are \emph{CCZ-equivalent} if there 
  is an affine permutation \(L\colon \mathbb{V}^2 \to \mathbb{V}^2\) such 
  that \(\Gamma_{f} = \call{L}{\Gamma_{g}}\).
\end{definition}

Clearly, CCZ-equivalence is an equivalence relation over a vector space \(\mathbb{V}\), hence it 
induces a partitioning of \(\mathbb{V}^2\) into equivalence classes.
An interesting fact is that all CCZ-equivalent functions share the same linear and differential 
properties.
Even more importantly for our purposes is that CCZ-equivalent functions are indistinguishable under 
constraint verification: checking the constraint system of any member of the class verifies 
the validity of the computation of any other member.
However, one thing that CCZ-equivalent functions do not share in general is their degree, hence we 
can use the one with the highest degree in the actual computation to provide the most strict security 
guarantees against algebraic attacks, while using the one with the lowest degree when building
the constraint system for the verification in the SNARK framework.

In a high-level, intuitive way we can say that CCZ-equivalence allows us to ``ignore the order'' 
in which the witnesses of a certain computation are obtained: reprising our usual example,
the verifier does not care that the prover first must know \(x\) in order to obtain \(y = x^{1/d}\), 
all it matters is that the prover knows both of them and that their relationship is correct.
In fact, even more generally, there is no way to know in which order someone actually got hold of 
the intermediate values of a computation, hence we might as well exploit this to our advantage in 
order to reduce the complexity of the protocol.

\subsection{The Generalized Triangular Dynamical System}
As we said, the design of third generation AO cryptographic primitives diverts from the plain 
SPN or Feistel constructions.
For this reason, we introduce the \emph{Generalized Triangular Dynamical System}~\cite{RoyS2022}, 
GTDS for short, an algebraic framework which generalizes many previous designs (such as Feistel, 
SPN, \Horst{}, \dots) and their instantiations (\Mimc{}, \Poseidon{}, \Griffin{}, \dots), and enables 
us to provide a systematic security analysis of the constructions derived from it.
In particular:
\begin{itemize}
  \item The input is split in branches like in earlier designs.
  \item The round function offers the strength of all the incorporated designs.
  \item It is secure against classical attacks like differential cryptanalysis.
  \item It is secure against interpolation attacks already at the first round.
  \item Its linear layer mixes all the branches through a circulant matrix with no zero entries.  
  \item It uses inverse exponents, but decouples them from the direct exponent. 
\end{itemize}

\begin{definition}[GTDS of \Arion~\cite{RoyS2022}]\label{def:gtds}
  Given a prime field \(\mathbb{F}_p\), a number of branches \(t \in \mathbb{N}\), the smallest 
  integer \(d_1\) such that \(\call{\gcd}{d_1, p - 1} = 1\), an arbitrary 
  integer \(d_2\) such that \(\call{\gcd}{d_2, p - 1} = 1\), some constants 
  \(\alpha_{1}, \beta_{1}, \gamma_1, \dots, \alpha_{t - 1}, \beta_{t - 1}, \gamma_{t - 1} \in \mathbb{F}_p\) 
  such that \(\forall i < t\colon \alpha_i^2 - 4\beta_i\) is a quadratic non-residue modulo 
  \(p\), let \(e = {1}/{d_2}\) and, for all \(i < t\) let:
  \begin{align*}
    & \call{g_i}{x}\colon \mathbb{F}_p \to \mathbb{F}_p = x^2 + \alpha_{i}x + \beta_{i} \\
    & \call{h_i}{x}\colon \mathbb{F}_p \to \mathbb{F}_p = x^2 + \gamma_{i}x
  \end{align*}
  Then, the GTDS of \Arion{} is the function 
  \(\call{F_{GTDS}}{\bm{x}}\colon \mathbb{F}_p^t \to \mathbb{F}_p^t\) such that:
  \[
    \call{F_{GTDS}}{\bm{x}}_i = \bm{y}_i = 
    \begin{cases}
      \bm{x}_i^{d_1}\call{g_i}{\sigma_{i+1, t}} + \call{h_i}{\sigma_{i+1, t}} & 1 \le i < t \\
      \bm{x}_i^e & i = t
    \end{cases}
  \]
  where \(\sigma_{i, k} = \sum_{j=i}^{k}{\bm{x}_j + \bm{y}_j}\).
\end{definition}

It can be shown~\cite{RoyS2022} that the GTDS is function is invertible.
An interesting detail which came up only in the later phases of the GTDS design is the decoupling 
of the inverse exponentiation from the direct exponentiation, in the sense that, instead of 
using \(x^d\) and \(x^{1/d}\), we use \(x^{d_1}\) and \(x^{1/d_2}\).
The rationale of this choice is that some security considerations about cryptographic constructions
over the GTDS depend on the size of \(d_2\): if it is too small, we would need more rounds to 
achieve the desired level of security, hence increasing the circuit complexity.
If \(d_2\) were to be equal to \(d_1\), the reduction in terms of number of rounds would be 
overcompensated by the increase of the complexity of a single round, but since \(d_2\) is only 
used in the last branch, the trade-off becomes more convenient, especially for bigger branch sizes.

Of course, \(d_2\) should require an optimal number of constraints.
Note that, since we have at our disposal intermediate results, the optimal number of operations 
required to exponentiate a number is not determined by the classical \emph{binary exponentiation} 
algorithm~\cite{Gueron2011}, but rather by \emph{addition chains}~\cite{BosC1990}.

\subsection{\Arion{} and \Arionhash{}}
Depending on our security and efficiency needs, we can instantiate the GTDS in many 
ways.
We design \Arion{} and \Arionhash{}~\textbf{\cite{RoyST2023}} to work over fields of size 
\(\approx 2^{256}\).
In order to achieve a \emph{degree overflow} in the first round, it can be shown that \(4e\), where 
\(e = {1}/{d_2}\), should be greater than \(p\). 
For BN254 and BLS12, this can be achieved by \(d_2 \in \Set{121, 123, 125, 161, 257}\).
For example, the optimal way to compute \(x^{121}\) is:
\begin{align*}
  & y = \Parens*{x^{2}}^{2} && z = \Parens*{y^{2}y}^{2} && x^{121} = \Parens*{z^{2}}^{2}zx
\end{align*}
It is not hard to see that numbers of the type \(2^k + 1\), for \(k \in \mathbb{N}\), are the 
most efficient to compute, and \(257\) is a particularly attractive candidate as it is also a 
prime number, and requires the same number of multiplications to be computed as the other candidates, 
(unfortunately, \(129 = 43 \cdot 3\) is not invertible neither in BN254 nor in BLS12).

To introduce mixing between the various branches, we use an \emph{affine layer} which employs a 
circulant matrix that has no zero entries and that is efficiently computable.
\begin{definition}[Affine layer of \Arion]
  The \emph{affine layer} of \Arion{} over a vector space \(\mathbb{F}_p^t\) is the function:
  \[\call{L}{\bm{x}, \bm{c}}\colon \Parens*{\mathbb{F}_p^n}^2 \to \mathbb{F}_p^n = 
  \call{\circulant}{1, \dots, t}\bm{x} + \bm{c}\]
\end{definition}

It is worth noting that the very simple matrix \(\call{\circulant}{1, \dots, t}\) is an MDS matrix
for any prime field \(\mathbb{F}_p\) such that \(p \ge 2^{39}\) and for values of
\(t \in \Iinterval{2}{12}\).
Furthermore, computing the matrix-vector product using \(\call{\circulant}{1, \dots, t}\) can be 
done in \(\BigO{t}\) time instead of the typical \(\BigO{t^2}\) required by a standard matrix-vector 
multiplication algorithm, by using \Cref{alg:circ_mult}.
\begin{algorithm}
  \begin{algorithmic}
    \Function{circ\_mul}{$\bm{v} \in \mathbb{F}_p^t$} %chktex 46
    \State{\(\bm{w} \gets \bm{0} \in \mathbb{F}_p^t\)}
    \State{\(\sigma \gets \sum_{i=1}^{t}{\bm{v}_i}\)}
    \State{\(\bm{w}_1 \gets \sigma + \sum_{i=1}^{t}{\Parens*{i - 1}\bm{v}_i}\)}
    \For{\(i \in \Iinterval{2}{t}\)}
    \State{\(\bm{w}_i \gets \bm{w}_{i-1} - \sigma + n\bm{v}_{i-1}\)}
    \EndFor{}
    \State{\Return{\(\bm{w}\)}}
    \EndFunction{}
  \end{algorithmic}
  \caption{Efficient evaluation of the matrix-vector product with 
    \(\call{\circulant}{1, \dots, t}\)}\label{alg:circ_mult}
\end{algorithm}

\begin{definition}[\Arion{} keyed permutation~\textbf{\cite{RoyST2023}}]
  Given a prime field \(\mathbb{F}_p\), a number of branches \(t \in \mathbb{N}\), a number 
  of rounds \(r \in \mathbb{N}\), some constants \(\bm{c}_1, \dots, \bm{c}_r \in \mathbb{F}_p^t\), 
  the \emph{\Arion{} keyed permutation} is the function \(\Arion = \Arion_r\), where:
  \[
    \call{\Arion_i}{\bm{x}, \bm{k}_0, \dots, \bm{k}_r}\colon 
      \Parens*{\mathbb{F}_p^t}^{r+2} \to \mathbb{F}_p^t = \bm{y}_i =
      \begin{cases}
        \call{L}{\bm{x}, \bm{0}} + \bm{k}_i & i = 0 \\
        \call{L}{\call{F_{GTDS}}{\bm{x}}, \bm{c}_i} + \bm{k}_i & 1 \le i \le r
      \end{cases}
  \]
\end{definition}

\begin{definition}[\Arionp{} unkeyed permutation]
  The \emph{\Arion{} unkeyed permutation} is the function:
  \[
    \call{\Arionp}{\bm{x}}\colon \mathbb{F}_p^t \to \mathbb{F}_p^t = 
      \call{\Arion}{\bm{x}, \bm{0}, \dots, \bm{0}}
  \]
\end{definition}

We can instantiate the hash function \Arionhash{} by applying the sponge construction to 
the \Arionp{} permutation.
Since it has been shown that, for a pseudorandom permutation \(P\) over a vector space 
\(\mathbb{F}_p^{t}\), a rate \(r\) and a capacity 
\(c\) such that \(t = r + c\), the sponge construction is indifferentiable from a random 
distribution up to \(\call{\min}{p^r, p^{c/2}}\) queries~\cite{BertoniDPV2008}, to provide 
\(\kappa \) bits of security, we must require that \(r \ge {\kappa}/{\call{\log}{p}}\) and 
\(c \ge {2\kappa}/{\call{\log}{p}}\).

As a padding scheme for a message \(m \in \Set{0, 1}^*\) whose length is not a multiple of the rate 
\(r\), we use \(\call{\Pad}{m} = m \concat 0^{\Parens*{-\abs{m} \bmod r}}\), and we replace the 
initial value \(v = 0^{\Parens*{t\abs{\Encode{p}}}}\) (where \(\Encode{x}\) denotes the binary 
encoding of an object \(x\)) with 
\(v' = \Encode{\abs{m}} \concat 0^{\Parens*{t-1}\abs{\Encode{p}}}\).
Note that we assume,  as basically any constructions does, that \(\abs{m} < p\), 
which should not be a problem as typically \(p \approx 2^{256}\), and we don't expect to hash 
messages of length \(\abs{m} > 2^{256}\).

\begin{table}
  \centering
  \caption{Instantiation parameters of \Arion{} and \Aarion{} for \(128\) bits of security and 
    primes \(p \geq 2^{60}\).}\label{tab:arion_instantiation}
  \begin{tabular}[t]{  c  c  c  c  }
      \toprule

      \phantom{ }\(d_1\)\phantom{ } & \phantom{ }\(t\)\phantom{ } & \phantom{ }\Arion{} \phantom{ } & \phantom{ }\Aarion{} \phantom{ } \\
      \midrule
      \multicolumn{2}{  c  }{} & \multicolumn{2}{ c  }{\phantom{ }Rounds\phantom{ }} \\
      \midrule
      \(3\) & \(3\) & \(6\) & \(5\) \\
      \(5\) & \(3\) & \(6\) & \(4\) \\

      \(3\) & \(4\) & \(6\) & \(4\) \\
      \(5\) & \(4\) & \(5\) & \(4\) \\

      \(3\) & \(5\) & \(5\) & \(4\) \\
      \(5\) & \(5\) & \(5\) & \(4\) \\

      \(3\) & \(6\) & \(5\) & \(4\) \\
      \(5\) & \(6\) & \(5\) & \(4\) \\

      \(3\) & \(8\) & \(4\) & \(4\) \\
      \(5\) & \(8\) & \(4\) & \(4\) \\
      \bottomrule
  \end{tabular}
\end{table}

In \Cref{tab:arion_instantiation} you can find the parameters for \Arion{} and 
its aggressive variant \Aarion{}, the parameters for the respective hash functions are respectively 
the same.
The aggressive variants have been parametrized in a way to provide the desired security level 
against all attacks but probabilistic Gr\"{o}bner basis attacks.
The choice to provide them anyway was dictated by the fact that, to the best of our knowledge, none 
of the competitor designs has been proved secure against such attacks, hence the aggressive 
versions provide the same guarantees in terms of security as the current state of the art.
For a detailed security analysis of \Arion, \Arionhash{} and their aggressive variants, refer 
to~\textbf{\cite{RoyST2023}}.

\section{Performance evaluation of \Arion}\label{sec:performance}
As we saw in \Cref{sec:gtds}, the concept of CCZ-equivalence introduced in the \Anemoi{} 
proposal~\cite{BouvierBCPSVW2022} plays an important role to build high degree permutations 
verifiable with low degree constraint systems.
In fact, it is possible to extend the concept of CCZ-equivalence by allowing any permutation.
\begin{definition}[\(\pi \)-equivalence]
  Given a vector space \(\mathbb{V}\) and two functions \(f,g\colon \mathbb{V} \to \mathbb{V}\), 
  \(f\) and \(g\) are \emph{\(\pi \)-equivalent} if there is a permutation 
  \(\pi\colon \mathbb{V}^2 \to \mathbb{V}^2\) such that \(\Gamma_{f} = \call{\pi}{\Gamma_{g}}\).
\end{definition}

Any function \(\pi\colon \mathbb{V}^2 \to \mathbb{V}^2\) can be decomposed into its 
\emph{projections} \(\pi_{\hat{x}}, \pi_{\hat{y}}\colon \mathbb{V}^2 \to \mathbb{V}\) such that:
\[
  \forall \bm{x}, \bm{y} \in \mathbb{V}\colon \call{\pi}{\bm{x}, \bm{y}} = 
    \Tuple{\call{\pi_{\hat{x}}}{\bm{x}, \bm{y}}, \call{\pi_{\hat{y}}}{\bm{x}, \bm{y}}}
\]

Thus, for any function \(f\colon \mathbb{V} \to \mathbb{V}\), we have that 
\(\call{\pi}{\bm{x}, \call{f}{\bm{x}}} = 
  \Tuple{\call{\pi_{\hat{x}}}{\bm{x}, \call{f}{\bm{x}}}, 
  \call{\pi_{\hat{y}}}{\bm{x}, \call{f}{\bm{y}}}}\).
Now, let \(\call{f_{\hat{x}}}{\bm{x}} = \call{\pi_{\hat{x}}}{\bm{x}, \call{f}{\bm{x}}}\) and 
\(\call{f_{\hat{y}}}{\bm{x}} = \call{\pi_{\hat{y}}}{\bm{x}, \call{f}{\bm{x}}}\), then 
\(\call{\pi}{\Gamma_f} = \Gamma_{f'}\) if and only if \(\pi_{\hat{x}}\) is a permutation, and 
in particular \(f' = f_{\hat{y}} \compose f_{\hat{x}}^{-1}\) is \(\pi \)-equivalent to \(f\).

Just like CCZ-equivalence, \(\pi \)-equivalence is also an equivalence relation, denoted 
\(\pieq \), which allows us to identify equiivalence classes of functions over \(\mathbb{V}\).
\begin{lemma}[\(\pi \)-equivalence of permutations]\label{lem:pi_equiv}
  All permutations over a vector space \(\mathbb{V}\) are \(\pi \)-equivalent.
\end{lemma}
\begin{proof}
  We just have to prove that, for every permutation \(f\colon \mathbb{V} \to \mathbb{V}\), it 
  is the case that \(f \pieq \fooid \), and the result will follow.
  Clearly, the function \(\call{\pi}{\bm{x}, \bm{y}}\colon \mathbb{V}^2 \to \mathbb{V}^2 = 
    \Tuple{\bm{x}, \call{f^{-1}}{\bm{y}}}\) is a permutation, and 
  \(\call{\pi}{\Gamma_f} = \Gamma_{\fooid}\).
\end{proof}

\begin{definition}[Alternative GTDS of \Arion]
  Given the same parameters as in \Cref{def:gtds}, the \emph{alternative GTDS of \Arion} is the
  function \(\tilde{F}_{GTDS}\colon \mathbb{F}_p^t \to \mathbb{F}_p^t\) such that:
  \[
    \call{\tilde{F}_{GTDS}}{\bm{x}}_i = \bm{y}_i =
    \begin{cases}
      \bm{x}_i^{d_1}\call{g_i}{\tilde{\sigma}_{i+1,t}} + 
        \call{h_i}{\tilde{\sigma}_{i+1,t}} & 1 \le i < t \\
      \bm{x}_i^{d_2} & i = t
    \end{cases}
  \]
  where \(\tilde{\tau}_{i, k} = \bm{x}_k + \bm{x}_{k}^{e} + \sum_{j=i}^{k-1}{\bm{x}_j + \bm{y}_j}\).

\end{definition}

\begin{proposition}[\(\pi \)-equivalence of GTDS]
  \(F_{GTDS} \pieq \tilde{F}_{GTDS}\).
\end{proposition}
\begin{proof}
  \(F_{GTDS}\) and \(\tilde{F}_{GTDS}\) are both permutations over the vector space \(\mathbb{F}_p^t\),
  therefore by \Cref{lem:pi_equiv} the claim follows.
\end{proof}

As a corollary, we have that verifying the constraint system of \(F_{GTDS}\) is equivalent to 
verifying the constraint system of \(\tilde{F}_{GTDS}\).
Computing the number of multiplicative constraints for \Arionhash{} is quite straightforward:
\begin{lemma}[R1CS constraints for \Arionhash]
  Given the \Arionhash{} function over a prime field \(\mathbb{F}_p\) with branch size 
  \(t\) and rate \(r\), let \(\call{\minmul}{x}\colon \mathbb{F}_p \to \mathbb{N}\) be the minimum
  number of field multiplications required to compute \(y^x\) for any \(y \in \mathbb{F}\).
  The number of R1CS constraints required by \Arionhash{} is:
  \[
    N_{\Arionhash} = 
      r\Parens*{\Parens*{n - 1}\Parens*{\call{\minmul}{d_1} + 2} + \call{\minmul}{d_2}}
  \]
\end{lemma}
\begin{proof}
  Consider the alternative GTDS \(\tilde{F}_{GTDS}\): we need \(\call{\minmul}{d_2}\) 
  multiplicative constraints in the last branch.
  In the remaining \(n - 1\) branches, we need \(\call{\minmul}{d_1}\) multiplications to 
  compute \(\bm{x}_i^{d_1}\), one multiplication for computing \(g_i\), one for computing \(h_i\), 
  and one to multiply \(g_i\) with \(\bm{x}_i^{d_1}\).
\end{proof}

For reference, the number of R1CS constraints required by \Poseidon{} and \Griffin{} 
over their respective parameters (see \Cref{def:poseidon} and \Cref{def:griffin}) are given by:
\begin{align*}
  & N_{\Poseidon} = \call{\minmul}{d}\Parens*{2tr_f + r_P} \\
  & N_{\Griffin} = 2r\Parens{\call{\minmul}{d} + t - 2}
\end{align*}

\begin{table}
  \centering
  \caption{R1CS constraint comparison over \(256\)-bit prime fields and \(128\) bits of security 
  with \(d_2 \in \Set{121, 123, 125, 161, 257}\).}\label{tab:arion_compare_muls}
  \begin{tabular}{  c c c c c c c  }
      \toprule
      \phantom{ }\(d_1\)\phantom{ } & \phantom{ }\(t\)\phantom{ } & \phantom{ }\Arionhash{}\phantom{ } & \phantom{ }\Aarionhash{}\phantom{ } & \phantom{ }\Griffin{}\phantom{ } & \phantom{ }\Anemoi{}\phantom{ } & \phantom{ }\Poseidon{}\phantom{ }             \\
      \midrule
      \multicolumn{2}{  c | }{} & \multicolumn{5}{ c }{Rounds} \\
      \midrule
      \(3\) & \(3\) & \(6\) & \(5\) & \(12\) &      & \phantom{ }\(r_f = 4,\ r_P = 84\)\phantom{ } \\
      \(5\) & \(3\) & \(6\) & \(4\) & \(12\) &      & \phantom{ }\(r_f = 4,\ r_P = 56\)\phantom{ } \\

      \(3\) & \(4\) & \(6\) & \(4\) & \(11\) & \(12\) & \phantom{ }\(r_f = 4,\ r_P = 84\)\phantom{ } \\
      \(5\) & \(4\) & \(5\) & \(4\) & \(11\) & \(12\) & \phantom{ }\(r_f = 4,\ r_P = 56\)\phantom{ } \\

      \(3\) & \(5\) & \(5\) & \(4\) &      &      & \phantom{ }\(r_f = 4,\ r_P = 84\)\phantom{ } \\
      \(5\) & \(5\) & \(5\) & \(4\) &      &      & \phantom{ }\(r_f = 4,\ r_P = 56\)\phantom{ } \\

      \(3\) & \(6\) & \(5\) & \(4\) &      & \(10\) & \phantom{ }\(r_f = 4,\ r_P = 84\)\phantom{ } \\
      \(5\) & \(6\) & \(5\) & \(4\) &      & \(10\) & \phantom{ }\(r_f = 4,\ r_P = 84\)\phantom{ } \\

      \(3\) & \(8\) & \(4\) & \(4\) & \(9\)  & \(10\) & \phantom{ }\(r_f = 4,\ r_P = 84\)\phantom{ } \\
      \(5\) & \(8\) & \(4\) & \(4\) & \(9\)  & \(10\) & \phantom{ }\(r_f = 4,\ r_P = 56\)\phantom{ } \\

      \midrule

      \multicolumn{2}{  c | }{} & \multicolumn{5}{ c  }{R1CS Constraints} \\
      \midrule

      \(3\) & \(3\) & \(102\) & \(85\)  & \(72\)  &       & \(216\) \\
      \(5\) & \(3\) & \(114\) & \(76\)  & \(96\)  &       & \(240\) \\

      \(3\) & \(4\) & \(126\) & \(84\)  & \(88\)  & \(96\)  & \(232\) \\
      \(5\) & \(4\) & \(120\) & \(96\)  & \(110\) & \(120\) & \(264\) \\

      \(3\) & \(5\) & \(120\) & \(100\) &       &       & \(248\) \\
      \(5\) & \(5\) & \(125\) & \(116\) &       &       & \(288\) \\

      \(3\) & \(6\) & \(145\) & \(116\) &       & \(120\) & \(264\) \\
      \(5\) & \(6\) & \(170\) & \(136\) &       & \(150\) & \(312\) \\

      \(3\) & \(8\) & \(148\) & \(148\) & \(144\) & \(160\) & \(296\) \\
      \(5\) & \(8\) & \(176\) & \(176\) & \(162\) & \(200\) & \(360\) \\

      \bottomrule
  \end{tabular}
\end{table}

\Cref{tab:arion_compare_muls} shows the number of constraints required for specific instantiations 
of \Arion, \Aarion, \Anemoi, \Poseidon{} and \Griffin{} over \(\approx 256\)-bit prime fields for a 
target \(128\) bits of security.

\subsection{\texttt{libsnark} implementation and experiments}



\section{Tree-like modes of hashing}\label{sec:tree_hash}
Consider an \(n\)-bit CHF \(H\), and suppose that a prover claims to know some message \(m\): 
the digest \(d = \call{H}{m}\) can be considered as a \emph{short binding commitment} for \(m\): 
By asking the prover to share the digest, whose size \(\abs{d} = n\) is typically considered to be 
\(\BigO{1}\) (or \(\BigO{\call{\log}{\abs{m}}}\) in some cases), a verifier is convinced that the 
prover does know \(m\) with probability \(\approx 1 - {1}/{2^n}\).
A modern standard CHF like SHA-256~\cite{Dang2015} produces digests of length at least \(256\) bits,
making the \(1 - {1}/{2^n}\) bound really hard to bruteforce through.
Note that the verifier needs not to know \(m\) in advance: the commitment \(d\) is (temporarily) 
appended to a public \emph{blockchain} and, at any point in the future, when the verifier becomes 
aware of some \(m'\) provided by the prover, if \(\call{H}{m'} = d\), the commitment can be 
approved or rejected.

Now, suppose that the prover wants to commit to a list of \(k\) messages: the simplest solution 
would be to publish the hash of every message, which would require to append \(\BigO{k}\) 
elements on the blockchain.
Another way would be for the prover to share \(\call{H}{\Tuple{m_1, \dots, m_k}}\): the 
communication cost would only be \(\BigO{1}\) but, in general, not all the messages belong to the 
same prover, so this method would not work, and we need a better solution.

\subsection{Merkle tree}
\begin{definition}[Binary Merkle tree~\cite{Merkle1988}]
	A \emph{binary Merkle tree (MT)} of height \(h \in \mathbb{N}\) over a \(2n\)-to-\(n\) compression 
	function \(C\), is the complete binary tree of height \(h\) such that, given a sequence of input 
	messages \(\Tuple{m_1, \dots, m_{2^{h-1}}}\) over \(\Set{0, 1}^{2n}\), produces an
	output digest \(d \in \Set{0, 1}^{n}\) in the following way:
	\begin{enumerate}
		\item The leaf nodes \(\nu_1, \dots, \nu_{2^{h-1}}\) contain 
					\(\call{C}{m_1}, \dots, \call{C}{m_{2^{h-1}}}\).
		\item Every other node \(\nu \) contains \(\call{C}{\nu_l, \nu_r}\), where \(\nu_l\) is
		      the left child of \(\nu \) and \(\nu_r\) is the right child of \(\nu \).
		\item The output digest \(d\) is the content of the root node. 
	\end{enumerate}
\end{definition}

The set of the sibling nodes visited in the path from a leaf of the tree to the root, including the 
leaf itself, is the \emph{authentication path} of the leaf.
By using Merkle trees, the prover only needs to send to the verifier, as a commitment for
some message \(m_i\) among \(n = 2^h\) messages, the contents of the co-path from the leaf 
containing \(m_i\) to the root, in addition plus the hash of \(m_i\): this requires
just \(\BigO{\call{\log}{n}}\) cost to validate the commitment.
Merkle trees bottom-up construction is very easy to parallelize, and they can be used in the 
multiple-provers scenario: each prover only needs to commit to the path from its own leaf to the 
root of the tree.
It is immediate to generalize the notion of binary Merkle tree to arbitrary arity.
\begin{proposition}[Security of Merkle tree mode of hash~\cite{Merkle1988}]
	Given a one-way \(tn\)-to-\(n\) compression function \(C\), the \(t\)-ary Merkle tree over 
	\(C\) is a cryptographic hash function.
\end{proposition}

\begin{example}\label{ex:merkle_tree}
	Consider the sequence of pre-hashed messages \(S = \Tuple{3, 4, 7, 7}\) and the 
	compression function 
	\(\call{C}{x, y}: \Tuple{x, y} \mapsto \Parens*{xy \bmod 13} + 1\) (for ease of exposition, 
	we work over integers instead of bit strings, but the two can be readily converted into one 
	another).
	\Cref{fig:merkle_tree} shows the contents of the associated Merkle Tree.
	Note that the real message is not stored in the Merkle Tree, but only the `first level' of hashes.
	The authentication path of the leaf labelled with \(3\) consists of the tuple \(\Tuple{3, 4, 11}\):
	by computing \(\call{H}{3, 4} = 13\) and then \(\call{H}{13, 11} = 1\) we can verify that the 
	commitment is respected.
\end{example}
\begin{figure}
	\centering
	\begin{tikzpicture}[node distance={32pt}, node/.style = {draw, circle},on grid=true]
		\node[node] (x1) {\(3\)};
		\node[node,draw=none] (n1) [right of=x1] {};
		\node[node] (x2) [right of=n1] {\(4\)};
		\node[node,shape=rectangle] (c1) [above of=n1] {\(C\)};
		\node[node,draw=none] (n2) [right of=x2] {};
		\node[node] (x3) [right of=n2] {\(7\)};
		\node[node,draw=none] (n3) [right of=x3] {};
		\node[node] (x4) [right of=n3] {\(7\)};
		\node[node,shape=rectangle] (c2) [above of=n3] {\(C\)};
		\node[node] (x5) [above of=c1] {\(13\)};
		\node[node,draw=none] (n3) [above of=n2] {};
		\node[node,draw=none] (n4) [above of=n3] {};
		\node[node] (x6) [above of=c2] {\(11\)};
		\node[node,shape=rectangle] (c3) [above of=n4] {\(C\)};
		\node[node] (x7) [above of=c3] {\(1\)};
		\draw[->] (x1) to (c1);
		\draw[->] (x2) to (c1);
		\draw[->] (x3) to (c2);
		\draw[->] (x4) to (c2);
		\draw[->] (c1) to (x5);
		\draw[->] (c2) to (x6);
		\draw[->] (x5) to (c3);
		\draw[->] (x6) to (c3);
		\draw[->] (c3) to (x7);
	\end{tikzpicture}
	\caption{Merkle tree of \Cref{ex:merkle_tree}.}\label{fig:merkle_tree}
\end{figure}

\subsection{Augmented Binary Tree}
The Merkle tree is the de-facto standard for blockchain applications, and basically for any 
scenario for which a `linear' hash function cannot be used.
In~\cite{Stam2008}, it was given a lower bound on the amount of queries necessary to obtain a 
collision for a \(\Parens*{m+s}\)-to-\(s\)-bit CHF \(H\) (the \(m\) is variable) built from a 
\(\Parens*{n+c}\)-to-\(n\)-bit OWCF \(C\): if \(H\) makes \(r\) queries to \(C\), it is possible 
to find a collision by making \(2^{\frac{nr + cr - m}{r + 1}}\) queries to \(H\).
By combining this result with the \(2^{s/2}\) upper bound of the birthday paradox, one can 
immediately obtain a tight bound \(m = \frac{2nr + 2cr -sr - s}{2}\) for the variable length \(m\) 
of the message.

\begin{definition}[Compactness~\cite{AndreevaBR2021}]
	The \emph{compactness} of an \(\Parens*{m+s}\)-to-\(s\)-bit hash function making \(r\) queries to 
	an underlying \(\Parens*{n+c}\)-to-\(n\)-bit one-way compression function is the value
	\(\alpha = \frac{2m}{2nr + 2cr -sr - s}\).
\end{definition}

\begin{example}\label{ex:mtree_compactness}
	Consider a \(2n\)-to-\(n\) bit OWCF and a Merkle Tree of height \(h\): the computation 
	of the tree is a \(\Parens*{2^{h-1}n}\)-to-\(n\)-bit hash function, and makes exactly 
	\(r = 2^{h-1} - 1\) queries to \(C\).
	We have \(s = c = n\) and \(m = 2^{h-1}n - n = nr\), therefore the compactness of the Merkle 
	Tree construction is:
	\[
		\alpha = \frac{2m}{2nr + 2cr -sr - s} = 
		\frac{2nr}{2nr + 2nr - nr - n} =
		\frac{2r}{3r - 1}
	\]
	Which tends to \(2/3\) when \(r\) tends to infinity.
\end{example}

\begin{definition}[Augmented Binary tRee~\cite{AndreevaBR2021}]
	An \emph{Augmented Binary tRee (ABR)} of height \(h \in \mathbb{N}\) over a 
	\(2n\)-to\(n\) compression function \(C\) is a complete binary tree of height \(h\) 
	augmented with \emph{middle} nodes such that, given a sequence of input messages
	\(S = \Tuple{m_1, \dots, m_{2^{h-1} + 2^{h-2}-1} \mid \forall i\colon m_i \in \Set{0, 1}^{*}}\), 
	it produces an output digest \(d \in \Set{0, 1}^n\) in the following way:
	\begin{enumerate}
		\item The leaf nodes \(\nu_{1}, \dots, \nu_{2^{h-1}}\) contain \(\call{C}{m_1}, \dots,
		      \call{C}{m_{2^{h-1}}}\).
		\item There are no middle nodes in the leaf layer.
		\item The middle nodes \(\nu_{2^{h-1}+1}, \dots, \nu_{\abs{S}}\) contain
		      \(\call{C}{m_{2^{h-1}+1}}, \dots, \call{C}{m_{\abs{S}}}\).
		\item Every other node \(\nu \) contains \(\call{C}{\nu_l \oplus \nu_m, \nu_r \oplus
		      \nu_m} \oplus \nu_r \), where \(\nu_l\) is the left child of \(\nu \), \(\nu_r\)
		      is the right child of \(\nu \), and \(\nu_m\) is the middle child of \(\nu \), or \(0\)
		      if \(\nu \) doesen't have a middle child.
	\end{enumerate}
\end{definition}

The authentication path of the ABR is similar to the one of the Merkle tree, but also includes 
the middle nodes encountered during the traversal.

\begin{proposition}[Security of ABR mode of hash~\cite{AndreevaBR2021}]
	Given a one-way \(2n\)-to-\(n\) compression function \(C\), the ABR over \(C\) is a cryptographic 
	hash function.
\end{proposition}

An ABR of height \(h\) can process 50\% more messages than a Merkle Tree of the same height, 
while performing the same number of queries to the underlying compression function, with the 
additional cost introduced by the intermediate \(\oplus \) operations being negligible in most 
scenarios.

\begin{example}
	Consider a \(2n\)-to-\(n\) bit OWCF and an ABR of height \(h\): the computation 
	of the tree is a \(\Parens*{2^{h-1} + 2^{h-2}-1}n\)-to-\(n\)-bit hash function, 
	and makes exactly \(r = 2^{h-1} - 1\) queries to \(C\).
	Like in \Cref{ex:mtree_compactness}, we have \(s = c = n\), but this time 
	\(m = \Parens*{2^{h-1} + 2^{h-2}-1}n - n = nr + {nr}/2 - n\), so the compactness of the ABR 
	construction is:
	\[
		\alpha = \frac{2m}{2nr + 2cr - sr - s} = 
		\frac{2nr + nr - 2n}{2nr + 2nr - nr - n} =
		\frac{3r - 2}{3r - 1}
	\]
	Which approaches \(1\) as \(r\) approaches infinity, meaning that the ABR construction achieves
	optimal compactness.
\end{example}

It is worth of notice that, while the ABR hash mode achieves collision resistance, it does not 
achieve \emph{indifferentiability} (a weaker notion of indistinguishability between Turing 
machines~\cite{MaurerRH2003}), hence a modified construction, called ABR+, was also proposed, 
although it does not achieve perfect compactness.
\begin{figure}
	\centering
	\begin{tikzpicture}[node distance={32pt}, node/.style = {draw, circle},on grid=true]
		\node[node] (x1) {\(3\)};
		\node[node,draw=none] (n1) [right of=x1] {};
		\node[node] (x2) [right of=n1] {\(4\)};
		\node[node,shape=rectangle] (c1) [above of=n1] {\(C\)};
		\node[node,draw=none] (n2) [right of=x2] {};
		\node[node] (x3) [right of=n2] {\(7\)};
		\node[node,draw=none] (n3) [right of=x3] {};
		\node[node] (x4) [right of=n3] {\(7\)};
		\node[node,shape=rectangle] (c2) [above of=n3] {\(C\)};
		\node[node,draw=none] (n4) [above of=n2] {};
		\node[node] (x8) [above of=n4] {\(10\)};
		\node[node] (x5) [above of=c1] {\(13\)};
		\node[node] (x6) [above of=c2] {\(11\)};
		\node[node,draw=none] (n5) [above of=x8] {};
		\node[node,shape=rectangle] (c3) [above of=n5] {\(C\)};
		\node[node,shape=rectangle] (a1) [below left of=c3] {\(\oplus \)};
		\node[node,shape=rectangle] (a2) [below right of=c3] {\(\oplus \)};
		\node[node,shape=rectangle] (a3) [above of=c3] {\(\oplus \)};
		\node[node] (x7) [above of=a3] {\(10\)};
		\draw[->] (x1) to (c1);
		\draw[->] (x2) to (c1);
		\draw[->] (x3) to (c2);
		\draw[->] (x4) to (c2);
		\draw[->] (c1) to (x5);
		\draw[->] (c2) to (x6);
		\draw[->] (x5) to (a1);
		\draw[->] (x6) to (a2);
		\draw[->] (x8) to (a1);
		\draw[->] (x8) to (a2);
		\draw[->] (a1) to (c3);
		\draw[->] (a2) to (c3);
		\draw[->] (c3) to (a3);
		\draw[->] (x5) [bend left] to (a3);
		\draw[->] (a3) to (x7);
	\end{tikzpicture}
	\caption{ABR of \Cref{ex:abr}.}\label{fig:abr}
\end{figure}
\begin{example}\label{ex:abr}
	Consider the same compression function \(C\) of \Cref{ex:merkle_tree}, and consider the 
	sequence of pre-hashed messages \(S' = \Tuple{3, 4, 7, 7, 10}\), in this case we interpret 
	\(x \oplus y \equiv \Parens*{x + y \bmod 13} + 1\).
	\Cref{fig:abr} shows the resulting ABR\@.
	The authentication path of the node labelled with \(3\) consists of the tuple 
	\(\Tuple{3, 4, 10, 11}\): by computing \(\call{H}{3, 4} = 13\) and then 
	\(\call{H}{13 \oplus 10, 11 \oplus 10} \oplus 13 = 10\) we can verify that the commitment is 
	respected.
\end{example}

\section{ZK-SNARK systems}
As we saw in \Cref{subsec:nizk}, researchers were able to construct ZK-NARK systems whose 
verification complexity was linear in the size of the problem instance, which is provided as a 
boolean circuit.
Furthermore, in the CRS model, by using a block cipher, it is also possible to have 
\emph{publicly verifiable} constructions~\cite{LapidotS1991}, meaning that \emph{any} verifier, 
not just the one that engages the protocol, is able to check the proof, which is encrypted with a 
\emph{proving key}, by using a public \emph{verification key}.

\begin{proposition}[Fiat-Shamir heuristic~\cite{FiatS1987}]
  Suppose a probabilistic I/O TM \(\mathcal{P}\) with access to a CHF \(H\) wants to prove its 
  knowledge of the discrete logarithm \(x = \call{\log}{y}\) for some value 
  \(y \in \mathbb{Z}_p\), where \(p\) is a large prime number.
  Then \(\mathcal{P}\) can sample a random value \(v\) from \(\Taperand \), compute the digest 
  \(d = \call{H}{p, y, p^v}\), the result \(r = {v - dx} \bmod \Parens*{p - 1}\), and finally 
  output the quadruple \(\Tuple{p, y, p^v, r}\).
  Any \textnormal{\textsc{PTIME}} TM \(\mathcal{V}\) with access to \(\Tuple{p, y, p^v, r}\) 
  and \(H\) can recompute \(d\) and check whether \(p^v = p^{r}y^{d}\)
  (If \(\mathcal{P}\) is not cheating, then \(p^{r}y^{d} = p^{v - dx}\Parens*{p^{x}}^d = 
  p^{v - dx}p^{dx} = p^{v - dx + dx} = p^v\)).
  Assuming that the discrete logarithm is hard and that true CHF exist, if equality holds 
  \(\mathcal{V}\) is convinced that \(\mathcal{P}\) knows \(x\) but is not able to retrieve it 
  except with negligible probability.
\end{proposition}

\begin{definition}[Succint proof]
  A \emph{succint proof} for a statement \(\sigma \) over a language \(L \subseteq \Set{0, 1}^*\) 
  is a proof \(\pi \) such that \(\abs{\pi} = \BigO{\call{\log}{\abs{\sigma}}}\).
\end{definition}

Similarly, one can define the notion of succint argument of knowledge, and in particular, a 
succint ZK-NARK system is called a ZK-SNARK system.
\begin{definition}[Probabilistically checkable proof system~\cite{BabaiFLS1991,FeigeGLSS1991}]
  A \emph{probabilistically checkable proof system} (PCP system) is an interactive proof system 
  \(\Tuple{\mathcal{P, V}}\) such that for any proof \(\pi \) provided by \(\mathcal{P}\):
  \(\exists k\colon \call{\Time}{\mathcal{V}} = \BigO{\call{\log^k}{\abs{\pi}}}\).
\end{definition}

In a PCP system, the prover \(\mathcal{P}\) constructs a proof \(\pi \) of size polynomial in the 
length of the original statement \(\sigma \); since the verifier \(\mathcal{V}\) is 
polylogarithmically bound to the size of the proof, it can only query a small portion of it, 
however, it is enough to get statistical completeness and soundness.

In~\cite{Kilian1992}, the author uses Merkle trees to have the prover commit to a proof \(\pi \), 
(the bits of \(\pi \) are the leaves and the root, whcih has constant size, is sent to the verifier). 
The verifier then queries a certain number of authentication paths, which have length
\(\BigO{\call{\log}{\abs{\pi}}}\), and decides whether to accept or reject.
In this sense, the protocol is therefore succint.
In~\cite{Micali2000}, the construction was extended and, by applying the Fiat-Shamir heuristic, it 
is possible to make the protocol non-interactive.

One of the first \emph{succint} ZK-NARK (ZK-SNARK) systems that didn't make explicit use of 
PCPs was devised in~\cite{Groth2010}, but had one important drawback: while the size of the proof 
is constant, the size of the CRS, and the computation that the prover has to perform is 
\emph{quadratic} in the size of the input circuit
(this bound wass slightly improved in~\cite{Lipmaa2011}).

However, by first transforming the circuit into \emph{quadratic span programs} (QSPs), 
the boolean equivalent of QAPs (\Cref{subsec:qap}), it was possible to reduce both the size of the 
CRS and the prover's computational complexity to linear, while still having succint proofs.
Since all these constructions make use of encryption based on the hardness of finding the discrete
logarithm of a number over a big finite field, dealing with boolean circuits and QSPs is not 
very efficient; although both polynomially sized boolean and arithmetic circuits are equivalent to 
\textsc{PTIME} Turing machines~\cite{PippengerF1979}, working over arithmetic programs, and hence 
using R1CSs~\cite{CramerD1998} and QAPs over QSPs, can greatly reduce the constant factors involved 
in such constructions, although this depends on the kind of input problem (numeric problems 
can exploit arithmetic circuits much better than, say, propositional problems).

\subsection{Pinocchio}
An important application of ZK-SNARK systems is in \emph{verifiable computation}.
Consider a client (say, a mobile phone) that wants to delegate to a server (say, a cloud provider) 
some computation, for which several inputs are required: some are provided by the client, 
and some by the server:
\begin{itemize}
  \item The client does not trust the server, so we would need a proof system, but since the server 
        is not computationally unbounded, an \emph{argument of knowledge} system will suffice.
  \item The server might have to interact with many clients or, similarly, many different clients
        might require the same computation, the system must be \emph{non-interactive}.
  \item Verifying the computation must be cheaper than performing it, otherwise the client wouldn't 
        have to ask the server in the first place, the system must provide \emph{succint} proofs.
  \item The server has too interests in to the client that the computation was correct, say to 
        avoid legal liability, but it is not willing to share its own inputs, so our system must
        be \emph{zero-knowledge}.
\end{itemize}
Clearly, among the various constructions we saw up to now, ZK-SNARK systems are the only one that 
can reasonably fulfill all these requirements.
However, all the constructions we saw, due to the high overheads involved 
(generating the CRS, building the QSP/QAP, generating the proof, etc.), were not efficient enough 
to make the whole process faster than just letting the client perform the computations by itself.

The first construction that was efficient enough to be practically usable was 
\emph{Pinocchio}~\cite{ParnoGHR2013}.


\subsection{Groth16}
\subsection{PLONK}

\section{\texttt{libsnark}}
In the last ten years, new improvements were put forward to reduce the size of the messages and 
the complexity of the computation, especially on the prover's side~\cite{Lipmaa2013} (for example, 
in~\cite{Groth2016} the size of the proof was reduced to just \(3\) field elements).
Furthermore, much effort has been put into making working implementations of ZK-SNARK 
systems, such as \texttt{libsnark}\footnote{\url{https://github.com/scipr-lab/libsnark}, you can 
also find a nice empirical comparison of the various protocols.}, 
a C\texttt{++}~\cite{Stroustrup2013} library which implements and refines several ZK-SNARK 
protocols~\cite{DanezisFGK2014,GrothM2017,BackesBFR2014,SassonCGTV2013}, although the core 
component (pre-processing ZK-SNARK, or PPZK-SNARK) is based mostly on the Pinocchio protocol and 
on~\cite{SassonCTV2013,Groth2016}, which are all extensions and improvements of the QAP (and QSP) 
model of~\cite{GennaroGPR2012}.

In order to implement some function \(f\) which can be expressed as an 
arithmetic expression \(\varphi \) (i.e.\ no variable-length loops or recursion) in 
\texttt{libsnark}, we must provide both the arithmetic circuit and the associated R1CS\@.
In many cases, deriving the R1CS from the arithmetic circuit is quite trivial, but there are 
instances where having them separate allows for some quite nice optimizations.
Although there has been work on \emph{compilers} that translate high-level code to R1CS 
constraints~\cite{EberhardtT2018,BellesBDM2022}, as it is often the case, this comes at a cost 
of flexibility.
Intuitively, we can divide the usage of \texttt{libsnark} in same three phases of the Pinocchio 
system (or really any other SNARK).

\subsubsection*{Preprocessing phase}
The first thing to do in \texttt{libsnark} is choosing, at compile time (or \emph{a priori}, in a 
theoretical interpretation), which bilinear group to use for the protocol. 
The standard choice is BN254~\cite{BarretoN2005}, but there are also other groups available, such 
as BLS12\footnote{\url{https://electriccoin.co/blog/new-snark-curve/}}~\cite{BonehLS2001}.
It is important to note that all these groups are paired to a prime field.
After choosing the group, we setup the \emph{protoboard} which, as the 
name suggests, it is the object where one places the components of the circuit:
\begin{lstlisting}[language=C++]
  libsnark::protoboard<Field> board;
  board.set_input_sizes(PUBLIC_N);
\end{lstlisting}
The template argument \texttt{Field} specifies the underlying field, and \texttt{PUBLIC\_N} 
specifies the number of output (i.e.\ public) variables in the circuit.
On the protoboard, we allocate \emph{variables} that will carry the input/intermediate/output
values of the circuit evaluation, together with an annotation (for debug).
However, it is usually much more convenient to only declare the input and output variables, and 
delegate the wiring to \emph{gadgets} which act as composable black-boxes:
\begin{lstlisting}[language=C++]
  libsnark::pb_variable<Field> output_var;
  libsnark::pb_variable<Field> input_var;

  output_var.allocate(board, "out");
  input_var.allocate(board, "in");
  FooGadget gadget{board, input_var, output_var};
\end{lstlisting}
Note that the first \texttt{PUBLIC\_N} variables allocated will be considered public, while the 
remaining ones will be private.
A typical gadget must provide two methods: one to generate the R1CS constraints, and one to 
compute the circuit evaluation given a circuit input, which will be used in the proving phase.
First, we generate the constraints:
\begin{lstlisting}[language=C++]
  gadget.generate_r1cs_constraints();
\end{lstlisting}
Internally, \texttt{gadget} allocates the required intermediate variables and constrains their 
linear combinations. 
A linear combination can be expressed quite naturally:
\begin{lstlisting}[language=C++] 
  libsnark::pb_linear_combination<Field> lc = a1*x1 + ... + an*xn;
\end{lstlisting}
An R1CS constraint of the type \(\bm{ab} = \bm{c}\) is expressed as:
\begin{lstlisting}[language=C++]
  libsnark::r1cs_constraint<Field> constraint{lc_a, lc_b, lc_c};
  board.add_r1cs_constraint(constraint);
\end{lstlisting}
Once all the constraints have been specified, the last thing to do is to convert the R1CS to a 
QAP and get the keys \(\ProverKey \) and \(\VerifierKey \).
The whole process is done transparently by the library:
\begin{lstlisting}[language=C++]
  auto keypair = libsnark::r1cs_ppzksnark_generator(board.get_constraint_system());
\end{lstlisting}
Now the constraint system and \(\VerifierKey \) can be made public, while \(\ProverKey \) is only 
known by the prover. 

\subsubsection*{Proving phase}
To generate a (valid) proof, the prover has to first provide a circuit input and compute the 
induced evaluation. 
At the highest level, this is done by setting the values of the input variables and letting the 
gadget generate the intermediate and the output values:
\begin{lstlisting}[language=C++]
  Field x; // some field element
  board.val(input_var) = x;
  gadget.generate_r1cs_witness(circuit_inputs);
\end{lstlisting}
Now, he can generate the proof, which is done transparently by the library by using the 
proving key and the primary (i.e.\ public) and auxiliary (i.e.\ private) inputs:
\begin{lstlisting}[language=C++]
  auto proof = libsnark::r1cs_ppzksnark_prover(keypair.pk, board.primary_input(), board.auxiliary_input());
\end{lstlisting}
Finally, the prover can make public the primary inputs of the protoboard together with the
proof.

\subsubsection*{Verification phase}
At this point, the verifier has to check the proof by using the public key, and there are two 
kinds of choices regarding how to perform such verification: 
\begin{itemize}
  \item \emph{weak} vs.\  \emph{strong} verification: in the former case, it is ok for the prover 
        to only provide some of the primary inputs, which will be zero-padded. 
        In the latter case, this is not considered acceptable.
  \item \emph{offline} vs.\  \emph{online} verification: in the former case, the verification key 
        is used `as is', in the latter, it is processed to get faster verification times when used 
        multiple times.
\end{itemize}
In any case, the whole thing is managed transparently by the library.
For example, if we choose strong, online verification:
\begin{lstlisting}[language=C++]
  auto proc_vk = libsnark::r1cs_ppzksnark_verifier_process_vk(keypair.vk);
  bool halt = libsnark::r1cs_ppzksnark_online_verifier_strong_IC(proc_vk, board.primary_input(), proof);
\end{lstlisting}
The verifier will accept the proof if \texttt{halt} is \texttt{true}, while it will reject if it 
is \texttt{false}.

\begin{example}
  Consider the R1CS in \Cref{ex:r1cs}.
  \Cref{lst:libsnark_example} shows the implementation of the corresponding gadget in 
  \texttt{libsnark}.
\end{example}

\begin{algorithm}
  \centering
  \begin{lstlisting}[caption={The \texttt{libsnark} gadget corresponding to the 
    arithmetic formula in \Cref{ex:arithmetic_formula}.}\label{lst:libsnark_example},language=C++]
    #include <libsnark/gadgetlib1/gadget.hpp>

    using namespace libsnark; // don't do this in real code please

    template<typename FieldT>
    class FooGadget : public gadget<FieldT>
    {
        using Var = pb_variable<FieldT>;

        const Var x1, x2, y;
        Var t1, t2, t3;

        FooGadget(protoboard<FieldT> &board, 
                  const std::string &ann,
                  const Var &x1, 
                  const Var &x2, 
                  const Var &y) : 
            gadget<FieldT>{board, ann}, x1{x1}, x2{x2}, y{y}
        {
            t1.allocate(board, "t1");
            t2.allocate(board, "t2");
            t3.allocate(board, "t3");
        }

        void generate_r1cs_constraints()
        {
            board.add_r1cs_constraint(r1cs_constraint<FieldT>{x1, x1, t1});
            board.add_r1cs_constraint(r1cs_constraint<FieldT>{t1, x1, 8 + 9*x2 + t2});
            board.add_r1cs_constraint(r1cs_constraint<FieldT>{x2, t2, y});
        }

        void generate_r1cs_witness()
        {
            board.val(t1) = board.val(x1) * board.val(x1);
            board.val(t2) = board.val(t1) * board.val(x1) + 4*board.val(x2) + 5;
            board.val(y)  = board.val(x2) * board.val(t2);
        }
    };
  \end{lstlisting}
\end{algorithm}


