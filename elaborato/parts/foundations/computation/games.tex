\section{Interactive proof systems}
Even though \NPcomplete{} problems are infeasible, they are \emph{efficient} to \emph{verify}: 
given an \emph{instance} \(\Pi \) of some \NPcomplete{} problem, and an additional \emph{witness}
string, we can build a deterministic TM that checks whether the witness \emph{proves} or 
not that the problem admits a positive answer.
\begin{example}
  Consider the problem \textsc{sat} of deciding whether a propositional logic formula \(\phi \) is 
  satisfiable, which is the most famous \NPcomplete{} problem~\cite{Cook1971}. 
  If we had a TM \(\mathcal{M}\) with access to an \emph{oracle} that, in \(\BigO{1}\) time, 
  provides a valid assignment for the variables in \(\phi \), it would be easy to 
  verify that \(\phi \) is indeed satisfiable.
  However, if the provided assignment was not valid, while \(\mathcal{M}\) would reject it, 
  there would be still no easy way to know whether \(\phi \) is actually satisfiable or not! 
\end{example}

Now, let's say we want to prove some theorem \(\Pi \): computationally, this is equivalent to 
deciding whether \(\Pi \) is word which belongs to the language of the valid propositions over some 
formal system (say, the ZFC set theory~\cite{FraenkelHL1973}).
A \emph{proof} of the theorem plays the same role of the \emph{witness} we discussed before: in 
general, verifying a proof for a theorem is (believed to be) much easier than finding the proof in 
the first place.
Hence, we can extend the logical/mathematical concept of theorem to the more computational concept 
of language: for example, by \NPtime{} theorem, we mean any language in \NPtime{}.

\clearpage
\begin{definition}[\textsc{NP} proof system~\cite{GoldwasserMR1989}]  
  A \textsc{NP} proof system for a language \(L\) is an interactive protocol 
  \(\mathcal{I} = \Tuple{\mathcal{P}, \mathcal{V}}\), where \(\mathcal{P}\) is the \emph{prover}
  and \(\mathcal{V}\) is the \emph{verifier}, such that:
  \begin{align*}
    & \forall w \in L\colon \call{\mathcal{I}_{\mathcal{V}}}{w} = \top & 
      \textnormal{(\emph{correctness})} \\
    & \forall w \notin L\colon \call{\mathcal{I}_{\mathcal{V}}}{w} = \bot & 
      \textnormal{(\emph{completeness})} \\
    & \exists k \in \mathbb{N}\colon \call{\Time}{\mathcal{V}} = \BigO{\abs{x}^k} & 
    \textnormal{(\emph{boundness})}
  \end{align*}
\end{definition}

In an \textsc{NP} proof system (\textsc{NP}PS), the common input tape of \(\mathcal{P}\) and 
\(\mathcal{V}\) contains some word \(w\) representing some statement: in typical scenario, the 
statement is provided by the prover himself, who wants to convince the verifier of the truthness of 
such statement.
During the protocol, \(\mathcal{P}\) and \(\mathcal{V}\) exchange messages through their 
communication tapes; at some point, \(\mathcal{P}\) sends to \(\mathcal{V}\) a candidate proof 
\(\pi \): the verifier checks the proof and, if the proof is valid, it is always convinced of its 
validity (correctness), hence it will accept. 
On the other hand, if the proof happens to be wrong (e.g.\ if \(\mathcal{P}\) is trying to deceive
\(\mathcal{V}\)), then the verifier will never be convinced by such a proof, and it will reject.
The polynomial bound on the execution time of \(\mathcal{V}\) is necessary to force cooperation, 
and avoid the case where \(\mathcal{V}\) simply ignores \(\mathcal{P}\) and computes the proof by 
itself.
\begin{definition}[Interactive proof system~\cite{GoldwasserMR1989}]  
  An interactive proof system for a language \(L\) is an interactive protocol 
  \(\mathcal{I} = \Tuple{\mathcal{P}, \mathcal{V}}\) such that, for any arbitrarily small 
  \(\epsilon \in \mathbb{R}_{+}\):
  \begin{align*}
    & \forall w \in L\colon \call{\Pr}{\call{\mathcal{I}_{\mathcal{V}}}{w} = \bot} < \abs{w}^{\epsilon}  
      & \textnormal{(\emph{probabilistic correctness})} \\
    & \forall w \notin L\colon \call{\Pr}{\call{\mathcal{I}_{\mathcal{V}}}{w} = \top} < \abs{w}^{\epsilon} 
      & \textnormal{(\emph{probabilistic completeness})} \\
    & \exists k \in \mathbb{N}\colon \call{\Time}{\mathcal{V}} = \BigO{\abs{x}^k} & 
    \textnormal{(\emph{boundness})}
  \end{align*}
\end{definition}

Clearly, the main difference between a NP proof system and an interactive proof system (IPS) lays 
in the probabilistic behaviour of the latter: there is a (negligible) probability that a verifier 
will be convinced of the truthness of a false statement, or of the falseness of a true statement, 
although the latter is not relevant in most instances, as the prover \emph{wants} the verifier 
to accept.
In fact, the proofs generated by the prover in an IPS are often called 
\emph{arguments of knowledge}, to denote their probabilistic nature. 
\begin{definition}[\textsc{IP} complexity class]
  The Interactive Polynomial class \textsc{IP} is the class of languages which are decided by some
  interactive proof system.
\end{definition}

Of course \(\textsc{NP} \subseteq \textsc{IP}\), as the verifier can check, with no error, a 
proof for any \textsc{NP-complete} problem statement in polynomial time.
Howvever, there are problems in \textsc{IP} which are not known to be in \textsc{NP}, such as 
the problems of deciding the order and the membership of a matrix group~\cite{BabaiS1984}, and 
deciding whether two graphs are not isomorphic~\cite{GoldreichMW1991}.

\subsection{Arthur-Merlin games}

In an interactive proof system, the coin tosses of the two parties are \emph{private}, and 
they can exchange any kind of message.
\begin{definition}[Arthur-Merlin games~\cite{BabaiS1984}]
  An Arthur-Merlin game is an interactive proof system \(\mathcal{I} = \Tuple{M, A}\) such that
  Arthur can only send coin tosses over \(\mathfrak{\Tapesend}\).
\end{definition}

In Arthur-Merlin games, the messages that the verifier (Arthur) can send to the prover (Merlin) are 
limited to the coin tosses from its random tape (in the sense that, whatever Arthur sends Merlin, 
he interprets it as a coin toss).

\begin{definition}[\textsc{AM} complexity class]
  The Arthur-Merlin complexity class \textsc{AM} is the class of languages which are decided by an
  Arthur-Merlin game.
\end{definition}

While it is obvious that \(\textsc{AM} \subseteq \textsc{IP}\), the converse was also shown to 
be true~\cite{BabaiS1984}: in particular, if we limit the number of communications between 
Arthur and Merlin to any constant \(k \in \mathbb{N}\), it is possible to build an equivalent 
game involving only \(2\) communications between them.


\subsection{Zero-Knowledge Interactive Protocols}

