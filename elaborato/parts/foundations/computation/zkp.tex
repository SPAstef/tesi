\section{Zero-Knowledge Protocols}\label{sec:zero_knowledge}
Suppose that two parties are executing an IARK system for some hard problem: the instance is places 
on the shared input tape, and also suppose that the secret in possession of the prover is simply 
a witness for the instance. 
All the prover has to do is send the witness to the verifier, which will in turn check it and 
decide whether to accept or not.
In this process, the verifier gained more knowledge than just the solvability of the problem: it 
also learned a solution, and not just any solution, but exactly the one available to the prover,
which should have been a secret.
To address this issue, researchers started exploring the field of so-called zero-knowledge 
proofs~\cite{GoldwasserMR1989,GoldreichMW1991}.

Informally, two random variables \(U\) and \(V\) that map words of some language 
\(L \subseteq \Set{0, 1}^{*}\) to words of \(\Set{0, 1}^{*}\) are 
\emph{perfectly indistinguishable} when no unbounded Turing machine is able to tell them apart,
are \emph{statistically indistinguishable} when no \textsc{PSPACE} Turing machine is able to 
tell them apart, and are \emph{computationally indistinguishable} when no \textsc{PTIME} Turing 
machine \(\mathcal{M}\) is able to tell them apart.
By `telling apart', we mean that the distribution of the words that are accepted/rejected 
by Turing machines respecting the imposed bounds is independent of \(U\) and \(V\): intuitively,
this means that \(U\) and \(V\) are interchangeable with each other and using one over the other 
does not give an `edge' to \(\mathcal{M}\)~\cite{GoldwasserM1984,GoldwasserMR1989,Yao1982}.
\begin{example}
  Consider the two random variables \(U, V: L \to \Set{0, 1}^{*}\) for some 
  \(L \subseteq \Set{0, 1}^{*}\), such that, for all words \(x \in L\) and all words 
  \(w \in \Set{0, 1}^{\abs{x}}\), it holds that:
  \begin{align*}
    & \call{\Pr}{\call{U}{x} = w} = 2^{-\abs{x}} &&
    \call{\Pr}{\call{V}{x} = w} = \begin{cases}
      0 & x = 0\dots0 \\
      2^{-\abs{x} + 1} & x = 1\dots1 \\
      2^{-\abs{x}} & \textrm{otherwise}
    \end{cases}
  \end{align*}
  \(U\) and \(V\) have \emph{almost} the same distribution, with the \(1\dots1\) string 
  happening twice as often in \(V\). 
  For increasingly longer strings, no Turing machine can tell the two distributions apart by 
  collecting a polynomial amount of samples, since 
  \(\sum_{w}{\abs{\call{\Pr}{\call{U}{x} = w} - \call{\Pr}{\call{V}{x} = w}}} = 2^{-\abs{x}+1}\),
  hence \(U\) and \(V\) are statistically indistinguishable.
\end{example}

\begin{definition}[Tape view]
  A \emph{tape view} is a random variable \(\View_{\mathcal{M}}\) that models the concatenation 
  of all the contents that are read/written by a halting Turing machine \(\mathcal{M}\) over its 
  tapes.
\end{definition}

For a deterministic, non-probabilistic Turing machine, the tape view variable is quite pointless, 
but it is a useful tool to model the behaviour of machines that exploit randomness, and especially 
for interactive protocols.
For example, if we have a Turing machine with one tape \(\Tape \), then:
\[
  \call{\Pr}{\call{\View_{\mathcal{M}}}{x} = w} = 
  \call{\Pr}{\call{\View_{\mathcal{M}, \Tape}}{x} = w} = 
  \call{\Pr}{\call{\mathcal{M}}{x} = w}
\]


\begin{definition}[Approximability]
  A random variable \(U\) is (perfectly, statistically, computationally) \emph{approximable} by a 
  probabilistic Turing machine \(\mathcal{M}\) over some language \(L\) if \(U\) and 
  \(\View_{\mathcal{M}}\) are (perfectly, statistically, computationally) indistinguishable.
\end{definition}

Note that for a random variable \(U\) and a halting PTM \(\mathcal{M}\) to be perfectly 
indistinguishable over some language \(L\), it must be the case that 
\(\forall x \in L\colon \call{\mathcal{M}}{x} = \call{\View_{\mathcal{M}}}{x} = \call{\mathcal{U}}{x}\).

\begin{definition}[Zero-knowledge interactive protocol]
  A (perfectly, statistically, computationally) \emph{Zero-knowledge interactive protocol} (ZKIP) 
  over a language \(L \subseteq \Set{0, 1}^{*}\) is an interactive protocol 
  \(\mathcal{I} = \Tuple{\mathcal{M}, \mathcal{M}'}\) such that, for every \(\mathcal{M'}\),
  \(\View_{\mathcal{M'}}\) is (perfectly, statistically, computationally) approximable by a 
  Turing machine \(\mathcal{M}''\) over the language 
  \(L' = \Set{\Tuple{x, h} \mid x \in L \wedge w \in \Set{0, 1}^*}\), where the string
  \(h\) represents the initial content of \(\Tapework'\).
\end{definition}

Naturally, a ZKIP which is also a proof system is a zero-knowledge proof system (ZKPS); similarly, 
if it is an interactive argument of knowledge system then it is a zero-knowledge interactive 
argument of knowledge system (ZK-IARK).
From now on, by zero-knowledge we mean computational zero-knowledge, as assuming a polynomial-time 
bounded adversary is an acceptable restriction in the real world.
The initial string \(h\) of a ZKIP can be interpreted as the \emph{history} of previous 
interactions with the prover, or some eavesdropped information from the interactions that the 
prover had with other verifiers.

A proof system being zero-knowledge basically means that, even for curious or malicious verifiers,
and even with additional knowledge on the behaviour of the prover, what can be computed is nothing 
more than what could have been computed in polynomial time, hence within the imposed computational 
power limits, without communicating with the prover.
While it is obvious that every problem solvable in probabilistic polynomial time 
(\textsc{PP}) has a zero-knowledge proof system (the prover does nothing and the verifier computes 
the solution by himself), it was proven that also all problems in \textsc{NP} have a 
ZKPS~\cite{GoldreichMW1991}. 
By assuming the existence of secure probabilistic encryption, it was finally shown that also 
all Arthur-Merlin games, and hence all problems in \textsc{IP}, have a ZKPS~\cite{BenorGGHKMR1990}.

\subsection{Non-interactive Zero-Knowledge}\label{subsec:nizk}
In many scenarios, especially ones involving multiple parties, interaction can be a problem as
the communication cost of bidirectional \(n\)-to-\(n\) grows quadratically.
Such cases are in fact of great interest for zero-knowledge systems: multiple parties can be 
both provers and/or verifiers, and their number might be huge.

For this reason, researches explored the possibility of having zero knowledge \emph{non-interactive} 
proof systems (ZK-NPS) or argument of knowledge systems (ZK-NARK).
Unfortunately, only the languages in \textsc{BPP}, that is languages decidable in 
probabilistic polynomial time with a bounded error allow for zero-knowledge non-interactive 
proofs~\cite{Oren1987,GoldreichK1996}. Such languages are of course trivial, as the verifier has 
enough power to do all the computation by itself without the need of the prover.

However, by introducing an initial \emph{preprocessing} phase, it is possible to regain the lost 
power~\cite{SantisMP1990}, and the most prominent technique to achieve non-interaction is the 
\emph{Common Reference String} (CRS) model~\cite{BlumFM1988} (sometimes also called 
\emph{common random string} model).
The main idea of the CRS model is that, before engaging in the protocol, the prover and 
the verifier have both obtained access to a shared string of random bits. 
In the simplest case, the string is generated by a \emph{trusted third party}, although in 
practice this is oftentimes not a viable solution as the whole point of zero-knowledge is having 
to deal with untrusted parties. 
To circumvent this problem, it is possible to generate the CRS by a \emph{majority vote}
between \(n\) authorities, which can be untrusted if picked singularly, but are assumed to be 
honest in their majority~\cite{GrothO2006}.
In fact, it was shown that it is possible, without losing zero-knowledge, to re-use multiple times 
a single CRS both by a single~\cite{BlumSMP1991} or multiple~\cite{FeigeLS1990} provers, although 
only for arguments of knowledge and not for proofs.

The first zero-knowledge systems, both interactive and non-interactive, were tailor-made for 
specific problems, such as the quadratic residuosity problem \textsc{qr}~\cite{GoldwasserMR1989}, 
the Hamiltonian path problem \textsc{hampath}~\cite{LapidotS1991}, or the \(3\)-\textsc{sat} 
problem~\cite{BlumSMP1991}.
Although for any \textsc{NP-complete} problem \textsc{prob} there is a polynomial-time 
algorithm~\cite{Karp1972} that converts every instance of \textsc{prob} to an instance 
of, say, \(3\)-\textsc{sat}, such reductions are often not trivial to devise and very expensive 
to apply.
For this reason, researchers started devising constructions to prove arbitrary NP statements 
embedded in the form of boolean circuits~\cite{Damgard1993}, which can neatly represent the 
computation of a Turing machine over any \textsc{NP-complete} problem~\cite{Cook1971}, and 
therefore remove the need to go through polynomial-time reductions.

The first of such systems~\cite{Damgard1993} required a CRS of size cubic in the length of the 
statement to be proven, although XOR and NOT gates didn't need to consume any bits from the CRS\@.
In the following years, many improvements were proposed, reducing the complexity of the 
constructions from cubic to sub-quadratic~\cite{BoyarBP1995} and eventually linear~\cite{CramerD1997}.
