\section{Interactive Turing machines}
In many scenarios, it is useful to extend Turing machines to include additional features, for 
example to represent the ability to access some source of randomness, or to communicate with an 
external environment to read inputs and produce outputs in an interactive manner.
\begin{definition}[Input/Output Turing machine]
  An input/output Turing machine is a quadruple \(\mathcal{M} = \Tuple{\Sigma, Q, q_0, \delta}\)
  where \(\Sigma \), \(Q\) and \(q_0\) are defined as in \Cref{def:turing_machine}, and:
  \[
    \delta\colon {\Parens*{Q \setminus \Set{\bot, \top}} \times \Sigma^2} \to 
    {Q \times \Sigma^2 \times \Set{\leftarrow, \rightarrow}^2}
  \]  
\end{definition}

The additional parameters in the transition function of an input/output Turing machine (I/O TM) 
account for two new tapes, namely the \emph{read-only input tape} \(\Tapein \) and the 
\emph{write-only output tape} \(\Tapeout \): now, depending on the state \(q\), the input symbol 
\(\sigma_{i}\) and the working symbol \(\sigma_{w}\), the machine overwrites \(\sigma_{w}\) with a 
new symbol \(\sigma'_{w}\) and moves left/right on \(\Tapework \), it writes a new symbol 
\(\sigma_{o}\) on \(\Tapeout \), where it can move only to the right, and it moves to the 
left/right on \(\Tapein \).
Additionally, in an I/O TM, the input string is placed on \(\Tapein \) instead of \(\Tapework \), 
which is instead blank at the beginning of the computation.
\begin{definition}[Probabilistic Turing machine]
  A probabilistic Turing machine is a quadruple \(\mathcal{M} = \Tuple{\Sigma, Q, q_0, \delta}\)
  where \(\Sigma \), \(Q\) and \(q_0\) are defined as in \Cref{def:turing_machine}, and:
  \[
    \delta\colon {\Parens*{Q \setminus \Set{\bot, \top}} \times \Sigma \times \Set{0, 1}} \to 
    {Q \times \Sigma \times \Set{\leftarrow, \rightarrow}}
  \]    
\end{definition}

In a probabilistic Turing machine (PTM), we have an additional \emph{read-only random tape} 
\(\Taperand \) which is populated with an infinite, uniformly random sequence of \emph{coin tosses} 
(zeros and ones), that are used by the transition function to decide what to do.
As for the writing tape of an I/O TM, the head on \(\Taperand \) can only move to the right.
\begin{definition}[Interactive Turing machine]
  An interactive Turing machine is a quadruple \(\mathcal{M} = \Tuple{\Sigma, Q, q_0, \delta}\)
  where \(\Sigma \), \(Q\) and \(q_0\) are defined as in \Cref{def:turing_machine}, and:
  \[
    \delta\colon {\Parens*{Q \setminus \Set{\bot, \top}} \times \Sigma^2} \to
    {Q \times \Sigma^2 \times \Set{\leftarrow, \rightarrow}}
  \]
\end{definition}

An interactive Turing machine (ITM) is quite similar to an I/O TM, as it also has two additional 
tapes, called the \emph{send tape} \(\Tapesend \) and the \emph{receive tape} \(\Taperec \).
However, unlike for the input tape \(\Tapein \) of an I/O TM, the head on \(\Taperec \) cannot move 
backwards.
\begin{remark}  
  Our definition of ITM differs slightly from the standard one in the 
  literature~\cite{GoldreichMW1991,GoldwasserMR1989}, but we find it to be more modular.
  In any case, from now on, we will say \emph{interactive Turing machine} to actually mean an 
 \emph{interactive, probabilistic, input/output Turing machine}.
\end{remark}

\begin{definition}[Interactive protocol]
  An interactive protocol is a pair \(\mathcal{I} = \Tuple{\mathcal{M}, \mathcal{M'}}\)
  where \(\mathcal{M}\) and \(\mathcal{M'}\) are interactive Turing machines such that 
  \(\Tapein = \Tapein'\), \(\Tapesend = \Taperec' \), \(\Taperec = \Tapesend'\), and their 
  state sets \(Q, Q'\) contain the special \emph{idle state} \(q_{idle} \in Q, Q'\).
\end{definition}

The computation of an interactive protocol (IP) over some string \(\overbar{\sigma}\), 
\(\call{\mathcal{I}}{\overbar{\sigma}}\), proceeds in the following manner: 
initially, the tapes \(\Tapesend \), \(\Taperec \), \(\Tapework \), \(\Tapework' \), \(\Tapeout \) 
and \(\Tapeout' \) are all empty (i.e.\ filled with blank symbols), the tapes \(\Taperand \) and 
\(\Taperand' \) are filled with random bits, and the tape \(\Tapein \) contains \(\overbar{\sigma}\).
The ITM \(\mathcal{M}\) is said to be \emph{active} and works normally until it transitions in the 
special state \(q_{idle}\), becoming \emph{inactive}.
When this happens, control passes to \(\mathcal{M}'\), which becomes active and works normally 
until it reaches its own idle state; control goes back to \(\mathcal{M}\), and the process repeats.
When one of the two machines halts, control passes over the other one until it also halts.
The protocol \emph{succeeds} if both machines halt in the accepting state \(\top \), and it 
\emph{fails} if at least one of them halts in the rejecting state \(\bot \).
To denote the final states reached by one of the machines at the end of the computation, 
we write \(\call{\mathcal{I}_{\mathcal{M}}}{\overbar{\sigma}}\) and 
\(\call{\mathcal{I}_{\mathcal{M}'}}{\overbar{\sigma}}\) respectively.
\Cref{fig:interactive_protocol} depicts the fundamental structure of an interactive protocol.

\begin{figure}
  \centering
  \begin{tikzpicture}[node distance=64pt,on grid,auto]
    \node[state,shape=rectangle,minimum height=16pt, minimum width=48pt] (in)  {\(\Tapein = \Tapein'\)};
    \node[state,shape=rectangle,minimum size=24pt,left =of in]   (m0)         {\(\mathcal{M}\)};
    \node[state,shape=rectangle,minimum size=24pt,right =of in] (m1)  {\(\mathcal{M}'\)};
    \node[state,shape=rectangle,minimum height=16pt, minimum width=48pt,above left =of m0] (p0)  {\(\Taperand \)};
    \node[state,shape=rectangle,minimum height=16pt, minimum width=48pt,above right =of m1] (p1)  {\(\Taperand' \)};
    \node[state,shape=rectangle,minimum height=16pt, minimum width=48pt,below left =of m0] (w0)  {\(\Tapework \)};
    \node[state,shape=rectangle,minimum height=16pt, minimum width=48pt,below right =of m1] (w1)  {\(\Tapework' \)};
    \node[state,shape=rectangle,minimum height=16pt, minimum width=48pt,above =of in] (s0)  {\(\Tapesend = \Taperec' \)};
    \node[state,shape=rectangle,minimum height=16pt, minimum width=48pt,below =of in] (r0)  {\(\Taperec = \Tapesend' \)};
    \node[state,shape=rectangle,minimum height=16pt, minimum width=48pt,left =of m0] (o0)  {\(\Tapeout \)};
    \node[state,shape=rectangle,minimum height=16pt, minimum width=48pt,right =of m1] (o1)  {\(\Tapeout' \)};
    \path[->]
    (in) edge (m0)
    (in) edge (m1)
    (p0) edge (m0)
    (p1) edge (m1)
    (w0) edge (m0)
    (m0) edge (w0)
    (w1) edge (m1)
    (m1) edge (w1)
    (m0) edge (s0)
    (s0) edge (m1)
    (m1) edge (r0)
    (r0) edge (m0)
    (m0) edge (o0)
    (m1) edge (o1)
    ;
  \end{tikzpicture}
  \caption{Visualization of an interactive protocol.}\label{fig:interactive_protocol}
\end{figure}
