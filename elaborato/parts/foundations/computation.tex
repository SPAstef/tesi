\chapter{Computational Background}
A \emph{computational model} (or model of computation) is any kind of system able to describe 
how to produce some \emph{output} given some \emph{input}~\cite{Savage1997}.
Different models do this in different ways, each one with its own strength and weaknesses in terms
of \emph{expressivness}, \emph{complexity} and \emph{succintness}.
Two historically important models of computations are Alonzo Church's 
\emph{\(\lambda \)-calculus}~\cite{Church1941} and Alan Turing's 
\emph{Turing machine} (TM)~\cite{Turing1950}. 
Among several equivalent models~\cite{Davis2004}, Turing machines became the standard model of 
computation.
\begin{definition}[Turing machine~\cite{Papadimitriou1994}]\label{def:turing_machine}
  A Turing machine is a quadruple \(\mathcal{M} = \Tuple{\Sigma, Q, q_0, \delta}\), where 
  the \emph{alphabet} \(\Sigma \) is a set of symbols such that \(\sqcup \in \Sigma \), the 
  \emph{state set} \(Q\) is a set of symbols such that \(\Set{\bot, \top} \subseteq Q\), 
  \(q_0 \in Q\) is the \emph{initial state}, and 
  \(\delta\colon {\Parens*{Q \setminus \Set{\bot, \top}} \times \Sigma} \to 
  {Q \times \Sigma \times \Set{\leftarrow, \rightarrow}}\) is the \emph{transition function}.
\end{definition}

By only requiring \(\delta \) to be a relation instead of a function, we obtain the so-called 
\emph{non-deterministic} Turing machine (NTM): given a state and an alphabet symbol, the machine 
can take different choiches at every step.
A TM \(\mathcal{M}\) manipulates a string \(\overbar{\sigma}\) over the alphabet 
\(\Sigma \setminus \Set{\sqcup}\) by placing it over an \emph{infinite, discrete working tape} 
\(\Tapework \), a total order isomorphic to \(\mathbb{Z}\).
The input string is positioned such that its first symbol is matched with the position \(0\) 
of the tape; all the positions before the first symbol and after the last symbol are filled 
with the \emph{blank} symbol \(\sqcup \).
The computation \(\call{\mathcal{M}}{\overbar{\sigma}}\) starts in the initial state \(q_0\) with 
the \emph{head} of the TM positioned over the position \(0\) of the tape, and proceeds according to 
the transition function: depending on the current state \(q\) and the symbol \(\sigma \) written in 
the current location of the head, it replaces \(\sigma \) with a new symbol \(\sigma'\), it moves 
the head to the left (\(\leftarrow \)) or to the right (\(\rightarrow \)) and it transitions into a 
new state \(q'\).
The computation \emph{terminates} whenever one of the two \emph{halting} states is reached: if
\(\call{\mathcal{M}}{\overbar{\sigma}} = \bot \), then the input string \(\overbar{\sigma}\) is 
\emph{rejected}, else if \(\call{\mathcal{M}}{\overbar{\sigma}} = \top \), then \(\overbar{\sigma}\) 
is \emph{accepted}.
It can also happen that the computation does not terminate: in such cases, we write 
\(\call{\mathcal{M}}{\overbar{\sigma}} = {\uparrow}\) and we say that the computation \emph{hangs}.

\section{Interactive Turing machines}
In many scenarios, it is useful to extend Turing machines to include additional features, for 
example to represent the ability to access some source of randomness, or to communicate with an 
external environment to read inputs and produce outputs in an interactive manner.
\begin{definition}[Input/Output Turing machine]
  An input/output Turing machine is a quadruple \(\mathcal{M} = \Tuple{\Sigma, Q, q_0, \delta}\)
  where \(\Sigma \), \(Q\) and \(q_0\) are defined as in \Cref{def:turing_machine}, and:
  \[
    \delta\colon {\Parens*{Q \setminus \Set{\bot, \top}} \times \Sigma^2} \to 
    {Q \times \Sigma^2 \times \Set{\leftarrow, \rightarrow}^2}
  \]  
\end{definition}

The additional parameters in the transition function of an input/output Turing machine (I/O TM) 
account for two new tapes, namely the \emph{read-only input tape} \(\Tapein \) and the 
\emph{write-only output tape} \(\Tapeout \): now, depending on the state \(q\), the input symbol 
\(\sigma_{i}\) and the working symbol \(\sigma_{w}\), the machine overwrites \(\sigma_{w}\) with a 
new symbol \(\sigma'_{w}\) and moves left/right on \(\Tapework \), it writes a new symbol 
\(\sigma_{o}\) on \(\Tapeout \), where it can move only to the right, and it moves to the 
left/right on \(\Tapein \).
Additionally, in an I/O TM, the input string is placed on \(\Tapein \) instead of \(\Tapework \), 
which is instead blank at the beginning of the computation.
\begin{definition}[Probabilistic Turing machine]
  A probabilistic Turing machine is a quadruple \(\mathcal{M} = \Tuple{\Sigma, Q, q_0, \delta}\)
  where \(\Sigma \), \(Q\) and \(q_0\) are defined as in \Cref{def:turing_machine}, and:
  \[
    \delta\colon {\Parens*{Q \setminus \Set{\bot, \top}} \times \Sigma \times \Set{0, 1}} \to 
    {Q \times \Sigma \times \Set{\leftarrow, \rightarrow}}
  \]    
\end{definition}

In a probabilistic Turing machine (PTM), we have an additional \emph{read-only random tape} 
\(\Taperand \) which is populated with an infinite, uniformly random sequence of \emph{coin tosses} 
(zeros and ones), that are used by the transition function to decide what to do.
As for the writing tape of an I/O TM, the head on \(\Taperand \) can only move to the right.
\begin{definition}[Interactive Turing machine]
  An interactive Turing machine is a quadruple \(\mathcal{M} = \Tuple{\Sigma, Q, q_0, \delta}\)
  where \(\Sigma \), \(Q\) and \(q_0\) are defined as in \Cref{def:turing_machine}, and:
  \[
    \delta\colon {\Parens*{Q \setminus \Set{\bot, \top}} \times \Sigma^2} \to
    {Q \times \Sigma^2 \times \Set{\leftarrow, \rightarrow}}
  \]
\end{definition}

An interactive Turing machine (ITM) is quite similar to an I/O TM, as it also has two additional 
tapes, called the \emph{send tape} \(\Tapesend \) and the \emph{receive tape} \(\Taperec \).
However, unlike for the input tape \(\Tapein \) of an I/O TM, the head on \(\Taperec \) cannot move 
backwards.
\begin{remark}  
  Our definition of ITM differs slightly from the standard one in the 
  literature~\cite{GoldreichMW1991,GoldwasserMR1989}, but we find it to be more modular.
  In any case, from now on, we will say \emph{interactive Turing machine} to actually mean an 
 \emph{interactive, probabilistic, input/output Turing machine}.
\end{remark}

\begin{definition}[Interactive protocol]
  An interactive protocol is a pair \(\mathcal{I} = \Tuple{\mathcal{M}, \mathcal{M'}}\)
  where \(\mathcal{M}\) and \(\mathcal{M'}\) are interactive Turing machines such that 
  \(\Tapein = \Tapein'\), \(\Tapesend = \Taperec' \), \(\Taperec = \Tapesend'\), and their 
  state sets \(Q, Q'\) contain the special \emph{idle state} \(q_{idle} \in Q, Q'\).
\end{definition}

The computation of an interactive protocol (IP) over some string \(\overbar{\sigma}\), 
\(\call{\mathcal{I}}{\overbar{\sigma}}\), proceeds in the following manner: 
initially, the tapes \(\Tapesend \), \(\Taperec \), \(\Tapework \), \(\Tapework' \), \(\Tapeout \) 
and \(\Tapeout' \) are all empty (i.e.\ filled with blank symbols), the tapes \(\Taperand \) and 
\(\Taperand' \) are filled with random bits, and the tape \(\Tapein \) contains \(\overbar{\sigma}\).
The ITM \(\mathcal{M}\) is said to be \emph{active} and works normally until it transitions in the 
special state \(q_{idle}\), becoming \emph{inactive}.
When this happens, control passes to \(\mathcal{M}'\), which becomes active and works normally 
until it reaches its own idle state; control goes back to \(\mathcal{M}\), and the process repeats.
When one of the two machines halts, control passes over the other one until it also halts.
The protocol \emph{succeeds} if both machines halt in the accepting state \(\top \), and it 
\emph{fails} if at least one of them halts in the rejecting state \(\bot \).
To denote the final states reached by one of the machines at the end of the computation, 
we write \(\call{\mathcal{I}_{\mathcal{M}}}{\overbar{\sigma}}\) and 
\(\call{\mathcal{I}_{\mathcal{M}'}}{\overbar{\sigma}}\) respectively.
\Cref{fig:interactive_protocol} depicts the fundamental structure of an interactive protocol.

\begin{figure}
  \centering
  \begin{tikzpicture}[node distance=64pt,on grid,auto]
    \node[state,shape=rectangle,minimum height=16pt, minimum width=48pt] (in)  {\(\Tapein = \Tapein'\)};
    \node[state,shape=rectangle,minimum size=24pt,left =of in]   (m0)         {\(\mathcal{M}\)};
    \node[state,shape=rectangle,minimum size=24pt,right =of in] (m1)  {\(\mathcal{M}'\)};
    \node[state,shape=rectangle,minimum height=16pt, minimum width=48pt,above left =of m0] (p0)  {\(\Taperand \)};
    \node[state,shape=rectangle,minimum height=16pt, minimum width=48pt,above right =of m1] (p1)  {\(\Taperand' \)};
    \node[state,shape=rectangle,minimum height=16pt, minimum width=48pt,below left =of m0] (w0)  {\(\Tapework \)};
    \node[state,shape=rectangle,minimum height=16pt, minimum width=48pt,below right =of m1] (w1)  {\(\Tapework' \)};
    \node[state,shape=rectangle,minimum height=16pt, minimum width=48pt,above =of in] (s0)  {\(\Tapesend = \Taperec' \)};
    \node[state,shape=rectangle,minimum height=16pt, minimum width=48pt,below =of in] (r0)  {\(\Taperec = \Tapesend' \)};
    \node[state,shape=rectangle,minimum height=16pt, minimum width=48pt,left =of m0] (o0)  {\(\Tapeout \)};
    \node[state,shape=rectangle,minimum height=16pt, minimum width=48pt,right =of m1] (o1)  {\(\Tapeout' \)};
    \path[->]
    (in) edge (m0)
    (in) edge (m1)
    (p0) edge (m0)
    (p1) edge (m1)
    (w0) edge (m0)
    (m0) edge (w0)
    (w1) edge (m1)
    (m1) edge (w1)
    (m0) edge (s0)
    (s0) edge (m1)
    (m1) edge (r0)
    (r0) edge (m0)
    (m0) edge (o0)
    (m1) edge (o1)
    ;
  \end{tikzpicture}
  \caption{Visualization of an interactive protocol.}\label{fig:interactive_protocol}
\end{figure}

\section{Problems and complexity}\label{sec:complexity}
Historically, the most important class of problems that have been analyzed are so-called
\emph{decision problems}, i.e.\ problems whose solution is a binary \emph{yes-or-no} 
answer~\cite{Sipser2013}.
This perfectly suits Turing machines as we can interpret their acceptance or rejection of the input 
string respectively as a yes and a no answer.

\begin{definition}[Kleene's closure]
  The Kleene's closure of a set \(S\) is the set \(S^* = \bigcup_{n \in \mathbb{N}}{S^n}\).
\end{definition}

As Turing machines operate over strings in \(\Sigma^*\), also called \emph{words}, they partition 
\(\Sigma^* \) into three \emph{languages} (a language is any set of strings): the language of 
accepted words, the language of rejected words and the language of hanging words.
\begin{definition}[Turing-recognizable language]
  A language \(L \subseteq \Sigma^*\) is recognized by some Turing machine \(\mathcal{M}\) if 
  \(\forall w \in L\colon \call{\mathcal{M}}{w} = \top \).
\end{definition}
\begin{definition}[Turing-decidable language]
  A language \(L \subseteq \Sigma^*\) is decided by some Turing machine \(\mathcal{M}\) if it is 
  recognized by \(\mathcal{M}\) and \(\forall w \notin L\colon \call{\mathcal{M}}{w} = \bot \).
\end{definition}

We denote the language recognized by a Turing machine \(\mathcal{M}\) with \(\call{L}{\mathcal{M}}\).
To solve an \emph{instance} \(\Pi \) of some decision problem \textsc{prob}, we first encode the 
instance into a string \(\Encode{\Pi} \in \Sigma^*\) such that 
\(\Encode{\Pi} \in \call{L}{\mathcal{M}}\) if and only if the answer to \(\Pi \) is `yes'.

The class of recognizable languages, called \textsc{RE}, strictly includes the class of decidable 
languages, called \textsc{DEC}~\cite{Turing1937}.
But even decidable languages are not all equal: their \emph{computational complexity}, that is,
the amount of some resource which is required by a Turing machine to decide membership words in 
function of their length, can vary wildly.
In general, we are only interested in the \emph{asymptotic behaviour} of the machine.
\begin{definition}[Big-O notation]
  Given two functions \(f, g\colon \mathbb{N} \to \mathbb{N}\), then \(f = \BigO{g}\) if 
  and only if \(\exists c,n\) such that \(\forall x \ge n\) \(\call{f}{x} \le c \cdot \call{g}{x}\).
\end{definition}

We write \(f = \BigOmega{g}\) when \(g = \BigO{f}\), and we write \(f = \BigTheta{g}\) when 
\(f = \BigO{g}\) and \(g = \BigO{f}\).
When there exists a TM \(\mathcal{M}\) for which some complexity metric \(\Complexity \) is 
upper-bounded at most by a polynomial function in the length \(n\) of the input word 
(i.e.\  \(\call{\Complexity}{\mathcal{M}} = \BigO{n^c}\) for some constant \(c \in \mathbb{N}\)), 
we say that deciding the language is \emph{feasible} w.r.t.\  \(\Complexity \). 
On the other hand, if \(\Complexity \) is upper-bounded at least by an 
exponential function (i.e.\  \(\call{\Complexity}{\mathcal{M}} = \BigO{c^{n}}\) for some constant 
\(c > 1\)), we say that the problem is \emph{infeasible} w.r.t.\  \(\Complexity \).
The standard complexity metrics are \emph{time} \(\Time \), that is the amount of transition steps 
a TM performs before halting, and \emph{space} \(\Space \), that is the amount of tape locations 
visited by a TM before halting\footnote{It is always the case that \(\Space \le \Time \).}.

Two of the most important 
\emph{complexity classes}\footnote{\url{https://complexityzoo.net/Complexity_Zoo}} 
are \textsc{PTIME} (\Ptime{} for short) and \textsc{NPTIME} (\NPtime{} for short), which are the 
classes of languages decidable respectively by a deterministic TM and a nondeterministic TM using 
at most polynomial time.
While we do not know if \(\Ptime \maybeequals \NPtime \), it is widely believed that 
\(\Ptime \subset \NPtime \): for a deterministic Turing machine, deciding \NPcomplete{} problems 
(i.e.\ the hardest problems in \NPtime{}) will generally take an exponential amount of 
time, and there is no known way in the physical world to build non-deterministic Turing machines.
Although quantum computers have been shown to be able to crack problems which are believed to 
be infeasible for standard computers, like integer factorization~\cite{Shor1994}, \NPcomplete{}
problems seem to be out of reach also for such powerful machines.

\section{Interactive proof systems}
Even though \NPcomplete{} problems are infeasible, they are \emph{efficient} to \emph{verify}: 
given an \emph{instance} \(\Pi \) of some \NPcomplete{} problem, and an additional \emph{witness}
string, we can build a deterministic TM that checks whether the witness \emph{proves} or 
not that the problem admits a positive answer.
\begin{example}
  Consider the problem \textsc{sat} of deciding whether a propositional logic formula \(\phi \) is 
  satisfiable, which is the most famous \NPcomplete{} problem~\cite{Cook1971}. 
  If we had a TM \(\mathcal{M}\) with access to an \emph{oracle} that, in \(\BigO{1}\) time, 
  provides a valid assignment for the variables in \(\phi \), it would be easy to 
  verify that \(\phi \) is indeed satisfiable.
  However, if the provided assignment was not valid, while \(\mathcal{M}\) would reject it, 
  there would be still no easy way to know whether \(\phi \) is actually satisfiable or not! 
\end{example}

Now, let's say we want to prove some theorem \(\Pi \): computationally, this is equivalent to 
deciding whether \(\Pi \) is word which belongs to the language of the valid propositions over some 
formal system (say, the ZFC set theory~\cite{FraenkelHL1973}).
A \emph{proof} of the theorem plays the same role of the \emph{witness} we discussed before: in 
general, verifying a proof for a theorem is (believed to be) much easier than finding the proof in 
the first place.
Hence, we can extend the logical/mathematical concept of theorem to the more computational concept 
of language: for example, by \NPtime{} theorem, we mean any language in \NPtime{}.

\clearpage
\begin{definition}[Interactive proof system~\cite{GoldwasserMR1989}]  
  An interactive proof system for a language \(L\) is an interactive protocol 
  \(\mathcal{I} = \Tuple{\mathcal{P}, \mathcal{V}}\), where \(\mathcal{P}\) is the \emph{prover}
  and \(\mathcal{V}\) is the \emph{verifier}, such that:
  \begin{align*}
    & \forall w \in L\colon \call{\mathcal{I}_{\mathcal{V}}}{w} = \top & 
      \textnormal{(\emph{correctness})} \\
    & \forall w \notin L\colon \call{\mathcal{I}_{\mathcal{V}}}{w} = \bot & 
      \textnormal{(\emph{completeness})} \\
    & \exists k \in \mathbb{N}\colon \call{\Time}{\mathcal{V}} = \BigO{\abs{x}^k} & 
    \textnormal{(\emph{boundness})}
  \end{align*}
\end{definition}

In an interactive proof system (IPS), the common input tape of \(\mathcal{P}\) and \(\mathcal{V}\)
contains some word \(w\) representing some statement: in typical scenario, the statement is 
provided by the prover himself, who wants to convince the verifier of the truthness of such 
statement.
During the protocol, \(\mathcal{P}\) and \(\mathcal{V}\) exchange messages through their 
communication tapes; at some point, \(\mathcal{P}\) sends to \(\mathcal{V}\) a candidate proof 
\(\pi \): the verifier checks the proof and, if the proof is valid, it is always convinced of its 
validity (correctness), hence it will accept. 
On the other hand, if the proof happens to be wrong (e.g.\ if \(\mathcal{P}\) is trying to deceive
\(\mathcal{V}\)), then the verifier will never be convinced by such a proof, and it will reject.
The polynomial bound on the execution time of \(\mathcal{V}\) is necessary to force cooperation, 
and avoid the case where \(\mathcal{V}\) simply ignores \(\mathcal{P}\) and computes the proof by 
itself.
\begin{definition}[Probabilistic interactive proof system~\cite{GoldwasserMR1989}]  
  A probabilistic interactive proof system for a language \(L\) is an interactive protocol 
  \(\mathcal{I} = \Tuple{\mathcal{P}, \mathcal{V}}\) such that, for any arbitrarily small 
  \(\epsilon \in \mathbb{R}_{+}\):
  \begin{align*}
    & \forall w \in L\colon \call{\Pr}{\call{\mathcal{I}_{\mathcal{V}}}{w} = \bot} < \epsilon  & 
      \textnormal{(\emph{probabilistic correctness})} \\
    & \forall w \notin L\colon \call{\Pr}{\call{\mathcal{I}_{\mathcal{V}}}{w} = \top} < \epsilon & 
      \textnormal{(\emph{probabilistic completeness})} \\
    & \exists k \in \mathbb{N}\colon \call{\Time}{\mathcal{V}} = \BigO{\abs{x}^k} & 
    \textnormal{(\emph{boundness})}
  \end{align*}
\end{definition}

\section{Zero-Knowledge Protocols}\label{sec:zero_knowledge}
Suppose that two parties are executing an IARK system for some hard problem: the instance is places 
on the shared input tape, and also suppose that the secret in possession of the prover is simply 
a witness for the instance. 
All the prover has to do is send the witness to the verifier, which will in turn check it and 
decide whether to accept or not.
In this process, the verifier gained more knowledge than just the solvability of the problem: it 
also learned a solution, and not just any solution, but exactly the one available to the prover,
which should have been a secret.
To address this issue, researchers started exploring the field of so-called zero-knowledge 
proofs~\cite{GoldwasserMR1989,GoldreichMW1991}.

Informally, two random variables \(U\) and \(V\) that map words of some language 
\(L \subseteq \Set{0, 1}^{*}\) to words of \(\Set{0, 1}^{*}\) are 
\emph{perfectly indistinguishable} when no unbounded Turing machine is able to tell them apart,
are \emph{statistically indistinguishable} when no \textsc{PSPACE} Turing machine is able to 
tell them apart, and are \emph{computationally indistinguishable} when no \textsc{PTIME} Turing 
machine \(\mathcal{M}\) is able to tell them apart.
By `telling apart', we mean that the distribution of the words that are accepted/rejected 
by Turing machines respecting the imposed bounds is independent from \(U\) and \(V\): intuitively,
this means that \(U\) and \(V\) are interchangable with each other and using one over the other 
does not give an `edge' to \(\mathcal{M}\)~\cite{GoldwasserM1984,GoldwasserMR1989,Yao1982}.
\begin{example}
  Consider the two random variables \(U, V: L \to \Set{0, 1}^{*}\) for some 
  \(L \subseteq \Set{0, 1}^{*}\), such that, for all words \(x \in L\) and all words 
  \(w \in \Set{0, 1}^{\abs{x}}\), it holds that:
  \begin{align*}
    & \call{\Pr}{\call{U}{x} = w} = 2^{-\abs{x}} &&
    \call{\Pr}{\call{V}{x} = w} = \begin{cases}
      0 & x = 0\dots0 \\
      2^{-\abs{x} + 1} & x = 1\dots1 \\
      2^{-\abs{x}} & \textrm{otherwise}
    \end{cases}
  \end{align*}
  \(U\) and \(V\) have \emph{almost} the same distribution, with the \(1\dots1\) string 
  happening twice as often in \(V\). 
  For increasingly longer strings, no Turing machine can tell the two distributions apart by 
  collecting a polynomial amount of samples, since 
  \(\sum_{w}{\abs{\call{\Pr}{\call{U}{x} = w} - \call{\Pr}{\call{V}{x} = w}}} = 2^{-\abs{x}+1}\),
  hence \(U\) and \(V\) are statistically indistinguishable.
\end{example}

\begin{definition}[Tape view]
  A \emph{tape view} is a random variable \(\View_{\mathcal{M}}\) that models the concatenation 
  of all the contents that are read/written by a halting Turing machine \(\mathcal{M}\) over its 
  tapes.
\end{definition}

For a deterministic, non-probabilistic Turing machine, the tape view variable is quite pointless, 
but it is a useful tool to model the behaviour of machines that exploit randomness, and expecially 
for interactive protocols.
For example, if we have a Turing machine with one tape \(\Tape \), then:
\[
  \call{\Pr}{\call{\View_{\mathcal{M}}}{x} = w} = 
  \call{\Pr}{\call{\View_{\mathcal{M}, \Tape}}{x} = w} = 
  \call{\Pr}{\call{\mathcal{M}}{x} = w}
\]


\begin{definition}[Approximability]
  A random variable \(U\) is (perfectly, statistically, computationally) \emph{approximable} by a 
  probabilistic Turing machine \(\mathcal{M}\) over some language \(L\) if \(U\) and 
  \(\View_{\mathcal{M}}\) are (perfectly, statistically, computationally) indistinguishable.
\end{definition}

Note that for a random variable \(U\) and a halting PTM \(\mathcal{M}\) to be perfectly 
indistinguishable over some language \(L\), it must be the case that 
\(\forall x \in L\colon \call{\mathcal{M}}{x} = \call{\View_{\mathcal{M}}}{x} = \call{\mathcal{U}}{x}\).

\begin{definition}[Zero-knowledge interactive protocol]
  A (perfectly, statistically, computationally) \emph{Zero-knowledge interactive protocol} (ZKIP) 
  over a language \(L \subseteq \Set{0, 1}^{*}\) is an interactive protocol 
  \(\mathcal{I} = \Tuple{\mathcal{M}, \mathcal{M}'}\) such that, for every \(\mathcal{M'}\),
  \(\View_{\mathcal{M'}}\) is (perfectly, statistically, computationally) approximable by a 
  Turing machine \(\mathcal{M}''\) over the language 
  \(L' = \Set{\Tuple{x, h} \mid x \in L \wedge w \in \Set{0, 1}^*}\), where the string
  \(h\) represents the initial content of \(\Tapework'\).
\end{definition}

Naturally, a ZKIP which is also a proof system is a zero-knowledge proof system (ZKPS); similarly, 
if it is an interactive argument of knowledge system then it is a zero-knowledge interactive 
argument of knowledge system (ZK-IARK).
From now on, by zero-knowledge we mean computational zero-knowledge, as assuming a polynomial-time 
bounded adversary is an acceptable restriction in the real world.
The initial string \(h\) of a ZKIP can be interpreted as the \emph{history} of previous 
interactions with the prover, or some eavesdropped information from the interactions that the 
prover had with other verifiers.

A proof system being zero-knowledge basically means that, even for curious or malicious verifiers,
and even with additional knowledge on the behaviour of the prover, what can be computed is nothing 
more than what could have been computed in polynomial time, hence within the imposed computational 
power limits, without communicating with the prover.
While it is obvious that every problem solvable in probabilistic polynomial time 
(\textsc{PP}) has a zero-knowledge proof system (the prover does nothing and the verifier computes 
the solution by himself), it was proven that also all problems in \textsc{NP} have a 
ZKPS~\cite{GoldreichMW1991}. 
By assuming the existance of secure probabilistic encryption, it was finally shown that also 
all Arthur-Merlin games, and hence all problems in \textsc{IP}, have a ZKPS~\cite{BenorGGHKMR1990}.

\subsection{Non interactive Zero-Knowledge}\label{subsec:nizk}
In many scenarios, especially ones involving multiple parties, interaction can be a problem as
the communication cost of bidirectional \(n\)-to-\(n\) grows quadratically.
Such cases are in fact of great interest for zero-knowledge systems: multiple parties can be 
both provers and/or verifiers, and their number might be huge.

For this reason, researches explored the possibility of having zero knowledge \emph{non-interactive} 
proof systems (ZK-NPS) or argument of knowledge systems (ZK-NARK).
Unfortunately, only the languages in \textsc{BPP}, that is languages decidable in 
probabilistic polynomial time with a bounded error allow for zero-knowledge non-interactive 
proofs~\cite{Oren1987,GoldreichK1996}. Such languages are of course trivial, as the verifier has 
enough power to do all the computation by itself without the need of the prover.

However, by introducing an initial \emph{preprocessing} phase, it is possible to regain the lost 
power~\cite{SantisMP1990}, and the most prominent technique to achieve non-interaction is the 
\emph{Common Reference String} (CRS) model~\cite{BlumFM1988} (sometimes also called 
\emph{common random string} model).
The main idea of the CRS model is that, before engaging in the protocol, the prover and 
the verifier have both obtained access to a shared string of random bits. 
In the simplest case, the string is generated by a \emph{trusted third party}, although in 
practice this is oftentimes not a viable solution as the whole point of zero-knowledge is having 
to deal with untrusted parties. 
To circumnent this problem, it is possible to generate the CRS by a \emph{majority vote}
between \(n\) authorities, which can be untrusted if picked singularly, but are assumed to be 
honest in their majority~\cite{GrothO2006}.
In fact, it was shown that it is possible, without losing zero-knowledge, to re-use multiple times 
a single CRS both by a single~\cite{BlumSMP1991} or multiple~\cite{FeigeLS1990} provers, although 
only for arguments of knowledge and not for proofs.

The first zero-knowledge systems, both interactive and non-interactive, were tailor-made for 
specific problems, such as the quadratic residuosity problem \textsc{qr}~\cite{GoldwasserMR1989}, 
the hamiltonian path problem \textsc{hampath}~\cite{LapidotS1991}, or the \(3\)-\textsc{sat} 
problem~\cite{BlumSMP1991}.
Although for any \textsc{NP-complete} problem \textsc{prob} there is a polynomial-time 
algorithm~\cite{Karp1972} that converts every instance of \textsc{prob} to an instance 
of, say, \(3\)-\textsc{sat}, such reductions are often not trivial to devise and very expensive 
to apply.
For this reason, researchers started devising constructions to prove arbitrary NP statements 
embedded in the form of boolean circuits~\cite{Damgard1993}, which can neatly represent the 
computation of a Turing machine over any \textsc{NP-complete} problem~\cite{Cook1971}, and 
therefore remove the need to go through polynomial-time reductions.

The first of such systems~\cite{Damgard1993} required a CRS of size cubic in the length of the 
statement to be proven, although XOR and NOT gates didn't need to consume any bits from the CRS\@.
In the following years, many improvements were proposed, reducing the complexity of the 
constructions from cubic to subquadratic~\cite{BoyarBP1995} and eventually 
linear~\cite{CramerD1997}.

