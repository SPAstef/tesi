\chapter{Conclusions and Future Work}\label{chap:conclusions}
In this work we have studied the history of zero-knowledge, succinct and non-interactive 
argument of knowledge protocols, a highly interdisciplinary field of study which involves advanced
concepts of mathematic, computer science and cryptography.
We have seen how such systems have seen an incredible improvement over the last ten years, to the 
point of finally becoming useful in many real-world scenarios, from private verifiable cloud 
computing to anonymous blockchain commitments.

We have seen how the best ZK-SNARK systems known today work over arithmetic models of 
computation, much different from the boolean ones algorithms are usually designed for.
Since the most prominent application of such systems is in cryptography, and particularly in 
Merkle tree commitment verification, we revised the very recent history and the state of the art of 
arithmetization-oriented cryptographic primitives.

By distilling the core ideas in this field and introducing new ones, we then formalized them in the 
so-called Generalized Triangular Dynamical System framework, which we used to extract a new family 
of cryptographic functions: \Arion{}.
We finally compared \Arion{} with the competitor designs, showing that we beat the state of the 
art when providing the same level of security guarantees, and by matching it when providing 
additional guarantees.

This work has been a very exciting journey, which is however just a glimpse of what could lie ahead.
New proof systems like the circuit-depth oriented transparent ZK-STARKs, or the Lagrange-bases 
oriented \(\Plonk \) offer new challenges, as they divert from the standard metrics used to measure 
the complexity of a good ZK-friendly function. 

Another interesting direction of research is the application of ZK-SNARK systems and the 
related cryptographic functions in a wide range of hardware devices: from high performance 
computing (HPC) machines (e.g.\ verifiable delegated computation), to personal computers 
(e.g.\ anonymous transactions) to embedded internet-of-things (IoT) chips 
(e.g.\ private household monitoring). 
