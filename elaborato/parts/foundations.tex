\part{Foundations}\label{part:foundation}
Zero Knowledge Proof (ZKP) systems are a relatively recent research topic: while the idea in itself,
like many other beautiful ideas, is simple and elegant, its formalization, and even more so its 
realization, is all but trivial.
A first rigorous description of what it means for a proof system to be \emph{Zero Knowledge} was
given by S.\ Goldwasser, S.\ Micali and C.\ Rackoff in 1985~\cite{GoldwasserMR1989} (the work was 
later updated in 1989).

To fully understand the properties of ZKP system, one needs to have an understanding of both 
fundamental and more advanced notions from the fields of group theory, computational theory and 
cryptographical theory. 
This is even more necessary for ZK-SNARK systems and ZK-friendly hash functions.
For this reason, in this first part of the work we will (hopefully) give an exhaustive description
of the tools required to have a better grasp of the results that will be presented in the 
second part.
\chapter{Mathematical Background}

