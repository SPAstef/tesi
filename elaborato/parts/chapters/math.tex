\chapter{Mathematical Background}
In this chapter we will introduce all the mathematical concepts behind ZKP and ZK-friendly 
functions.
While we decided, for completeness, to include even some of the more fundamental notions, we still
expect the reader to have at least a rough idea of these concepts.
\Cref{sec:prime_fields} will introduce prime fields, cyclic groups other related notions.  

\section{Finite algebra}\label{sec:prime_fields}
In algebra, a \emph{tuple} consisting of one or more \emph{sets} together with one or more 
\emph{operations} over the sets is called an \emph{algebraic structure}.
Such structures can be organized according to a quite wide taxonomy, depending on whether 
they satisfy certain properties or not. 
We will denote sets with capital letters (e.g. \(S\)), a generic operation with a circled dot 
\(\odot \) and algebraic structures with blackboard bold letters (e.g. \(\mathbb{A}\)).
We will also denote elements of a set with lowercase letters \(a, b, c, \dots \) and variables 
over a set with lowercase letters \(x, y, z, \dots \).
Finally, we will often use the term algebra to mean algebraic structure, whenever we belive the 
meaning to be clear from the context.
\begin{remark}
  Some letters will be used reserved to denote some common algebraic structures. 
  In particular, \(\mathbb{B}\) will denote the boolean algebra, while \(\mathbb{N}, \mathbb{Z}, 
  \mathbb{Q}, \mathbb{R}\) and \(\mathbb{C}\) will denote, respectively, the 
  natural, integer, rational, real and complex numbers.
\end{remark}

We will denote the \emph{cardinality} of set \(S\) with \(\abs{S}\), and use the same notation for 
the \emph{arity} of an operation: for example, if \(\odot \) is a binary operation, like integer 
addition\footnote{if considered as a relation, addition would actually be ternary.}, then 
\(\abs{\odot} = 2\).
As a common abuse of notation in literature, when an algebraic structure \(\mathbb{A}\) has exactly 
one \emph{underlying set} \(A\) we will identify the two, e.g.\ by writing \(x \in \mathbb{A}\) to 
mean \(x \in A\).

For computer science applications, algebras with a finite number of elements play a central role.
\begin{definition}[Finite algebra]
  Given an algebraic structure \(\mathbb{A}\), the \emph{order} \(\abs{\mathbb{A}}\) denotes the 
  number of elements inside (the underlying set of) \(\mathbb{A}\).
  If \(\abs{\mathbb{A}} \in \mathbb{N}\), then \(\mathbb{A}\) is a finite algebra.
\end{definition}

Elements of differents algebraic structures can be associated through \emph{morphisms}.
\begin{definition}[Homomorphism]
  Given two algebras \(\mathbb{A} = \Tuple{A, \odot_{1}, \dots, \odot_{n}}\), 
  \(\mathbb{A}' = \Tuple{A', \odot'_{1}, \dots, \odot'_{n}}\) such that 
  \(\forall i \le n\colon \abs{\odot_i} = \abs{\odot'_i} = a_i\), an homomorphism is a function 
  \(h\colon A \to A'\) such that:
  \begin{align*}
    & \forall i \le n, \forall x_1,\dots, x_{a_{i}} \in A\colon 
    h(\odot_{i}(x_{1}, \dots, x_{a_{i}})) = \odot'_{i}(h(x_{1}), \dots, h(x_{a_{i}})) && 
    \textnormal{(linearity)}
  \end{align*}
  We say that \(\mathbb{A}\) is homomorphic to \(\mathbb{A}'\) through \(h\).
\end{definition}

\begin{definition}[Isomorphism]
  An isomorphism is an homomorphism whose inverse is also an homomorphism.
  Given two algebras \(\mathbb{A}\) and \(\mathbb{A}'\), if there exists some isomorphism 
  \(h\colon \mathbb{A} \to \mathbb{A}'\) we write \(\mathbb{A} \cong \mathbb{A}'\).
\end{definition}

\begin{definition}[Endomorphism, Automorphism]
  An endomorphism is a homomorphism from an algebraic structure \(\mathbb{A}\) to itself.
  An automorphism is an endomorphism which is also an isomorphism.
\end{definition}

\subsection{Groups}
We will now introduce some important classes of algebraic structures equipped with one fundamental 
operation. 
\begin{definition}[Monoid]
  A monoid is a pair \(\mathbb{M} = \Tuple{M, \odot} \), where \(M\) is the 
  underlying set and \(\odot\colon M \times M \to M\) is the \emph{composition} 
  operation, such that the following properties are satisfied: 
  \begin{align*}
    & \forall x,y \in M\colon x \odot \left(y \odot z\right) = \left(x \odot y\right) \odot z
      && \textnormal{(\emph{associativity})} \\
    & \exists \algid \in M\colon \forall x \in M\colon x \odot \algid = x
      && \textnormal{(\emph{identity element})}
  \end{align*}
  \(\mathbb{M}\) is a \emph{commutative (or abelian) monoid}, if it also holds that:
  \begin{align*}
    & \forall x,y \in M\colon x \odot y = y \odot x && (\emph{commutativity})
  \end{align*}
  Finally:
  \begin{equation}\label{eq:exponentiation}    
    \forall x \in \mathbb{M}, \forall k \in \mathbb{N}\colon x^{k} = 
    \begin{cases}
      \algid & k = 0 \\
      x^{k-1} \odot x & k > 0
    \end{cases}
  \end{equation}
\end{definition}

\begin{definition}[Cyclic Monoid]
  A cyclic monoid is a monoid \(\mathbb{M} = \Tuple{M, \odot}\) such that:
  \[\exists g \in M\colon \mathbb{M} = \gengroup{g} = 
  \Tuple{\Set{g^k}_{k \in \mathbb{N}}, \odot} \]  
\end{definition}

\begin{definition}[Group]
  A group is a monoid \(\mathbb{G} = \Tuple{G, \odot} \), such that: 
  \begin{align*}    
    & \forall x \in G\colon \exists x^{-1} \in G\colon x \odot x^{-1} = \algid
    && \textnormal{(\emph{inverse element})}
  \end{align*}
  With the notion of inverse element, we can rewrite and extend \Cref{eq:exponentiation} 
  for groups as follows:
  \[
    \forall x \in \mathbb{G},\forall k \in \mathbb{Z}\colon x^k =
    \begin{cases}
      x^{k-1} \odot x & k \ge 0 \\
      x^{k+1} \odot x^{-1} & k < 0
    \end{cases}
  \]
  If \(\mathbb{G}\) is also a commutative (resp.\ cyclic) monoid, then it is a 
  commutative (resp.\ cyclic) group.
  A group \(\mathbb{G}' = \Tuple{G', \odot'}\) is a \emph{subgroup} of \(\mathbb{G}\) if 
  \(G' \subseteq G\) and \(\odot' \subseteq \odot \).
\end{definition}

The identity element \(\algid \) of a monoid is typically denoted with \(1\) in 
numeric algebras, when the composition operation resembles standard multiplication, or by \(0\) 
when the composition operation resembles standard addition. 
We use the notation \(\algid_{\mathbb{A}}\) (or \(1_{\mathbb{A}}\), 
\(0_{\mathbb{A}}\)) to specify the algebra over which we intend to pick the identity element, 
dropping the subscript when \(\mathbb{A}\) is clear from the context.

It is important to stress that one should be careful not to confuse the symbol and the name of an 
operation or of a special element with its semantics: the syntax to denote the inverse element 
\(x^{-1}\) of a group is reminiscent of standard multiplication inversion, but this is not the 
case in general.
When the composition operation resembles standard addition, the inverse is more likely denoted 
with \(-x\).

\begin{example}
  The algebra \(\mathbb{A} = \mathbb{Z} \setminus \Set{\times}\) (i.e.\ integer numbers 
  without multiplication) is an abelian group: 
  addition is associative and commutative, the identity element is 
  \(\algid_{\mathbb{A}} = 0\), and every number \(x\) has an inverse \(x^{-1} = -x\) 
  (e.g. \({42}^{-1} = -42\)). 
\end{example}

\begin{example}\label{ex:endo_group}
  Given a commutative group \(\mathbb{G} = \Tuple{G, \odot}\), consider the algebra 
  \(\Endset{\mathbb{G}}_{+} = \Tuple{H, +}\), where \(H\) is the set endomorphisms over 
  \(\mathbb{G}\) and \(+\colon H \times H \to H\) is defined as: 
  \(\left(h_1 + h_2\right)\left(x\right) \equiv \left(h_1 + h_2\right)\left(x\right)\).

  \(\Endset{\mathbb{G}}_{+}\) is a commutative group: \(+\) is associative and 
  commutative, the identity element is \(\algid_{\Endset{\mathbb{G}}_{+}} = z\), where
  \(z\) is the zero endomorphism (i.e. \(\forall x \in G\colon z\left(x\right) = 
  \algid_{\mathbb{G}}\)); finally, every homomorphism \(h \in H\) has an inverse 
  \(h^{-1} = -h\) such that \(\forall x \in G\colon \left(-h\right)\left(x\right) = 
  {h\left(x\right)}^{-1}\) (in this example, using the \(h^{-1}\) notation causes confusion with 
  the inverse function!).
\end{example}

\begin{example}\label{ex:endo_monoid}
  Consider now the algebra \(\Endset{\mathbb{G}}_{\circ} = \Tuple{H, \circ}\) 
  where \(\mathbb{G}\) and \(H\) are defined as in \Cref{ex:endo_group}, and 
  \(\circ\colon H \times H \to H\) is function composition: 
  \(\left(h_1 \circ h_2\right)\left(x\right) \equiv h_1\left(h_2\left(x\right)\right)\).
  
  \(\Endset{\mathbb{G}}_{\circ}\) is a monoid: function composition is associative, and the 
  identity element is \(\algid_{\Endset{\mathbb{G}}_{\circ}} = \fooid \), where \(\fooid \) is the
  identity endomorphism (i.e. \(\forall x \in G\colon \fooid\left(x\right) = x\)). 
\end{example}

\subsection{Fields}
Many algebraic structures rely on two fundamental operations, called \emph{addition} and 
\emph{multiplication}: two important types of such structures are \emph{rings} and \emph{fields}.
\begin{definition}[Ring]
  A ring is a triple \(\mathbb{O} = \Tuple{O, \oplus, \otimes}\) where \(O\) is the 
  underlying set, \(\oplus\colon O \times O \to O\) is the \emph{addition} operation and 
  \(\otimes\colon O \times O \to O\) is the \emph{multiplication} operation, sukh that the 
  following properties are satisfied:
  \begin{align*}
    & \mathbb{O}_{\oplus} = \mathbb{O} \setminus \Set{\otimes}
      \textnormal{ is an abelian group} \\
    & \mathbb{O}_{\otimes} = \mathbb{O} \setminus \Set{\oplus} 
      \textnormal{ is a monoid} \\
    & \forall x,y,z \in O\colon x \otimes \left(y \oplus z\right) = 
      \left(x \otimes y\right) \oplus \left(x \otimes z\right) && (\emph{left distributivity})\\
    & \forall x,y,z \in O\colon \left(y \oplus z\right) \otimes x = 
      \left(y \otimes x\right) \oplus \left(z \otimes x\right) && (\emph{right distributivity})
  \end{align*}
  If \(\mathbb{O}_{\otimes}\) is a commutative monoid, then \(\mathbb{O}\) is a 
  \emph{commutative (abelian) ring}.
  %(in this case, either one of the two distributivity properties becomes redundant).
\end{definition}

Given a ring \(\mathbb{O}\) and an element \(x \in \mathbb{O}\), 
we denote its inverse w.r.t.\ addition as \(-x\), while maintaining the notation \(x^{-1}\) for 
the multiplicative inverse.
Furthermore, the identity element w.r.t.\ addition, denoted \(\algid_{\oplus}\), will also be 
denoted as \(0\), while the identity element w.r.t.\ multiplication, denoted 
\(\algid_{\otimes}\), will maintain its alternative notation as \(1\).

\begin{definition}[Field]
  A field is a commutative ring \(\mathbb{F} = \Tuple{F, \oplus, \otimes}\) such that
  \(0 \neq 1\) and \(\mathbb{F}_{\otimes} \setminus \Set{0}\) is a commutative group.
\end{definition}

Fields are one of the most important and studied algebraic structures: the algebra of real numbers 
\(\mathbb{R}\) is a field, as is the algebra of complex numbers \(\mathbb{C}\).
\begin{remark}
  Given the set of integers \(Z_q = \Set{0, \dots, q-1}\), we denote with \(\oplus_q\) 
  integer sum modulo \(q\), and with \(\otimes_q\) integer multiplication modulo \(q\).
  The algebra \(\mathbb{Z}_q = \Tuple{Z_q, \oplus_q, \otimes_q}\) is a finite ring for any 
  \(q \in \mathbb{N}\), and it is a finite field if and only if \(q\) is prime.
\end{remark}

\begin{example}
  Boolean circuits with \textsc{xor} and \textsc{and} gates behave like elements of the boolean 
  field \(\mathbb{B} = \Tuple{\Set{\bot, \top}, \bitxor, \bitand} \).
  It is easy to show that \(\mathbb{B} \cong \mathbb{Z}_2\).
  Similarly, \(k\)-bit unsigned integers sum and multiplication work as in \(\mathbb{Z}_{2^k}\).
\end{example}

\begin{example}
  Given an abelian group \(\mathbb{G}\), the algebra \(\mathbb{H}_{\mathbb{G}} = 
  \Endset{\mathbb{G}} = \Endset{\mathbb{G}}_{+} \cup \Endset{\mathbb{G}}_{\circ}\) 
  is the \emph{endomorphism ring} of \(\mathbb{G}\): \(\Endset{\mathbb{G}}_{+}\) is an abelian 
  group, \(\Endset{\mathbb{G}}_{\circ}\) is a monoid 
  (cfr.\ \Cref{ex:endo_group,ex:endo_monoid}), and it is easy to show that \(\circ \)
  distributes over \(+\) both on the left and the right.
\end{example}

\subsection{Vector spaces}
It is possible to naturally extend the underlying operations of a field over tuples of field 
elements to obtain a \emph{vector space}.
\begin{definition}[Vector space]
  A vector space is a quadruple \(\mathbb{V} = \Tuple{V, \mathbb{F}, +, \odot}\) where 
  \(V\) is the underlying vector set, \(\mathbb{F}\) is the underlying field, 
  \(+\colon V \times V \to V\) is the \emph{vector addition} operation and 
  \(\odot\colon \mathbb{F} \times V \to V\) is the \emph{scalar multiplication} operation, such 
  that \(\mathbb{V}_{+} = \Tuple{V, +}\) is a commutative group, and \(\odot \) is an 
  homomorphism between \(\mathbb{F}\) and \(\Endset{\mathbb{V}_{+}}\).
\end{definition} 
