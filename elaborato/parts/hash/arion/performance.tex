\section{Performance evaluation of \Arion}\label{sec:performance}
As we saw in \Cref{sec:gtds}, the concept of CCZ-equivalence introduced in the \Anemoi{} 
proposal~\cite{BouvierBCPSVW2022} plays an important role to build high degree permutations 
verifiable with low degree constraint systems.
In fact, it is possible to extend the concept of CCZ-equivalence by allowing any permutation.
\begin{definition}[\(\pi \)-equivalence]
  Given a vector space \(\mathbb{V}\) and two functions \(f,g\colon \mathbb{V} \to \mathbb{V}\), 
  \(f\) and \(g\) are \emph{\(\pi \)-equivalent} if there is a permutation 
  \(\pi\colon \mathbb{V}^2 \to \mathbb{V}^2\) such that \(\Gamma_{f} = \call{\pi}{\Gamma_{g}}\).
\end{definition}

Any function \(\pi\colon \mathbb{V}^2 \to \mathbb{V}^2\) can be decomposed into its 
\emph{projections} \(\pi_{\hat{x}}, \pi_{\hat{y}}\colon \mathbb{V}^2 \to \mathbb{V}\) such that:
\[
  \forall \bm{x}, \bm{y} \in \mathbb{V}\colon \call{\pi}{\bm{x}, \bm{y}} = 
    \Tuple{\call{\pi_{\hat{x}}}{\bm{x}, \bm{y}}, \call{\pi_{\hat{y}}}{\bm{x}, \bm{y}}}
\]

Thus, for any function \(f\colon \mathbb{V} \to \mathbb{V}\), we have that 
\(\call{\pi}{\bm{x}, \call{f}{\bm{x}}} = 
  \Tuple{\call{\pi_{\hat{x}}}{\bm{x}, \call{f}{\bm{x}}}, 
  \call{\pi_{\hat{y}}}{\bm{x}, \call{f}{\bm{y}}}}\).
Now, let \(\call{f_{\hat{x}}}{\bm{x}} = \call{\pi_{\hat{x}}}{\bm{x}, \call{f}{\bm{x}}}\) and 
\(\call{f_{\hat{y}}}{\bm{x}} = \call{\pi_{\hat{y}}}{\bm{x}, \call{f}{\bm{x}}}\), then 
\(\call{\pi}{\Gamma_f} = \Gamma_{f'}\) if and only if \(\pi_{\hat{x}}\) is a permutation, and 
in particular \(f' = f_{\hat{y}} \compose f_{\hat{x}}^{-1}\) is \(\pi \)-equivalent to \(f\).

Just like CCZ-equivalence, \(\pi \)-equivalence is also an equivalence relation, denoted 
\(\pieq \), which allows us to identify equiivalence classes of functions over \(\mathbb{V}\).
\begin{lemma}[\(\pi \)-equivalence of permutations]\label{lem:pi_equiv}
  All permutations over a vector space \(\mathbb{V}\) are \(\pi \)-equivalent.
\end{lemma}
\begin{proof}
  We just have to prove that, for every permutation \(f\colon \mathbb{V} \to \mathbb{V}\), it 
  is the case that \(f \pieq \fooid \), and the result will follow.
  Clearly, the function \(\call{\pi}{\bm{x}, \bm{y}}\colon \mathbb{V}^2 \to \mathbb{V}^2 = 
    \Tuple{\bm{x}, \call{f^{-1}}{\bm{y}}}\) is a permutation, and 
  \(\call{\pi}{\Gamma_f} = \Gamma_{\fooid}\).
\end{proof}

\begin{definition}[Alternative GTDS of \Arion]
  Given the same parameters as in \Cref{def:gtds}, the \emph{alternative GTDS of \Arion} is the
  function \(\tilde{F}_{GTDS}\colon \mathbb{F}_p^t \to \mathbb{F}_p^t\) such that:
  \[
    \call{\tilde{F}_{GTDS}}{\bm{x}}_i = \bm{y}_i =
    \begin{cases}
      \bm{x}_i^{d_1}\call{g_i}{\tilde{\sigma}_{i+1,t}} + 
        \call{h_i}{\tilde{\sigma}_{i+1,t}} & 1 \le i < t \\
      \bm{x}_i^{d_2} & i = t
    \end{cases}
  \]
  where \(\tilde{\tau}_{i, k} = \bm{x}_k + \bm{x}_{k}^{e} + \sum_{j=i}^{k-1}{\bm{x}_j + \bm{y}_j}\).

\end{definition}

\begin{proposition}[\(\pi \)-equivalence of GTDS]
  \(F_{GTDS} \pieq \tilde{F}_{GTDS}\).
\end{proposition}
\begin{proof}
  \(F_{GTDS}\) and \(\tilde{F}_{GTDS}\) are both permutations over the vector space \(\mathbb{F}_p^t\),
  therefore by \Cref{lem:pi_equiv} the claim follows.
\end{proof}

As a corollary, we have that verifying the constraint system of \(F_{GTDS}\) is equivalent to 
verifying the constraint system of \(\tilde{F}_{GTDS}\).
Computing the number of multiplicative constraints for \Arionhash{} is quite straightforward:
\begin{lemma}[R1CS constraints for \Arionhash]
  Given the \Arionhash{} function over a prime field \(\mathbb{F}_p\) with branch size 
  \(t\) and rate \(r\), let \(\call{\minmul}{x}\colon \mathbb{F}_p \to \mathbb{N}\) be the minimum
  number of field multiplications required to compute \(y^x\) for any \(y \in \mathbb{F}\).
  The number of R1CS constraints required by \Arionhash{} is:
  \[
    N_{\Arionhash} = 
      r\Parens*{\Parens*{n - 1}\Parens*{\call{\minmul}{d_1} + 2} + \call{\minmul}{d_2}}
  \]
\end{lemma}
\begin{proof}
  Consider the alternative GTDS \(\tilde{F}_{GTDS}\): we need \(\call{\minmul}{d_2}\) 
  multiplicative constraints in the last branch.
  In the remaining \(n - 1\) branches, we need \(\call{\minmul}{d_1}\) multiplications to 
  compute \(\bm{x}_i^{d_1}\), one multiplication for computing \(g_i\), one for computing \(h_i\), 
  and one to multiply \(g_i\) with \(\bm{x}_i^{d_1}\).
\end{proof}

For reference, the number of R1CS constraints required by \Poseidon{} and \Griffin{} 
over their respective parameters (see \Cref{def:poseidon} and \Cref{def:griffin}) are given by:
\begin{align*}
  & N_{\Poseidon} = \call{\minmul}{d}\Parens*{2tr_f + r_P} \\
  & N_{\Griffin} = 2r\Parens{\call{\minmul}{d} + t - 2}
\end{align*}

\begin{table}
  \centering
  \caption{R1CS constraint comparison over \(256\)-bit prime fields and \(128\) bits of security 
  with \(d_2 \in \Set{121, 123, 125, 161, 257}\).}\label{tab:arion_compare_muls}
  \begin{tabular}{  c c c c c c c  }
      \toprule
      \phantom{ }\(d_1\)\phantom{ } & \phantom{ }\(t\)\phantom{ } & \phantom{ }\Arionhash{}\phantom{ } & \phantom{ }\Aarionhash{}\phantom{ } & \phantom{ }\Griffin{}\phantom{ } & \phantom{ }\Anemoi{}\phantom{ } & \phantom{ }\Poseidon{}\phantom{ }             \\
      \midrule
      \multicolumn{2}{  c | }{} & \multicolumn{5}{ c }{Rounds} \\
      \midrule
      \(3\) & \(3\) & \(6\) & \(5\) & \(12\) &      & \phantom{ }\(r_f = 4,\ r_P = 84\)\phantom{ } \\
      \(5\) & \(3\) & \(6\) & \(4\) & \(12\) &      & \phantom{ }\(r_f = 4,\ r_P = 56\)\phantom{ } \\

      \(3\) & \(4\) & \(6\) & \(4\) & \(11\) & \(12\) & \phantom{ }\(r_f = 4,\ r_P = 84\)\phantom{ } \\
      \(5\) & \(4\) & \(5\) & \(4\) & \(11\) & \(12\) & \phantom{ }\(r_f = 4,\ r_P = 56\)\phantom{ } \\

      \(3\) & \(5\) & \(5\) & \(4\) &      &      & \phantom{ }\(r_f = 4,\ r_P = 84\)\phantom{ } \\
      \(5\) & \(5\) & \(5\) & \(4\) &      &      & \phantom{ }\(r_f = 4,\ r_P = 56\)\phantom{ } \\

      \(3\) & \(6\) & \(5\) & \(4\) &      & \(10\) & \phantom{ }\(r_f = 4,\ r_P = 84\)\phantom{ } \\
      \(5\) & \(6\) & \(5\) & \(4\) &      & \(10\) & \phantom{ }\(r_f = 4,\ r_P = 84\)\phantom{ } \\

      \(3\) & \(8\) & \(4\) & \(4\) & \(9\)  & \(10\) & \phantom{ }\(r_f = 4,\ r_P = 84\)\phantom{ } \\
      \(5\) & \(8\) & \(4\) & \(4\) & \(9\)  & \(10\) & \phantom{ }\(r_f = 4,\ r_P = 56\)\phantom{ } \\

      \midrule

      \multicolumn{2}{  c | }{} & \multicolumn{5}{ c  }{R1CS Constraints} \\
      \midrule

      \(3\) & \(3\) & \(102\) & \(85\)  & \(72\)  &       & \(216\) \\
      \(5\) & \(3\) & \(114\) & \(76\)  & \(96\)  &       & \(240\) \\

      \(3\) & \(4\) & \(126\) & \(84\)  & \(88\)  & \(96\)  & \(232\) \\
      \(5\) & \(4\) & \(120\) & \(96\)  & \(110\) & \(120\) & \(264\) \\

      \(3\) & \(5\) & \(120\) & \(100\) &       &       & \(248\) \\
      \(5\) & \(5\) & \(125\) & \(116\) &       &       & \(288\) \\

      \(3\) & \(6\) & \(145\) & \(116\) &       & \(120\) & \(264\) \\
      \(5\) & \(6\) & \(170\) & \(136\) &       & \(150\) & \(312\) \\

      \(3\) & \(8\) & \(148\) & \(148\) & \(144\) & \(160\) & \(296\) \\
      \(5\) & \(8\) & \(176\) & \(176\) & \(162\) & \(200\) & \(360\) \\

      \bottomrule
  \end{tabular}
\end{table}

\Cref{tab:arion_compare_muls} shows the number of constraints required for specific instantiations 
of \Arion, \Aarion, \Anemoi, \Poseidon{} and \Griffin{} over \(\approx 256\)-bit prime fields for a 
target \(128\) bits of security.

\subsection{Implementations and Experiments}
In order to experiment with the cryptographic primitives and the constructions discussed 
throughrout this thesis, we wrote the C\texttt{++} 
\texttt{zkp-hash}\footnote{\url{https://github.com/SPAstef/zkp_hash}} library, by building on
\texttt{libsnark}.
In particualar, our library is meant to improve and extend \texttt{libsnark}, by providing: 
\begin{itemize}
  \item A native and ZK-SNARK implementation of SHA-512.
  \item Native and ZK-SNARK implementation of \Mimchash-\(256\) and \Mimchash-\(512\).
  \item Templates for native and Zk-SNARK implementations of \Poseidon, \Griffin{} 
        and \Arionhash.
  \item A template for native and ZK-SNARK fixed-size Merkle tree authentication paths with 
        arbitrary arity and node size, for arbitrary hash functions.
  \item A template for native and ZK-SNARK fixed-size ABR authenticatin paths with arbitrary node 
        size over arbitrary \(2n\)/\(n\) hash functions.
  \item Additional utilities to simplify working with \texttt{libsnark}.
\end{itemize}

We would like to discuss a bit more in detail our Merkle tree implementation, both to highlight some 
important security considerations that might not be obvious at first, as well as an optimization 
for its construction, which we are not aware of being applied in any other implementation.

Let's consider, for the sake of simplicity, a binary Merkle tree over a field \(\mathbb{F}_p\) 
where each node contains only one field element. 
The circuit of the tree takes in input the value \(x\) contained in the node that the prover claims 
to hold, together with the values \(y_1, \dots, y_{h-1}\) along the alleged authentication path, 
and by using the sub-circuit (gadget) of the underlying compression function \(C\), it constrains 
the middle values \(m_1, \dots, m_h\) to be:
\begin{equation}\label{eq:bad_constraint}
  \Parens*{m_1 = x} \land
  \forall i < h\colon m_{i+1} =
  \begin{cases}
    \call{C}{m_{i}, y_{i}} & \textnormal{\(m_{i}\) is a left leaf} \\
    \call{C}{y_{i}, m_{i}} & \textnormal{\(m_{i}\) is a right leaf}
  \end{cases}
\end{equation}
where \(m_{h}\) will be the root value. 
However, there is a big issue in this naive construction: the left/right choice is made 
transparently and the verifier, by simply inspecting the circuit layout, would be able to know 
the location of the prover's node and therefore its value!
So we must \emph{embed} the choice in the circuit (in the \texttt{libsnark} implementation this is 
left to the user, which we believe to be unnecessarily complicated and error-prone).

In our construction, each node is labeled with a left-to-right bottom-to-top \(0\)-based index 
\(n\): the leftmost leaf will have index \(n = 0\) and the root will have index \(n = 2^{h} - 2\), 
where \(h\) is the height of the tree. 
The index will be an additional input variable to the circuit. 
Note that the base-\(2\) representation of the index, from the LSB to the MSB, shows the path from 
the leaf to the root: if the \(i\)th bit of \(n\) is \(0\), it means that we are on a left node, if 
it is \(1\) it means that we are on a right node.
By introducing a set of \emph{selection variables} \(s_1, \dots, s_{h-1}\), a set of 
\emph{left variables} \(l_1, \dots, l_{h-1}\) and a set of \emph{right variables} 
\(r_0, \dots, r_{h-1}\), we can then replace \Cref{eq:bad_constraint} with the following:
\begin{align*}
  & n = \sum_{i=1}^{h}{s_{i}2^{i-1}} \\
  & m_1 = x \\
  & \forall i < h\colon
  \begin{cases}  
    l_i = \Parens*{1 - s_{i}}m_{i} + s_{i}y_{i} \\ 
    r_i = s_{i}m_{i} + \Parens*{1 - s_{i}}y_{i} \\
    m_{i+1} = \call{C}{l_i, m_i}
  \end{cases}
\end{align*}
Where the \(i\)th selection variable will have to be set to the value of the \(i\)th bit of the 
index. If any of the selection variables is set to an incorrect value, the verification will 
fail, hence we have been able to hide the path from the verifier without compromising the 
correctness of the construction.

\subsubsection*{From binary to \(t\)-ary}
To work with \(t\)-ary trees of height \(h\) using the \emph{minimal amount of constraints}, we 
need to make some changes to the previous reasoning.
The authentication path will now consist of the node \(x\) and a set of \(t-1\)-dimensional 
vectors \(\bm{y}_1, \dots, \bm{y}_{h-1}\), where \(\bm{y}_i\) will contain the sibling nodes at 
the \(i\)th level of the tree, and the compression function will be \(t\)-to-\(1\).
The base-\(t\) representation of the index, from LSD to MSD, will again show the path from the leaf 
to the root, where the \(i\)th digit will tell us which of the \(t\) children we are on.
We introduce a set of \(t\)-dimensional selection variables \(\bm{s}_1, \dots, \bm{s}_{h-1}\), a
a set of \(t\)-dimensional children variables \(\bm{c}_1, \dots, \bm{c}_{h-1}\), and introduce 
the following constraints:
\begin{align*}
  & n = \sum_{i=1}^{h}{t^{i-1}\sum_{j=1}^{t}{j\bm{s}_{i,j}2^{t-1}}} \\
  & \bm{c}_{1,1} = x \land \forall i \in \Iinterval{2}{t}\colon \bm{c}_{1,i} = \bm{y}_{i-1} \\
  & \forall i < h, \forall j \le t\colon 
  \begin{cases}
    \bm{s}_{i,j}\Parens*{m_{i+1} - \bm{y}_{i,j}} = \bm{c}_{i,j} - \bm{y}_{i,j} \\
    m_{i+1} = \call{C}{\bm{c}_i}
  \end{cases}
\end{align*}

To better understand the constraints, let's focus on a single level \(i\): the vector 
\(\bm{s}_i\) will contain all zeros except in the position dictated by the \(i\)th digit of 
\(n\) when represented in base \(t\). 
In an electronic jargon, we can say that the selector blocks all the signals coming from the 
previous layer but in one position, and the dangling wires will be powered by the sibling signals 
provided by the prover.
To make sure that the prover cannot cheat, the \(i\)th digit of \(n\) will have to match the 
value of the \(i\)th bit of \(\bm{s}_i\) weighted by its position \(j\).
Further extending the scheme to allow for single nodes to contain multiple field elements is 
quite straightforward.

\subsubsection*{\Arion{} vs.\  \Griffin{} vs.\  \Poseidon{}}
\begin{table}
  \centering
  \caption{Performance of various hash functions for generating a proof of membership in a Merkle 
      tree accumulator over BN254. Proving times are in milliseconds.}\label{tab:runtimes}
  \resizebox{480pt}{!}{
      \begin{tabular}{  c  c c c  c c c  c c c  c c c  }
          \toprule
          & \multicolumn{12}{ c  }{Time (ms)} \\
          \midrule

          \phantom{ }Height\phantom{ } & & \Arionhash{} &  &  &  \Aarionhash{}  & & & \Griffin{} & & & \Poseidon{} & \\
          \midrule
          \(d = 5\) & \phantom{ }\(n = 3\) & \(n = 4\) & \(n = 8\)\phantom{ } & \phantom{ }\(n = 3\) & \(n = 4\) & \(n = 8\)\phantom{ } & \phantom{ }\(n = 3\)  & \(n = 4\)  & \(n = 8\)\phantom{ } & \phantom{ }\(n = 3\)  & \(n = 4\)  & \(n = 8\)\phantom{ }  \\
          \midrule
          \(4\)   & \(101\)  & \(103\)  & \(142\)  & \(73\)  & \(87\)  & \(143\)  & \(88\)  & \(99\) & \(133\) & \(186\)  & \(212\)  & \(274\)  \\
          \(8\)   & \(211\) & \(216\) & \(294\)  & \(145\) & \(177\) & \(294\)  & \(181\) & \(209\) & \(270\) & \(386\)  & \(417\)  & \(566\)  \\
          \(16\)  & \(392\) & \(401\) & \(554\)  & \(278\) & \(334\) & \(553\)  & \(338\) & \(387\) & \(505\) & \(745\)  & \(805\)  & \(1095\) \\
          \(32\)  & \(730\) & \(751\) & \(1046\) & \(509\) & \(646\) & \(1047\) & \(622\) & \(727\) & \(980\) & \(1422\) & \(1550\) & \(2111\) \\
          \bottomrule
      \end{tabular}
  }
\end{table}

Using our library, we compared \Arion{} with \Griffin{} and \Poseidon{}.
The comparative result of our experiments are given in \Cref{tab:runtimes}. 
All experiments were run on a system with an Intel Core i7--11800H CPU and 32GB RAM on a Clear Linux 
instance, using the \texttt{g++-12} compiler with \texttt{-O3 -march=native} flags.
Our result shows that \Arionhash{} significantly outperforms \Poseidon{} showing \(2\)x efficiency 
improvement, and \Aarionhash{} is considerably faster than \Griffin{} for the most common
real-world parameter choices (i.e.\  \(n = 3\) in Merkle tree hashing mode).
