\section{\Arion: A new ZK-friendly permutation}\label{sec:gtds}
The constructions that we saw in \Cref{sec:sota} are prominent examples of different 
\emph{generations} of \emph{Arithmetization Oriented} (AO) designs.
For example, \Mimc{} is an example of a Gen-I design: its main purpose was mostly to demonstrate 
that it was indeed possible to to construct secure and efficient cryptographic primitives by 
stacking simple, low-degree round functions.
On the other hand, the \Hades{} framework, its derivative \Poseidon{} and other similar 
constructions like \Rescue{} are examples of Gen-II designs: by tweaking the SPN and 
Feistel constructions, their purpose was to massimely improve the efficiency over Gen-I designs.
Finally, \Griffin{} is an example of a Gen-III design: the underlying \Horst{} scheme is neither 
a ``pure'' Feistel nor SPN design and, by deviating from such standard constructions, its authors 
were able to improve the efficiency even further. 
Furthermore, in third generation designs it was shown that one does not necessarily need to use 
round functions with a low degree, as long as the resulting constraint system is not affected 
negatively\footnote{As it is often the case in research, the separation line is a bit blurry, 
as \Rescue, which we said to be a Gen-II design, already used inverse exponentiations}.

Other from \Griffin, there is another very recent Gen-III construction, called 
\Anemoi~\cite{BouvierBCPSVW2022} and based on the \Flystel{} design, which has some common points 
with the \Horst{} construction.
A very interesting fact used in \Flystel{} was the notion of CCZ-equivalence~\cite{CarletCZ1998},
a concept which generalizes the intuition behind the idea that there is ``no difference'' 
between, say, using \(x^{d}\) and \(x^{\frac{1}{d}}\).
\begin{definition}[Affine function]
  An \emph{affine function} over an \(n\)-dimensional vector space \(\mathbb{F}^n\) is a function
  \(\call{f}{\bm{x}}\colon \mathbb{F}^n \to \mathbb{F}^n = \bm{Mx} + q\), where 
  \(\bm{M} \in \mathbb{F}^{n \times n}\) and \(q \in \mathbb{F}\).
\end{definition}
\begin{definition}[Function graph]
  Given a set \(S\), the \emph{function graph} of a function \(f\colon S \to S\) is the pair 
  \(\Gamma = \Tuple{S, E}\) where \(E = \Set{\Tuple{x, \call{f}{x}} \mid x \in S}\).
\end{definition}
\begin{definition}[Induced permutation]
  Given a \emph{function graph} \(\Gamma = \Tuple{S, E}\) and a permutation \(P\colon S^2 \to S^2\), 
  the \emph{induced permutation} of \(\Gamma \) by \(P\) is the function graph 
  \(\call{P}{\Gamma} = \Tuple{S, E'}\) where \(E' = \Set{\call{P}{e} \mid e \in E}\).
\end{definition}
\begin{definition}[CCZ equivalence~\cite{BouvierBCPSVW2022,CarletCZ1998}]
  Given a vector space \(\mathbb{V}\) and two functions 
  \(f,g\colon \mathbb{V} \to \mathbb{V}\), \(f\) and \(g\) are \emph{CCZ-equivalent} if there 
  is an affine permutation \(L\colon \mathbb{V}^2 \to \mathbb{V}^2\) such 
  that \(\Gamma_{f} = \call{L}{\Gamma_{g}}\).
\end{definition}

Clearly, CCZ-equivalence is an equivalence relation over a vector space \(\mathbb{V}\), hence it 
induces a partitioning of \(\mathbb{V}^2\) into equivalence classes.
An interesting fact is that all CCZ-equivalent functions share the same linear and differential 
properties.
Even more importantly for our purposes is that CCZ-equivalent functions are indistinguishable under 
constraint verification: checking the constraint system of any member of the class verifies 
the validity of the computation of any other member.
However, one thing that CCZ-equivalent functions do not share in general is their degree, hence we 
can use the one with highest degree in the actual computation to provide the most strict security 
guarantees against algebraic attacks, while using the one with the lowest degree when building
the constraint system for the verification in the SNARK framework.

In an high-level, intuitive way we can say that CCZ-equivalence allows us to ``ignore the order'' 
in which the witnesses of a certain computation are obtained: reprising our usual example,
the verifier does not care that the prover first must know \(x\) in order to obtain \(y = x^{1/d}\), 
all it matters is that the prover knows both of them and that their relationship is correct.
In fact, even more generally, there is no way to know in which order someone actually got hold of 
the intermediate values of a computation, hence we might as well exploit this to our advantage in 
order to reduce the complexity of the protocol.

\subsection{The Generalized Triangular Dynamical System}
As we said, the design of third generation AO cryprographic primitives diverts from the plain 
SPN or Feistel constructions.
For this reason, we introduce the \emph{Generalized Triangular Dynamical System}~\cite{RoyS2022}, 
GTDS for short, an algebraic framework which generalizes many previous designs (such as Feistel, 
SPN, and \Horst) and enables us to provide a systematic security analysis 
of the constructions derived from it.
In particular:
\begin{itemize}
  \item The input is split in branches like in earlier designs.
  \item The round function offers the strength of all the incorporated designs.
  \item It is secure against classical attacks like differential cryptanalysis.
  \item It is secure against interpolation attacks already at the first round.
  \item Its linear layer mixes all the branches through a circulant matrix with no zero entries.  
  \item It uses inverse exponents, but decouples them from the direct exponent. 
\end{itemize}

\begin{definition}[GTDS of \Arion~\cite{RoyS2022}]\label{def:gtds}
  Given a prime field \(\mathbb{F}_p\), a number of branches \(t \in \mathbb{N}\), the smallest 
  integer \(d_1\) such that \(\call{\gcd}{d_1, \call{\totient}{p}} = 1\), an arbitrary 
  integer \(d_2\) such that \(\call{\gcd}{d_2, \call{\totient}{p}} = 1\), some constants 
  \(\alpha_{1}, \beta_{1}, \gamma_1, \dots, \alpha_{t - 1}, \beta_{t - 1}, \gamma_{t - 1} \in \mathbb{F}_p\) 
  such that \(\forall i < t\colon \alpha_i^2 - 4\beta_i\) is a quadratic non-residue modulo 
  \(p\), let \(e = {1}/{d_2}\) and, for all \(i < t\) let:
  \begin{align*}
    & \call{g_i}{x}\colon \mathbb{F}_p \to \mathbb{F}_p = x^2 + \alpha_{i}x + \beta_{i} \\
    & \call{h_i}{x}\colon \mathbb{F}_p \to \mathbb{F}_p = x^2 + \gamma_{i}x
  \end{align*}
  Then, the GTDS of \Arion{} is the function 
  \(\call{F_{GTDS}}{\bm{x}}\colon \mathbb{F}_p^t \to \mathbb{F}_p^t\) such that:
  \[
    \call{F_{GTDS}}{\bm{x}}_i = \bm{y}_i = 
    \begin{cases}
      \bm{x}_i^{d_1}\call{g_i}{\sigma_{i+1, t}} + \call{h_i}{\sigma_{i+1, t}} & 1 \le i < t \\
      \bm{x}_i^e & i = t
    \end{cases}
  \]
  where \(\sigma_{i, k} = \sum_{j=i+1}^{k}{\bm{x}_j + \bm{y}_j}\).
\end{definition}

It can be shown~\cite{RoyS2022} that the GTDS is function is invertible.
An interesting detail which came up only in the later phases of the GTDS design is the decoupling 
of the inverse exponentiation from the direct exponentiation, in the sense that, instead of 
using \(x^d\) and \(x^{1/d}\), we use \(x^{d_1}\) and \(x^{1/d_2}\).
The rationale of this choice is that some security considerations about cryptographic constructions
over the GTDS depend on the size of \(d_2\): if it is too small, we would need more rounds to 
achieve the desired level of security, hence increasing the circuit complexity.
If \(d_2\) were to be equal to \(d_1\), the reduction in terms of number of rounds would be 
overcompensated by the increase of the complexity of a single round, but since \(d_2\) is only 
used in the last branch, the tradeoff becomes more convenient, especially for bigger branch sizes.

Of course, \(d_2\) should require an optimal number of constraints.
Note that, since we have at our disposal intermediate results, the optimal number of oeprations 
required to exponentiate a number is not determined by the classical \emph{binary exponentiation} 
algorithm~\cite{Gueron2011}, but rather by \emph{addition chains}~\cite{BosC1990}.

\subsection{\Arion{} and \Arionhash{}}
Depending on our security and efficiency needs, we can instantiate the GTDS in many different 
ways.
We design \Arion{} and \Arionhash{}~\textbf{\cite{RoyST2023}} to work over fields of size 
\(\approx 2^{256}\).
In order to achieve a \emph{degree overflow} in the first round, it can be shown that \(4e\), where 
\(e = {1}/{d_2}\), should be greaater than \(p\). 
For BN254 and BLS12, this can be achieved by \(d_2 \in \Set{121, 123, 125, 161, 257}\).
For example, the optimal way to compute \(x^{121}\) is:
\begin{align*}
  & y = \Parens*{x^{2}}^{2} && z = \Parens*{y^{2}y}^{2} && x^{121} = \Parens*{z^{2}}^{2}zx
\end{align*}
It is not hard to see that numbers of the type \(2^k + 1\), for \(k \in \mathbb{N}\), are the 
most efficient to compute, and \(257\) is a particularly attractive candidate as it is also a 
prime number, and requires the same number of multiplications to be computed as the other candidates, 
(unfortunately, \(129 = 43 \cdot 3\) is not invertible neither in BN254 nor in BLS12).

To introduce mixing between the various branches, we use an \emph{affine layer} which employs a 
circulant matrix that has no zero entries and that is efficiently computable.
\begin{definition}[Affine layer of \Arion]
  The \emph{affine layer} of \Arion{} over a vector space \(\mathbb{F}_p^t\) is the function:
  \[\call{L}{\bm{x}, \bm{c}}\colon \Parens*{\mathbb{F}_p^n}^2 \to \mathbb{F}_p^n = 
  \call{\circulant}{1, \dots, t}\bm{x} + \bm{c}\]
\end{definition}

It is worth noting that the very simple matrix \(\call{\circulant}{1, \dots, t}\) is an MDS matrix
for any prime field \(\mathbb{F}_p\) such that \(p \ge 2^{39}\) and for values of
\(t \in \Iinterval{2}{12}\).
Furthermore, computing the matrix-vector product using \(\call{\circulant}{1, \dots, t}\) can be 
done in \(\BigO{t}\) time instead of the typical \(\BigO{t^2}\) required by a standard matrix-vector 
multiplication algorithm, by using \Cref{alg:circ_mult}.
\begin{algorithm}
  \begin{algorithmic}
    \Function{circ\_mul}{$\bm{v} \in \mathbb{F}_p^t$}
    \State{\(\bm{w} \gets \bm{0} \in \mathbb{F}_p^t\)}
    \State{\(\sigma \gets \sum_{i=1}^{t}{\bm{v}_i}\)}
    \State{\(\bm{w}_1 \gets \sigma + \sum_{i=1}^{t}{\Parens*{i - 1}\bm{v}_i}\)}
    \For{\(i \in \Iinterval{2}{t}\)}
    \State{\(\bm{w}_i \gets \bm{w}_{i-1} - \sigma + n\bm{v}_{i-1}\)}
    \EndFor{}
    \State{\Return{\(\bm{w}\)}}
    \EndFunction{}
  \end{algorithmic}
  \caption{Efficient evaluation of the matrix-vector product with 
    \(\call{\circulant}{1, \dots, t}\)}\label{alg:circ_mult}
\end{algorithm}

\begin{definition}[\Arion{} keyed permutation~\textbf{\cite{RoyST2023}}]
  Given a prime field \(\mathbb{F}_p\), a number of branches \(t \in \mathbb{N}\), a number 
  of rounds \(r \in \mathbb{N}\), some constants \(\bm{c}_1, \dots, \bm{c}_r \in \mathbb{F}_p^t\), 
  the \emph{\Arion{} keyed permutation} is the function \(\Arion = \Arion_r\), where:
  \[
    \call{\Arion_i}{\bm{x}, \bm{k}_0, \dots, \bm{k}_r}\colon 
      \Parens*{\mathbb{F}_p^t}^{r+2} \to \mathbb{F}_p^t = \bm{y}_i =
      \begin{cases}
        \call{L}{\bm{x}, \bm{0}} + \bm{k}_i & i = 0 \\
        \call{L}{\call{F_{GTDS}}{\bm{x}}, \bm{c}_i} + \bm{k}_i & 1 \le i \le r
      \end{cases}
  \]
\end{definition}

\begin{definition}[\Arionp{} unkeyed permutation]
  The \emph{\Arion{} unkeyed permutation} is the function:
  \[
    \call{\Arionp}{\bm{x}}\colon \mathbb{F}_p^t \to \mathbb{F}_p^t = 
      \call{\Arion}{\bm{x}, \bm{0}, \dots, \bm{0}}
  \]
\end{definition}

We can instantiate the hash function \Arionhash{} by applying the sponge construction to 
the \Arionp{} permutation.
Since it has been shown that, for a pseudorandom permutation \(P\) over a vector space 
\(\mathbb{F}_p^{t}\), a rate \(r\) and a capacity 
\(c\) such that \(t = r + c\), the sponge construction is indifferentiable from a random 
distribution up to \(\call{\min}{p^r, p^{c/2}}\) queries~\cite{BertoniDPV2008}, to provide 
\(\kappa \) bits of security, we must require that \(r \ge {\kappa}/{\call{\log}{p}}\) and 
\(c \ge {2\kappa}/{\call{\log}{p}}\).

As a padding scheme for a message \(m \in \Set{0, 1}^*\) whose length is not a multiple of the rate 
\(r\), we use \(\call{\Pad}{m} = m \concat 0^{\Parens*{-\abs{m} \bmod r}}\), and we replace the 
initial value \(v = 0^{\Parens*{t\abs{\Encode{p}}}}\) (where \(\Encode{x}\) denotes the binary 
encoding of an object \(x\)) with 
\(v' = \Encode{\abs{m}} \concat 0^{\Parens*{t-1}\abs{\Encode{p}}}\).
Note that we assume,  as basically any constructions does, that \(\abs{m} < p\), 
which should not be a problem as typically \(p \approx 2^{256}\) and we don't expect to hash 
messages of length \(\abs{m} > 2^{256}\).

\begin{table}
  \centering
  \caption{Instantiation parameters of \Arion{} and \Aarion{} for \(128\) bits of security and 
    primes \(p \geq 2^{60}\).}\label{tab:arion_instantiation}
  \begin{tabular}[t]{  c  c  c  c  }
      \toprule

      \phantom{ }\(d_1\)\phantom{ } & \phantom{ }\(t\)\phantom{ } & \phantom{ }\Arion{} \phantom{ } & \phantom{ }\Aarion{} \phantom{ } \\
      \midrule
      \multicolumn{2}{  c  }{} & \multicolumn{2}{ c  }{\phantom{ }Rounds\phantom{ }} \\
      \midrule
      \(3\) & \(3\) & \(6\) & \(5\) \\
      \(5\) & \(3\) & \(6\) & \(4\) \\

      \(3\) & \(4\) & \(6\) & \(4\) \\
      \(5\) & \(4\) & \(5\) & \(4\) \\

      \(3\) & \(5\) & \(5\) & \(4\) \\
      \(5\) & \(5\) & \(5\) & \(4\) \\

      \(3\) & \(6\) & \(5\) & \(4\) \\
      \(5\) & \(6\) & \(5\) & \(4\) \\

      \(3\) & \(8\) & \(4\) & \(4\) \\
      \(5\) & \(8\) & \(4\) & \(4\) \\
      \bottomrule
  \end{tabular}
\end{table}

In \Cref{tab:arion_instantiation} you can find the parameters for \Arion{} and 
its aggressive variant \Aarion{}, the parameters for the respective hash functions are respectively 
the same.
The aggressive variants have been parametrized in a way to provide the desired security level 
against all attacks but probabilistic Gr\"{o}bner basis attacks.
The choice to provide them anyway was dictated by the fact that, to the best of our knowledge, none 
of the competitor designs has been proved secure against such attacks, hence the aggressive 
versions provide the same guarantees in terms of security as the current state of the art.
For a detailed security analysis of \Arion, \Arionhash{} and their aggressive variants, refer 
to~\textbf{\cite{RoyST2023}}.

\subsection{Experimental results}
