\section{\Arion{} and \ArionHash{}}\label{sec:gtds}
The constructions that we saw in \Cref{sec:sota} are prominent examples of different 
\emph{generations} of \emph{Arithmetization Oriented} (AO) designs.
For example, \Mimc{} is an example of a Gen-I design: its main purpose was mostly to demonstrate 
that it was indeed possible to to construct secure and efficient cryptographic primitives by 
stacking simple, low-degree round functions.
On the other hand, the \Hades{} framework, its derivative \Poseidon{} and other similar 
constructions like \Rescue{} are examples of Gen-II designs: by tweaking the SPN and 
Feistel constructions, their purpose was to massimely improve the efficiency over Gen-I designs.
Finally, \Griffin{} is an example of a Gen-III design: the underlying \Horst{} scheme is neither 
a ``pure'' Feistel nor SPN design and, by deviating from such standard constructions, its authors 
were able to improve the efficiency even further. 
Furthermore, in third generation designs it was shown that one does not necessarily need to use 
round functions with a low degree, as long as the resulting constraint system is not affected 
negatively\footnote{As it is often the case in research, the separation line is a bit blurry, 
as \Rescue, which we said to be a Gen-II design, already used inverse exponentiations}.

Other from \Griffin, there is another very recent Gen-III construction, called 
\Anemoi~\cite{BouvierBCPSVW2022} and based on the \Flystel{} design, which has some common points 
with the \Horst{} construction.
A very interesting fact used in \Flystel{} was the notion of CCZ-equivalence~\cite{CarletCZ1998},
a concept which generalizes the intuition behind the idea that there is ``no difference'' 
between, say, using \(x^{d}\) and \(x^{\frac{1}{d}}\).
\begin{definition}[Affine function]
  An \emph{affine function} over an \(n\)-dimensional vector space \(\mathbb{F}^n\) is a function
  \(\call{f}{\bm{x}}\colon \mathbb{F}^n \to \mathbb{F}^n = \bm{Mx} + q\), where 
  \(\bm{M} \in \mathbb{F}^{n \times n}\) and \(q \in \mathbb{F}\).
\end{definition}
\begin{definition}[Function graph]
  Given a set \(S\), the \emph{function graph} of a function \(f\colon S \to S\) is the pair 
  \(\Gamma = \Tuple{S, E}\) where \(E = \Set{\Tuple{x, \call{f}{x}} \mid x \in S}\).
\end{definition}
\begin{definition}[Induced permutation]
  Given a \emph{function graph} \(\Gamma = \Tuple{S, E}\) and a permutation \(P\colon S^2 \to S^2\), 
  the \emph{induced permutation} of \(\Gamma \) by \(P\) is the function graph 
  \(\call{P}{\Gamma} = \Tuple{S, E'}\) where \(E' = \Set{\call{P}{e} \mid e \in E}\).
\end{definition}
\begin{definition}[CCZ equivalence~\cite{BouvierBCPSVW2022,CarletCZ1998}]
  Given a vector space \(\mathbb{V}\) and two functions 
  \(f,g\colon \mathbb{V} \to \mathbb{V}\), \(f\) and \(g\) are \emph{CCZ-equivalent} if there 
  is an affine permutation \(L\colon \mathbb{V}^2 \to \mathbb{V}^2\) such 
  that \(\Gamma_{f} = \call{L}{\Gamma_{g}}\).
\end{definition}

Clearly, CCZ-equivalence is an equivalence relation over a vector space \(\mathbb{V}\), hence it 
induces a partitioning of \(\mathbb{V}^2\) into equivalence classes.
An interesting fact is that all CCZ-equivalent functions share the same linear and differential 
properties.
Even more importantly for our purposes is that CCZ-equivalent functions are indistinguishable under 
constraint verification: checking the constraint system of any member of the class verifies 
the validity of the computation of any other member.
However, one thing that CCZ-equivalent functions do not share in general is their degree, hence we 
can use the one with highest degree in the actual computation to provide the most strict security 
guarantees against algebraic attacks, while using the one with the lowest degree when building
the constraint system for the verification in the SNARK framework.

In an high-level, intuitive way we can say that CCZ-equivalence allows us to ``ignore the order'' 
in which the witnesses of a certain computation are obtained: reprising our usual example,
the verifier does not care that the prover first must know \(x\) in order to obtain \(y = x^{1/d}\), 
all it matters is that the prover knows both of them and that their relationship is correct.
In fact, even more generally, there is no way to know in which order someone actually got hold of 
the intermediate values of a computation, hence we might as well exploit this to our advantage in 
order to reduce the complexity of the protocol.

\subsection{The Generalized Triangular Dynamical System}
As we said, the design of third generation AO cryprographic primitives diverts from the plain 
SPN or Feistel constructions.
For this reason, we introduce the \emph{Generalized Triangular Dynamical System}~\cite{RoyS2022}, 
GTDS for short, an algebraic framework which generalizes many previous designs (such as Feistel, 
SPN, and \Horst) and enables us to provide a systematic security analysis 
of the constructions derived from it\footnote{The GTDS design idea is due to Matthias Steiner.}.
In particular:
\begin{itemize}
  \item The input is split in branches like in earlier designs.
  \item The round function offers the strength of all the incorporated designs.
  \item It is secure against classical attacks like differential cryptanalysis.
  \item It is secure against interpolation attacks already at the first round.
  \item Its linear layer mixes all the branches through a circulant matrix with no zero entries.  
  \item It uses inverse exponents, but decouples them from the direct exponent. 
\end{itemize}

\begin{definition}[GTDS of \Arion]\label{def:gtds}
  Given a prime field \(\mathbb{F}_p\), a number of branches \(t \in \mathbb{N}\), the smallest 
  integer \(d_1\) such that \(\call{\gcd}{d_1, \call{\totient}{p}} = 1\), an arbitrary 
  integer \(d_2\) such that \(\call{\gcd}{d_2, \call{\totient}{p}} = 1\), some constants 
  \(\alpha_{1}, \beta_{1}, \gamma_1, \dots, \alpha_{t - 1}, \beta_{t - 1}, \gamma_{t - 1} \in \mathbb{F}_p\) 
  such that \(\forall i < t\colon \alpha_i^2 - 4\beta_i\) is a quadratic non-residue modulo 
  \(p\), let \(e = {1}/{d_2}\) and, for all \(i < t\) let:
  \begin{align*}
    & \call{g_i}{x}\colon \mathbb{F}_p \to \mathbb{F}_p = x^2 + \alpha_{i}x + \beta_{i} \\
    & \call{h_i}{x}\colon \mathbb{F}_p \to \mathbb{F}_p = x^2 + \gamma_{i}x
  \end{align*}
  Then, the GTDS of \Arion{} is the function 
  \(\call{F}{\bm{x}}\colon \mathbb{F}_p^t \to \mathbb{F}_p^t\) such that:
  \[
    \call{F}{\bm{x}}_i = \bm{y}_i = 
    \begin{cases}
      \bm{x}_i^{d_1}\call{g_i}{\sigma_{i+1, t}} + \call{h_i}{\sigma_{i+1, t}} & 1 \le i < t \\
      \bm{x}_i^e & i = t
    \end{cases}
  \]
  where \(\sigma_{i, k} = \sum_{j=i+1}^{k}{\bm{x}_j + \bm{y}_j}\).
\end{definition}

It can be shown~\cite{RoyS2022} that the GTDS is function is invertible.
An interesting detail which came up only in the later phases of the GTDS design is the decoupling 
of the inverse exponentiation from the direct exponentiation, in the sense that, instead of 
using \(x^d\) and \(x^{1/d}\), we use \(x^{d_1}\) and \(x^{1/d_2}\).
The rationale of this choice is that some security considerations about cryptographic constructions
over the GTDS depend on the size of \(d_2\): if it is too small, we would need more rounds to 
achieve the desired level of security, hence increasing the circuit complexity.
If \(d_2\) were to be equal to \(d_1\), the reduction in terms of number of rounds would be 
overcompensated by the increase of the complexity of a single round, but since \(d_2\) is only 
used in the last branch, the tradeoff becomes more convenient, especially for bigger branch sizes.

\subsection{Security of \Arion{}}
