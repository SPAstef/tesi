\chapter{\Arion: Cryptographic Primitives from Generalized Triangular Dynamical Systems}\label{chap:arion}
One of the most important applications of zero-knowledge verifiable computation lies in digital 
currency transactions over the blockchain infrastructure.
An example of ZK-SNARK applied in the real world is the ZCash cryptocurrency~\cite{SassonCGGMTV2014}, 
which is inspired by the more famous Bitcoin~\cite{NarayananBFMG2016}, and was devised by some of 
the authors of \texttt{libsnark} (which frames the zero-knowledge backend of the currency).

As we discussed in \Cref{sec:tree_hash}, the fundamental component of a block chain is the 
Merkle tree, which uses one-way compression functions in order to produce the binding 
commitment.
In a digital currency scenario, the leaves of the Merkle tree consist of the details of some 
transaction, typical informations include the ID of the sender, the ID of the recipient, and the 
amount of currency to be transferred. 
Without a zero-knowledge framework in place, when one wants to verify whether an user did abide to 
their commitment, the only possible solution is to ask the user to disclose his transaction, 
together with the authentication path, and check that the tree commitment is respected. 
When using currencies like Bitcoin or Ethereum\footnote{\url{https://ethereum.org/}}, anyone 
can see the details of every single transaction being performed on the relative 
blockchain, meaning that there is no privacy whatsoever\footnote{For example, on 
\url{https://etherscan.io/} you can see the transactions on the Ethereum blockchain. %It is 
%curious how privacy has often been foisted as a feature of mainstream cryptocurrencies while, 
%on the contrary, any bank offers much more privacy!
}.
However, if we translate the Merkle tree computation in an equivalent circuit, it is possible to 
apply a zero-knowledge scheme that allows a verifier to be sure (with overwhelming probability) 
of the validity of a transaction without actually having to see it!
Since a Merkle tree applies over and over the underlying compression function, the problem of 
creating a circuit for the former immediately reduces to the problem of creating a circuit for the 
latter.

In \Cref{sec:sota} we will review the evolution of the state of the art concerning zero-knowledge 
friendly compression functions.
Then, in \Cref{sec:gtds}, we present a new algebraic framework to represent cryptographic 
primitives, the \emph{Generalized Triangular Dynamic System}, and apply it to construct the 
\Arion{} block cipher and the \Arionhash{} hash function, which we compare to the state of 
the art in the \texttt{libsnark} framework, showing extremely competitive results.
\section{State of the art}\label{sec:sota}
The standard compression function used in Merkle trees is usually one of the SHA-2 or SHA-3 
functions~\cite{Dang2015}: this is certainly the most sensible choice in a \emph{native} 
environment (i.e.\ no zero-knowledge), as SHA is specifically designed to be fast in both software 
and hardware~\cite{DaddaMO2004,MichailAKTG2012} implementations, and is the most studied hash 
function from a security standpoint (e.g.\ for SHA-2 
see~\cite{KhovratovichRS2012,GuoLRW2010,DobraunigEM2016}).

However, when working with arithmetic circuits over a prime field \(\mathbb{F}_p\), SHA has a lot 
of issues: the underlying operations being performed are bitwise XOR, bitwise AND, 
bit shifts/rotations and additions modulo \(2^{32}\).
While shifts and rotations come at no cost, as they basically consist in a renaming of the circuit 
wires/variables, bitwise operations and addition which is not modulo \(p\) have to be simulated 
bit-by-bit, and the overhead introduced in such a translation is huge.
For example, for SHA-256, over a bilinear group like BN254 for which 
\(\abs{\mathbb{F}_p} \approx 2^{256}\), we would need \(256\) input variables each holding a
\(256\)-bit integer to simulate the behaviour of every single bit during the SHA computation; 
clearly, this is decisely suboptimal.

\begin{example}
  Suppose we are given two strings \(a, b \in \Set{0, 1}^{n}\), and we want to compute 
  \(a \bitxor b\).
  By interpreting them as vectors \(\bm{v}, \bm{w} \in \mathbb{F}_{p}^{n}\), we can simulate 
  bitwise XOR by computing, \(\forall i \le n\):
  \[\bm{v}_{i} \bitxor \bm{w}_{i} = \bm{v}_{i} + \bm{w}_{i} - 2\bm{v}_{i}\bm{w}_{i}\]
  that is, every XOR operation requires one multiplication gate.
  Similarly, bitwise AND and non-native addition also require multiplications to be simulated.
  Furthermore, we must guarantee that the values \(\bm{v}_i\) and \(\bm{w}_i\) are boolean, as 
  in principle they could assume any value in \(\mathbb{F}_p\), so we must also add constraints of 
  the kind \(\bm{v}_{i}\Parens*{\bm{v}_i - 1} = 0\).
\end{example}

\subsection{\Mimc}
In an effort to find secure cryptographic designs that could be efficient in zero-knowledge 
settings, called \emph{zk-friendly} designs, researchers began to study the properties of 
permutations that make use of a low number of multiplications 
(\emph{multiplicative complexity})~\cite{AlbrechtRSTZ2016}.

One of the first constructions over finite fields was the \emph{Minimal Multiplicative Complexity}
(\Mimc) family of cryptographic permutations~\cite{AlbrechtGRRT2016}.
The idea of \Mimc{}, reprising an older proposal~\cite{NybergK1995}, is to use a very simple 
polynomial permutation as its core component, and by repeating it for an adequate number of rounds,
obtain a secure construction.
\begin{definition}[\Mimc{} keyed permutation]
  Given a finite field \(\mathbb{F}_p\), a number of rounds 
  \(r = \Ceil*{\frac{\call{\log}{p}}{\call{\log}{3}}}\), some constants 
  \(c_1, \dots, c_r \in \mathbb{F}_p\) and a set of functions 
  \(f_1, \dots, f_r\colon \mathbb{F}_p \times \mathbb{F}_p \to \mathbb{F}_p\) such that 
  \(\forall i \le r\colon \call{f_i}{x, k} = x^3 + k + c_i\), the \emph{\Mimc{} keyed permutation}
  is defined as:
  \[
    \call{E_{\Mimc}}{x, k}\colon \mathbb{F}_p \times \mathbb{F}_p \to \mathbb{F}_p = 
    \call{\Parens*{f_r \compose \dots \compose f_1}}{x, k} + k
  \]
\end{definition}

The \Mimc{} keyed permutation is also called \Mimc-\(n/n\). 
By applying the Feistel construction on the \Mimc{} permutation, one obtains Feistel \Mimc, 
or \Mimc-\(2n/n\).
Finally, by applying the sponge construction, one can obtain the \Mimchash{} hash function.
In alternative, it is also possible to build an hash function using first the Davies-Meyer 
construction to obtain a one-way compression function, and then the Merkle-Damg\"{a}rd construction
to obtain an hash function.

There are some important observations to be made on the \Mimc{} construction.
First, the round permutation uses a low degree polynomial, but it is repeated for a high number of 
rounds: for example, if the size of the underlying field is \(\approx 2^{256}\), the number of 
rounds will be \(r = 162\). 
Note that \(x^3\) might not actually induce a permutation over \(\mathbb{F}_p\), as in general 
\(3\) is not coprime with \(\call{\totient}{p}\) (in fact, in the underlying fields of both BN254 
and BLS12, \(3\) is a factor of \(p - 1\)).
In such cases, one should modify the definition to consider the smallest prime number \(d\) such 
that \(\call{\gcd}{d, \call{\totient}{p}} = 1\), and reduce the number of rounds to
\(r = \Ceil*{\frac{\call{\log}{p}}{\call{\log}{d}}}\).

A second observation is that \(r\) must be chosen to thwart many different types of cryptanalysis 
techniques: since the \Mimc{} permutation corresponds to the 
polynomial \(p = \Parens*{x^3 + k + c_1}\dots\Parens*{x^3 + k + c_r}\) 
(which has degree \(\call{\deg}{p} = 3^r\)), in addition to the traditional \emph{brute-force}, 
\emph{meet-in-the-middle}~\cite{DiffieH1977}, \emph{differential}~\cite{BihamS1991} and 
\emph{linear}~\cite{Matsui1994} attacks, one must also consider \emph{algebraic attacks}, 
which exploit the inherent nature of this type of constructions.

In fact, traditional attacks don't tend to pose a major threat to these kinds of constructions:
brute force is clearly too expensive and meet-in-the-middle is also infeasible both due to the high 
number of rounds and to the huge degree of the inverse permutation (usually \(1/3 \gg 3\)).
The permutation \(x^3\) is not approximable by a linear function~~\cite{AbdelraheemABL2012}, 
hence linear attacks are not a threat, and since it can be easily shown that any arbitrary input 
difference \(\delta_{in} \) propagates to any arbitrary output difference \(\delta_{out} \) with a 
probability of at most \({2}/{2^n}\), differential attacks are also ineffective~\cite{Nyberg1994}.

On the side of algebraic cryptanalysis, one might attempt an \emph{interpolation attack}, which 
uses Lagrange interpolation to find a polynomial \(\tilde{p}\) which behaves like a keyless version 
of \(p\)~\cite{JakobsenK1997}.
This attack's complexity depends solely on \(\call{\deg}{p}\) (in fact, an interpolation can be 
computed in \(\BigO{n\call{\log}{n}}\), where \(n = \call{\deg}{p}\)~\cite{Stoss1985}), hence we 
must be sure that the degree of \(p\) also grows exponentially round by round (as it is the case).
Another kind of algebraic attack is the \emph{GCD attack}: by using two plaintext/ciphertext pairs,
once can compute their greatest common divisor which will allow to easily retrieve the secret key.
Again, computing the GCD depends almost linearly on the degree of the polynomial, hence one must 
again be sure that the degree grows exponentially.

\subsubsection*{\Mimc{} vs.\ SHA}
Performing computations over large prime fields is extremely expensive, as addition and 
multiplication cannot be performed by a single CPU intruction, but must be emulated (multiplication 
in particular is extremely costly, even with clever implementations like the Montgomery 
form~\cite{Montgomery1985}): for example, on an x86 architecture, a standard software 
implementation of SHA-256 is more than \(100\) times faster than \Mimc{} over the BN254 curve 
(implemented using \texttt{libff}\footnote{https://github.com/scipr-lab/libff}, the underlying 
arithmetic library of \texttt{libsnark}) to compress a 512 bit input to a 256 bit output.
However, when translated into an arithmetic circuit, SHA-256 requires about \(25000\) 
multiplications, while \Mimc{} requires only \(640\), making it about \(40\) times (furthermore, 
designing and optimizing the SHA-256 circuit is much harder than for \Mimc{}). 

\subsection{Poseidon}
After the design of \Mimc{}, a natural question that followed was whether it was possible to increase
the complexity of the round function and reduce the number of rounds without compromising on 
security.
In particular, researchers started exploring the design of \emph{algebraic frameworks} that could 
abstract from the details involved in the construction of a specific primitive, by still providing 
security guarantees for their instantiatios.

In~\cite{GrassiLRRS2019}, the authors present the \Hades{} design, a generalization
of the \emph{substitution-permutation network} (SPN) approach used in many famous cryptographic 
primitives such as AES~\cite{DaemenR1999}.
In fact, the \Hades{} design has the same three main steps of the classic SPN, where the 
message and the key are interpreted as vectors in \(\mathbb{F}_{p}^t\) for some prime field 
\(\mathbb{F}_p\) and some arbitrary \(t \in \mathbb{N}\):
\begin{enumerate}
  \item \emph{AddKey}: add the key to the message.
  \item \emph{SubWords}: apply a substitution function \(S\colon \mathbb{F}_{p} \to \mathbb{F}_{p}\) 
        to the elements of the message (\emph{non-linear layer}).
  \item \emph{MixLayer}: apply a permutation function \(M\colon \mathbb{F}_{p}^t \to \mathbb{F}_{p}^t\) 
        to the message (\emph{linear layer}).
\end{enumerate}
The substitution function \(S\) is defined as \(\call{S}{x} = x^d\), where \(d\) is the smallest 
integer such that \(\call{\gcd}{d, p} = 1\), while the permutation function \(M\) is defined as 
\(\call{M}{\bm{x}} = \bm{Mx}\), where \(\bm{M}\) is a \emph{maximum distance separable} (MDS) 
matrix (i.e.\ the determinant of all its submatrices are non-zero, this for example minimizes the 
effectiveness of differential attacks~\cite{MacwilliamsS1977,RijmenD1996}).

The peculiarity of \Hades{} is that only the first and last few rounds apply the substitution
function to all elements: the middle rounds are \emph{partial}, meaning that \(S\) is only applied 
to some of the elements of the message.
This allows to use the first and last part of the construction as an argument for dealing with 
classical attacks, while also considering the middle rounds when dealing with algebraic attacks.
Additionally, in the design of \Hades{}, new kind of algebraic attacks were 
considered~\cite{BeyneEtAl2020}, the most important of which is certainly the 
\emph{Gr\"{o}bner basis attack} 
(a generalization of the concept of gaussian elimination~\cite{CoxLO2015,Lazard1983}).
In a Gr\"{o}bner attack, we view the function under attack as a system of polynomial equations 
for which we want to find a solution.
This kind of attack, and in particular its probabilistic version~\cite{FaugereGHR2014}, is 
(at the moment of this writing) the most efficient known attack against algebraic constructions 
(note however that the security of \Hades{} was only proven against the deterministic 
version).

A family of hash functions that builds on the \Hades{} design is \Poseidon~\cite{GrassiKRRS2021}.
\begin{definition}[\Poseidon{} permutation]
  Given a finite field \(\mathbb{F}_p\), a number of branches \(t \in \mathbb{N}\), a number of 
  full rounds \(r_f \in \mathbb{N}\), a number of partial rounds \(r_{P} \in \mathbb{N}\), 
  an MDS matrix \(\bm{M} \in \mathbb{F}_{p}^{t \times t}\), and some constants 
  \(\bm{c}_{1}, \dots, \bm{c}_{r} \in \mathbb{F}_{p}^{t}\), where \(r = 2r_f + r_P\), let the 
  \emph{full SBOX function} be:
  \[
    \call{S_{f}}{\bm{x}}\colon \mathbb{F}_{p}^{t} \to \mathbb{F}_{p}^{t} = 
    {\begin{pmatrix}
      \bm{x}_{1}^{d} & \cdots & \bm{x}_{t}^{d}
    \end{pmatrix}}^{\transpose}
  \]
  and the \emph{partial SBOX function} be:
  \[
    \call{S_{P}}{\bm{x}}\colon \mathbb{F}_{p}^{t} \to \mathbb{F}_{p}^{t} = 
    {\begin{pmatrix}
      \bm{x}_{1}^{d} & \bm{x}_{2} & \cdots & \bm{x}_{t}
    \end{pmatrix}}^{\transpose}
  \]
  where \(d\) is the smallest prime number such that \(\call{\gcd}{x, \call{\totient}{p}} = 1\).
  Then, let the full \(i\)th round function be 
  \(\call{f_{i}}{\bm{x}}\colon \mathbb{F}_{p}^{t} \to \mathbb{F}_{p}^{t} = 
  \bm{M}\call{S_{f}}{\bm{x} + \bm{c}_{i}}\)
  and the partial \(i\)th round function be
  \(\call{P_{i}}{\bm{x}}\colon \mathbb{F}_{p}^{t} \to \mathbb{F}_{p}^{t} = 
  \bm{M}\call{S_{P}}{\bm{x} + \bm{c}_{i}}\).
  The \emph{\Poseidon{} permutation} is the function:
  \[
    P_{\Poseidon} = 
    f_{r} \compose \cdots \compose f_{r_f + r_P + 1} \compose 
    P_{r_f + r_P} \compose \cdots \compose P_{r_f + 1} \compose 
    f_{r_f} \compose \cdots \compose f_{1}
  \]
\end{definition}

The \Poseidon{} permutation can be easily used in a sponge construction to obtain a
one-way compression function.
For example, by setting \(t = 3\), we immediately obtain a \(2/1\) compression function, 
or by setting \(t = 5\) we get a \(4/1\) compression function.
For the same level of security \(s\), the \Poseidon-\(s\) arithmetic circuit is much cheaper than 
\Mimc{} in terms of multiplications. 
For reference, the \Poseidon-\(128\) \(2/1\) compression function requires \(276\) 
multiplications, compared to the \(640\) required by \Mimc{}.

Furthermore, \Poseidon{} was also designed with other systems other than the more traditional 
ZK-SNARK constructions in mind such as \(\Plonk \)~\cite{GabizonWC2019}, 
Bulletproofs~\cite{BunzBBPWM2017} and ZK-STARKs~\cite{SassonBHR2018} 
(\emph{transparent} ZK-SNARKs, i.e.\ they do not require a trusted setup phase).
For example, in the \(\Plonk \) system the cost of the ZK-SNARK construction depends also 
on the number of additions, while in ZK-STARKs it depends on the \emph{depth} of the circuit.
These systems, which are much more recent, do not have consolidated libraries which support them, 
and their analysis is out of the scope of this work.

\subsection{Griffin}
Many new designs were proposed in the last few years, such as 
\textsc{Ciminion}~\cite{DobraunigGGK2021}, which exploits \emph{Toffoli gates}~\cite{Toffoli1980} 
to improve the construction, and \Rescue~\cite{AlyABDS2019}, which uses inverse 
exponentiations (i.e.\  \(x^{1/d}\)) in the substitution layer.

An extremely recent design is \Griffin~\cite{GrassiHRSWW2022}, which deviates from the
established SPN approach by instead using an extension of the Feistel construction, called the 
\emph{\Horst{} scheme}: instead of the mapping 
\(\Tuple{x, y} \mapsto \Tuple{x, y \oplus \call{F}{x}}\) for some function \(F\) over the 
underlying field \(\mathbb{F}_p\), it uses the mapping 
\(\Tuple{x, y} \mapsto \Tuple{x, y \otimes \call{G}{x} \oplus \call{F}{x}}\), where \(G\) is an 
additional function such that \(\forall x \in \mathbb{F}_p\colon \call{G}{x} \neq 0\), in order 
to maintain invertibility (note that if \(\forall x\colon \call{G}{x} = 1\), the \Horst{} scheme
degenerates into the Feistel construction).

A quirk of \Griffin{} is the matrix used in the linear layer (which is basically the same of 
\Poseidon): it is a \(t \times t\) MDS circulant matrix, where \(t\) is the number of branches, 
but is well-defined only when \(t = 3\) or \(t \equiv 0 \pmod{4}\).

\begin{definition}[Griffin permutation]
  Given a finite field \(\mathbb{F}_p\), a number of branches \(t \in \mathbb{N}\) such that 
  \(t = 3\) or \(t \equiv 0 \pmod{4}\), a number of rounds \(r \in \mathbb{N}\), some constants 
  \(\bm{c}_{1}, \dots, \bm{c}_{r} \in \mathbb{F}_{p}^{t}\), some pairwise-distinct 
  constants \(\gamma_3, \dots, \gamma_t \in \mathbb{F}_{p}\), and some constants 
  \(\alpha_3, \beta_3, \dots, \alpha_t, \beta_t \in \mathbb{F}_{p}\) such that 
  \(\forall i \in \Iinterval*{3}{t}\colon \alpha_i \neq \beta_i\) and 
  \(\alpha_{i}^{2} - 4\beta_{i}\) is a quadratic non-residue modulo \(p\), 
  let the linear functions \(L_3, \dots, L_t\) be such that:
  \[\call{L_i}{x, y, z}\colon \mathbb{F}_p^3 \to \mathbb{F}_p = \gamma_{i}x + y + z\]
  Furthermore, let the \emph{non-linear layer} be the function:
  \[
    \call{S}{\bm{x}}_i\colon \mathbb{F}_{p}^{t} \to \mathbb{F}_{p}^{t} = \bm{y}_i =
    \begin{cases}
      x_{1}^{1/d}                                                         & i = 1         \\
      x_{2}^{d}                                                           & i = 2         \\
      x_{i}\Parens*{\call{L_i}{\bm{y}_0, \bm{y}_1, 0}^{2} + 
      \alpha_{i}\call{L_i}{\bm{y}_0, \bm{y}_1, 0} + \beta_i}              & i = 3         \\
      x_{i}\Parens*{\call{L_i}{\bm{y}_0, \bm{y}_1, \bm{x}_{i-1}}^{2} + 
      \alpha_{i}\call{L_i}{\bm{y}_0, \bm{y}_1, \bm{x}_{i-1}} + \beta_i}   & 4 \le i \le t
    \end{cases}
  \]
  where \(d\) is the smallest prime number such that \(\call{\gcd}{x, \call{\totient}{p}} = 1\).\\
  Finally, let the matrix \(\bm{M}_t \in \mathbb{F}_p^{t \times t}\) be:
  \[
    \bm{M}_t = 
    \begin{cases}
      \call{\circulant}{2, 1, 1} & t = 3 \\  
      \call{\circulant}{3, 2, 1, 1} & t = 4 \\
      \call{\diag}{\underbrace{2, \dots, 2}_{t}}
      \call{\diag}{\underbrace{\bm{M}_{4}, \dots, \bm{M}_4}_{s}}
      \call{\circulant}{\underbrace{\bm{I}_4, \dots, \bm{I}_4}_{s}}
      & t = 4s
    \end{cases}
  \]
  and the \(i\)th round function be
  \(\call{f_i}{\bm{x}}\colon \mathbb{F}_{p}^{t} \to \mathbb{F}_{p}^{t} = 
    \bm{M}_{t}\call{S}{\bm{x}} + \bm{c}_i\).
  The \emph{\Griffin{} permutation} is the function:
  \[
    P_{\Griffin} = f_{r} \compose \cdots \compose f_{1} \compose \bm{M}_{t}
  \]
\end{definition}

The quadratic non-residuosity requirement for the pairs \(\alpha_i, \beta_i\) is a condition 
imposed in order to make sure that the non-linear layer \(S\) does not output zero from the third 
branch onwards, unless the input is also zero.

Another important observation about the non-linear layer is the usage of the inverse exponentiation.
As we already observed when discussing \Mimc, usually \(1/d \gg d\), which means that the degree 
of the permutation grows much faster. 
While this also entails that computing \(x^{\frac{1}{d}}\) will be slower, from the constraint 
system point of view there is absolutely no difference: since \(y = x^{\frac{1}{d}} \iff x = y^{d}\),
one can simply write the constraints in reverse, introducing no extra cost in the R1CS and, 
in turn, in the QAP\@!

Similarly to \Poseidon, the \Griffin{} permutation can be very easily extended to a compression 
function due to the embedded sponge mechanism. The fact that \(t\) must be either \(3\) or a 
multiple of \(4\) is a bit unfortunate though, as usually one would like to use powers of \(2\) as 
input size of a compression function: except for \(2\)-to-\(1\), \Griffin{} does not allow any 
such instantiation, (when \(t = 4s\), we have a \(\Parens*{4s-1}\)-to-\(1\) compression function).

From a performance point of view, \Griffin{} is extremely competitive: for reference, the 
\(2\)-to-\(1\) instantiation only needs \(96\) R1CS constraints, compared to the \(276\) required 
by \Poseidon.
In fact, although it is a very recent design, i.e.\ it is subject to changes, \Griffin{} will be 
our target competitor in terms of pure performance in ZK-SNARK settings.

\section{\Arion{} and \ArionHash{}}\label{sec:gtds}
\subsection{The Generalized Dynamic Triangular System}
\subsection{Security of \Arion{}}

