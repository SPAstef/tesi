\chapter{\Arion: Cryptographic Primitives from Generalized Triangular Dynamical Systems}\label{chap:arion}
One of the most important applications of zero-knowledge verifiable computation lies in digital 
currency transactions over the blockchain infrastructure.
An example of ZK-SNARK applied in the real world is the ZCash cryptocurrency~\cite{SassonCGGMTV2014}, 
which is inspired by the more famous Bitcoin~\cite{NarayananBFMG2016}, and was devised by some of 
the authors of \texttt{libsnark} (which frames the zero-knowledge backend of the currency).

As we discussed in \Cref{sec:tree_hash}, the fundamental component of a block chain is the 
Merkle tree, which uses one-way compression functions in order to produce the binding 
commitment.
In a digital currency scenario, the leaves of the Merkle tree consist of the details of some 
transaction, typical informations include the ID of the sender, the ID of the recipient, and the 
amount of currency to be transferred. 
Without a zero-knowledge framework in place, when one wants to verify whether an user did abide to 
their commitment, the only possible solution is to ask the user to disclose his transaction, 
together with the authentication path, and check that the tree commitment is respected. 
When using currencies like Bitcoin or Ethereum\footnote{\url{https://ethereum.org/}}, anyone 
can see the details of every single transaction being performed on the relative 
blockchain, meaning that there is no privacy whatsoever\footnote{For example, on 
\url{https://etherscan.io/} you can see the transactions on the Ethereum blockchain. %It is 
%curious how privacy has often been foisted as a feature of mainstream cryptocurrencies while, 
%on the contrary, any bank offers much more privacy!
}.
However, if we translate the Merkle tree computation in an equivalent circuit, it is possible to 
apply a zero-knowledge scheme that allows a verifier to be sure (with overwhelming probability) 
of the validity of a transaction without actually having to see it!
Since a Merkle tree applies over and over the underlying compression function, the problem of 
creating a circuit for the former immediately reduces to the problem of creating a circuit for the 
latter.

In \Cref{sec:sota} we will review the evolution of the state of the art concerning zero-knowledge 
friendly compression functions.
Then, in \Cref{sec:gtds}, we present a new algebraic framework to represent cryptographic 
primitives, the \emph{Generalized Triangular Dynamic System}, and apply it to construct the 
\Arion{} block cipher and the \Arionhash{} hash function.
Finally, in \Cref{sec:performance}, we compare our new construction to the state of the art using 
the \texttt{libsnark} library, showing extremely competitive results.
\section{State of the art}\label{sec:sota}
The standard compression function used in Merkle trees is usually one of the SHA-2 or SHA-3 
functions~\cite{Dang2015}: this is certainly the most sensible choice in a \emph{native} 
environment (i.e.\ no zero-knowledge), as SHA is specifically designed to be fast in both software 
and hardware~\cite{DaddaMO2004,MichailAKTG2012} implementations, and is the most studied hash 
function from a security standpoint (e.g.\ for SHA-2 
see~\cite{KhovratovichRS2012,GuoLRW2010,DobraunigEM2016}).

However, when working with arithmetic circuits over a prime field \(\mathbb{F}_p\), SHA has a lot 
of issues: the underlying operations being performed are bitwise XOR, bitwise AND, 
bit shifts/rotations and additions modulo \(2^{32}\).
While shifts and rotations come at no cost, as they basically consist in a renaming of the circuit 
wires/variables, bitwise operations and addition which is not modulo \(p\) have to be simulated 
bit-by-bit, and the overhead introduced in such a translation is huge.
For example, for SHA-256, over a bilinear group like BN254 for which 
\(\abs{\mathbb{F}_p} \approx 2^{256}\), we would need \(256\) input variables each holding a
\(256\)-bit integer to simulate the behaviour of every single bit during the SHA computation; 
clearly, this is decisely suboptimal.

\begin{example}
  Suppose we are given two strings \(a, b \in \Set{0, 1}^{n}\), and we want to compute 
  \(a \bitxor b\).
  By interpreting them as vectors \(\bm{v}, \bm{w} \in \mathbb{F}_{p}^{n}\), we can simulate 
  bitwise XOR by computing, \(\forall i \le n\):
  \[\bm{v}_{i} \bitxor \bm{w}_{i} = \bm{v}_{i} + \bm{w}_{i} - 2\bm{v}_{i}\bm{w}_{i}\]
  that is, every XOR operation requires one multiplication gate.
  Similarly, bitwise AND and non-native addition also require multiplications to be simulated.
  Furthermore, we must guarantee that the values \(\bm{v}_i\) and \(\bm{w}_i\) are boolean, as 
  in principle they could assume any value in \(\mathbb{F}_p\), so we must also add constraints of 
  the kind \(\bm{v}_{i}\Parens*{\bm{v}_i - 1} = 0\).
\end{example}

\subsection{\Mimc}
In an effort to find secure cryptographic designs that could be efficient in zero-knowledge 
settings, called \emph{zk-friendly} designs, researchers began to study the properties of 
permutations that make use of a low number of multiplications 
(\emph{multiplicative complexity})~\cite{AlbrechtRSTZ2016}.

One of the first constructions over finite fields was the \emph{Minimal Multiplicative Complexity}
(\Mimc) family of cryptographic permutations~\cite{AlbrechtGRRT2016}.
The idea of \Mimc{}, reprising an older proposal~\cite{NybergK1995}, is to use a very simple 
polynomial permutation as its core component, and by repeating it for an adequate number of rounds,
obtain a secure construction.
\begin{definition}[\Mimc{} keyed permutation]
  Given a finite field \(\mathbb{F}_p\), a number of rounds 
  \(r = \Ceil*{\frac{\call{\log}{p}}{\call{\log}{3}}}\), some constants 
  \(c_1, \dots, c_r \in \mathbb{F}_p\) and a set of functions 
  \(f_1, \dots, f_r\colon \mathbb{F}_p \times \mathbb{F}_p \to \mathbb{F}_p\) such that 
  \(\forall i \le r\colon \call{f_i}{x, k} = x^3 + k + c_i\), the \emph{\Mimc{} keyed permutation}
  is defined as:
  \[
    \call{E_{\Mimc}}{x, k}\colon \mathbb{F}_p \times \mathbb{F}_p \to \mathbb{F}_p = 
    \call{\Parens*{f_r \compose \dots \compose f_1}}{x, k} + k
  \]
\end{definition}

The \Mimc{} keyed permutation is also called \Mimc-\(n/n\). 
By applying the Feistel construction on the \Mimc{} permutation, one obtains Feistel \Mimc, 
or \Mimc-\(2n/n\).
Finally, by applying the sponge construction, one can obtain the \Mimchash{} hash function.
In alternative, it is also possible to build an hash function using first the Davies-Meyer 
construction to obtain a one-way compression function, and then the Merkle-Damg\"{a}rd construction
to obtain an hash function.

There are some important observations to be made on the \Mimc{} construction.
First, the round permutation uses a low degree polynomial, but it is repeated for a high number of 
rounds: for example, if the size of the underlying field is \(\approx 2^{256}\), the number of 
rounds will be \(r = 162\). 
Note that \(x^3\) might not actually induce a permutation over \(\mathbb{F}_p\), as in general 
\(3\) is not coprime with \(\call{\totient}{p}\) (in fact, in the underlying fields of both BN254 
and BLS12, \(3\) is a factor of \(p - 1\)).
In such cases, one should modify the definition to consider the smallest prime number \(d\) such 
that \(\call{\gcd}{d, \call{\totient}{p}} = 1\), and reduce the number of rounds to
\(r = \Ceil*{\frac{\call{\log}{p}}{\call{\log}{d}}}\).

A second observation is that \(r\) must be chosen to thwart many different types of cryptanalysis 
techniques: since the \Mimc{} permutation corresponds to the 
polynomial \(p = \Parens*{x^3 + k + c_1}\dots\Parens*{x^3 + k + c_r}\) 
(which has degree \(\call{\deg}{p} = 3^r\)), in addition to the traditional \emph{brute-force}, 
\emph{meet-in-the-middle}~\cite{DiffieH1977}, \emph{differential}~\cite{BihamS1991} and 
\emph{linear}~\cite{Matsui1994} attacks, one must also consider \emph{algebraic attacks}, 
which exploit the inherent nature of this type of constructions.

In fact, traditional attacks don't tend to pose a major threat to these kinds of constructions:
brute force is clearly too expensive and meet-in-the-middle is also infeasible both due to the high 
number of rounds and to the huge degree of the inverse permutation (usually \(1/3 \gg 3\)).
The permutation \(x^3\) is not approximable by a linear function~~\cite{AbdelraheemABL2012}, 
hence linear attacks are not a threat, and since it can be easily shown that any arbitrary input 
difference \(\delta_{in} \) propagates to any arbitrary output difference \(\delta_{out} \) with a 
probability of at most \({2}/{2^n}\), differential attacks are also ineffective~\cite{Nyberg1994}.

On the side of algebraic cryptanalysis, one might attempt an \emph{interpolation attack}, which 
uses Lagrange interpolation to find a polynomial \(\tilde{p}\) which behaves like a keyless version 
of \(p\)~\cite{JakobsenK1997}.
This attack's complexity depends solely on \(\call{\deg}{p}\) (in fact, an interpolation can be 
computed in \(\BigO{n\call{\log}{n}}\), where \(n = \call{\deg}{p}\)~\cite{Stoss1985}), hence we 
must be sure that the degree of \(p\) also grows exponentially round by round (as it is the case).
Another kind of algebraic attack is the \emph{GCD attack}: by using two plaintext/ciphertext pairs,
once can compute their greatest common divisor which will allow to easily retrieve the secret key.
Again, computing the GCD depends almost linearly on the degree of the polynomial, hence one must 
again be sure that the degree grows exponentially.

\subsubsection*{\Mimc{} vs.\ SHA}
Performing computations over large prime fields is extremely expensive, as addition and 
multiplication cannot be performed by a single CPU intruction, but must be emulated (multiplication 
in particular is extremely costly, even with clever implementations like the Montgomery 
form~\cite{Montgomery1985}): for example, on an x86 architecture, a standard software 
implementation of SHA-256 is more than \(100\) times faster than \Mimc{} over the BN254 curve 
(implemented using \texttt{libff}\footnote{https://github.com/scipr-lab/libff}, the underlying 
arithmetic library of \texttt{libsnark}) to compress a 512 bit input to a 256 bit output.
However, when translated into an arithmetic circuit, SHA-256 requires about \(25000\) 
multiplications, while \Mimc{} requires only \(640\), making it about \(40\) times (furthermore, 
designing and optimizing the SHA-256 circuit is much harder than for \Mimc{}). 

\subsection{Poseidon}
After the design of \Mimc{}, a natural question that followed was whether it was possible to increase
the complexity of the round function and reduce the number of rounds without compromising on 
security.
In particular, researchers started exploring the design of \emph{algebraic frameworks} that could 
abstract from the details involved in the construction of a specific primitive, by still providing 
security guarantees for their instantiatios.

In~\cite{GrassiLRRS2019}, the authors present the \Hades{} design, a generalization
of the \emph{substitution-permutation network} (SPN) approach used in many famous cryptographic 
primitives such as AES~\cite{DaemenR1999}.
In fact, the \Hades{} design has the same three main steps of the classic SPN, where the 
message and the key are interpreted as vectors in \(\mathbb{F}_{p}^t\) for some prime field 
\(\mathbb{F}_p\) and some arbitrary \(t \in \mathbb{N}\):
\begin{enumerate}
  \item \emph{AddKey}: add the key to the message.
  \item \emph{SubWords}: apply a substitution function \(S\colon \mathbb{F}_{p} \to \mathbb{F}_{p}\) 
        to the elements of the message (\emph{non-linear layer}).
  \item \emph{MixLayer}: apply a permutation function \(M\colon \mathbb{F}_{p}^t \to \mathbb{F}_{p}^t\) 
        to the message (\emph{linear layer}).
\end{enumerate}
The substitution function \(S\) is defined as \(\call{S}{x} = x^d\), where \(d\) is the smallest 
integer such that \(\call{\gcd}{d, p} = 1\), while the permutation function \(M\) is defined as 
\(\call{M}{\bm{x}} = \bm{Mx}\), where \(\bm{M}\) is a \emph{maximum distance separable} (MDS) 
matrix (i.e.\ the determinant of all its submatrices are non-zero, this for example minimizes the 
effectiveness of differential attacks~\cite{MacwilliamsS1977,RijmenD1996}).

The peculiarity of \Hades{} is that only the first and last few rounds apply the substitution
function to all elements: the middle rounds are \emph{partial}, meaning that \(S\) is only applied 
to some of the elements of the message.
This allows to use the first and last part of the construction as an argument for dealing with 
classical attacks, while also considering the middle rounds when dealing with algebraic attacks.
Additionally, in the design of \Hades{}, new kind of algebraic attacks were 
considered~\cite{BeyneEtAl2020}, the most important of which is certainly the 
\emph{Gr\"{o}bner basis attack} 
(a generalization of the concept of gaussian elimination~\cite{CoxLO2015,Lazard1983}).
In a Gr\"{o}bner attack, we view the function under attack as a system of polynomial equations 
for which we want to find a solution.
This kind of attack, and in particular its probabilistic version~\cite{FaugereGHR2014}, is 
(at the moment of this writing) the most efficient known attack against algebraic constructions 
(note however that the security of \Hades{} was only proven against the deterministic 
version).

A family of hash functions that builds on the \Hades{} design is \Poseidon~\cite{GrassiKRRS2021}.
\begin{definition}[\Poseidon{} permutation]
  Given a finite field \(\mathbb{F}_p\), a number of branches \(t \in \mathbb{N}\), a number of 
  full rounds \(r_f \in \mathbb{N}\), a number of partial rounds \(r_{P} \in \mathbb{N}\), 
  an MDS matrix \(\bm{M} \in \mathbb{F}_{p}^{t \times t}\), and some constants 
  \(\bm{c}_{1}, \dots, \bm{c}_{r} \in \mathbb{F}_{p}^{t}\), where \(r = 2r_f + r_P\), let the 
  \emph{full SBOX function} be:
  \[
    \call{S_{f}}{\bm{x}}\colon \mathbb{F}_{p}^{t} \to \mathbb{F}_{p}^{t} = 
    {\begin{pmatrix}
      \bm{x}_{1}^{d} & \cdots & \bm{x}_{t}^{d}
    \end{pmatrix}}^{\transpose}
  \]
  and the \emph{partial SBOX function} be:
  \[
    \call{S_{P}}{\bm{x}}\colon \mathbb{F}_{p}^{t} \to \mathbb{F}_{p}^{t} = 
    {\begin{pmatrix}
      \bm{x}_{1}^{d} & \bm{x}_{2} & \cdots & \bm{x}_{t}
    \end{pmatrix}}^{\transpose}
  \]
  where \(d\) is the smallest prime number such that \(\call{\gcd}{x, \call{\totient}{p}} = 1\).
  Then, let the full \(i\)th round function be 
  \(\call{f_{i}}{\bm{x}}\colon \mathbb{F}_{p}^{t} \to \mathbb{F}_{p}^{t} = 
  \bm{M}\call{S_{f}}{\bm{x} + \bm{c}_{i}}\)
  and the partial \(i\)th round function be
  \(\call{P_{i}}{\bm{x}}\colon \mathbb{F}_{p}^{t} \to \mathbb{F}_{p}^{t} = 
  \bm{M}\call{S_{P}}{\bm{x} + \bm{c}_{i}}\).
  The \emph{\Poseidon{} permutation} is the function:
  \[
    P_{\Poseidon} = 
    f_{r} \compose \cdots \compose f_{r_f + r_P + 1} \compose 
    P_{r_f + r_P} \compose \cdots \compose P_{r_f + 1} \compose 
    f_{r_f} \compose \cdots \compose f_{1}
  \]
\end{definition}

The \Poseidon{} permutation can be easily used in a sponge construction to obtain a
one-way compression function.
For example, by setting \(t = 3\), we immediately obtain a \(2/1\) compression function, 
or by setting \(t = 5\) we get a \(4/1\) compression function.
For the same level of security \(s\), the \Poseidon-\(s\) arithmetic circuit is much cheaper than 
\Mimc{} in terms of multiplications. 
For reference, the \Poseidon-\(128\) \(2/1\) compression function requires \(276\) 
multiplications, compared to the \(640\) required by \Mimc{}.

Furthermore, \Poseidon{} was also designed with other systems other than the more traditional 
ZK-SNARK constructions in mind such as \(\Plonk \)~\cite{GabizonWC2019}, 
Bulletproofs~\cite{BunzBBPWM2017} and ZK-STARKs~\cite{SassonBHR2018} 
(\emph{transparent} ZK-SNARKs, i.e.\ they do not require a trusted setup phase).
For example, in the \(\Plonk \) system the cost of the ZK-SNARK construction depends also 
on the number of additions, while in ZK-STARKs it depends on the \emph{depth} of the circuit.
These systems, which are much more recent, do not have consolidated libraries which support them, 
and their analysis is out of the scope of this work.

\subsection{Griffin}
Many new designs were proposed in the last few years, such as 
\textsc{Ciminion}~\cite{DobraunigGGK2021}, which exploits \emph{Toffoli gates}~\cite{Toffoli1980} 
to improve the construction, and \Rescue~\cite{AlyABDS2019}, which uses inverse 
exponentiations (i.e.\  \(x^{1/d}\)) in the substitution layer.

An extremely recent design is \Griffin~\cite{GrassiHRSWW2022}, which deviates from the
established SPN approach by instead using an extension of the Feistel construction, called the 
\emph{\Horst{} scheme}: instead of the mapping 
\(\Tuple{x, y} \mapsto \Tuple{x, y \oplus \call{F}{x}}\) for some function \(F\) over the 
underlying field \(\mathbb{F}_p\), it uses the mapping 
\(\Tuple{x, y} \mapsto \Tuple{x, y \otimes \call{G}{x} \oplus \call{F}{x}}\), where \(G\) is an 
additional function such that \(\forall x \in \mathbb{F}_p\colon \call{G}{x} \neq 0\), in order 
to maintain invertibility (note that if \(\forall x\colon \call{G}{x} = 1\), the \Horst{} scheme
degenerates into the Feistel construction).

A quirk of \Griffin{} is the matrix used in the linear layer (which is basically the same of 
\Poseidon): it is a \(t \times t\) MDS circulant matrix, where \(t\) is the number of branches, 
but is well-defined only when \(t = 3\) or \(t \equiv 0 \pmod{4}\).

\begin{definition}[Griffin permutation]
  Given a finite field \(\mathbb{F}_p\), a number of branches \(t \in \mathbb{N}\) such that 
  \(t = 3\) or \(t \equiv 0 \pmod{4}\), a number of rounds \(r \in \mathbb{N}\), some constants 
  \(\bm{c}_{1}, \dots, \bm{c}_{r} \in \mathbb{F}_{p}^{t}\), some pairwise-distinct 
  constants \(\gamma_3, \dots, \gamma_t \in \mathbb{F}_{p}\), and some constants 
  \(\alpha_3, \beta_3, \dots, \alpha_t, \beta_t \in \mathbb{F}_{p}\) such that 
  \(\forall i \in \Iinterval*{3}{t}\colon \alpha_i \neq \beta_i\) and 
  \(\alpha_{i}^{2} - 4\beta_{i}\) is a quadratic non-residue modulo \(p\), 
  let the linear functions \(L_3, \dots, L_t\) be such that:
  \[\call{L_i}{x, y, z}\colon \mathbb{F}_p^3 \to \mathbb{F}_p = \gamma_{i}x + y + z\]
  Furthermore, let the \emph{non-linear layer} be the function:
  \[
    \call{S}{\bm{x}}_i\colon \mathbb{F}_{p}^{t} \to \mathbb{F}_{p}^{t} = \bm{y}_i =
    \begin{cases}
      x_{1}^{1/d}                                                         & i = 1         \\
      x_{2}^{d}                                                           & i = 2         \\
      x_{i}\Parens*{\call{L_i}{\bm{y}_0, \bm{y}_1, 0}^{2} + 
      \alpha_{i}\call{L_i}{\bm{y}_0, \bm{y}_1, 0} + \beta_i}              & i = 3         \\
      x_{i}\Parens*{\call{L_i}{\bm{y}_0, \bm{y}_1, \bm{x}_{i-1}}^{2} + 
      \alpha_{i}\call{L_i}{\bm{y}_0, \bm{y}_1, \bm{x}_{i-1}} + \beta_i}   & 4 \le i \le t
    \end{cases}
  \]
  where \(d\) is the smallest prime number such that \(\call{\gcd}{x, \call{\totient}{p}} = 1\).\\
  Finally, let the matrix \(\bm{M}_t \in \mathbb{F}_p^{t \times t}\) be:
  \[
    \bm{M}_t = 
    \begin{cases}
      \call{\circulant}{2, 1, 1} & t = 3 \\  
      \call{\circulant}{3, 2, 1, 1} & t = 4 \\
      \call{\diag}{\underbrace{2, \dots, 2}_{t}}
      \call{\diag}{\underbrace{\bm{M}_{4}, \dots, \bm{M}_4}_{s}}
      \call{\circulant}{\underbrace{\bm{I}_4, \dots, \bm{I}_4}_{s}}
      & t = 4s
    \end{cases}
  \]
  and the \(i\)th round function be
  \(\call{f_i}{\bm{x}}\colon \mathbb{F}_{p}^{t} \to \mathbb{F}_{p}^{t} = 
    \bm{M}_{t}\call{S}{\bm{x}} + \bm{c}_i\).
  The \emph{\Griffin{} permutation} is the function:
  \[
    P_{\Griffin} = f_{r} \compose \cdots \compose f_{1} \compose \bm{M}_{t}
  \]
\end{definition}

The quadratic non-residuosity requirement for the pairs \(\alpha_i, \beta_i\) is a condition 
imposed in order to make sure that the non-linear layer \(S\) does not output zero from the third 
branch onwards, unless the input is also zero.

Another important observation about the non-linear layer is the usage of the inverse exponentiation.
As we already observed when discussing \Mimc, usually \(1/d \gg d\), which means that the degree 
of the permutation grows much faster. 
While this also entails that computing \(x^{\frac{1}{d}}\) will be slower, from the constraint 
system point of view there is absolutely no difference: since \(y = x^{\frac{1}{d}} \iff x = y^{d}\),
one can simply write the constraints in reverse, introducing no extra cost in the R1CS and, 
in turn, in the QAP\@!

Similarly to \Poseidon, the \Griffin{} permutation can be very easily extended to a compression 
function due to the embedded sponge mechanism. The fact that \(t\) must be either \(3\) or a 
multiple of \(4\) is a bit unfortunate though, as usually one would like to use powers of \(2\) as 
input size of a compression function: except for \(2\)-to-\(1\), \Griffin{} does not allow any 
such instantiation, (when \(t = 4s\), we have a \(\Parens*{4s-1}\)-to-\(1\) compression function).

From a performance point of view, \Griffin{} is extremely competitive: for reference, the 
\(2\)-to-\(1\) instantiation only needs \(96\) R1CS constraints, compared to the \(276\) required 
by \Poseidon.
In fact, although it is a very recent design, i.e.\ it is subject to changes, \Griffin{} will be 
our target competitor in terms of pure performance in ZK-SNARK settings.

\section{\Arion{} and \ArionHash{}}\label{sec:gtds}
\subsection{The Generalized Dynamic Triangular System}
\subsection{Security of \Arion{}}

\section{Performance evaluation of \Arion}\label{sec:performance}
As we saw in \Cref{sec:gtds}, the concept of CCZ-equivalence introduced in the \Anemoi{} 
proposal~\cite{BouvierBCPSVW2022} plays an important role to build high degree permutations 
verifiable with low degree constraint systems.
In fact, it is possible to extend the concept of CCZ-equivalence by allowing any permutation.
\begin{definition}[\(\pi \)-equivalence]
  Given a vector space \(\mathbb{V}\) and two functions \(f,g\colon \mathbb{V} \to \mathbb{V}\), 
  \(f\) and \(g\) are \emph{\(\pi \)-equivalent} if there is a permutation 
  \(\pi\colon \mathbb{V}^2 \to \mathbb{V}^2\) such that \(\Gamma_{f} = \call{\pi}{\Gamma_{g}}\).
\end{definition}

Any function \(\pi\colon \mathbb{V}^2 \to \mathbb{V}^2\) can be decomposed into its 
\emph{projections} \(\pi_{\hat{x}}, \pi_{\hat{y}}\colon \mathbb{V}^2 \to \mathbb{V}\) such that:
\[
  \forall \bm{x}, \bm{y} \in \mathbb{V}\colon \call{\pi}{\bm{x}, \bm{y}} = 
    \Tuple{\call{\pi_{\hat{x}}}{\bm{x}, \bm{y}}, \call{\pi_{\hat{y}}}{\bm{x}, \bm{y}}}
\]

Thus, for any function \(f\colon \mathbb{V} \to \mathbb{V}\), we have that 
\(\call{\pi}{\bm{x}, \call{f}{\bm{x}}} = 
  \Tuple{\call{\pi_{\hat{x}}}{\bm{x}, \call{f}{\bm{x}}}, 
  \call{\pi_{\hat{y}}}{\bm{x}, \call{f}{\bm{y}}}}\).
Now, let \(\call{f_{\hat{x}}}{\bm{x}} = \call{\pi_{\hat{x}}}{\bm{x}, \call{f}{\bm{x}}}\) and 
\(\call{f_{\hat{y}}}{\bm{x}} = \call{\pi_{\hat{y}}}{\bm{x}, \call{f}{\bm{x}}}\), then 
\(\call{\pi}{\Gamma_f} = \Gamma_{f'}\) if and only if \(\pi_{\hat{x}}\) is a permutation, and 
in particular \(f' = f_{\hat{y}} \compose f_{\hat{x}}^{-1}\) is \(\pi \)-equivalent to \(f\).

Just like CCZ-equivalence, \(\pi \)-equivalence is also an equivalence relation, denoted 
\(\pieq \), which allows us to identify equiivalence classes of functions over \(\mathbb{V}\).
\begin{lemma}[\(\pi \)-equivalence of permutations]\label{lem:pi_equiv}
  All permutations over a vector space \(\mathbb{V}\) are \(\pi \)-equivalent.
\end{lemma}
\begin{proof}
  We just have to prove that, for every permutation \(f\colon \mathbb{V} \to \mathbb{V}\), it 
  is the case that \(f \pieq \fooid \), and the result will follow.
  Clearly, the function \(\call{\pi}{\bm{x}, \bm{y}}\colon \mathbb{V}^2 \to \mathbb{V}^2 = 
    \Tuple{\bm{x}, \call{f^{-1}}{\bm{y}}}\) is a permutation, and 
  \(\call{\pi}{\Gamma_f} = \Gamma_{\fooid}\).
\end{proof}

\begin{definition}[Alternative GTDS of \Arion]
  Given the same parameters as in \Cref{def:gtds}, the \emph{alternative GTDS of \Arion} is the
  function \(\tilde{F}_{GTDS}\colon \mathbb{F}_p^t \to \mathbb{F}_p^t\) such that:
  \[
    \call{\tilde{F}_{GTDS}}{\bm{x}}_i = \bm{y}_i =
    \begin{cases}
      \bm{x}_i^{d_1}\call{g_i}{\tilde{\sigma}_{i+1,t}} + 
        \call{h_i}{\tilde{\sigma}_{i+1,t}} & 1 \le i < t \\
      \bm{x}_i^{d_2} & i = t
    \end{cases}
  \]
  where \(\tilde{\tau}_{i, k} = \bm{x}_k + \bm{x}_{k}^{e} + \sum_{j=i}^{k-1}{\bm{x}_j + \bm{y}_j}\).

\end{definition}

\begin{proposition}[\(\pi \)-equivalence of GTDS]
  \(F_{GTDS} \pieq \tilde{F}_{GTDS}\).
\end{proposition}
\begin{proof}
  \(F_{GTDS}\) and \(\tilde{F}_{GTDS}\) are both permutations over the vector space \(\mathbb{F}_p^t\),
  therefore by \Cref{lem:pi_equiv} the claim follows.
\end{proof}

As a corollary, we have that verifying the constraint system of \(F_{GTDS}\) is equivalent to 
verifying the constraint system of \(\tilde{F}_{GTDS}\).
Computing the number of multiplicative constraints for \Arionhash{} is quite straightforward:
\begin{lemma}[R1CS constraints for \Arionhash]
  Given the \Arionhash{} function over a prime field \(\mathbb{F}_p\) with branch size 
  \(t\) and rate \(r\), let \(\call{\minmul}{x}\colon \mathbb{F}_p \to \mathbb{N}\) be the minimum
  number of field multiplications required to compute \(y^x\) for any \(y \in \mathbb{F}\).
  The number of R1CS constraints required by \Arionhash{} is:
  \[
    N_{\Arionhash} = 
      r\Parens*{\Parens*{n - 1}\Parens*{\call{\minmul}{d_1} + 2} + \call{\minmul}{d_2}}
  \]
\end{lemma}
\begin{proof}
  Consider the alternative GTDS \(\tilde{F}_{GTDS}\): we need \(\call{\minmul}{d_2}\) 
  multiplicative constraints in the last branch.
  In the remaining \(n - 1\) branches, we need \(\call{\minmul}{d_1}\) multiplications to 
  compute \(\bm{x}_i^{d_1}\), one multiplication for computing \(g_i\), one for computing \(h_i\), 
  and one to multiply \(g_i\) with \(\bm{x}_i^{d_1}\).
\end{proof}

For reference, the number of R1CS constraints required by \Poseidon{} and \Griffin{} 
over their respective parameters (see \Cref{def:poseidon} and \Cref{def:griffin}) are given by:
\begin{align*}
  & N_{\Poseidon} = \call{\minmul}{d}\Parens*{2tr_f + r_P} \\
  & N_{\Griffin} = 2r\Parens{\call{\minmul}{d} + t - 2}
\end{align*}

\begin{table}
  \centering
  \caption{R1CS constraint comparison over \(256\)-bit prime fields and \(128\) bits of security 
  with \(d_2 \in \Set{121, 123, 125, 161, 257}\).}\label{tab:arion_compare_muls}
  \begin{tabular}{  c c c c c c c  }
      \toprule
      \phantom{ }\(d_1\)\phantom{ } & \phantom{ }\(t\)\phantom{ } & \phantom{ }\Arionhash{}\phantom{ } & \phantom{ }\Aarionhash{}\phantom{ } & \phantom{ }\Griffin{}\phantom{ } & \phantom{ }\Anemoi{}\phantom{ } & \phantom{ }\Poseidon{}\phantom{ }             \\
      \midrule
      \multicolumn{2}{  c | }{} & \multicolumn{5}{ c }{Rounds} \\
      \midrule
      \(3\) & \(3\) & \(6\) & \(5\) & \(12\) &      & \phantom{ }\(r_f = 4,\ r_P = 84\)\phantom{ } \\
      \(5\) & \(3\) & \(6\) & \(4\) & \(12\) &      & \phantom{ }\(r_f = 4,\ r_P = 56\)\phantom{ } \\

      \(3\) & \(4\) & \(6\) & \(4\) & \(11\) & \(12\) & \phantom{ }\(r_f = 4,\ r_P = 84\)\phantom{ } \\
      \(5\) & \(4\) & \(5\) & \(4\) & \(11\) & \(12\) & \phantom{ }\(r_f = 4,\ r_P = 56\)\phantom{ } \\

      \(3\) & \(5\) & \(5\) & \(4\) &      &      & \phantom{ }\(r_f = 4,\ r_P = 84\)\phantom{ } \\
      \(5\) & \(5\) & \(5\) & \(4\) &      &      & \phantom{ }\(r_f = 4,\ r_P = 56\)\phantom{ } \\

      \(3\) & \(6\) & \(5\) & \(4\) &      & \(10\) & \phantom{ }\(r_f = 4,\ r_P = 84\)\phantom{ } \\
      \(5\) & \(6\) & \(5\) & \(4\) &      & \(10\) & \phantom{ }\(r_f = 4,\ r_P = 84\)\phantom{ } \\

      \(3\) & \(8\) & \(4\) & \(4\) & \(9\)  & \(10\) & \phantom{ }\(r_f = 4,\ r_P = 84\)\phantom{ } \\
      \(5\) & \(8\) & \(4\) & \(4\) & \(9\)  & \(10\) & \phantom{ }\(r_f = 4,\ r_P = 56\)\phantom{ } \\

      \midrule

      \multicolumn{2}{  c | }{} & \multicolumn{5}{ c  }{R1CS Constraints} \\
      \midrule

      \(3\) & \(3\) & \(102\) & \(85\)  & \(72\)  &       & \(216\) \\
      \(5\) & \(3\) & \(114\) & \(76\)  & \(96\)  &       & \(240\) \\

      \(3\) & \(4\) & \(126\) & \(84\)  & \(88\)  & \(96\)  & \(232\) \\
      \(5\) & \(4\) & \(120\) & \(96\)  & \(110\) & \(120\) & \(264\) \\

      \(3\) & \(5\) & \(120\) & \(100\) &       &       & \(248\) \\
      \(5\) & \(5\) & \(125\) & \(116\) &       &       & \(288\) \\

      \(3\) & \(6\) & \(145\) & \(116\) &       & \(120\) & \(264\) \\
      \(5\) & \(6\) & \(170\) & \(136\) &       & \(150\) & \(312\) \\

      \(3\) & \(8\) & \(148\) & \(148\) & \(144\) & \(160\) & \(296\) \\
      \(5\) & \(8\) & \(176\) & \(176\) & \(162\) & \(200\) & \(360\) \\

      \bottomrule
  \end{tabular}
\end{table}

\Cref{tab:arion_compare_muls} shows the number of constraints required for specific instantiations 
of \Arion, \Aarion, \Anemoi, \Poseidon{} and \Griffin{} over \(\approx 256\)-bit prime fields for a 
target \(128\) bits of security.

\subsection{Implementations and Experiments}
In order to experiment with the cryptographic primitives and the constructions discussed 
throughrout this thesis, we wrote the C\texttt{++} 
\texttt{zkp-hash}\footnote{\url{https://github.com/SPAstef/zkp_hash}} library, by building on
\texttt{libsnark}.
In particualar, our library is meant to improve and extend \texttt{libsnark}, by providing: 
\begin{itemize}
  \item A native and ZK-SNARK implementation of SHA-512.
  \item Native and ZK-SNARK implementation of \Mimchash-\(256\) and \Mimchash-\(512\).
  \item Templates for native and Zk-SNARK implementations of \Poseidon, \Griffin{} 
        and \Arionhash.
  \item A template for native and ZK-SNARK fixed-size Merkle tree authentication paths with 
        arbitrary arity and node size, for arbitrary hash functions.
  \item A template for native and ZK-SNARK fixed-size ABR authenticatin paths with arbitrary node 
        size over arbitrary \(2n\)/\(n\) hash functions.
  \item Additional utilities to simplify working with \texttt{libsnark}.
\end{itemize}

We would like to discuss a bit more in detail our Merkle tree implementation, both to highlight some 
important security considerations that might not be obvious at first, as well as an optimization 
for its construction, which we are not aware of being applied in any other implementation.

Let's consider, for the sake of simplicity, a binary Merkle tree over a field \(\mathbb{F}_p\) 
where each node contains only one field element. 
The circuit of the tree takes in input the value \(x\) contained in the node that the prover claims 
to hold, together with the values \(y_1, \dots, y_{h-1}\) along the alleged authentication path, 
and by using the sub-circuit (gadget) of the underlying compression function \(C\), it constrains 
the middle values \(m_1, \dots, m_h\) to be:
\begin{equation}\label{eq:bad_constraint}
  \Parens*{m_1 = x} \land
  \forall i < h\colon m_{i+1} =
  \begin{cases}
    \call{C}{m_{i}, y_{i}} & \textnormal{\(m_{i}\) is a left leaf} \\
    \call{C}{y_{i}, m_{i}} & \textnormal{\(m_{i}\) is a right leaf}
  \end{cases}
\end{equation}
where \(m_{h}\) will be the root value. 
However, there is a big issue in this naive construction: the left/right choice is made 
transparently and the verifier, by simply inspecting the circuit layout, would be able to know 
the location of the prover's node and therefore its value!
So we must \emph{embed} the choice in the circuit (in the \texttt{libsnark} implementation this is 
left to the user, which we believe to be unnecessarily complicated and error-prone).

In our construction, each node is labeled with a left-to-right bottom-to-top \(0\)-based index 
\(n\): the leftmost leaf will have index \(n = 0\) and the root will have index \(n = 2^{h} - 2\), 
where \(h\) is the height of the tree. 
The index will be an additional input variable to the circuit. 
Note that the base-\(2\) representation of the index, from the LSB to the MSB, shows the path from 
the leaf to the root: if the \(i\)th bit of \(n\) is \(0\), it means that we are on a left node, if 
it is \(1\) it means that we are on a right node.
By introducing a set of \emph{selection variables} \(s_1, \dots, s_{h-1}\), a set of 
\emph{left variables} \(l_1, \dots, l_{h-1}\) and a set of \emph{right variables} 
\(r_0, \dots, r_{h-1}\), we can then replace \Cref{eq:bad_constraint} with the following:
\begin{align*}
  & n = \sum_{i=1}^{h}{s_{i}2^{i-1}} \\
  & m_1 = x \\
  & \forall i < h\colon
  \begin{cases}  
    l_i = \Parens*{1 - s_{i}}m_{i} + s_{i}y_{i} \\ 
    r_i = s_{i}m_{i} + \Parens*{1 - s_{i}}y_{i} \\
    m_{i+1} = \call{C}{l_i, m_i}
  \end{cases}
\end{align*}
Where the \(i\)th selection variable will have to be set to the value of the \(i\)th bit of the 
index. If any of the selection variables is set to an incorrect value, the verification will 
fail, hence we have been able to hide the path from the verifier without compromising the 
correctness of the construction.

\subsubsection*{From binary to \(t\)-ary}
To work with \(t\)-ary trees of height \(h\) using the \emph{minimal amount of constraints}, we 
need to make some changes to the previous reasoning.
The authentication path will now consist of the node \(x\) and a set of \(t-1\)-dimensional 
vectors \(\bm{y}_1, \dots, \bm{y}_{h-1}\), where \(\bm{y}_i\) will contain the sibling nodes at 
the \(i\)th level of the tree, and the compression function will be \(t\)-to-\(1\).
The base-\(t\) representation of the index, from LSD to MSD, will again show the path from the leaf 
to the root, where the \(i\)th digit will tell us which of the \(t\) children we are on.
We introduce a set of \(t\)-dimensional selection variables \(\bm{s}_1, \dots, \bm{s}_{h-1}\), a
a set of \(t\)-dimensional children variables \(\bm{c}_1, \dots, \bm{c}_{h-1}\), and introduce 
the following constraints:
\begin{align*}
  & n = \sum_{i=1}^{h}{t^{i-1}\sum_{j=1}^{t}{j\bm{s}_{i,j}2^{t-1}}} \\
  & \bm{c}_{1,1} = x \land \forall i \in \Iinterval{2}{t}\colon \bm{c}_{1,i} = \bm{y}_{i-1} \\
  & \forall i < h, \forall j \le t\colon 
  \begin{cases}
    \bm{s}_{i,j}\Parens*{m_{i+1} - \bm{y}_{i,j}} = \bm{c}_{i,j} - \bm{y}_{i,j} \\
    m_{i+1} = \call{C}{\bm{c}_i}
  \end{cases}
\end{align*}

To better understand the constraints, let's focus on a single level \(i\): the vector 
\(\bm{s}_i\) will contain all zeros except in the position dictated by the \(i\)th digit of 
\(n\) when represented in base \(t\). 
In an electronic jargon, we can say that the selector blocks all the signals coming from the 
previous layer but in one position, and the dangling wires will be powered by the sibling signals 
provided by the prover.
To make sure that the prover cannot cheat, the \(i\)th digit of \(n\) will have to match the 
value of the \(i\)th bit of \(\bm{s}_i\) weighted by its position \(j\).
Further extending the scheme to allow for single nodes to contain multiple field elements is 
quite straightforward.

\subsubsection*{\Arion{} vs.\  \Griffin{} vs.\  \Poseidon{}}
\begin{table}
  \centering
  \caption{Performance of various hash functions for generating a proof of membership in a Merkle 
      tree accumulator over BN254. Proving times are in milliseconds.}\label{tab:runtimes}
  \resizebox{480pt}{!}{
      \begin{tabular}{  c  c c c  c c c  c c c  c c c  }
          \toprule
          & \multicolumn{12}{ c  }{Time (ms)} \\
          \midrule

          \phantom{ }Height\phantom{ } & & \Arionhash{} &  &  &  \Aarionhash{}  & & & \Griffin{} & & & \Poseidon{} & \\
          \midrule
          \(d = 5\) & \phantom{ }\(n = 3\) & \(n = 4\) & \(n = 8\)\phantom{ } & \phantom{ }\(n = 3\) & \(n = 4\) & \(n = 8\)\phantom{ } & \phantom{ }\(n = 3\)  & \(n = 4\)  & \(n = 8\)\phantom{ } & \phantom{ }\(n = 3\)  & \(n = 4\)  & \(n = 8\)\phantom{ }  \\
          \midrule
          \(4\)   & \(101\)  & \(103\)  & \(142\)  & \(73\)  & \(87\)  & \(143\)  & \(88\)  & \(99\) & \(133\) & \(186\)  & \(212\)  & \(274\)  \\
          \(8\)   & \(211\) & \(216\) & \(294\)  & \(145\) & \(177\) & \(294\)  & \(181\) & \(209\) & \(270\) & \(386\)  & \(417\)  & \(566\)  \\
          \(16\)  & \(392\) & \(401\) & \(554\)  & \(278\) & \(334\) & \(553\)  & \(338\) & \(387\) & \(505\) & \(745\)  & \(805\)  & \(1095\) \\
          \(32\)  & \(730\) & \(751\) & \(1046\) & \(509\) & \(646\) & \(1047\) & \(622\) & \(727\) & \(980\) & \(1422\) & \(1550\) & \(2111\) \\
          \bottomrule
      \end{tabular}
  }
\end{table}

Using our library, we compared \Arion{} with \Griffin{} and \Poseidon{}.
The comparative result of our experiments are given in \Cref{tab:runtimes}. 
All experiments were run on a system with an Intel Core i7--11800H CPU and 32GB RAM on a Clear Linux 
instance, using the \texttt{g++-12} compiler with \texttt{-O3 -march=native} flags.
Our result shows that \Arionhash{} significantly outperforms \Poseidon{} showing \(2\)x efficiency 
improvement, and \Aarionhash{} is considerably faster than \Griffin{} for the most common
real-world parameter choices (i.e.\  \(n = 3\) in Merkle tree hashing mode).

