\chapter{Cryptographic Primitives from Generalized Triangular Dynamical Systems}\label{chap:arion}
One of the most important applications of zero-knowledge verifiable computation lies in digital 
currency transactions over the blockchain infrastructure.
An example of ZK-SNARK applied in the real world is the ZCash cryptocurrency~\cite{SassonCGGMTV2014}, 
which is inspired by the more famous Bitcoin~\cite{NarayananBFMG2016}, and was devised by the 
authors of \texttt{libsnark} (which frames the zero-knowledge backend of the currency).

As we discussed in \Cref{sec:tree_hash}, the fundamental component of a blockchain is the 
Merkle tree, which uses one-way compression functions in order to produce the binding 
commitment.
In a digital currency scenario, the leaves of the Merkle tree consist of the details of some 
transaction, typical information include the ID of the sender, the ID of the recipient, and the 
amount of currency to be transferred. 
Without a zero-knowledge framework in place, when one wants to verify whether a user did abide to 
their commitment, the only possible solution is to ask the user to disclose his transaction, 
together with the authentication path, and check that the tree commitment is respected. 
When using currencies like Bitcoin or Ethereum\footnote{\url{https://ethereum.org/}}, anyone 
can see the details of every single transaction being performed on the relative 
blockchain, meaning that there is no privacy whatsoever\footnote{For example, on 
\url{https://etherscan.io/} you can see the transactions on the Ethereum blockchain. %It is 
%curious how privacy has often been foisted as a feature of mainstream cryptocurrencies while, 
%on the contrary, any bank offers much more privacy!
}.
However, if we translate the Merkle tree computation in an equivalent circuit, it is possible to 
apply a zero-knowledge scheme that allows a verifier to be sure (with overwhelming probability) 
of the validity of a transaction without actually having to see it!
Since a Merkle tree applies over and over the underlying compression function, the problem of 
creating a circuit for the former immediately reduces to the problem of creating a circuit for the 
latter.

In \Cref{sec:sota} we will review the evolution of the state of the art concerning zero-knowledge 
friendly compression functions.
Then, in \Cref{sec:gtds}, we present a new algebraic framework to represent cryptographic 
primitives, the \emph{Generalized Triangular Dynamic System}, and apply it to construct the 
\Arion{} block cipher and the \Arionhash{} hash function.
Finally, in \Cref{sec:performance}, we compare our new construction to the state of the art using 
the \texttt{libsnark} library, showing extremely competitive results.
\section{State of the art}\label{sec:sota}
The standard compression function used in Merkle trees is usually one of the SHA-2 or SHA-3 
functions~\cite{Dang2015}: this is certainly the most sensible choice in a \emph{native} 
environment, as SHA is specifically designed to be fast in both software and 
hardwaare~\cite{DaddaMO2004,MichailAKTG2012} implementations, and is the most studied hash function 
from a security standpoint (e.g.\ for SHA-2 see~\cite{KhovratovichRS2012,GuoLRW2010,DobraunigEM2016}).

However, when working with arithmetic circuits over a prime field \(\mathbb{F}_p\), SHA has a lot 
of issues: the underlying operations being performed are bitwise XOR, bitwise AND, 
bit shifts/rotations and additions modulo \(2^{32}\).
While shifts and rotations come at no cost, as they basically consist in a renaming of the circuit 
wires/variables, bitwise operations and addition which is not modulo \(p\) have to be simulated 
bit-by-bit, and the overhead introduced in such a translation is huge.
For example, for SHA-256, over a bilinear group like BN254 for which 
\(\abs{\mathbb{F}_p} \approx 2^{256}\), we would need \(256\) input variables each holding a
\(256\)-bit integer to simulate the behaviour of every single bit during the SHA computation; 
clearly, this is decisely suboptimal.

\begin{example}
  Suppose we are given two strings \(a, b \in \Set{0, 1}^{n}\), and we want to compute 
  \(a \bitxor b\).
  By interpreting them as vectors \(\bm{v}, \bm{w} \in \mathbb{F}_{p}^{n}\), we can simulate 
  bitwise XOR by computing, \(\forall i \le n\):
  \[\bm{v}_{i} \bitxor \bm{w}_{i} = \bm{v}_{i} + \bm{w}_{i} - 2\bm{v}_{i}\bm{w}_{i}\]
  that is, every XOR operation requires one multiplication gate.
  Similarly, bitwise AND and non-native addition also require multiplications to be simulated.
  Furthermore, we must guarantee that the values \(\bm{v}_i\) and \(\bm{w}_i\) are boolean, as 
  in principle they could assume any value in \(\mathbb{F}_p\), so we must also add constraints of 
  the kind \(\bm{v}_{i}\Parens*{\bm{v}_i - 1} = 0\).
\end{example}

\subsection{MiMC}
In an effort to find secure cryptographic designs that could be efficient in zero-knowledge 
settings, called \emph{zk-friendly} designs, researchers began to study the properties of 
permutations that make use of a low number of multiplications 
(\emph{multiplicative complexity})~\cite{AlbrechtRSTZ2016}.

One of the first constructions over finite fields was the \emph{Minimal Multiplicative Complexity}
(MiMC) family of cryptographic permutations~\cite{AlbrechtGRRT2016}.
The idea of MiMC, reprising an older proposal~\cite{NybergK1995}, is to use a very simple 
polynomial permutation as its core component, and by repeating it for an adequate number of rounds,
obtain a secure construction.
\begin{definition}[MiMC keyed permutation]
  Given a finite field \(\mathbb{F}_p\), a number of rounds 
  \(r = \Ceil*{\frac{\call{\log}{p}}{\call{\log}{3}}}\), some constants 
  \(c_1, \dots, c_r \in \mathbb{F}_p\) and a set of functions 
  \(f_1, \dots, f_r\colon \mathbb{F}_p \times \mathbb{F}_p \to \mathbb{F}_p\) such that 
  \(\forall i \le r\colon \call{f_i}{x, k} = x^3 + k + c_i\), the \emph{MiMC keyed permutation}
  is defined as:
  \[
    \call{E_{MiMC}}{x, k}\colon \mathbb{F}_p \times \mathbb{F}_p \to \mathbb{F}_p = 
    \call{\Parens*{f_r \compose \dots \compose f_1}}{x, k} + k
  \]
\end{definition}

The MiMC keyed permutation is also called MiMC-\(n/n\). 
By applying the Feistel construction on the MiMC permutation, one obtains the Feistel MiMC function, 
or MiMC-\(2n/n\).
Finally, by applying the sponge construction, one can obtain the MiMC hash function.
In alternative, it is also possible to build an hash function using first the Davies-Meyer 
construction to obtain a one-way compression function, and then the Merkle-Damg\"{a}rd construction
to obtain an hash function.

There are some important observations to be made on the MiMC construction.
First, the round permutation uses a low degree polynomial, but it is repeated for a high number of 
rounds: for example, if the size of the underlying field is \(\approx 2^{256}\), the number of 
rounds will be \(r = 162\). 
Note that \(x^3\) might not actually induce a permutation over \(\mathbb{F}_p\), as in general 
\(3\) is not coprime with \(\call{\totient}{p}\) (in fact, in the underlying fields of both BN254 
and BLS12, \(3\) is a factor of \(p - 1\)).
In such cases, one should modify the definition to consider the smallest prime number \(d\) such 
that \(\call{\gcd}{d, \call{\totient}{p}} = 1\), and reduce the number of rounds to
\(r = \Ceil*{\frac{\call{\log}{p}}{\call{\log}{d}}}\).

A second observation is that \(r\) must be chosen to thwart many different types of cryptanalysis 
techniques: since the MiMC permutation corresponds to the 
polynomial \(p = \Parens*{x^3 + k + c_1}\dots\Parens*{x^3 + k + c_r}\) 
(which has degree \(\call{\deg}{p} = 3^r\)), in addition to the traditional \emph{brute-force}, 
\emph{meet-in-the-middle}~\cite{DiffieH1977}, \emph{differential}~\cite{BihamS1991} and 
\emph{linear}~\cite{Matsui1994} attacks, one must also consider \emph{algebraic attacks}, 
which exploit the inherent nature of this type of constructions.

In fact, traditional attacks don't tend to pose a major threat to these kinds of constructions:
brute force is clearly too expensive and meet-in-the-middle is also infeasible both due to the high 
number of rounds and to the huge degree of the inverse permutation (usually \(1/3 \gg 3\)).
The permutation \(x^3\) is not approximable by a linear function~~\cite{AbdelraheemABL2012}, 
hence linear attacks are not a threat, and since it can be easily shown that any arbitrary input 
difference \(\delta_{in} \) propagates to any arbitrary output difference \(\delta_{out} \) with a 
probability of at most \({2}/{2^n}\), differential attacks are also ineffective~\cite{Nyberg1994}.

On the side of algebraic cryptanalysis, one might attempt an \emph{interpolation attack}, which 
uses Lagrange interpolation to find a polynomial \(\tilde{p}\) which behaves like a keyless version 
of \(p\)~\cite{JakobsenK1997}.
This attack's complexity depends solely on \(\call{\deg}{p}\) (in fact, an interpolation can be 
computed in \(\BigO{n\call{\log}{n}}\), where \(n = \call{\deg}{p}\)~\cite{Stoss1985}), hence we 
must be sure that the degree of \(p\) also grows exponentially round by round (as it is the case).
Another kind of algebraic attack is the \emph{GCD attack}: by using two plaintext/ciphertext pairs,
once can compute their greatest common divisor which will allow to easily retrieve the secret key.
Again, computing the GCD depends almost linearly on the degree of the polynomial, hence one must 
again be sure that the degree grows exponentially.

\subsubsection*{MiMC vs.\ SHA-256}

\subsection{Poseidon}
\subsection{Griffin}
\subsection{Other designs}

\section{\Arion{}: A new ZK-friendly permutation}\label{sec:gtds}
The constructions that we saw in \Cref{sec:sota} are prominent examples of different 
\emph{generations} of \emph{Arithmetization Oriented} (AO) designs.
For example, \Mimc{} is an example of a Gen-I design: its main purpose was mostly to demonstrate 
that it was indeed possible to construct secure and efficient cryptographic primitives by 
stacking simple, low-degree round functions.
On the other hand, the \Hades{} framework, its derivative \Poseidon{} and other similar 
constructions like \Rescue{} are examples of Gen-II designs: by tweaking the SPN and 
Feistel constructions, their purpose was to massively improve the efficiency over Gen-I designs.
Finally, \Griffin{} is an example of a Gen-III design: the underlying \Horst{} scheme is neither 
a ``pure'' Feistel nor SPN design and, by deviating from such standard constructions, its authors 
were able to improve the efficiency even further. 
Furthermore, in third generation designs it was shown that one does not necessarily need to use 
round functions with a low degree, as long as the resulting constraint system is not affected 
negatively\footnote{As it is often the case in research, the separation line is a bit blurry, 
as \Rescue, which we said to be a Gen-II design, already used inverse exponentiations}.

Other from \Griffin, there is another very recent Gen-III construction, called 
\Anemoi~\cite{BouvierBCPSVW2022} and based on the \Flystel{} design, which has some common points 
with the \Horst{} construction.
A very interesting fact used in \Flystel{} was the notion of CCZ-equivalence~\cite{CarletCZ1998},
a concept which generalizes the intuition behind the idea that there is ``no difference'' 
between, say, using \(x^{d}\) and \(x^{\frac{1}{d}}\).
\begin{definition}[Affine function]
  An \emph{affine function} over an \(n\)-dimensional vector space \(\mathbb{F}^n\) is a function
  \(\call{f}{\bm{x}}\colon \mathbb{F}^n \to \mathbb{F}^n = \bm{Mx} + q\), where 
  \(\bm{M} \in \mathbb{F}^{n \times n}\) and \(q \in \mathbb{F}^n\).
\end{definition}
\begin{definition}[Function graph]
  Given a set \(S\), the \emph{function graph} of a function \(f\colon S \to S\) is the pair 
  \(\Gamma = \Tuple{S, E}\) where \(E = \Set{\Tuple{x, \call{f}{x}} \mid x \in S}\).
\end{definition}
\begin{definition}[Induced permutation]
  Given a \emph{function graph} \(\Gamma = \Tuple{S, E}\) and a permutation \(P\colon S^2 \to S^2\), 
  the \emph{induced permutation} of \(\Gamma \) by \(P\) is the function graph 
  \(\call{P}{\Gamma} = \Tuple{S, E'}\) where \(E' = \Set{\call{P}{e} \mid e \in E}\).
\end{definition}
\begin{definition}[CCZ equivalence~\cite{BouvierBCPSVW2022,CarletCZ1998}]
  Given a vector space \(\mathbb{V}\) and two functions 
  \(f,g\colon \mathbb{V} \to \mathbb{V}\), \(f\) and \(g\) are \emph{CCZ-equivalent} if there 
  is an affine permutation \(L\colon \mathbb{V}^2 \to \mathbb{V}^2\) such 
  that \(\Gamma_{f} = \call{L}{\Gamma_{g}}\).
\end{definition}

Clearly, CCZ-equivalence is an equivalence relation over a vector space \(\mathbb{V}\), hence it 
induces a partitioning of \(\mathbb{V}^2\) into equivalence classes.
An interesting fact is that all CCZ-equivalent functions share the same linear and differential 
properties.
Even more importantly for our purposes is that CCZ-equivalent functions are indistinguishable under 
constraint verification: checking the constraint system of any member of the class verifies 
the validity of the computation of any other member.
However, one thing that CCZ-equivalent functions do not share in general is their degree, hence we 
can use the one with the highest degree in the actual computation to provide the most strict security 
guarantees against algebraic attacks, while using the one with the lowest degree when building
the constraint system for the verification in the SNARK framework.

In a high-level, intuitive way we can say that CCZ-equivalence allows us to ``ignore the order'' 
in which the witnesses of a certain computation are obtained: reprising our usual example,
the verifier does not care that the prover first must know \(x\) in order to obtain \(y = x^{1/d}\), 
all it matters is that the prover knows both of them and that their relationship is correct.
In fact, even more generally, there is no way to know in which order someone actually got hold of 
the intermediate values of a computation, hence we might as well exploit this to our advantage in 
order to reduce the complexity of the protocol.

\subsection{The Generalized Triangular Dynamical System}
As we said, the design of third generation AO cryptographic primitives diverts from the plain 
SPN or Feistel constructions.
For this reason, we introduce the \emph{Generalized Triangular Dynamical System}~\cite{RoyS2022}, 
GTDS for short, an algebraic framework which generalizes many previous designs (such as Feistel, 
SPN, \Horst{}, \dots) and their instantiations (\Mimc{}, \Poseidon{}, \Griffin{}, \dots), and enables 
us to provide a systematic security analysis of the constructions derived from it.
In particular:
\begin{itemize}
  \item The input is split in branches like in earlier designs.
  \item The round function offers the strength of all the incorporated designs.
  \item It is secure against classical attacks like differential cryptanalysis.
  \item It is secure against interpolation attacks already at the first round.
  \item Its linear layer mixes all the branches through a circulant matrix with no zero entries.  
  \item It uses inverse exponents, but decouples them from the direct exponent. 
\end{itemize}

\begin{definition}[GTDS of \Arion~\cite{RoyS2022}]\label{def:gtds}
  Given a prime field \(\mathbb{F}_p\), a number of branches \(t \in \mathbb{N}\), the smallest 
  integer \(d_1\) such that \(\call{\gcd}{d_1, p - 1} = 1\), an arbitrary 
  integer \(d_2\) such that \(\call{\gcd}{d_2, p - 1} = 1\), some constants 
  \(\alpha_{1}, \beta_{1}, \gamma_1, \dots, \alpha_{t - 1}, \beta_{t - 1}, \gamma_{t - 1} \in \mathbb{F}_p\) 
  such that \(\forall i < t\colon \alpha_i^2 - 4\beta_i\) is a quadratic non-residue modulo 
  \(p\), let \(e = {1}/{d_2}\) and, for all \(i < t\) let:
  \begin{align*}
    & \call{g_i}{x}\colon \mathbb{F}_p \to \mathbb{F}_p = x^2 + \alpha_{i}x + \beta_{i} \\
    & \call{h_i}{x}\colon \mathbb{F}_p \to \mathbb{F}_p = x^2 + \gamma_{i}x
  \end{align*}
  Then, the GTDS of \Arion{} is the function 
  \(\call{F_{GTDS}}{\bm{x}}\colon \mathbb{F}_p^t \to \mathbb{F}_p^t\) such that:
  \[
    \call{F_{GTDS}}{\bm{x}}_i = \bm{y}_i = 
    \begin{cases}
      \bm{x}_i^{d_1}\call{g_i}{\sigma_{i+1, t}} + \call{h_i}{\sigma_{i+1, t}} & 1 \le i < t \\
      \bm{x}_i^e & i = t
    \end{cases}
  \]
  where \(\sigma_{i, k} = \sum_{j=i}^{k}{\bm{x}_j + \bm{y}_j}\).
\end{definition}

It can be shown~\cite{RoyS2022} that the GTDS is function is invertible.
An interesting detail which came up only in the later phases of the GTDS design is the decoupling 
of the inverse exponentiation from the direct exponentiation, in the sense that, instead of 
using \(x^d\) and \(x^{1/d}\), we use \(x^{d_1}\) and \(x^{1/d_2}\).
The rationale of this choice is that some security considerations about cryptographic constructions
over the GTDS depend on the size of \(d_2\): if it is too small, we would need more rounds to 
achieve the desired level of security, hence increasing the circuit complexity.
If \(d_2\) were to be equal to \(d_1\), the reduction in terms of number of rounds would be 
overcompensated by the increase of the complexity of a single round, but since \(d_2\) is only 
used in the last branch, the trade-off becomes more convenient, especially for bigger branch sizes.

Of course, \(d_2\) should require an optimal number of constraints.
Note that, since we have at our disposal intermediate results, the optimal number of operations 
required to exponentiate a number is not determined by the classical \emph{binary exponentiation} 
algorithm~\cite{Gueron2011}, but rather by \emph{addition chains}~\cite{BosC1990}.

\subsection{\Arion{} and \Arionhash{}}
Depending on our security and efficiency needs, we can instantiate the GTDS in many 
ways.
We design \Arion{} and \Arionhash{}~\textbf{\cite{RoyST2023}} to work over fields of size 
\(\approx 2^{256}\).
In order to achieve a \emph{degree overflow} in the first round, it can be shown that \(4e\), where 
\(e = {1}/{d_2}\), should be greater than \(p\). 
For BN254 and BLS12, this can be achieved by \(d_2 \in \Set{121, 123, 125, 161, 257}\).
For example, the optimal way to compute \(x^{121}\) is:
\begin{align*}
  & y = \Parens*{x^{2}}^{2} && z = \Parens*{y^{2}y}^{2} && x^{121} = \Parens*{z^{2}}^{2}zx
\end{align*}
It is not hard to see that numbers of the type \(2^k + 1\), for \(k \in \mathbb{N}\), are the 
most efficient to compute, and \(257\) is a particularly attractive candidate as it is also a 
prime number, and requires the same number of multiplications to be computed as the other candidates, 
(unfortunately, \(129 = 43 \cdot 3\) is not invertible neither in BN254 nor in BLS12).

To introduce mixing between the various branches, we use an \emph{affine layer} which employs a 
circulant matrix that has no zero entries and that is efficiently computable.
\begin{definition}[Affine layer of \Arion]
  The \emph{affine layer} of \Arion{} over a vector space \(\mathbb{F}_p^t\) is the function:
  \[\call{L}{\bm{x}, \bm{c}}\colon \Parens*{\mathbb{F}_p^n}^2 \to \mathbb{F}_p^n = 
  \call{\circulant}{1, \dots, t}\bm{x} + \bm{c}\]
\end{definition}

It is worth noting that the very simple matrix \(\call{\circulant}{1, \dots, t}\) is an MDS matrix
for any prime field \(\mathbb{F}_p\) such that \(p \ge 2^{39}\) and for values of
\(t \in \Iinterval{2}{12}\).
Furthermore, computing the matrix-vector product using \(\call{\circulant}{1, \dots, t}\) can be 
done in \(\BigO{t}\) time instead of the typical \(\BigO{t^2}\) required by a standard matrix-vector 
multiplication algorithm, by using \Cref{alg:circ_mult}.
\begin{algorithm}
  \begin{algorithmic}
    \Function{circ\_mul}{$\bm{v} \in \mathbb{F}_p^t$} %chktex 46
    \State{\(\bm{w} \gets \bm{0} \in \mathbb{F}_p^t\)}
    \State{\(\sigma \gets \sum_{i=1}^{t}{\bm{v}_i}\)}
    \State{\(\bm{w}_1 \gets \sigma + \sum_{i=1}^{t}{\Parens*{i - 1}\bm{v}_i}\)}
    \For{\(i \in \Iinterval{2}{t}\)}
    \State{\(\bm{w}_i \gets \bm{w}_{i-1} - \sigma + n\bm{v}_{i-1}\)}
    \EndFor{}
    \State{\Return{\(\bm{w}\)}}
    \EndFunction{}
  \end{algorithmic}
  \caption{Efficient evaluation of the matrix-vector product with 
    \(\call{\circulant}{1, \dots, t}\)}\label{alg:circ_mult}
\end{algorithm}

\begin{definition}[\Arion{} keyed permutation~\textbf{\cite{RoyST2023}}]
  Given a prime field \(\mathbb{F}_p\), a number of branches \(t \in \mathbb{N}\), a number 
  of rounds \(r \in \mathbb{N}\), some constants \(\bm{c}_1, \dots, \bm{c}_r \in \mathbb{F}_p^t\), 
  the \emph{\Arion{} keyed permutation} is the function \(\Arion = \Arion_r\), where:
  \[
    \call{\Arion_i}{\bm{x}, \bm{k}_0, \dots, \bm{k}_r}\colon 
      \Parens*{\mathbb{F}_p^t}^{r+2} \to \mathbb{F}_p^t = \bm{y}_i =
      \begin{cases}
        \call{L}{\bm{x}, \bm{0}} + \bm{k}_i & i = 0 \\
        \call{L}{\call{F_{GTDS}}{\bm{x}}, \bm{c}_i} + \bm{k}_i & 1 \le i \le r
      \end{cases}
  \]
\end{definition}

\begin{definition}[\Arionp{} unkeyed permutation]
  The \emph{\Arion{} unkeyed permutation} is the function:
  \[
    \call{\Arionp}{\bm{x}}\colon \mathbb{F}_p^t \to \mathbb{F}_p^t = 
      \call{\Arion}{\bm{x}, \bm{0}, \dots, \bm{0}}
  \]
\end{definition}

We can instantiate the hash function \Arionhash{} by applying the sponge construction to 
the \Arionp{} permutation.
Since it has been shown that, for a pseudorandom permutation \(P\) over a vector space 
\(\mathbb{F}_p^{t}\), a rate \(r\) and a capacity 
\(c\) such that \(t = r + c\), the sponge construction is indifferentiable from a random 
distribution up to \(\call{\min}{p^r, p^{c/2}}\) queries~\cite{BertoniDPV2008}, to provide 
\(\kappa \) bits of security, we must require that \(r \ge {\kappa}/{\call{\log}{p}}\) and 
\(c \ge {2\kappa}/{\call{\log}{p}}\).

As a padding scheme for a message \(m \in \Set{0, 1}^*\) whose length is not a multiple of the rate 
\(r\), we use \(\call{\Pad}{m} = m \concat 0^{\Parens*{-\abs{m} \bmod r}}\), and we replace the 
initial value \(v = 0^{\Parens*{t\abs{\Encode{p}}}}\) (where \(\Encode{x}\) denotes the binary 
encoding of an object \(x\)) with 
\(v' = \Encode{\abs{m}} \concat 0^{\Parens*{t-1}\abs{\Encode{p}}}\).
Note that we assume,  as basically any constructions does, that \(\abs{m} < p\), 
which should not be a problem as typically \(p \approx 2^{256}\), and we don't expect to hash 
messages of length \(\abs{m} > 2^{256}\).

\begin{table}
  \centering
  \caption{Instantiation parameters of \Arion{} and \Aarion{} for \(128\) bits of security and 
    primes \(p \geq 2^{60}\).}\label{tab:arion_instantiation}
  \begin{tabular}[t]{  c  c  c  c  }
      \toprule

      \phantom{ }\(d_1\)\phantom{ } & \phantom{ }\(t\)\phantom{ } & \phantom{ }\Arion{} \phantom{ } & \phantom{ }\Aarion{} \phantom{ } \\
      \midrule
      \multicolumn{2}{  c  }{} & \multicolumn{2}{ c  }{\phantom{ }Rounds\phantom{ }} \\
      \midrule
      \(3\) & \(3\) & \(6\) & \(5\) \\
      \(5\) & \(3\) & \(6\) & \(4\) \\

      \(3\) & \(4\) & \(6\) & \(4\) \\
      \(5\) & \(4\) & \(5\) & \(4\) \\

      \(3\) & \(5\) & \(5\) & \(4\) \\
      \(5\) & \(5\) & \(5\) & \(4\) \\

      \(3\) & \(6\) & \(5\) & \(4\) \\
      \(5\) & \(6\) & \(5\) & \(4\) \\

      \(3\) & \(8\) & \(4\) & \(4\) \\
      \(5\) & \(8\) & \(4\) & \(4\) \\
      \bottomrule
  \end{tabular}
\end{table}

In \Cref{tab:arion_instantiation} you can find the parameters for \Arion{} and 
its aggressive variant \Aarion{}, the parameters for the respective hash functions are respectively 
the same.
The aggressive variants have been parametrized in a way to provide the desired security level 
against all attacks but probabilistic Gr\"{o}bner basis attacks.
The choice to provide them anyway was dictated by the fact that, to the best of our knowledge, none 
of the competitor designs has been proved secure against such attacks, hence the aggressive 
versions provide the same guarantees in terms of security as the current state of the art.
For a detailed security analysis of \Arion, \Arionhash{} and their aggressive variants, refer 
to~\textbf{\cite{RoyST2023}}.

\section{Performance evaluation of \Arion}\label{sec:performance}
As we saw in \Cref{sec:gtds}, the concept of CCZ-equivalence introduced in the \Anemoi{} 
proposal~\cite{BouvierBCPSVW2022} plays an important role to build high degree permutations 
verifiable with low degree constraint systems.
In fact, it is possible to extend the concept of CCZ-equivalence by allowing any permutation.
\begin{definition}[\(\pi \)-equivalence]
  Given a vector space \(\mathbb{V}\) and two functions \(f,g\colon \mathbb{V} \to \mathbb{V}\), 
  \(f\) and \(g\) are \emph{\(\pi \)-equivalent} if there is a permutation 
  \(\pi\colon \mathbb{V}^2 \to \mathbb{V}^2\) such that \(\Gamma_{f} = \call{\pi}{\Gamma_{g}}\).
\end{definition}

Any function \(\pi\colon \mathbb{V}^2 \to \mathbb{V}^2\) can be decomposed into its 
\emph{projections} \(\pi_{\hat{x}}, \pi_{\hat{y}}\colon \mathbb{V}^2 \to \mathbb{V}\) such that:
\[
  \forall \bm{x}, \bm{y} \in \mathbb{V}\colon \call{\pi}{\bm{x}, \bm{y}} = 
    \Tuple{\call{\pi_{\hat{x}}}{\bm{x}, \bm{y}}, \call{\pi_{\hat{y}}}{\bm{x}, \bm{y}}}
\]

Thus, for any function \(f\colon \mathbb{V} \to \mathbb{V}\), we have that 
\(\call{\pi}{\bm{x}, \call{f}{\bm{x}}} = 
  \Tuple{\call{\pi_{\hat{x}}}{\bm{x}, \call{f}{\bm{x}}}, 
  \call{\pi_{\hat{y}}}{\bm{x}, \call{f}{\bm{y}}}}\).
Now, let \(\call{f_{\hat{x}}}{\bm{x}} = \call{\pi_{\hat{x}}}{\bm{x}, \call{f}{\bm{x}}}\) and 
\(\call{f_{\hat{y}}}{\bm{x}} = \call{\pi_{\hat{y}}}{\bm{x}, \call{f}{\bm{x}}}\), then 
\(\call{\pi}{\Gamma_f} = \Gamma_{f'}\) if and only if \(\pi_{\hat{x}}\) is a permutation, and 
in particular \(f' = f_{\hat{y}} \compose f_{\hat{x}}^{-1}\) is \(\pi \)-equivalent to \(f\).

Just like CCZ-equivalence, \(\pi \)-equivalence is also an equivalence relation, denoted 
\(\pieq \), which allows us to identify equiivalence classes of functions over \(\mathbb{V}\).
\begin{lemma}[\(\pi \)-equivalence of permutations]\label{lem:pi_equiv}
  All permutations over a vector space \(\mathbb{V}\) are \(\pi \)-equivalent.
\end{lemma}
\begin{proof}
  We just have to prove that, for every permutation \(f\colon \mathbb{V} \to \mathbb{V}\), it 
  is the case that \(f \pieq \fooid \), and the result will follow.
  Clearly, the function \(\call{\pi}{\bm{x}, \bm{y}}\colon \mathbb{V}^2 \to \mathbb{V}^2 = 
    \Tuple{\bm{x}, \call{f^{-1}}{\bm{y}}}\) is a permutation, and 
  \(\call{\pi}{\Gamma_f} = \Gamma_{\fooid}\).
\end{proof}

\begin{definition}[Alternative GTDS of \Arion]
  Given the same parameters as in \Cref{def:gtds}, the \emph{alternative GTDS of \Arion} is the
  function \(\tilde{F}_{GTDS}\colon \mathbb{F}_p^t \to \mathbb{F}_p^t\) such that:
  \[
    \call{\tilde{F}_{GTDS}}{\bm{x}}_i = \bm{y}_i =
    \begin{cases}
      \bm{x}_i^{d_1}\call{g_i}{\tilde{\sigma}_{i+1,t}} + 
        \call{h_i}{\tilde{\sigma}_{i+1,t}} & 1 \le i < t \\
      \bm{x}_i^{d_2} & i = t
    \end{cases}
  \]
  where \(\tilde{\tau}_{i, k} = \bm{x}_k + \bm{x}_{k}^{e} + \sum_{j=i}^{k-1}{\bm{x}_j + \bm{y}_j}\).

\end{definition}

\begin{proposition}[\(\pi \)-equivalence of GTDS]
  \(F_{GTDS} \pieq \tilde{F}_{GTDS}\).
\end{proposition}
\begin{proof}
  \(F_{GTDS}\) and \(\tilde{F}_{GTDS}\) are both permutations over the vector space \(\mathbb{F}_p^t\),
  therefore by \Cref{lem:pi_equiv} the claim follows.
\end{proof}

As a corollary, we have that verifying the constraint system of \(F_{GTDS}\) is equivalent to 
verifying the constraint system of \(\tilde{F}_{GTDS}\).
Computing the number of multiplicative constraints for \Arionhash{} is quite straightforward:
\begin{lemma}[R1CS constraints for \Arionhash]
  Given the \Arionhash{} function over a prime field \(\mathbb{F}_p\) with branch size 
  \(t\) and rate \(r\), let \(\call{\minmul}{x}\colon \mathbb{F}_p \to \mathbb{N}\) be the minimum
  number of field multiplications required to compute \(y^x\) for any \(y \in \mathbb{F}\).
  The number of R1CS constraints required by \Arionhash{} is:
  \[
    N_{\Arionhash} = 
      r\Parens*{\Parens*{n - 1}\Parens*{\call{\minmul}{d_1} + 2} + \call{\minmul}{d_2}}
  \]
\end{lemma}
\begin{proof}
  Consider the alternative GTDS \(\tilde{F}_{GTDS}\): we need \(\call{\minmul}{d_2}\) 
  multiplicative constraints in the last branch.
  In the remaining \(n - 1\) branches, we need \(\call{\minmul}{d_1}\) multiplications to 
  compute \(\bm{x}_i^{d_1}\), one multiplication for computing \(g_i\), one for computing \(h_i\), 
  and one to multiply \(g_i\) with \(\bm{x}_i^{d_1}\).
\end{proof}

For reference, the number of R1CS constraints required by \Poseidon{} and \Griffin{} 
over their respective parameters (see \Cref{def:poseidon} and \Cref{def:griffin}) are given by:
\begin{align*}
  & N_{\Poseidon} = \call{\minmul}{d}\Parens*{2tr_f + r_P} \\
  & N_{\Griffin} = 2r\Parens{\call{\minmul}{d} + t - 2}
\end{align*}

\begin{table}
  \centering
  \caption{R1CS constraint comparison over \(256\)-bit prime fields and \(128\) bits of security 
  with \(d_2 \in \Set{121, 123, 125, 161, 257}\).}\label{tab:arion_compare_muls}
  \begin{tabular}{  c c c c c c c  }
      \toprule
      \phantom{ }\(d_1\)\phantom{ } & \phantom{ }\(t\)\phantom{ } & \phantom{ }\Arionhash{}\phantom{ } & \phantom{ }\Aarionhash{}\phantom{ } & \phantom{ }\Griffin{}\phantom{ } & \phantom{ }\Anemoi{}\phantom{ } & \phantom{ }\Poseidon{}\phantom{ }             \\
      \midrule
      \multicolumn{2}{  c | }{} & \multicolumn{5}{ c }{Rounds} \\
      \midrule
      \(3\) & \(3\) & \(6\) & \(5\) & \(12\) &      & \phantom{ }\(r_f = 4,\ r_P = 84\)\phantom{ } \\
      \(5\) & \(3\) & \(6\) & \(4\) & \(12\) &      & \phantom{ }\(r_f = 4,\ r_P = 56\)\phantom{ } \\

      \(3\) & \(4\) & \(6\) & \(4\) & \(11\) & \(12\) & \phantom{ }\(r_f = 4,\ r_P = 84\)\phantom{ } \\
      \(5\) & \(4\) & \(5\) & \(4\) & \(11\) & \(12\) & \phantom{ }\(r_f = 4,\ r_P = 56\)\phantom{ } \\

      \(3\) & \(5\) & \(5\) & \(4\) &      &      & \phantom{ }\(r_f = 4,\ r_P = 84\)\phantom{ } \\
      \(5\) & \(5\) & \(5\) & \(4\) &      &      & \phantom{ }\(r_f = 4,\ r_P = 56\)\phantom{ } \\

      \(3\) & \(6\) & \(5\) & \(4\) &      & \(10\) & \phantom{ }\(r_f = 4,\ r_P = 84\)\phantom{ } \\
      \(5\) & \(6\) & \(5\) & \(4\) &      & \(10\) & \phantom{ }\(r_f = 4,\ r_P = 84\)\phantom{ } \\

      \(3\) & \(8\) & \(4\) & \(4\) & \(9\)  & \(10\) & \phantom{ }\(r_f = 4,\ r_P = 84\)\phantom{ } \\
      \(5\) & \(8\) & \(4\) & \(4\) & \(9\)  & \(10\) & \phantom{ }\(r_f = 4,\ r_P = 56\)\phantom{ } \\

      \midrule

      \multicolumn{2}{  c | }{} & \multicolumn{5}{ c  }{R1CS Constraints} \\
      \midrule

      \(3\) & \(3\) & \(102\) & \(85\)  & \(72\)  &       & \(216\) \\
      \(5\) & \(3\) & \(114\) & \(76\)  & \(96\)  &       & \(240\) \\

      \(3\) & \(4\) & \(126\) & \(84\)  & \(88\)  & \(96\)  & \(232\) \\
      \(5\) & \(4\) & \(120\) & \(96\)  & \(110\) & \(120\) & \(264\) \\

      \(3\) & \(5\) & \(120\) & \(100\) &       &       & \(248\) \\
      \(5\) & \(5\) & \(125\) & \(116\) &       &       & \(288\) \\

      \(3\) & \(6\) & \(145\) & \(116\) &       & \(120\) & \(264\) \\
      \(5\) & \(6\) & \(170\) & \(136\) &       & \(150\) & \(312\) \\

      \(3\) & \(8\) & \(148\) & \(148\) & \(144\) & \(160\) & \(296\) \\
      \(5\) & \(8\) & \(176\) & \(176\) & \(162\) & \(200\) & \(360\) \\

      \bottomrule
  \end{tabular}
\end{table}

\Cref{tab:arion_compare_muls} shows the number of constraints required for specific instantiations 
of \Arion, \Aarion, \Anemoi, \Poseidon{} and \Griffin{} over \(\approx 256\)-bit prime fields for a 
target \(128\) bits of security.

\subsection{\texttt{libsnark} implementation and experiments}


