%% Le lingue utilizzate, che verranno passate come opzioni al pacchetto babel. Come sempre, l'ultima indicata sar� quella primaria.
%% Se si utilizzano una o pi� lingue diverse da "italian" o "english", leggere le istruzioni in fondo.
\def\thudbabelopt{english}
%% Valori ammessi per target: bach (tesi triennale), mst (tesi magistrale), phd (tesi di dottorato).
\documentclass[target=mst]{thud}

%% --- Informazioni sulla tesi ---
%% Per tutti i tipi di tesi
% Scommentare quello di interesse, o mettete quello che vi pare
\course{Artificial Intelligence and Cybersecurity}
%\course{Internet of Things, Big Data e Web}
%\course{Matematica}
%\course{Comunicazione Multimediale e Tecnologie dell'Informazione}
\title{Zero-Knowledge friendly cryptographic permutation: theory and implementation}
\author{Stefano Trevisani}
\supervisor{Dr.\ Arnab Roy \and Prof.\ Alberto Policriti}
\cosupervisor{Prof.\ Elisabeth Oswald}
\tutor{Msc. Matthias Steiner}
%% Campi obbligatori: \title, \author e \course.
%% Altri campi disponibili: \reviewer, \tutor, \chair, \date (anno accademico, calcolato in automatico), \rights
%% Con \supervisor, \cosupervisor, \reviewer e \tutor si possono indicare pi� nomi separati da \and.
%% Per le sole tesi di dottorato:
%\phdnumber{313}
%\cycle{XXIX}
%\contacts{Via della Sintassi Astratta, 0/1\\65536 Gigatera --- Italia\\+39 0123 456789\\\texttt{http://www.example.com}\\\texttt{inbox@example.com}}

%% --- Pacchetti consigliati ---
%% pdfx: per generare il PDF/A per l'archiviazione. Necessario solo per la versione finale
%\usepackage[a-1b]{pdfx}
%% hyperref: Regola le impostazioni della creazione del PDF... pi� tante altre cose. Ricordarsi di usare l'opzione pdfa.
%% tocbibind: Inserisce nell'indice anche la lista delle figure, la bibliografia, ecc.
\usepackage{amsmath}
\usepackage{amssymb}
\usepackage{amsthm}
\usepackage{bm}
\usepackage{booktabs}
\usepackage[justification=centering]{caption}
\usepackage{comment}
\usepackage[inline]{enumitem}
\usepackage[T1]{fontenc}
\usepackage[pdfa]{hyperref}
\usepackage{cleveref}
\usepackage{lmodern}
%\usepackage{mathematica}
%\usepackage[frak=euler]{mathalpha}
\usepackage{mathtools}
\usepackage{mleftright}
\usepackage{stmaryrd}
\usepackage{subfig}
\usepackage{tikz}
\usetikzlibrary{automata,positioning,shapes.multipart}

\mleftright{}
\hypersetup{hidelinks}

\theoremstyle{definition}
\newtheorem{definition}{Definition}[chapter]
\newtheorem{example}{Example}[chapter]

\theoremstyle{plain}
\newtheorem{theorem}{Theorem}[chapter]
\newtheorem{lemma}{Lemma}[chapter]
\newtheorem{corollary}{Corollary}[chapter]
\newtheorem{proposition}{Proposition}[chapter]

\theoremstyle{remark}
\newtheorem{remark}{Remark}[chapter]
%\usepackage[export]{adjustbox}

%% --- Stili di pagina disponibili (comando \pagestyle) ---
%% sfbig (predefinito): Apertura delle parti e dei capitoli col numero grande; titoli delle parti e dei capitoli e intestazioni di pagina in sans serif.
%% big: Come "sfbig", solo serif.
%% plain: Apertura delle parti e dei capitoli tradizionali di LaTeX; intestazioni di pagina come "big".

\newcommand{\range}[2]{{{#1}\dots{#2}}}
\newcommand{\overbar}[1]{\mkern 1.5mu\overline{\mkern-1.5mu#1\mkern-1.5mu}\mkern 1.5mu} %chktex 1

\makeatletter
\newcommand*{\transpose}{{\mathpalette\@transpose{}}} %chktex 21
\newcommand*{\@transpose}[2]{\raisebox{\depth}{\(\m@th#1\intercal \)}}

\DeclarePairedDelimiter{\Pipes}{\lvert}{\rvert}
\DeclarePairedDelimiter{\Angular}{\langle}{\rangle}
\DeclarePairedDelimiter{\Braces}{\{}{\}} %chktex 21
\DeclarePairedDelimiter{\Brackets}{[}{]} %chktex 9
\DeclarePairedDelimiter{\Parens}{(}{)}   %chktex 9

\newcommand{\Set}[1]{\Braces*{#1}}
\newcommand{\Tuple}[1]{\Parens*{#1}}
\newcommand{\Encode}[1]{\Angular*{#1}}
\newcommand{\call}[2]{#1\Parens*{#2}}
\newcommand{\extend}[2]{#1\Brackets*{#2}}
\newcommand{\abs}[1]{\Pipes*{#1}}
\newcommand{\explicit}[1]{\widehat{#1}}
\newcommand{\id}{\mathrm{e}}
\newcommand{\inv}[1]{\overbar{#1}}
\newcommand{\fooid}{\mathrm{id}}
\newcommand{\gengroup}[1]{\left\langle#1\right\rangle}
\newcommand{\bitand}{\mathbin{\textnormal{\textsc{and}}}}
\newcommand{\bitxor}{\mathbin{\textnormal{\textsc{xor}}}}
\newcommand{\Endset}[1]{\mathrm{End}\left(#1\right)}


\newcommand{\BigO}[1]{\call{\mathcal{O}}{#1}}
\newcommand{\SmallO}[1]{\call{o}{#1}}
\newcommand{\BigOmega}[1]{\call{\Omega}{#1}}
\newcommand{\SmallOmega}[1]{\call{\omega}{#1}}
\newcommand{\BigTheta}[1]{\call{\Theta}{#1}}
\newcommand{\Complexity}{\mathsf{C}}
\newcommand{\Time}{\mathsf{T}}
\newcommand{\Space}{\mathsf{S}}


\newcommand{\then}{\Rightarrow}

\newcommand{\Ptime}{\textnormal{\textsc{P}}}
\newcommand{\NPtime}{\textnormal{\textsc{NP}}}
\newcommand{\NPcomplete}{\textnormal{\textsc{NP-complete}}}

\newcommand{\Stamp}{\texttt{Stamp}}
\newcommand{\Blocc}{\texttt{Blocc}}

\newcommand{\Tape}{\mathfrak{T}}
\newcommand{\Tapework}{\mathfrak{W}}
\newcommand{\Tapein}{\mathfrak{I}}
\newcommand{\Tapeout}{\mathfrak{O}}
\newcommand{\Taperand}{\mathfrak{P}}
\newcommand{\Tapesend}{\mathfrak{S}}
\newcommand{\Taperec}{\mathfrak{R}}

\newcommand{\maybeequals}{\mathrel{\overset{?}{=}}}


\begin{document}
\maketitle

%% Dedica (opzionale)
%\begin{dedication}
%	To <somebody\_special>,\par for being with me all the way through.
%\end{dedication}

%% Ringraziamenti (opzionali)
%\acknowledgements{
%Sed vel lorem a arcu faucibus aliquet eu semper tortor. 
%Aliquam dolor lacus, semper vitae ligula sed, blandit iaculis leo. 
%Nam pharetra lobortis leo nec auctor. 
%Pellentesque habitant morbi tristique senectus et netus et malesuada fames ac turpis egestas. 
%Fusce ac risus pulvinar, congue eros non, interdum metus. 
%Mauris tincidunt neque et aliquam imperdiet. 
%Aenean ac tellus id nibh pellentesque pulvinar ut eu lacus. 
%Proin tempor facilisis tortor, et hendrerit purus commodo laoreet. 
%Quisque sed augue id ligula consectetur adipiscing. 
%Vestibulum libero metus, lacinia ac vestibulum eu, varius non arcu. 
%Nam et gravida velit.
%}

%% Sommario (opzionale)
\abstract{
	Zero Knowledge (ZK) proof systems have been an increasingly studied subject in the last 40 years.
	In the last decade, the efficiency of the proposed frameworks, along with the processing power of 
	computing devices, has improved to the point of making ZK computation feasible in real-world 
	scenarios.
	One of the primary applications lies in hash-tree commitment verification, and in this past 
	five years there has been intense research in proposing ZK-friendly cryptographic 
	primitives.

	In this work, we begin by studying the history of ZK systems and reviewing the state of the art 
	concerning ZK-friendly cryptographic permutations. 
	We then present a novel, generic algebraic framework to design cryptographic permutations 
	and we apply it to construct a new permutation.
	Finally, we implement our permutation together with the reviewed ones in the Groth16 ZK-SNARK 
	framework and compare their efficiency for Merkle-Tree commitment verification.
}
%% Indice
\tableofcontents

%% Lista delle tabelle (se presenti)
%\listoftables

%% Lista delle figure (se presenti)
%\listoffigures

%% Corpo principale del documento
\mainmatter%

%% Parte
%% La suddivisione in parti � opzionale; solitamente sono sufficienti i capitoli.
\chapter{Introduction}
An important research branch of cryptography which emerged in the last fourty years is the study of
\emph{Zero Knowledge Interactive} (ZK-I) protocols, and more specifically zero knowledge proof 
systems (ZKP)~\cite{GoldwasserMR1989}.
The main idea behind ZKP systems is to have two (or, in some cases, more) parties, where 
one is the \emph{prover} and the other is the \emph{verifier}: in a classical proof system, the prover
must be able to convince the verifier that a certain statement is true, when this is indeed the case, 
but the verifier cannot be fooled if the statement is actually false.
In a ZKP system we also require that the verifier does not get any useful additional information 
(i.e.\ knowledge) other than the truth, or lack thereof, of the statement.
This additional requirement is particularly interesting when dealing with statements that are 
notoriously (believed to be) hard to prove, so that the verifier would not be realistically able to 
prove them in a reasonable amount of time.
As a simple example, a prover would like to show that a propositional logic formula is satisfiable 
(an instance of the famous \textsc{sat} problem) without revealing the satisfying assignment to the
verifier.  

Along the years, additional interesting and useful properties have been added to extend and improve 
the capabilities of ZKP systems.
For example, we would like to have a \emph{Non-interactive} (ZK-NP) protocol, to minimize the amount 
of required communication and have it happen only at the beginning and at the end of the protocol.
We could also want to relax the soundness requirement so that it is guaranteed only against 
computationally bounded provers: in this case, instead of `proof' we use the term 
\emph{ARgument of Knowledge}, and hence we can have ZK-IARK/ZK-NARK systems.
More recently, there has been a research effort towards reducing the length of the ARK by ensuring 
that it is constant size or at most bounded by a logarithmic function in the length of the theorem 
statement: such systems are said to be \emph{Succint}. 
Implementations of ZK-SNARK system, like Pinocchio~\cite{ParnoGHR2013} or Groth16~\cite{Groth2016}, 
represent the current state of the art (SoTA) of ZKP systems, and allow to generate proofs to verify any 
computation representable by means of \emph{bounded arithmetic circuits}.
A major downside of ZK-SNARK protocols is their need of a trusted third party (TTP) to setup the 
system, hence current research is studying \emph{Transparent} systems (ZK-STARK) to address this 
issue~\cite{SassonBHR2018}.

An especially useful application of ZKP systems is proving knowledge of a preimage 
for a cryptographic hash function digest (a.k.a.\ commitment).
Many data integrity systems, such as blockchains, rely on Merkle Trees~\cite{Merkle1979} to 
ensure efficient commitment validation, especially in dynamic environments. 
In Merkle Trees, an hash function is applied in a bottom-up fashion: the leaves will contain the 
data owned by some parties, while the root will contain the tree commitment.
In a non-ZK setting, a prover would send the verifier his leaf together with the co-path, 
the verifier would then recompute the tree commitment and compare it with the public one and be 
convinced whether or not the prover does actually own the leaf. 
On the other hand, in a ZK-SNARK setting, we first have to represent the computation through a 
bounded arithmetic circuit, i.e.\ we are allowed to use exclusively a constant number of additions 
and multiplications over some suitable finite field.
The circuit, together with a \emph{proving key} provided by a TTP, and some private and public data, 
is then used by the prover to generate a proof which is sent to the verifier, who in turn uses a 
\emph{verification key}, again provided by the same TTP, to assert whether the circuit computation
was performed correctly. 

While the various ZK-SNARK (or ZK-STARK) frameworks differ in the details, it is intuitive to see
that the complexity of generating the proof (which dominates the cost of the protocol) must depend 
on the size of the circuit, which in turn depends on the amount of multiplications and additions 
performed in the computation: in the case of Merkle Tree commitment verification, most of the 
computation consists in iterating the underlying hash function. 
Since the finite field over which ZK-SNARK frameworks works is typically a huge prime field 
(\(\approx 2^{256}\) elements), traditional hash functions like MD5~\cite{Rivest1990} or 
SHA~\cite{Dang2015}, which are designed to be extremely efficient on classical boolean circuits, 
become extremely inefficient in the ZK case.

It is no wonder then, that in the last years researchers began to study so-called ZK-friendly 
cryptographic permutation (ZKFCP) designs that exploit the features of large prime fields to be 
efficient when translated into airhtmetic circuit, fundamentally resulting in a one-to-one mapping.
Being a new research topic, these designs have seen a rapid series of 
improvements~\cite{AlbrechtGRRT2016,GrassiKRRS2021,GrassiHRSWW2022} in the last three years:
in a two-part series of papers undergoing publication, we presented an algebraic 
framework, called \emph{Generalized Triangular Dynamical System} (GTDS), which allows to express
many of the existing cryptographic permutation designs and eases the construction of new ones, while
at the same time giving strong security guarantees, and we then applied it to devise the \Arion{}
blockcipher and the \ArionHash{} hash function.
Using the \texttt{libsnark}\footnote{\url{https://github.com/scipr-lab/libsnark}} library (an 
implementation of the Groth16 framework), we implemented our hash function, along with other 
competitor hash functions and a hash-agnostic variable-arity Merkle Tree circuit template, in a 
\texttt{C++} project which we then used to compare their real-world performance for same-level 
security gaurantees in various scenarios.

\section*{Structure of this thesis}
This work is organized as follows: \Cref{chap:math} presents the mathematical background for 
both current ZK-SNARK systems and our hash function.
\Cref{chap:computation} presents the computational background of ZK-SNARK systems, and their origin.
\Cref{chap:crypto} presents the cryptographical background of ZK-SNARK systems, their latest 
evolutions and their most important real-world applications.
In \Cref{chap:arion}, we review the history and the state of the art concerning ZK-friendly hash 
functions; we present the GTDS framework and its instantiation in the form of the \Arion{} block 
cipher and the \ArionHash{} hash function; and we study the implementation and performance 
of the latter by comparing it to the state of the art. 
Finally, in \Cref{chap:conclusions}, we draw our conclusion and explore future directions of the 
work.


\part{Foundations}\label{part:foundation}
Zero Knowledge Proof (ZKP) systems are a relatively recent research topic: while the idea in itself,
like many other beautiful ideas, is simple and elegant, its formalization, and even more so its 
realization, is all but trivial.
A first rigorous description of what it means for a proof system to be \emph{Zero Knowledge} was
given by S.\ Goldwasser, S.\ Micali and C.\ Rackoff in 1985~\cite{GoldwasserMR1989} (the work was 
later updated in 1989).

To fully understand the properties of ZKP system, one needs to have an understanding of both 
fundamental and more advanced notions from the fields of group theory, computational theory and 
cryptographical theory. 
This is even more necessary for ZK-SNARK systems and ZK-friendly hash functions.
For this reason, in this first part of the work we will (hopefully) give an exhaustive description
of the tools required to have a better grasp of the results that will be presented in the 
second part.
\chapter{Mathematical Background}


%% Capitolo
%\part{Zero Knowledge friendly permutations}\label{part:zk-hash}
\chapter{\ArionHash{}: a ZK-friendly hash function}\label{chap:arion}
\section{State of the art}
\subsection{MiMC}
\subsection{Poseidon}
\subsection{Griffin}
\subsection{Other designs}
\section{\Arion{} and \ArionHash{}}
\subsection{The Generalized Dynamic Triangular System}
\subsection{Security of \Arion{}}
\section{Implementing \ArionHash{}}
\subsection{Performance of \ArionHash{}}

\chapter{Conclusions}\label{chap:conclusions}

%% Fine dei capitoli normali, inizio dei capitoli-appendice (opzionali)
%\appendix

%\part{Appendix}

%% Parte conclusiva del documento; tipicamente per riassunto, bibliografia e/o indice analitico.
\backmatter%

%% Riassunto (opzionale)
%\summary
%Maecenas tempor elit sed arcu commodo, dapibus sagittis leo egestas. Praesent at ultrices urna. Integer et nibh in augue mollis facilisis sit amet eget magna. Fusce at porttitor sapien. Phasellus imperdiet, felis et molestie vulputate, mauris sapien tincidunt justo, in lacinia velit nisi nec ipsum. Duis elementum pharetra lorem, ut pellentesque nulla congue et. Sed eu venenatis tellus, pharetra cursus felis. Sed et luctus nunc. Aenean commodo, neque a aliquam bibendum, mauris augue fringilla justo, et scelerisque odio mi sit amet diam. Nulla at placerat nibh, nec rutrum urna. Donec ut egestas magna. Aliquam erat volutpat. Phasellus vestibulum justo sed purus mattis, vitae lacinia magna viverra. Nulla rutrum diam dui, vel semper mi mattis ac. Vestibulum ante ipsum primis in faucibus orci luctus et ultrices posuere cubilia Curae; Donec id vestibulum lectus, eget tristique est.

%% Bibliografia (praticamente obbligatoria)
\bibliographystyle{plain_\languagename}%% Carica l'omonimo file .bst, dove \languagename � la lingua attiva.
%% Nel caso in cui si usi un file .bib (consigliato)
\bibliography{biblio.bib}
%% Nel caso di bibliografia manuale, usare l'environment thebibliography.

%% Per l'indice analitico, usare il pacchetto makeidx (o analogo).

\end{document}

\begin{comment}
--- Istruzioni per l'aggiunta di nuove lingue ---
Per ogni nuova lingua utilizzata aggiungere nel preambolo il seguente spezzone:
\addto\captionsitalian{%
\def\abstractname{Sommario}%
\def\acknowledgementsname{Ringraziamenti}%
\def\authorcontactsname{Contatti dell'autore}%
\def\candidatename{Candidato}%
\def\chairname{Direttore}%
\def\conclusionsname{Conclusioni}%
\def\cosupervisorname{Co-relatore}%
\def\cosupervisorsname{Co-relatori}%
\def\cyclename{Ciclo}%
\def\datename{Anno accademico}%
\def\indexname{Indice analitico}%
\def\institutecontactsname{Contatti dell'Istituto}%
\def\introductionname{Introduzione}%
\def\prefacename{Prefazione}%
\def\reviewername{Controrelatore}%
\def\reviewersname{Controrelatori}%
%% Anno accademico
\def\shortdatename{A.A.}%
\def\summaryname{Riassunto}%
\def\supervisorname{Relatore}%
\def\supervisorsname{Relatori}%
\def\thesisname{Tesi di \expandafter\ifcase\csname thud@target\endcsname Laurea\or Laurea Magistrale\or Dottorato\fi}%
\def\tutorname{Tutor aziendale%
	\def\tutorsname{Tutor aziendali}%
}
sostituendo a "italian" (nella 1a riga) il nome della lingua e traducendo le varie voci.
\end{comment}
