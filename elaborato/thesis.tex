%% Le lingue utilizzate, che verranno passate come opzioni al pacchetto babel. Come sempre, l'ultima indicata sar� quella primaria.
%% Se si utilizzano una o pi� lingue diverse da "italian" o "english", leggere le istruzioni in fondo.
\def\thudbabelopt{english}
%% Valori ammessi per target: bach (tesi triennale), mst (tesi magistrale), phd (tesi di dottorato).
\documentclass[target=mst]{thud}

%% --- Informazioni sulla tesi ---
%% Per tutti i tipi di tesi
% Scommentare quello di interesse, o mettete quello che vi pare
\course{Artificial Intelligence and Cybersecurity}
%\course{Internet of Things, Big Data e Web}
%\course{Matematica}
%\course{Comunicazione Multimediale e Tecnologie dell'Informazione}
\title{Cryptographic Primitives for Zero-Knowledge: Theory and Implementation}
\author{Stefano Trevisani}
\supervisor{Dr.\ Arnab Roy}
\cosupervisor{Prof.\ Alberto Policriti \and Prof.\ Elisabeth Oswald}
\tutor{Msc.\ Matthias Steiner}
%% Campi obbligatori: \title, \author e \course.
%% Altri campi disponibili: \reviewer, \tutor, \chair, \date (anno accademico, calcolato in automatico), \rights
%% Con \supervisor, \cosupervisor, \reviewer e \tutor si possono indicare pi� nomi separati da \and.
%% Per le sole tesi di dottorato:
%\phdnumber{313}
%\cycle{XXIX}
%\contacts{Via della Sintassi Astratta, 0/1\\65536 Gigatera --- Italia\\+39 0123 456789\\\texttt{http://www.example.com}\\\texttt{inbox@example.com}}

%% --- Pacchetti consigliati ---
%% pdfx: per generare il PDF/A per l'archiviazione. Necessario solo per la versione finale
%\usepackage[a-1b]{pdfx}
%% hyperref: Regola le impostazioni della creazione del PDF... pi� tante altre cose. Ricordarsi di usare l'opzione pdfa.
%% tocbibind: Inserisce nell'indice anche la lista delle figure, la bibliografia, ecc.
\usepackage{algorithm}
\usepackage[noend]{algpseudocode}
\usepackage{amsmath}
\usepackage{amssymb}
\usepackage{amsthm}
\usepackage{bm}
\usepackage{booktabs}
\usepackage[justification=centering]{caption}
\usepackage{comment}
\usepackage[inline]{enumitem}
\usepackage[T1]{fontenc}
\usepackage[pdfa]{hyperref}
\usepackage{lmodern}
\usepackage{listings}
%\usepackage{mathematica}
%\usepackage[frak=euler]{mathalpha}
\usepackage{mathtools}
%\usepackage{minted}
\usepackage{mleftright}
%\usepackage{sourcecodepro}
\usepackage{stmaryrd}
\usepackage{subfig}
\usepackage{tikz}
\usepackage{cleveref}

\lstset{basicstyle=\footnotesize\ttfamily,breaklines=true,frame=single}
%\setminted{fontsize=\footnotesize}
\usetikzlibrary{automata,positioning,shapes.multipart}

\mleftright{}
\hypersetup{hidelinks,linkcolor = blue,
urlcolor  = blue,
citecolor = blue,
anchorcolor = blue}

\theoremstyle{definition}
\newtheorem{definition}{Definition}[chapter]
\newtheorem{example}{Example}[chapter]

\theoremstyle{plain}
\newtheorem{theorem}{Theorem}[chapter]
\newtheorem{lemma}{Lemma}[chapter]
\newtheorem{corollary}{Corollary}[chapter]
\newtheorem{proposition}{Proposition}[chapter]

\theoremstyle{remark}
\newtheorem{remark}{Remark}[chapter]
%\usepackage[export]{adjustbox}

%% --- Stili di pagina disponibili (comando \pagestyle) ---
%% sfbig (predefinito): Apertura delle parti e dei capitoli col numero grande; titoli delle parti e dei capitoli e intestazioni di pagina in sans serif.
%% big: Come "sfbig", solo serif.
%% plain: Apertura delle parti e dei capitoli tradizionali di LaTeX; intestazioni di pagina come "big".

\newcommand{\Stamp}{\texttt{Stamp}}
\newcommand{\Blocc}{\texttt{Blocc}}
\newcommand{\abs}[1]{\left\lvert#1\right\rvert}
\newcommand{\gengroup}[1]{\left\langle#1\right\rangle}
\newcommand{\bitand}{\mathbin{\textnormal{\textsc{and}}}}
\newcommand{\bitxor}{\mathbin{\textnormal{\textsc{xor}}}}
\newcommand{\algid}{\mathrm{e}}
\newcommand{\Endset}[1]{\mathrm{End}\left(#1\right)}
\newcommand{\Set}[1]{\left\{#1\right\}} %chktex 21
\newcommand{\Tuple}[1]{\left(#1\right)} %chktex 21
\newcommand{\fooid}{\mathrm{id}}


\begin{document}
\maketitle

%% Dedica (opzionale)
%\begin{dedication}
%	To <somebody\_special>,\par for being with me all the way through.
%\end{dedication}

%% Ringraziamenti (opzionali)
%\acknowledgements{
%Sed vel lorem a arcu faucibus aliquet eu semper tortor. 
%Aliquam dolor lacus, semper vitae ligula sed, blandit iaculis leo. 
%Nam pharetra lobortis leo nec auctor. 
%Pellentesque habitant morbi tristique senectus et netus et malesuada fames ac turpis egestas. 
%Fusce ac risus pulvinar, congue eros non, interdum metus. 
%Mauris tincidunt neque et aliquam imperdiet. 
%Aenean ac tellus id nibh pellentesque pulvinar ut eu lacus. 
%Proin tempor facilisis tortor, et hendrerit purus commodo laoreet. 
%Quisque sed augue id ligula consectetur adipiscing. 
%Vestibulum libero metus, lacinia ac vestibulum eu, varius non arcu. 
%Nam et gravida velit.
%}

%% Sommario (opzionale)
\abstract{
	Zero Knowledge (ZK) proof systems have been a subject of study of increasing interest in the last 
	40 years.
	In the last decade, the efficiency of the proposed frameworks, along with the processing power of 
	computing devices, has improved to the point of making ZK computation feasible in real-world 
	scenarios.
	One of the primary applications lies in hash-tree commitment verification, and in this past 
	five years there has been intense research in proposing ZK-friendly cryptographic 
	primitives.

	In this work, we begin by studying the history of ZK systems and reviewing the state of the art 
	concerning ZK-friendly cryptographic permutations. 
	We then apply a new generic algebraic framework, the Generalized Triangular Dynamical System, 
	to design a cryptographic permutation called \Arion{}.
	Finally, we implement our permutation together with the reviewed ones in the Groth16 ZK-SNARK 
	framework and compare their efficiency for Merkle tree commitment verification.
}
%% Indice
\tableofcontents

%% Lista delle tabelle (se presenti)
%\listoftables

%% Lista delle figure (se presenti)
%\listoffigures

%% Corpo principale del documento
\mainmatter%

%% Parte
%% La suddivisione in parti � opzionale; solitamente sono sufficienti i capitoli.
\chapter{Introduction}
An important research branch of cryptography which emerged in the last fourty years is the study of
\emph{Zero Knowledge Interactive} (ZK-I) protocols, and more specifically zero knowledge proof 
systems (ZKP)~\cite{GoldwasserMR1989}.
The main idea behind ZKP systems is to have two (or, in some cases, more) parties, where 
one is the \emph{prover} and the other is the \emph{verifier}: in a classical proof system, the prover
must be able to convince the verifier that a certain statement is true, when this is indeed the case, 
but the verifier cannot be fooled if the statement is actually false.
In a ZKP system we also require that the verifier does not get any useful additional information 
(i.e.\ knowledge) other than the truth, or lack thereof, of the statement.
This additional requirement is particularly interesting when dealing with statements that are 
notoriously (believed to be) hard to prove, so that the verifier would not be realistically able to 
prove them in a reasonable amount of time.
As a simple example, a prover would like to show that a propositional logic formula is satisfiable 
(an instance of the famous \textsc{sat} problem) without revealing the satisfying assignment to the
verifier.  

Along the years, additional interesting and useful properties have been added to extend and improve 
the capabilities of ZKP systems.
For example, we would like to have a \emph{Non-interactive} (ZK-NP) protocol, to minimize the amount 
of required communication and have it happen only at the beginning and at the end of the protocol.
We could also want to relax the soundness requirement so that it is guaranteed only against 
computationally bounded provers: in this case, instead of `proof' we use the term 
\emph{ARgument of Knowledge}, and hence we can have ZK-IARK/ZK-NARK systems.
More recently, there has been a research effort towards reducing the length of the ARK by ensuring 
that it is constant size or at most bounded by a logarithmic function in the length of the theorem 
statement: such systems are said to be \emph{Succint}. 
Implementations of ZK-SNARK system, like Pinocchio~\cite{ParnoGHR2013} or Groth16~\cite{Groth2016}, 
represent the current state of the art of ZKP systems, and allow to generate proofs to verify any 
computation representable by means of \emph{bounded arithmetic circuits}.
A major downside of ZK-SNARK protocols is their need of a trusted third party (TTP) to setup the 
system, hence current research is studying \emph{Transparent} systems (ZK-STARK) to address this 
issue~\cite{SassonBHR2018}.

An especially useful application of ZKP systems is proving knowledge of a preimage 
for a cryptographic hash function digest (a.k.a.\ commitment).
Many data integrity systems, such as blockchains, rely on Merkle Trees~\cite{Merkle1979} to 
ensure efficient commitment validation, especially in dynamic environments. 
In Merkle Trees, an hash function is applied in a bottom-up fashion: the leaves will contain the 
data owned by some parties, while the root will contain the tree commitment.
In a non-ZK setting, a prover would send the verifier his leaf together with the co-path, 
the verifier would then recompute the tree commitment and compare it with the public one and be 
convinced whether or not the prover does actually own the leaf. 
On the other hand, in a ZK-SNARK setting, we first have to represent the computation through a 
bounded arithmetic circuit, i.e.\ we are allowed to use exclusively a constant number of additions 
and multiplications over some suitable finite field.
The circuit, together with a \emph{proving key} provided by a TTP, and some private and public data, 
is then used by the prover to generate a proof which is sent to the verifier, who in turn uses a 
\emph{verification key}, again provided by the same TTP, to assert whether the circuit computation
was performed correctly. 

While the various ZK-SNARK (or ZK-STARK) frameworks differ in the details, it is intuitive to see
that the complexity of generating the proof (which dominates the cost of the protocol) must depend 
on the size of the circuit, which in turn depends on the amount of multiplications and additions 
performed in the computation: in the case of Merkle Tree commitment verification, most of the 
computation consists in iterating the underlying hash function. 
Since the finite field over which ZK-SNARK frameworks works is typically a huge prime field 
(\(\approx 2^{256}\) elements), traditional hash functions like MD5~\cite{Rivest1990} or 
SHA~\cite{Dang2015}, which are designed to be extremely efficient on classical boolean circuits, 
become extremely inefficient in the ZK case.

It is no wonder then, that in the last years researchers began to study so-called ZK-friendly 
cryptographic permutation (ZKFCP) designs that exploit the features of large prime fields to be 
efficient when translated into airhtmetic circuit, fundamentally resulting in a one-to-one mapping.
Being a new research topic, these designs have seen a rapid series of 
improvements~\cite{AlbrechtGRRT2016,GrassiKRRS2021,GrassiHRSWW2022} in the last three years:
in a two-part series of papers undergoing publication, we presented an algebraic 
framework, called \emph{Generalized Triangular Dynamical System} (GTDS), which allows to express
many of the existing cryptographic permutation designs and eases the construction of new ones, while
at the same time giving strong security guarantees, and we then applied it to devise the \Arion{}
blockcipher and the \Arionhash{} hash function.
Using the \texttt{libsnark}\footnote{\url{https://github.com/scipr-lab/libsnark}} library (an 
implementation of the Groth16 framework), we implemented our hash function, along with other 
competitor hash functions and a hash-agnostic variable-arity Merkle Tree circuit template, in a 
\texttt{C++} project which we then used to compare their real-world performance for same-level 
security gaurantees in various scenarios.

\section*{Structure of this thesis}
This work is organized as follows: \Cref{chap:math} presents the mathematical background for 
both current ZK-SNARK systems and our hash function.
\Cref{chap:computation} presents the computational background of ZK-SNARK systems, and their origin.
\Cref{chap:crypto} presents the cryptographical background of ZK-SNARK systems, their latest 
evolutions and their most important real-world applications.
In \Cref{chap:arion}, we review the history and the state of the art concerning ZK-friendly hash 
functions; we present the GTDS framework and its instantiation in the form of the \Arion{} block 
cipher and the \Arionhash{} hash function; and we study the implementation and performance 
of the latter by comparing it to the state of the art. 
Finally, in \Cref{chap:conclusions}, we draw our conclusion and explore future directions of the 
work.

\part{Foundations}\label{part:foundation}
Zero Knowledge Proof (ZKP) systems are a relatively recent research topic: while the idea in itself,
like many other beautiful ideas, is simple and elegant, its formalization, and even more so its 
realization, is all but trivial.
A first rigorous description of what it means for a proof system to be \emph{Zero Knowledge} was
given by S.\ Goldwasser, S.\ Micali and C.\ Rackoff in 1985~\cite{GoldwasserMR1989} (the work was 
later updated in 1989).

To fully understand the properties of ZKP system, one needs to have an understanding of both 
fundamental and more advanced notions from the fields of group theory, computational theory and 
cryptographical theory. 
This is even more necessary for ZK-SNARK systems and ZK-friendly hash functions.
For this reason, in this first part of the work we will (hopefully) give an exhaustive description
of the tools required to have a better grasp of the results that will be presented in the 
second part.
\chapter{Mathematical Background}\label{chap:math}
In this chapter we will introduce all the mathematical concepts behind ZKP and ZK-friendly 
functions.
While we decided, for completeness, to include even some of the more fundamental notions, we still
expect the reader to have at least a rough idea of these concepts.
\Cref{sec:prime_fields} will introduce prime fields, cyclic groups other related notions.  

\section{Finite algebra}\label{sec:prime_fields}
In algebra, a \emph{tuple} consisting of one or more \emph{sets} together with one or more 
\emph{operations} over the sets is called an \emph{algebraic structure}.
Such structures can be organized according to a quite wide taxonomy, depending on whether 
they satisfy certain properties or not. 
We will denote sets with capital letters (e.g.\  \(S, T, U, \dots \)), a generic operation with a 
circled dot \(\odot \) and algebraic structures with blackboard bold letters 
(e.g.\  \(\mathbb{A}, \mathbb{B}, \mathbb{C}, \dots \)).
We will also denote elements of a set with lowercase letters (e.g.\  \(a, b, c, \dots \)) and 
variables over a set with lowercase letters (e.g.\  \(x, y, z, \dots \)).
Finally, we will often use the term algebra to mean algebraic structure, whenever we belive the 
meaning to be clear from the context.
\begin{remark}
  Some symbols will be reserved to denote some common algebraic structures. 
  In particular, \(\mathbb{B}\) will denote the boolean algebra, while \(\mathbb{N}, \mathbb{Z}, 
  \mathbb{Q}, \mathbb{R}\) and \(\mathbb{C}\) will denote, respectively, the 
  natural, integer, rational, real and complex numbers.
\end{remark}

We will denote the \emph{cardinality} of set \(S\) with \(\abs{S}\), and use the same notation for the \emph{order} of an algebraic structure and for the \emph{arity} of an operation: for example, if \(\odot \) is a binary operation, like integer 
addition\footnote{if considered as a relation, addition would be ternary.}, then 
\(\abs{\odot} = 2\).
When an algebraic structure \(\mathbb{A}\) has exactly one \emph{underlying set} \(A\), we will identify the two, e.g.\ by writing \(x \in \mathbb{A}\) to mean \(x \in A\).

\begin{definition}[Finite algebra]
  A finite algebra is an algebraic structure \(\mathbb{A}\) such that \(\abs{\mathbb{A}} \in \mathbb{N}\).
\end{definition}
\begin{definition}[Subalgebra]
  An algebraic structure \(\mathbb{A} = \Tuple{A, \odot_1, \dots, \odot_n}\) is a subalgebra of an algebraic structure \(\mathbb{A}' = \Tuple{A', \odot'_1, \dots, \odot'_m}\), for some \(n \le m\), if \(A \subseteq A'\) and \(\forall i \le n\colon \odot_i \subseteq \odot'_i\).
\end{definition}

Elements of differents algebraic structures can be associated through \emph{morphisms}.
\begin{definition}[Homomorphism]
  Given two algebras \(\mathbb{A} = \Tuple{A, \odot_{1}, \dots, \odot_{n}}\), 
  \(\mathbb{A}' = \Tuple{A', \odot'_{1}, \dots, \odot'_{n}}\) such that 
  \(\forall i \le n\colon \abs{\odot_i} = \abs{\odot'_i} = a_i\), an homomorphism is a map 
  \(h\colon A \to A'\) such that:
  \begin{align*}
    & \forall i \le n, \forall x_1,\dots, x_{a_{i}} \in A\colon 
    h(\odot_{i}(x_{1}, \dots, x_{a_{i}})) = \odot'_{i}(h(x_{1}), \dots, h(x_{a_{i}})) && 
    \textnormal{(linearity)}
  \end{align*}
  We say that \(\mathbb{A}\) is homomorphic to \(\mathbb{A}'\) through \(h\).
\end{definition}

\begin{definition}[Isomorphism]
  An isomorphism is a bijective homomorphism.
\end{definition}

Given two algebras \(\mathbb{A}\) and \(\mathbb{A}'\), if they are isomorphic through some map
\(h\), we write \(\mathbb{A} \cong_h \mathbb{A}'\), or more succintly \(\mathbb{A} \cong \mathbb{A}'\).

\begin{definition}[Endomorphism, Automorphism]
  An endomorphism is a homomorphism from an algebraic structure \(\mathbb{A}\) to itself.
  An automorphism is an endomorphism which is also an isomorphism.
\end{definition}

\subsection{Groups}
We will now introduce some important classes of algebraic structures equipped with one fundamental 
operation. 
\begin{definition}[Monoid]
  A monoid is a pair \(\mathbb{M} = \Tuple{M, \odot} \), where \(M\) is the 
  underlying set and \(\odot\colon M \times M \to M\) is the \emph{composition} 
  operation, such that the following properties are satisfied: 
  \begin{align*}
    & \forall x,y \in M\colon x \odot \Parens*{y \odot z} = \Parens*{x \odot y} \odot z
      && \textnormal{(\emph{associativity})} \\
    & \exists \algid \in M\colon \forall x \in M\colon x \odot \algid = x
      && \textnormal{(\emph{identity element})}
  \end{align*}
  \(\mathbb{M}\) is a \emph{commutative (or abelian) monoid}, if it also holds that:
  \begin{align*}
    & \forall x,y \in M\colon x \odot y = y \odot x && (\emph{commutativity})
  \end{align*}
  Finally:
  \begin{equation}\label{eq:exponentiation}    
    \forall x \in \mathbb{M}, \forall k \in \mathbb{N}\colon x^{k} = 
    \begin{cases}
      \algid & k = 0 \\
      x^{k-1} \odot x & k > 0
    \end{cases}
  \end{equation}
\end{definition}

\begin{definition}[Cyclic Monoid]
  A cyclic monoid is a monoid \(\mathbb{M} = \Tuple{M, \odot}\) which has a 
  \emph{generator element} \(g\) such that:
  \[\mathbb{M} = \gengroup{g} = \Tuple{\Set{g^k \mid k \in \mathbb{N}}, \odot} \]
\end{definition}

\begin{definition}[Group]
  A group is a monoid \(\mathbb{G} = \Tuple{G, \odot} \), such that: 
  \begin{align*}    
    & \forall x \in G\colon \exists x^{-1} \in G\colon x \odot x^{-1} = \algid
    && \textnormal{(\emph{inverse element})}
  \end{align*}
  With the notion of inverse element, we can rewrite and extend \Cref{eq:exponentiation} 
  for groups as follows:
  \[
    \forall x \in \mathbb{G},\forall k \in \mathbb{Z}\colon x^k =
    \begin{cases}
      x^{k-1} \odot x & k \ge 0 \\
      x^{k+1} \odot x^{-1} & k < 0
    \end{cases}
  \]
  If \(\mathbb{G}\) is also a commutative (resp.\ cyclic) monoid, then it is a 
  commutative (resp.\ cyclic) group.
\end{definition}

The identity element \(\algid \) of a monoid is typically denoted with \(1\) in 
numeric algebras, when the composition operation resembles standard multiplication, or by \(0\) 
when the composition operation resembles standard addition. 
We use the notation \(\algid_{\mathbb{A}}\) (or \(1_{\mathbb{A}}\), 
\(0_{\mathbb{A}}\)) to specify the algebra over which we intend to pick the identity element, 
dropping the subscript when \(\mathbb{A}\) is clear from the context.

It is important to stress that one should be careful not to confuse the symbol and the name of an 
operation or of a special element with its semantics: the syntax to denote the inverse element 
\(x^{-1}\) of a group is reminiscent of standard multiplication inversion, but this is not the 
case in general.
In fact, when the composition operation resembles standard addition, the inverse is more likely 
denoted with \(-x\).
With this clear in mind, to slim the notation we will often be using the same symbols to denote 
even quite different operations (`overloading'), whose semantics should be clear from the 
associated operands. 
\begin{example}
  The algebra \(\mathbb{A} = \mathbb{Z} \setminus \Set{\times}\) (i.e.\ integer numbers 
  without multiplication) is an abelian group: 
  addition is associative and commutative, the identity element is 
  \(\algid_{\mathbb{A}} = 0\), and every number \(x\) has an inverse \(x^{-1} = -x\) 
  (e.g.\  \({42}^{-1} = -42\)). 
\end{example}

\begin{example}\label{ex:endo_group}
  Given a commutative group \(\mathbb{G} = \Tuple{G, \odot}\), consider the algebra 
  \(\Endset{\mathbb{G}}_{+} = \Tuple{H, +}\), where \(H\) is the set of endomorphisms over 
  \(\mathbb{G}\) and \(+\colon H \times H \to H\) is such that 
  \(\forall h_1, h_2 \in H, \forall x \in G\colon \call{\Parens*{h_1 + h_2}}{x} = 
  \call{h_1}{x} + \call{h_2}{x}\).

  \(\Endset{\mathbb{G}}_{+}\) is a commutative group: \(+\) is both associative and 
  commutative, the identity element is \(\algid_{\Endset{\mathbb{G}}_{+}} = z\), where
  \(z\) is the zero endomorphism (i.e.\  \(\forall x \in G\colon \call{z}{x} = 
  \algid_{\mathbb{G}}\)); finally, every homomorphism \(h \in H\) has an inverse 
  \(h^{-1} = -h\) such that \(\forall x \in G\colon \call{\Parens*{-h}}{x} = 
  \call{h}{x}^{-1}\) (in this example, using the \(h^{-1}\) notation causes confusion with 
  the inverse function!).
\end{example}

\begin{example}\label{ex:endo_monoid}
  Consider now the algebra \(\Endset{\mathbb{G}}_{\circ} = \Tuple{H, \circ}\) 
  where \(\mathbb{G}\) and \(H\) are defined as in \Cref{ex:endo_group}, and 
  \(\circ\colon H \times H \to H\) is defined as function composition: 
  \(\call{\Parens*{h_1 \circ h_2}}{x} = \call{h_1}{\call{h_2}{x}}\).
  
  \(\Endset{\mathbb{G}}_{\circ}\) is a monoid: function composition is associative, and the 
  identity element is \(\algid_{\Endset{\mathbb{G}}_{\circ}} = \fooid \), where \(\fooid \) is the
  identity endomorphism (i.e.\  \(\forall x \in G\colon \call{\fooid}{x} = x\)). 
\end{example}

\subsection{Fields}
Many algebraic structures rely on two fundamental operations, called \emph{addition} and 
\emph{multiplication}: two important types of such structures are \emph{rings} and \emph{fields}.
\begin{definition}[Ring]
  A ring is a triple \(\mathbb{O} = \Tuple{O, \oplus, \otimes}\) where \(O\) is the 
  underlying set, \(\oplus\colon O \times O \to O\) is the \emph{addition} operation and 
  \(\otimes\colon O \times O \to O\) is the \emph{multiplication} operation, such that the 
  following properties are satisfied:
  \begin{align*}
    & \mathbb{O}_{\oplus} = \mathbb{O} \setminus \Set{\otimes}
      \textnormal{ is an abelian group} \\
    & \mathbb{O}_{\otimes} = \mathbb{O} \setminus \Set{\oplus} 
      \textnormal{ is a monoid} \\
    & \forall x,y,z \in O\colon x \otimes \Parens*{y \oplus z} = 
      \Parens*{x \otimes y} \oplus \Parens*{x \otimes z} && (\emph{left distributivity})\\
    & \forall x,y,z \in O\colon \Parens*{y \oplus z} \otimes x = 
      \Parens*{y \otimes x} \oplus \Parens*{z \otimes x} && (\emph{right distributivity})
  \end{align*}
  If \(\mathbb{O}_{\otimes}\) is a commutative monoid, then \(\mathbb{O}\) is a 
  \emph{commutative (abelian) ring}.
  %(in this case, either one of the two distributivity properties becomes redundant).
\end{definition}

Given a ring \(\mathbb{O}\) and an element \(x \in \mathbb{O}\), 
we denote its inverse w.r.t.\ addition as \(-x\), while maintaining the notation \(x^{-1}\) for 
the multiplicative inverse.
Furthermore, the identity element w.r.t.\ addition, denoted \(\algid_{\oplus}\), will also be 
denoted as \(0\), while the identity element w.r.t.\ multiplication, denoted 
\(\algid_{\otimes}\), will maintain its alternative notation as \(1\).

\begin{definition}[Field]
  A field is a commutative ring \(\mathbb{F} = \Tuple{F, \oplus, \otimes}\) such that
  \(0 \neq 1\) and \(\mathbb{F}_{\otimes} \setminus \Set{0}\) is a commutative group.
\end{definition}

Fields are one of the most important and studied algebraic structures: the algebra of real numbers 
\(\mathbb{R}\) is a field, as is the algebra of complex numbers \(\mathbb{C}\).
Given the set of integers \(Z_q = \Set{0, \dots, q-1}\), we denote with \(\oplus_q\) integer sum 
modulo \(q\), and with \(\otimes_q\) integer multiplication modulo \(q\).
Furthermore, we will denote with \(\gengroup{g}_q\) the cyclic group generated by \(g\) under 
the operation \(\otimes_q\).
The algebra \(\mathbb{Z}_q = \Tuple{Z_q, \oplus_q, \otimes_q}\) is a finite ring 
\(\forall q \in \mathbb{N}\), and it is a finite field if and only if \(q\) is prime.
\begin{definition}[Discrete logartithm]
  The discrete logartithm over some cyclic group \(\gengroup{g}\) of order \(q\) is the function:
  \[\call{\log_g}{g^x}\colon \gengroup{g} \to \mathbb{Z}_q = x\]
\end{definition}

When the group genreator is clear from the context, we simply write \(\log \) instead of \(\log_g\).
Typically, cyclic groups are obtained as the subset of a larger finite field 
(see \Cref{ex:cyclic_group}). 
\begin{example}
  Boolean circuits with \textsc{xor} and \textsc{and} gates behave like elements of the boolean 
  field \(\mathbb{B} = \Tuple{\Set{\bot, \top}, \bitxor, \bitand} \).
  It is easy to show that \(\mathbb{B} \cong \mathbb{Z}_2\).
  Similarly, \(k\)-bit unsigned integers sum and multiplication work as in \(\mathbb{Z}_{2^k}\).
\end{example}

\begin{example}
  Given an abelian group \(\mathbb{G}\), the algebra \(\mathbb{H}_{\mathbb{G}} = 
  \Endset{\mathbb{G}} = \Endset{\mathbb{G}}_{+} \cup \Endset{\mathbb{G}}_{\circ}\) 
  is the \emph{endomorphism ring} of \(\mathbb{G}\): \(\Endset{\mathbb{G}}_{+}\) is an abelian 
  group, \(\Endset{\mathbb{G}}_{\circ}\) is a monoid 
  (cfr.\ \Cref{ex:endo_group,ex:endo_monoid}), and it is easy to show that \(\circ \)
  distributes over \(+\) both on the left and the right.
\end{example}

\begin{example}\label{ex:cyclic_group}
  Consider the cyclic group 
  \(\mathbb{G} = \gengroup{2}_{23} = \Tuple{\Set{1, 2, 3, 4, 6, 8, 9, 12, 13, 16, 18}, \otimes_{23}}\)
  which has order \(\abs{G} = 11\).
  Let's show that \(\log = \log_2\) is an homomorphism between \(\mathbb{G}\) and 
  \(\mathbb{Z}_{11,\oplus}\): for any two elements \(x, y \in \mathbb{G}\), 
  we have:
  \[
    x \otimes_{23} y = 2^{\call{\log}{x}} \otimes_{23} 2^{\call{\log}{y}} = 
    2^{\call{\log}{x} \oplus_{11} \call{\log}{y}}
  \]
  Since \(\log_2\) is a bijection, it is also an isomorphism.
  In fact, one can show that \(\forall q \in \mathbb{N}\) and \(\forall g < q\) such 
  that \(\call{\gcd}{g, q} = 1\) (otherwise \(\gengroup{g}_{q}\) would not be a group), then
  \(\mathbb{G} = \gengroup{g}_{q} \cong_{\log_g} \mathbb{Z}_{\abs{\mathbb{G}},\oplus}\).
\end{example}

\subsection{Vector spaces}
All the algebraic structures we have seen in the previous section operate on an underlying set 
whose elements we consider to be, in some sense, atomic.
On the other hand, many objects interact with each other exhibiting a multi-dimensional behaviour 
(e.g.\ physical forces).
The standard structure to deal with such objects are \emph{vector spaces}.
\begin{definition}[Module]
  A module is a quadruple \(\mathbb{M} = \Tuple{M, \mathbb{O}, +, \odot}\) where 
  \(M\) is the underlying vector set, \(\mathbb{O} = \Tuple{O, \oplus, \otimes}\) is the underlying 
  scalar ring, \(+\colon M \times M \to M\) is the \emph{module addition} operation and 
  \(\odot\colon O \times M \to M\) is the \emph{scalar multiplication} operation, such 
  that \(\mathbb{M}_{+} = \Tuple{M, +}\) is a commutative group and \(\odot \) is an 
  homomorphism between \(\mathbb{O}\) and \(\Endset{\mathbb{M}_{+}}\).
\end{definition} 

\begin{definition}[Vector space]
  A vector space is a module \(\mathbb{V} = \Tuple{V, \mathbb{F}, +, \odot}\) such that the 
  underlying scalar ring \(\mathbb{F}\) is a field.
\end{definition} 

The most common vector space is the one of \(n\)-dimensional \emph{column vectors} over a 
field \(\mathbb{F} = \Tuple{F, \oplus, \otimes}\) such that \(\mathbb{F}^n = 
\Tuple{F^n, \mathbb{F}, +, \odot}\), where \(+\) is entry-wise field addition between 
column vectors and \(\odot \) is element-wise field multiplication of scalars with column vectors.
We will denote elements of a column vector space \(\mathbb{V}\) with bold letters 
(e.g.\  \(\bm{u}, \bm{v}, \bm{w}, \dots \)), and elements of the dual \emph{row vector} space 
\(\mathbb{V}^{\transpose}\) with 
(e.g.\  \(\bm{u}^{\transpose}, \bm{v}^{\transpose}, \bm{w}^{\transpose}, \dots \)); 
finally, we denote the \(i\)th element of a column vector \(\bm{v}\) with \(\bm{v}_i\).
\begin{definition}[Dot product]
  Given a field \(\mathbb{F} = \Tuple{F, \oplus, \otimes}\) and an \(n\)-dimensional vector space 
  \(\mathbb{V} = \Tuple{V, \mathbb{F}, +, \odot}\), the dot product operation is the map:
  \[
    \bm{v} \cdot \bm{w}\colon \mathbb{V} \times \mathbb{V} \to \mathbb{F} = 
    \bigoplus_{i = 1}^{n}{\bm{v}_i \otimes \bm{w}_i}
  \]
\end{definition}

Another important vector space is the one of \(\Parens*{n \times m}\)-dimensional \emph{matrices} 
over some base field \(\mathbb{F}\): 
\(\mathbb{F}^{n \times m} = \Tuple{{\Parens*{F^n}}^m, \mathbb{F}, +, \odot}\), where \(+\) is 
element-wise field addition between matrices, and \(\odot \) is element-wise field multiplication 
of scalars with matrices.
We will denote elements of a matrix space \(\mathbb{M}\) with bold capital letters 
(e.g.\  \(\bm{A}, \bm{B}, \bm{C}, \dots \)), we denote the \(i\)th row of a matrix \(\bm{M}\) with 
\(\bm{M}_{i}\), 
and the \(j\)th element of the \(i\)th row with \(\bm{M}_{i,j}\).

From now on, for vectors we will only deal with column vector space extensions of the kind 
\(\mathbb{F}^n\) and row vector space extensions of the kind 
\(\Parens*{\mathbb{F}^m}^{\transpose}\) for some base field \(\mathbb{F}\) and some 
\(n, m \in \mathbb{N}\).
Similarly, we will only deal with matrix space extensions of the kind \(\mathbb{F}^{n \times m}\).
Therefore, the \(i\)th column of a matrix will always be an element of 
\(\mathbb{F}^{n} \cong \mathbb{F}^{n \times 1}\), and the \(j\)th row of a matrix will always be an
element of \(\Parens*{\mathbb{F}^m}^{\transpose} \cong \mathbb{F}^{1 \times m}\).

\begin{definition}[Transpose matrix]
  The transpose of a matrix \(\bm{M} \in \mathbb{F}^{n \times m}\) is the matrix:
  \[\bm{M}^{\transpose} \mid 
  \forall i \le n, \forall j \le m\colon \bm{M}^{\transpose}_{i,j} = \bm{M}_{j,i}\]
\end{definition}

Therefore, given a matrix \(\bm{M}\), we can denote the \(i\)th column with 
\(\bm{M}^{\transpose}_{i}\).

\begin{definition}[Matrix concatenation]
  Given two matrices \(\bm{A} \in \mathbb{F}^{n \times m_1}\) and 
  \(\bm{B} \in \mathbb{F}^{n \times m_2}\), their row-wise concatenation is the matrix 
  \(\bm{C} = 
  \begin{pmatrix*}
    \bm{A} & \bm{B}
  \end{pmatrix*}
    \in \mathbb{F}^{n \times \Parens*{m_1 + m_2}}\), such that:
  \[\forall i \le n\colon \Parens*{\forall j \le m_1 \colon \bm{C}_{i,j} = \bm{A}_{i,j}} \wedge 
  \Parens*{\forall j \le m_2\colon \bm{C}_{i,j} = \bm{B}_{i,j}}\]
  And their column-wise concatenation is the matrix \(
  \begin{pmatrix*}
    \bm{A}; \bm{B}
  \end{pmatrix*} =
  {\begin{pmatrix*}
    \bm{A}^{\transpose} & \bm{B}^{\transpose}
  \end{pmatrix*}}^{\transpose}
  \).
\end{definition}

\begin{definition}[Matrix multiplication]
  Matrix multiplication over a base field \(\mathbb{F}\) and some \(m, n_1, n_2 \in \mathbb{N}\), 
  is the map:
  \[
    \bm{A}\bm{B}\colon 
    \mathbb{F}^{n_1 \times m} \times \mathbb{F}^{m \times n_2} \to \mathbb{F}^{n_1 \times n_2} \mid 
    \forall i \le n_1, \forall j \le n_2\colon 
    \Parens*{\bm{A}\bm{B}}_{i,j} = \bm{A}_{i} \cdot \bm{B}^{\transpose}_{j}
  \] 
\end{definition}

\begin{definition}[Linear map]
   A linear map is a homomorphism between two modules.
\end{definition}
\begin{definition}[\(k\)-linear map]
  Given \(k\) vector spaces \(\mathbb{V}_1, \dots, \mathbb{V}_k, \mathbb{W}\) over the same scalar 
  field \(\mathbb{F}\), a map 
  \(f\colon \mathbb{V}_1 \times \cdots \times \mathbb{V}_k \to \mathbb{W}\) 
  is \(k\)-linear if, \(\forall i \in \mathbb{N}\), all the maps resulting by fixing all but the 
  \(i\)th argument are linear maps.
\end{definition}

As we will see, bilinear (\(2\)-linear) maps are a fundamental component of modern ZK-SNARK systems.

\subsection{Polynomials}
The last fundamental object that we will need are polynomials and their relative algebras. 
\begin{definition}[Monovariate polynomial ring]
  A monovariate polynomial ring over a field \(\mathbb{F}\) is the triple 
  \(\extend{\mathbb{F}}{x} = \Tuple{\extend{F}{x}, +, \cdot}\) where \(\extend{F}{x}\) is the set 
  of monovariate polynomials over \(F\) in the indeterminate \(x\), 
  \(+\colon \extend{F}{x} \times \extend{F}{x} \to \extend{F}{x}\) is the \emph{polynomial addition}
  operation and \(\cdot\colon \extend{F}{x} \times \extend{F}{x} \to \extend{F}{x}\) is the
  \emph{polynomial multiplication} operation, such that all the properties of a ring are satisfied.
\end{definition}

We will denote polynomials with lowercase letters (e.g.\  \(p, q, r, \dots \)), and the degree of 
some polynomial \(p\) with \(\call{\deg}{p}\).

Given two vectors \(\bm{x}, \bm{y} \in \mathbb{F}^n\), by using 
\emph{Lagrange interpolation}~\cite{Waring1779}:
\[
  \call{L}{\bm{x}, \bm{y}}\colon \mathbb{F}^n \times \mathbb{F}^n \to \extend{\mathbb{F}}{x} = 
  \sum_{i}{\bm{y}_i\prod_{j \neq i}{\frac{x - \bm{x}_i}{\bm{x}_i - \bm{x}_j}}}
\]
we can build the unique polynomial of degree \(n - 1\) which, \(\forall i \le n\) assumes 
value \(\bm{y}_i\) at point \(\bm{x}_i\).
We can extend the Lagrange interpolation function to a matrix space \(\mathbb{F}^{n \times m}\)
by applying \(L\) separately to each row, as follows:
\[
  \call{L}{\bm{X}, \bm{Y}}\colon \mathbb{F}^{n \times m} \times \mathbb{F}^{n \times m} \to 
  \extend{\mathbb{F}^n}{x} = 
  \begin{pmatrix*} 
    \call{L}{\bm{X}_1, \bm{Y}_1} & \cdots & \call{L}{\bm{X}_n, \bm{Y}_n}
  \end{pmatrix*}
\]

\section{Arithmetic Programs}
Suppose we have some algebra \(\mathbb{A}\): we can represent and deal with finite sequences of 
operations, called \emph{expressions}, between elements of \(\mathbb{A}\) and/or variables over 
\(\mathbb{A}\).

For example, given the expression \(x^2 + x + 1\) over \(\mathbb{R}\left[x\right]\), we might be 
interested to know what is the \emph{evaluation} of the expression given some value for \(x\).

\begin{definition}[Arithmetic formula]
  Given an algebraic structure \(\mathbb{A} = \Tuple{A, \odot_1, \dots, \odot_n}\), an explicit 
  arithmetic formula over \(\mathbb{A}\) is any expression \(\varphi \) of the kind:
  \begin{align*}
    & \varphi \equiv a && \textnormal{with \(a\) constant over \(A\)} \\ 
    & \varphi \equiv x && \textnormal{with \(x\) variable over \(A\)} \\
    & \varphi \equiv \call{\odot_i}{\varphi_1, \dots, \varphi_{\abs{\odot_i}}} && 
    \textnormal{with \(\varphi_1, \dots, \varphi_{\abs{\odot_i}}\) formulae over \(A\)}
  \end{align*}
  Additionally, an implicit (or succint) arithmetic formula also allows expressions involving 
  exponentiation:
  \begin{align*}
    & \varphi \equiv \call{\odot_i^k}{\varphi_1} && 
    \textnormal{\(\forall k \in \mathbb{N}\), with \(\varphi_1\) formula over \(A\)}
  \end{align*}
  where \(\odot_i^k\) represents the \(k\)th power of \(\odot_i \). 
\end{definition}

\begin{remark}
  It is always possible to translate an implicit formula into an equivalent explicit one.
  We denote the explicit version of an implciit formula \(\varphi \) with \(\explicit{\varphi}\).
  For some particular structures, such as fields, this translation can be specialized. 
\end{remark}

From now on, we will only deal with arithmetic formulae over some field (or ring) \(\mathbb{F}\), 
in which case implicit arithmetic expressions are equivalent to multi-variate polynomials.
\begin{example}\label{ex:arithmetic_formula}
  Consider the finite field \(\mathbb{Z}_{13}\).
  For ease of notation, we will use \(+\), juxtaposition and superscripting to denote, respectively,
  field addition, field multiplication, and exponentiation w.r.t.\ field multiplication. 
  A possible implicit arithemtic formula over \(\mathbb{Z}_{13}\) is the following expression:
  \[\varphi = x_{2}\Parens*{x_{1}^{3} + 4x_{2} + 5}\]
  Since in a finite field multiplication by a constant is simply repeated addition, i.e. 
  \(cx = \call{+^{c}}{x}\), the explicit version of \(\varphi \) then is:
  \[\explicit{\varphi} = x_{2}\Parens*{x_{1}x_{1}x_{1} + x_{2} + x_{2} + x_{2} + x_{2} + 5}\]
\end{example} 

\subsection{Arithmetic circuits}
It is possible to visually represent an arithmetic formula using a particualr kind of labeled 
\emph{directed acyclic graph} (DAG), called the \emph{arithmetic circuit}.
\begin{definition}[Arithmetic circuit]
  An arithmetic circuit over an algebra \(\mathbb{A} = \Tuple{A, \odot_1, \dots, \odot_n}\) and a 
  set of variables \(X\) over \(\mathbb{A}\) is a triple \(\mathcal{G} = \Tuple{V, E, L}\) where 
  \(V\) is the set of \emph{vertices}, \(E \subseteq V \times V\) is the set of \emph{edges}, and 
  \(L\colon V \to A \cup X \cup \Parens*{\Set{\odot_1, \dots, \odot_n} \times \mathbb{N}}\) is 
  the vertex \emph{labeling map}, such that, \(\forall v \in V\):
  \begin{align*}
    & \call{L}{v} \in A && \implies \nexists w \in V\colon \Tuple{w, v} \in E
    && \textnormal{(no in-edges for constant nodes)} \\
    & \call{L}{v} \in X && \implies \nexists w \in V\colon \Tuple{w, v} \in E
    && \textnormal{(no in-edges for variable nodes)} \\
    & \forall i \le n\colon \odot_i \in \call{L}{v} && \implies \abs{\Set{\Tuple{w, v}}_{w \in V} \cap E} = \abs{\odot_i}
    && \textnormal{(exactly \(\abs{\odot_i}\) in-edges for \(\odot_i\) nodes)}
  \end{align*}
\end{definition}

As an abuse of notation, we will sometimes identify a node \(v\) with its label \(\call{L}{v}\).
Given any explicit arithmetic formula \(\varphi \) over an algebra \(\mathbb{A}\) and a set of 
variables \(X\), we can build the corresponding arithmetic circuit \(\mathcal{G} = \Tuple{V, E, L}\) 
in the following way: for every distinct (i.e.\ ignoring repetitions) variable \(x\) appearing 
in \(\varphi \), we add a vertex \(v\) with label \(\call{L}{v} = x\); 
for every distinct constant \(c\) appearing in \(\varphi \) we add a vertex \(v\) with label 
\(\call{L}{v} = c\); 
finally, for every occurence \(i\) of some operation \(\odot \) in \(\varphi \), we add a vertex
\(v\) with label \(\call{L}{v} = \odot_{i}\).
Furthermore, we can partition \(V\) as follows:
\begin{itemize}
  \item \emph{Constant vertices}: 
    \(\mathcal{G}_{const} = \Set{v \mid \call{L}{v} \in \mathbb{A}}\) .
  \item \emph{Variable vertices}: 
    \(\mathcal{G}_{var} = \Set{v \mid \call{L}{v} \in X}\).
  \item \emph{Operation vertices}: 
    \(\mathcal{G}_{\odot_i} = \Set{v \mid \odot_i \in \call{L}{v}}\).
  \item \emph{Input vertices}: 
    \(\mathcal{G}_{in} = \mathcal{G}_{const} \cup \mathcal{G}_{var}\).
  \item \emph{Output vertices}: 
    \(\mathcal{G}_{out} = \Set{v \mid \nexists w \in V\colon \Tuple{v, w} \in E}\).
  \item \emph{I/O vertices}: \(\mathcal{G}_{IO} = \mathcal{G}_{in} \cup \mathcal{G}_{out}\).
\end{itemize}

To build the set of edges \(E\), for every operation occuring in \(\varphi \), we connect the 
vertices representing the operands to the vertex representing said operation, e.g.\ if we have 
the formula \(\Parens*{x \odot y} \odot z\) we add the edges \(\Tuple{x, \odot_1}\), 
\(\Tuple{y, \odot_1}\) and \(\Tuple{z, \odot_2}\).
We also consider operation nodes as holding the intermediate values of the computation: in the 
previous example, we will also have the edge \(\Tuple{\odot_1, \odot_2}\), where \(\odot_1 \) 
represents the intermediate value \(x \odot y\).
The fact that we store all the intermediate values of a computation is something that can be 
greatly exploited to optimize the design of a circuit.

\begin{example}
  \Cref{fig:arithmetic_circuit} shows the arithmetic circuit derived from the formula shown in 
  \Cref{ex:arithmetic_formula}.
  We can see the two variable vertices \(x_1\) and \(x_2\) which are also input vertices, the 
  constant vertex \(5\), which is an input vertex too, the
  addition vertices \(\oplus_1, \dots, \oplus_5\) and the multiplication vertices 
  \(\otimes_1, \otimes_2, \otimes_3\), of which the latter is also an output vertex.
\end{example}

\begin{figure}
	\centering
	\begin{tikzpicture}[node distance={48pt}, node/.style = {draw, circle}]
		\node[node] (x1) {\(x_1\)};
		\node[node] (x2) [below of=x1] {\(x_2\)};
		\node[node] (m1) [right of=x1] {\(\otimes_1\)};
		\node[node] (m2) [right of=m1] {\(\otimes_2\)};
		\node[node] (a1) [right of=x2] {\(\oplus_1\)};
		\node[node] (a2) [right of=a1] {\(\oplus_2\)};
		\node[node] (a3) [right of=a2] {\(\oplus_3\)};
		\node[node] (a4) [right of=m2] {\(\oplus_4\)};
		\node[node] (a5) [right of=a4] {\(\oplus_5\)};
		\node[node] (5) [below of=a5] {\(5\)};
		\node[node] (m3) [right of=5] {\(\otimes_3\)};
		\draw[->] (x1) to [bend left] (m1);
		\draw[->] (x1) to [bend right] (m1);
		\draw[->] (x1) to [bend left] (m2);
		\draw[->] (m1) to [bend right] (m2);
		\draw[->] (x2) to [bend left] (a1);
		\draw[->] (x2) to [bend left] (a2);
		\draw[->] (x2) to [bend left] (a3);
		\draw[->] (x2) to [bend right] (a1);
		\draw[->] (a1) to [bend right] (a2);
		\draw[->] (a2) to [bend right] (a3);
		\draw[->] (m2) to (a4);
		\draw[->] (a3) to (a4);
		\draw[->] (a4) to (a5);
		\draw[->] (5) to (a5);
		\draw[->] (a5) to (m3);
		\draw[->] (x2) to [bend right] (m3);

	\end{tikzpicture}
	\caption{Arithmetic circuit of the formula shown in 
    \Cref{ex:arithmetic_formula}.}\label{fig:arithmetic_circuit}
\end{figure}

Since arithmetic circuits contain no cycles, they can only be used to represent a fixed number 
of operations (aka \emph{bounded computations}).
In general though, this is not really a big issue, as oftentimes we can easily synthesize circuits 
\emph{on-the-fly}.

Every arithmetic circuit can be then associated with a set of \emph{circuit assignments}.
\begin{definition}[Circuit assignment]
  A circuit assignment over an arithmetic circuit \(\mathcal{G} = \Tuple{V, E, L}\) is a triple 
  \(\mathcal{A}_{\mathcal{G}} = \Tuple{V, E, L'}\) such that, 
  \(\forall v \in V\colon \call{L'}{v} \in A\)
\end{definition}


\chapter{An Overview of Proof Systems}\label{chap:computation}
Proving theorems is an activity which has always been considered highly intellectual, and 
even today most theorems are proven ``by hand''. 
But what does it really mean to prove a theorem? Is finding a proof for a theorem really harder 
than checking whether said proof is valid? And is it possible to prove to someone that a theorem is 
true, without revealing why that is so?

In this chapter, after introducing the required concepts \emph{computation and complexity theory} 
in \Cref{sec:interactive_tm} and \Cref{sec:complexity}, we give an overview of 
\emph{interactive proof systems} in \Cref{sec:interactive_proof_systems}, and the history 
of their \emph{zero-knowledge} counterparts in \Cref{sec:zero_knowledge}.

\section{Interactive Turing machines}\label{sec:interactive_tm}
A \emph{computational model} (or model of computation) is any kind of system able to describe 
how to produce some \emph{output} given some \emph{input}~\cite{Savage1997}.
Different models do this in different ways, each one with its own strength and weaknesses in terms
of \emph{expressivness}, \emph{complexity} and \emph{succintness}.
Two historically important models of computations are Alonzo Church's 
\emph{\(\lambda \)-calculus}~\cite{Church1941} and Alan Turing's 
\emph{Turing machine} (TM)~\cite{Turing1950}. 
Among several equivalent models~\cite{Davis2004}, Turing machines became the standard model of 
computation.
\begin{definition}[Turing machine~\cite{Papadimitriou1994}]\label{def:turing_machine}
  A Turing machine is a quadruple \(\mathcal{M} = \Tuple{\Sigma, Q, q_0, \delta}\), where 
  the \emph{alphabet} \(\Sigma \) is a set of symbols such that \(\sqcup \in \Sigma \), the 
  \emph{state set} \(Q\) is a set of symbols such that \(\Set{\bot, \top} \subseteq Q\), 
  \(q_0 \in Q\) is the \emph{initial state}, and 
  \(\delta\colon {\Parens*{Q \setminus \Set{\bot, \top}} \times \Sigma} \to 
  {Q \times \Sigma \times \Set{\leftarrow, \rightarrow}}\) is the \emph{transition function}.
\end{definition}

By only requiring \(\delta \) to be a relation instead of a function, we obtain the so-called 
\emph{non-deterministic} Turing machine (NTM): given a state and an alphabet symbol, the machine 
can take different choiches at every step.
A TM \(\mathcal{M}\) manipulates a string \(\overbar{\sigma}\) over the alphabet 
\(\Sigma \setminus \Set{\sqcup}\) by placing it over an \emph{infinite, discrete working tape} 
\(\Tapework \), a total order isomorphic to \(\mathbb{Z}\).
The input string is positioned such that its first symbol is matched with the position \(0\) 
of the tape; all the positions before the first symbol and after the last symbol are filled 
with the \emph{blank} symbol \(\sqcup \).
The computation \(\call{\mathcal{M}}{\overbar{\sigma}}\) starts in the initial state \(q_0\) with 
the \emph{head} of the TM positioned over the position \(0\) of the tape, and proceeds according to 
the transition function: depending on the current state \(q\) and the symbol \(\sigma \) written in 
the current location of the head, it replaces \(\sigma \) with a new symbol \(\sigma'\), it moves 
the head to the left (\(\leftarrow \)) or to the right (\(\rightarrow \)) and it transitions into a 
new state \(q'\).
The computation \emph{terminates} whenever one of the two \emph{halting} states is reached: if
\(\call{\mathcal{M}}{\overbar{\sigma}} = \bot \), then the input string \(\overbar{\sigma}\) is 
\emph{rejected}, else if \(\call{\mathcal{M}}{\overbar{\sigma}} = \top \), then \(\overbar{\sigma}\) 
is \emph{accepted}.
It can also happen that the computation does not terminate: in such cases, we write 
\(\call{\mathcal{M}}{\overbar{\sigma}} = {\uparrow}\) and we say that the computation \emph{hangs}.

In many scenarios, it is useful to extend Turing machines to include additional features, for 
example to represent the ability to access some source of randomness, or to communicate with an 
external environment to read inputs and produce outputs in an interactive manner.
\begin{definition}[Input/Output Turing machine]
  An input/output Turing machine is a quadruple \(\mathcal{M} = \Tuple{\Sigma, Q, q_0, \delta}\)
  where \(\Sigma \), \(Q\) and \(q_0\) are defined as in \Cref{def:turing_machine}, and:
  \[
    \delta\colon {\Parens*{Q \setminus \Set{\bot, \top}} \times \Sigma^2} \to 
    {Q \times \Sigma^2 \times \Set{\leftarrow, \rightarrow}^2}
  \]  
\end{definition}

The additional parameters in the transition function of an input/output Turing machine (I/O TM) 
account for two new tapes, namely the \emph{read-only input tape} \(\Tapein \) and the 
\emph{write-only output tape} \(\Tapeout \): now, depending on the state \(q\), the input symbol 
\(\sigma_{i}\) and the working symbol \(\sigma_{w}\), the machine overwrites \(\sigma_{w}\) with a 
new symbol \(\sigma'_{w}\) and moves left/right on \(\Tapework \), it writes a new symbol 
\(\sigma_{o}\) on \(\Tapeout \), where it can move only to the right, and it moves to the 
left/right on \(\Tapein \).
Additionally, in an I/O TM, the input string is placed on \(\Tapein \) instead of \(\Tapework \), 
which is instead blank at the beginning of the computation.
\begin{definition}[Probabilistic Turing machine]
  A probabilistic Turing machine is a quadruple \(\mathcal{M} = \Tuple{\Sigma, Q, q_0, \delta}\)
  where \(\Sigma \), \(Q\) and \(q_0\) are defined as in \Cref{def:turing_machine}, and:
  \[
    \delta\colon {\Parens*{Q \setminus \Set{\bot, \top}} \times \Sigma \times \Set{0, 1}} \to 
    {Q \times \Sigma \times \Set{\leftarrow, \rightarrow}}
  \]    
\end{definition}

In a probabilistic Turing machine (PTM), we have an additional \emph{read-only random tape} 
\(\Taperand \) which is populated with an infinite, uniformly random sequence of \emph{coin tosses} 
(zeros and ones), that are used by the transition function to decide what to do.
As for the writing tape of an I/O TM, the head on \(\Taperand \) can only move to the right.
\begin{definition}[Interactive Turing machine]
  An interactive Turing machine is a quadruple \(\mathcal{M} = \Tuple{\Sigma, Q, q_0, \delta}\)
  where \(\Sigma \), \(Q\) and \(q_0\) are defined as in \Cref{def:turing_machine}, and:
  \[
    \delta\colon {\Parens*{Q \setminus \Set{\bot, \top}} \times \Sigma^2} \to
    {Q \times \Sigma^2 \times \Set{\leftarrow, \rightarrow}}
  \]
\end{definition}

An interactive Turing machine (ITM) is quite similar to an I/O TM, as it also has two additional 
tapes, called the \emph{send tape} \(\Tapesend \) and the \emph{receive tape} \(\Taperec \).
However, unlike for the input tape \(\Tapein \) of an I/O TM, the head on \(\Taperec \) cannot move 
backwards.
\begin{remark}  
  Our definition of ITM differs slightly from the standard one in the 
  literature~\cite{GoldreichMW1991,GoldwasserMR1989}, but we find it to be more modular.
  In any case, from now on, we will say \emph{interactive Turing machine} to actually mean an 
 \emph{interactive, probabilistic, input/output Turing machine}.
\end{remark}

\begin{definition}[Interactive protocol]
  An interactive protocol is a pair \(\mathcal{I} = \Tuple{\mathcal{M}, \mathcal{M'}}\)
  where \(\mathcal{M}\) and \(\mathcal{M'}\) are interactive Turing machines such that 
  \(\Tapein = \Tapein'\), \(\Tapesend = \Taperec' \), \(\Taperec = \Tapesend'\), and their 
  state sets \(Q, Q'\) contain the special \emph{idle state} \(q_{idle} \in Q, Q'\).
\end{definition}

The computation of an interactive protocol (IP) over some string \(\overbar{\sigma}\), 
\(\call{\mathcal{I}}{\overbar{\sigma}}\), proceeds in the following manner: 
initially, the tapes \(\Tapesend \), \(\Taperec \), \(\Tapework \), \(\Tapework' \), \(\Tapeout \) 
and \(\Tapeout' \) are all empty (i.e.\ filled with blank symbols), the tapes \(\Taperand \) and 
\(\Taperand' \) are filled with random bits, and the tape \(\Tapein \) contains \(\overbar{\sigma}\).
The ITM \(\mathcal{M}\) is said to be \emph{active} and works normally until it transitions in the 
special state \(q_{idle}\), becoming \emph{inactive}.
When this happens, control passes to \(\mathcal{M}'\), which becomes active and works normally 
until it reaches its own idle state; control goes back to \(\mathcal{M}\), and the process repeats.
When one of the two machines halts, control passes over the other one until it also halts.
The protocol \emph{succeeds} if both machines halt in the accepting state \(\top \), and it 
\emph{fails} if at least one of them halts in the rejecting state \(\bot \).
To denote the final states reached by one of the machines at the end of the computation, 
we write \(\call{\mathcal{I}_{\mathcal{M}}}{\overbar{\sigma}}\) and 
\(\call{\mathcal{I}_{\mathcal{M}'}}{\overbar{\sigma}}\) respectively.
\Cref{fig:interactive_protocol} depicts the fundamental structure of an interactive protocol.

\begin{figure}
  \centering
  \begin{tikzpicture}[node distance=64pt,on grid,auto]
    \node[state,shape=rectangle,minimum height=16pt, minimum width=48pt] (in)  {\(\Tapein = \Tapein'\)};
    \node[state,shape=rectangle,minimum size=24pt,left =of in]   (m0)         {\(\mathcal{M}\)};
    \node[state,shape=rectangle,minimum size=24pt,right =of in] (m1)  {\(\mathcal{M}'\)};
    \node[state,shape=rectangle,minimum height=16pt, minimum width=48pt,above left =of m0] (p0)  {\(\Taperand \)};
    \node[state,shape=rectangle,minimum height=16pt, minimum width=48pt,above right =of m1] (p1)  {\(\Taperand' \)};
    \node[state,shape=rectangle,minimum height=16pt, minimum width=48pt,below left =of m0] (w0)  {\(\Tapework \)};
    \node[state,shape=rectangle,minimum height=16pt, minimum width=48pt,below right =of m1] (w1)  {\(\Tapework' \)};
    \node[state,shape=rectangle,minimum height=16pt, minimum width=48pt,above =of in] (s0)  {\(\Tapesend = \Taperec' \)};
    \node[state,shape=rectangle,minimum height=16pt, minimum width=48pt,below =of in] (r0)  {\(\Taperec = \Tapesend' \)};
    \node[state,shape=rectangle,minimum height=16pt, minimum width=48pt,left =of m0] (o0)  {\(\Tapeout \)};
    \node[state,shape=rectangle,minimum height=16pt, minimum width=48pt,right =of m1] (o1)  {\(\Tapeout' \)};
    \path[->]
    (in) edge (m0)
    (in) edge (m1)
    (p0) edge (m0)
    (p1) edge (m1)
    (w0) edge (m0)
    (m0) edge (w0)
    (w1) edge (m1)
    (m1) edge (w1)
    (m0) edge (s0)
    (s0) edge (m1)
    (m1) edge (r0)
    (r0) edge (m0)
    (m0) edge (o0)
    (m1) edge (o1)
    ;
  \end{tikzpicture}
  \caption{Visualization of an interactive protocol.}\label{fig:interactive_protocol}
\end{figure}

\section{Problems and complexity}\label{sec:complexity}
Historically, the most important class of problems that have been analyzed are so-called
\emph{decision problems}, i.e.\ probles whose solution is a binary \emph{yes-or-no} 
answer~\cite{Sipser2013}.
This perfectly suits Turing machiens as we can interpret their acceptance or rejection of the input 
string respectively as a yes and a no answer.

\begin{definition}[Kleene's closure]
  The Kleene's closure of a set \(S\) is the set \(S^* = \bigcup_{n \in \mathbb{N}}{S^n}\).
\end{definition}

As Turing machine operate over strings in \(\Sigma^*\), also called \emph{words}, they partition 
\(\Sigma^* \) into three \emph{languages} (a language is any set of strings): the language of 
accepted words, the languages of rejected words and the language of hanging words.
\begin{definition}[Turing-recognizable language]
  A language \(L \subseteq \Sigma^*\) is recognized by some Turing machine \(\mathcal{M}\) if 
  \(\forall w \in L\colon \call{\mathcal{M}}{w} = \top \).
\end{definition}
\begin{definition}[Turing-decidable language]
  A language \(L \subseteq \Sigma^*\) is decided by some Turing machine \(\mathcal{M}\) if it is 
  recognized by \(\mathcal{M}\) and \(\forall w \notin L\colon \call{\mathcal{M}}{w} = \bot \).
\end{definition}

We denote the language recognized by a Turing machine \(\mathcal{M}\) with \(\call{L}{\mathcal{M}}\).
To solve an \emph{instance} \(\Pi \) of some decision problem \textsc{prob}, we first encode the 
instance into a string \(\Encode{\Pi} \in \Sigma^*\) such that 
\(\Encode{\Pi} \in \call{L}{\mathcal{M}}\) if and only if the answer to \(\Pi \) is `yes'.

The class of recognizable languages, called \textsc{RE}, strictly includes the class of decidable 
languages, called \textsc{DEC}~\cite{Turing1937}.
But even decidable languages are not all equal: their \emph{computational complexity}, that is,
the amount of some resource which is required by a Turing machine to decide membership words in 
function of their length, can vary wildly.
In general, we are only interested in the \emph{asymptotic behaviour} of the machine.
\begin{definition}[Big-O notation]
  Given two functions \(f, g\colon \mathbb{N} \to \mathbb{N}\), then \(f = \BigO{g}\) if 
  and only if \(\exists c,n\) such that \(\forall x \ge n\) \(\call{f}{x} \le c \cdot \call{g}{x}\).
\end{definition}

We write \(f = \BigOmega{g}\) when \(g = \BigO{f}\), and we write \(f = \BigTheta{g}\) when 
\(f = \BigO{g}\) and \(g = \BigO{f}\).
When there exists a TM \(\mathcal{M}\) for which some complexity metric \(\Complexity \) is 
upper-bounded at most by a polynomial function in the length \(n\) of the input word 
(i.e.\  \(\call{\Complexity}{\mathcal{M}} = \BigO{n^c}\) for some constant \(c \in \mathbb{N}\)), 
we say that deciding the language is \emph{feasible} w.r.t.\  \(\Complexity \). 
On the other hand, if \(\Complexity \) is upper-bounded at least by an 
exponential function (i.e.\  \(\call{\Complexity}{\mathcal{M}} = \BigO{c^{n}}\) for some constant 
\(c > 1\)), we say that the problem is \emph{infeasible} w.r.t.\  \(\Complexity \).
The standard complexity metrics are \emph{time} \(\Time \), that is the amount of transition steps 
a TM performs before halting, and \emph{space} \(\Space \), that is the amount of tape locations 
visited by a TM before halting\footnote{It is always the case that \(\Space \le \Time \).}.

Two of the most important 
\emph{complexity classes}\footnote{\url{https://complexityzoo.net/Complexity_Zoo}} 
are \textsc{PTIME} (\Ptime{} for short) and \textsc{NPTIME} (\NPtime{} for short), which are the 
classes of languages decidable respectively by a deterministic TM and a nondeterministic TM using 
at most polynomial time.
While we do not know if \(\Ptime \maybeequals \NPtime \), it is widely believed that 
\(\Ptime \subset \NPtime \): for a deterministic Turing machine, deciding \NPcomplete{} problems 
(i.e.\ the hardest problems in \NPtime{}) will generally take an exponential amount of 
time, and there is no known way in the physical world to build non-deterministic Turing machines.
Although quantum computers have been shown to be able to crack problems which are believed to 
be infeasible for standard computers, like integer factorization~\cite{Shor1994}, \NPcomplete{}
problems seem to be out of reach also for such powerful machines.

\section{Interactive proof systems}
Even though \NPcomplete{} problems are infeasible, they are \emph{efficient} to \emph{verify}: 
given an \emph{instance} \(\Pi \) of some \NPcomplete{} problem, and an additional \emph{witness}
string, we can build a deterministic TM that checks whether the witness \emph{proves} or 
not that the problem admits a positive answer.
\begin{example}
  Consider the problem \textsc{sat} of deciding whether a propositional logic formula \(\phi \) is 
  satisfiable, which is the most famous \NPcomplete{} problem~\cite{Cook1971}. 
  If we had a TM \(\mathcal{M}\) with access to an \emph{oracle} that, in \(\BigO{1}\) time, 
  provides a valid assignment for the variables in \(\phi \), it would be easy to 
  verify that \(\phi \) is indeed satisfiable.
  However, if the provided assignment was not valid, while \(\mathcal{M}\) would reject it, 
  there would be still no easy way to know whether \(\phi \) is actually satisfiable or not! 
\end{example}

Now, let's say we want to prove some theorem \(\Pi \): computationally, this is equivalent to 
deciding whether \(\Pi \) is word which belongs to the language of the valid propositions over some 
formal system (say, the ZFC set theory~\cite{FraenkelHL1973}).
A \emph{proof} of the theorem plays the same role of the \emph{witness} we discussed before: in 
general, verifying a proof for a theorem is (believed to be) much easier than finding the proof in 
the first place.
Hence, we can extend the logical/mathematical concept of theorem to the more computational concept 
of language: for example, by \NPtime{} theorem, we mean any language in \NPtime{}.

\clearpage
\begin{definition}[Interactive proof system~\cite{GoldwasserMR1989}]  
  An interactive proof system for a language \(L\) is an interactive protocol 
  \(\mathcal{I} = \Tuple{\mathcal{P}, \mathcal{V}}\), where \(\mathcal{P}\) is the \emph{prover}
  and \(\mathcal{V}\) is the \emph{verifier}, such that:
  \begin{align*}
    & \forall w \in L\colon \call{\mathcal{I}_{\mathcal{V}}}{w} = \top & 
      \textnormal{(\emph{correctness})} \\
    & \forall w \notin L\colon \call{\mathcal{I}_{\mathcal{V}}}{w} = \bot & 
      \textnormal{(\emph{completeness})} \\
    & \exists k \in \mathbb{N}\colon \call{\Time}{\mathcal{V}} = \BigO{\abs{x}^k} & 
    \textnormal{(\emph{boundness})}
  \end{align*}
\end{definition}

In an interactive proof system (IPS), the common input tape of \(\mathcal{P}\) and \(\mathcal{V}\)
contains some word \(w\) representing some statement: in typical scenario, the statement is 
provided by the prover himself, who wants to convince the verifier of the truthness of such 
statement.
During the protocol, \(\mathcal{P}\) and \(\mathcal{V}\) exchange messages through their 
communication tapes; at some point, \(\mathcal{P}\) sends to \(\mathcal{V}\) a candidate proof 
\(\pi \): the verifier checks the proof and, if the proof is valid, it is always convinced of its 
validity (correctness), hence it will accept. 
On the other hand, if the proof happens to be wrong (e.g.\ if \(\mathcal{P}\) is trying to deceive
\(\mathcal{V}\)), then the verifier will never be convinced by such a proof, and it will reject.
The polynomial bound on the execution time of \(\mathcal{V}\) is necessary to force cooperation, 
and avoid the case where \(\mathcal{V}\) simply ignores \(\mathcal{P}\) and computes the proof by 
itself.
\begin{definition}[Probabilistic interactive proof system~\cite{GoldwasserMR1989}]  
  A probabilistic interactive proof system for a language \(L\) is an interactive protocol 
  \(\mathcal{I} = \Tuple{\mathcal{P}, \mathcal{V}}\) such that, for any arbitrarily small 
  \(\epsilon \in \mathbb{R}_{+}\):
  \begin{align*}
    & \forall w \in L\colon \call{\Pr}{\call{\mathcal{I}_{\mathcal{V}}}{w} = \bot} < \epsilon  & 
      \textnormal{(\emph{probabilistic correctness})} \\
    & \forall w \notin L\colon \call{\Pr}{\call{\mathcal{I}_{\mathcal{V}}}{w} = \top} < \epsilon & 
      \textnormal{(\emph{probabilistic completeness})} \\
    & \exists k \in \mathbb{N}\colon \call{\Time}{\mathcal{V}} = \BigO{\abs{x}^k} & 
    \textnormal{(\emph{boundness})}
  \end{align*}
\end{definition}

\section{Zero-Knowledge Protocols}
Suppose that two parties are executing an IARK system for some hard problem: the instance is places 
on the shared input tape, and also suppose that the secret in possession of the prover is simply 
a witness for the instance. 
All the prover has to do is send the witness to the verifier, which will in turn check it and 
decide whether to accept or not.
In this process, the verifier gained more knowledge than just the solvability of the problem: it 
also learned a solution, and not just any solution, but exactly the one available to the prover,
which should have been a secret.
To address this issue, researchers started exploring the field of so-called zero-knowledge 
proofs~\cite{GoldwasserMR1989,GoldreichMW1991}.

Informally, two random variables \(U\) and \(V\) that map words of some language 
\(L \subseteq \Set{0, 1}^{*}\) to words of \(\Set{0, 1}^{*}\) are 
\emph{perfectly indistinguishable} when no unbounded Turing machine is able to tell them apart,
are \emph{statistically indistinguishable} when no \textsc{PSPACE} Turing machine is able to 
tell them apart, and are \emph{computationally indistinguishable} when no \textsc{PTIME} Turing 
machine \(\mathcal{M}\) is able to tell them apart.
By `telling apart', we mean that the distribution of the words that are accepted/rejected 
by Turing machines respecting the imposed bounds is independent from \(U\) and \(V\): intuitively,
this means that \(U\) and \(V\) are interchangable with each other and using one over the other 
does not give an `edge' to \(\mathcal{M}\)~\cite{GoldwasserM1984,GoldwasserMR1989,Yao1982}.
\begin{example}
  Consider the two random variables \(U, V: L \to \Set{0, 1}^{*}\) for some 
  \(L \subseteq \Set{0, 1}^{*}\), such that, for all words \(x \in L\) and all words 
  \(w \in \Set{0, 1}^{\abs{x}}\), it holds that:
  \begin{align*}
    & \call{\Pr}{\call{U}{x} = w} = 2^{-\abs{x}} &&
    \call{\Pr}{\call{V}{x} = w} = \begin{cases}
      0 & x = 0\dots0 \\
      2^{-\abs{x} + 1} & x = 1\dots1 \\
      2^{-\abs{x}} & \textrm{otherwise}
    \end{cases}
  \end{align*}
  \(U\) and \(V\) have \emph{almost} the same distribution, with the \(1\dots1\) string 
  happening twice as often in \(V\). 
  For increasingly longer strings, no Turing machine can tell the two distributions apart by 
  collecting a polynomial amount of samples, since 
  \(\sum_{w}{\abs{\call{\Pr}{\call{U}{x} = w} - \call{\Pr}{\call{V}{x} = w}}} = 2^{-\abs{x}+1}\),
  hence \(U\) and \(V\) are statistically indistinguishable.
\end{example}

\begin{definition}[Tape view]
  A \emph{tape view} is a random variable \(\View_{\mathcal{M}}\) that models the concatenation 
  of all the contents that are read/written by a halting Turing machine \(\mathcal{M}\) over its 
  tapes.
\end{definition}

For a deterministic, non-probabilistic Turing machine, the tape view variable is quite pointless, 
but it is a useful tool to model the behaviour of machines that exploit randomness, and expecially 
for interactive protocols.
For example, if we have a Turing machine with one tape \(\Tape \), then:
\[
  \call{\Pr}{\call{\View_{\mathcal{M}}}{x} = w} = 
  \call{\Pr}{\call{\View_{\mathcal{M}, \Tape}}{x} = w} = 
  \call{\Pr}{\call{\mathcal{M}}{x} = w}
\]


\begin{definition}[Approximability]
  A random variable \(U\) is (perfectly, statistically, computationally) \emph{approximable} by a 
  probabilistic Turing machine \(\mathcal{M}\) over some language \(L\) if \(U\) and 
  \(\View_{\mathcal{M}}\) are (perfectly, statistically, computationally) indistinguishable.
\end{definition}

Note that for a random variable \(U\) and a halting PTM \(\mathcal{M}\) to be perfectly 
indistinguishable over some language \(L\), it must be the case that 
\(\forall x \in L\colon \call{\mathcal{M}}{x} = \call{\View_{\mathcal{M}}}{x} = \call{\mathcal{U}}{x}\).

\begin{definition}[Zero-knowledge interactive protocol]
  A (perfectly, statistically, computationally) \emph{Zero-knowledge interactive protocol} (ZKIP) 
  over a language \(L \subseteq \Set{0, 1}^{*}\) is an interactive protocol 
  \(\mathcal{I} = \Tuple{\mathcal{M}, \mathcal{M}'}\) such that, for every \(\mathcal{M'}\),
  \(\View_{\mathcal{M'}}\) is (perfectly, statistically, computationally) approximable by a 
  Turing machine \(\mathcal{M}''\) over the language 
  \(L' = \Set{\Tuple{x, h} \mid x \in L \wedge w \in \Set{0, 1}^*}\), where the string
  \(h\) represents the initial content of \(\Tapework'\).
\end{definition}

Naturally, a ZKIP which is also a proof system is a zero-knowledge proof system (ZKPS); similarly, 
if it is an interactive argument of knowledge system then it is a zero-knowledge interactive 
argument of knowledge system (ZK-IARK).
From now on, by zero-knowledge we mean computational zero-knowledge, as assuming a polynomial-time 
bounded adversary is an acceptable restriction in the real world.
The initial string \(h\) of a ZKIP can be interpreted as the \emph{history} of previous 
interactions with the prover, or some eavesdropped information from the interactions that the 
prover had with other verifiers.

A proof system being zero-knowledge basically means that, even for curious or malicious verifiers,
and even with additional knowledge on the behaviour of the prover, what can be computed is nothing 
more than what could have been computed in polynomial time, hence within the imposed computational 
power limits, without communicating with the prover.
While it is obvious that every problem solvable in probabilistic polynomial time 
(\textsc{PP}) has a zero-knowledge proof system (the prover does nothing and the verifier computes 
the solution by himself), it was proven that also all problems in \textsc{NP} have a 
ZKPS~\cite{GoldreichMW1991}. 
By assuming the existance of secure probabilistic encryption, it was finally shown that also 
all Arthur-Merlin games, and hence all problems in \textsc{IP}, have a ZKPS~\cite{BenorGGHKMR1990}.

\subsection{Non interactive Zero-Knowledge}\label{subsec:nizk}
In many scenarios, especially ones involving multiple parties, interaction can be a problem as
the communication cost of bidirectional \(n\)-to-\(n\) grows quadratically.
Such cases are in fact of great interest for zero-knowledge systems: multiple parties can be 
both provers and/or verifiers, and their number might be huge.

For this reason, researches explored the possibility of having zero knowledge \emph{non-interactive} 
proof systems (ZK-NPS) or argument of knowledge systems (ZK-NARK).
Unfortunately, only the languages in \textsc{BPP}, that is languages decidable in 
probabilistic polynomial time with a bounded error allow for zero-knowledge non-interactive 
proofs~\cite{Oren1987,GoldreichK1996}. Such languages are of course trivial, as the verifier has 
enough power to do all the computation by itself without the need of the prover.

However, by introducing an initial \emph{preprocessing} phase, it is possible to regain the lost 
power~\cite{SantisMP1990}, and the most prominent technique to achieve non-interaction is the 
\emph{Common Reference String} (CRS) model~\cite{BlumFM1988} (sometimes also called 
\emph{common random string} model).
The main idea of the CRS model is that, before engaging in the protocol, the prover and 
the verifier have both obtained access to a shared string of random bits. 
In the simplest case, the string is generated by a \emph{trusted third party}, although in 
practice this is oftentimes not a viable solution as the whole point of zero-knowledge is having 
to deal with untrusted parties. 
To circumnent this problem, it is possible to generate the CRS by a \emph{majority vote}
between \(n\) authorities, which can be untrusted if picked singularly, but are assumed to be 
honest in their majority~\cite{GrothO2006}.
In fact, it was shown that it is possible, without losing zero-knowledge, to re-use multiple times 
a single CRS both by a single~\cite{BlumSMP1991} or multiple~\cite{FeigeLS1990} provers, although 
only for arguments of knowledge and not for proofs.

The first zero-knowledge systems, both interactive and non-interactive, were tailor-made for 
specific problems, such as the quadratic residuosity problem \textsc{qr}~\cite{GoldwasserMR1989}, 
the hamiltonian path problem \textsc{hampath}~\cite{LapidotS1991}, or the \(3\)-\textsc{sat} 
problem~\cite{BlumSMP1991}.
Although for any \textsc{NP-complete} problem \textsc{prob} there is a polynomial-time 
algorithm~\cite{Karp1972} that converts every instance of \textsc{prob} to an instance 
of, say, \(3\)-\textsc{sat}, such reductions are often not trivial to devise and very expensive 
to apply.
For this reason, researchers started devising constructions to prove arbitrary NP statements 
embedded in the form of boolean circuits~\cite{Damgard1993}, which can neatly represent the 
computation of a Turing machine over any \textsc{NP-complete} problem~\cite{Cook1971}, and 
therefore remove the need to go through polynomial-time reductions.

The first of such systems~\cite{Damgard1993} required a CRS of size cubic in the length of the 
statement to be proven, although XOR and NOT gates didn't need to consume any bits from the CRS\@.
In the following years, many improvements were proposed, reducing the complexity of the 
constructions from cubic to subquadratic~\cite{BoyarBP1995} and eventually 
linear~\cite{CramerD1997}.


\chapter{Cryptographical Background}\label{chap:crypto}
The main application of Zero Knowledge proof systems has been, arguably unsurprisingly, in the 
cryptography field.
The possibility of two or more parties to cooperate and exchange information one with another in a 
zero-knowledge manner is the fundamental idea behind many branches of cryptography such as 
\emph{Multi Party Computation} (MPC)~\cite{Yao1982-2} and \emph{Fully Homomorphic Encryption} 
(FHE)~\cite{ArmknechtEtAl2015}.

The main application of Zero knowledge protocols has been in \emph{blockchain} infrastructures, 
with the cryptocurrency \emph{ZCash} being the most prominent example~\cite{SassonCGGMTV2014}.
In a public blockchain, a user (the prover) wants to convince the other users (the verifiers) that 
he posseses some data: to this end, he exhibits a \emph{commitment}, typically a short message 
which can be easily computed when knowing the original data, but for which it is hard to find a 
\emph{collision}.
In this scenario, the prover would like to be able to convince the verifiers of the validity of 
its commitment, without having to hand them out the original data.

\input{parts/hash/arion.tex}
\section{Tree-like modes of hashing}\label{sec:tree_hash}
Consider an \(n\)-bit CHF \(H\), and suppose that a prover claims to know some message \(m\): 
the digest \(d = \call{H}{m}\) can be considered as a \emph{short binding commitment} for \(m\): 
By asking the prover to share the digest, whose size \(\abs{d} = n\) is typically considered to be 
\(\BigO{1}\) (or \(\BigO{\call{\log}{\abs{m}}}\) in some cases), a verifier is convinced that the 
prover does know \(m\) with probability \(\approx 1 - {1}/{2^n}\).
A modern standard CHF like SHA-256~\cite{Dang2015} produces digests of length at least \(256\) bits,
making the \(1 - {1}/{2^n}\) bound really hard to bruteforce through.
Note that the verifier needs not to know \(m\) in advance: the commitment \(d\) is (temporarily) 
appended to a public \emph{blockchain} and, at any point in the future, when the verifier becomes 
aware of some \(m'\) provided by the prover, if \(\call{H}{m'} = d\), the commitment can be 
approved or rejected.

Now, suppose that the prover wants to commit to a list of \(k\) messages: the simplest solution 
would be to publish the hash of every message, which would require to append \(\BigO{k}\) 
elements on the blockchain.
Another way would be for the prover to share \(\call{H}{\Tuple{m_1, \dots, m_k}}\): the 
communication cost would only be \(\BigO{1}\) but, in general, not all the messages belong to the 
same prover, so this method would not work, and we need a better solution.

\subsection{Merkle tree}
\begin{definition}[Binary Merkle tree~\cite{Merkle1988}]
	A \emph{binary Merkle tree (MT)} of height \(h \in \mathbb{N}\) over a \(2n\)-to-\(n\) compression 
	function \(C\), is the complete binary tree of height \(h\) such that, given a sequence of input 
	messages \(\Tuple{m_1, \dots, m_{2^{h-1}}}\) over \(\Set{0, 1}^{2n}\), produces an
	output digest \(d \in \Set{0, 1}^{n}\) in the following way:
	\begin{enumerate}
		\item The leaf nodes \(\nu_1, \dots, \nu_{2^{h-1}}\) contain 
					\(\call{C}{m_1}, \dots, \call{C}{m_{2^{h-1}}}\).
		\item Every other node \(\nu \) contains \(\call{C}{\nu_l, \nu_r}\), where \(\nu_l\) is
		      the left child of \(\nu \) and \(\nu_r\) is the right child of \(\nu \).
		\item The output digest \(d\) is the content of the root node. 
	\end{enumerate}
\end{definition}

The set of the sibling nodes visited in the path from a leaf of the tree to the root, including the 
leaf itself, is the \emph{authentication path} of the leaf.
By using Merkle trees, the prover only needs to send to the verifier, as a commitment for
some message \(m_i\) among \(n = 2^h\) messages, the contents of the co-path from the leaf 
containing \(m_i\) to the root, in addition plus the hash of \(m_i\): this requires
just \(\BigO{\call{\log}{n}}\) cost to validate the commitment.
Merkle trees bottom-up construction is very easy to parallelize, and they can be used in the 
multiple-provers scenario: each prover only needs to commit to the path from its own leaf to the 
root of the tree.
It is immediate to generalize the notion of binary Merkle tree to arbitrary arity.
\begin{proposition}[Security of Merkle tree mode of hash~\cite{Merkle1988}]
	Given a one-way \(tn\)-to-\(n\) compression function \(C\), the \(t\)-ary Merkle tree over 
	\(C\) is a cryptographic hash function.
\end{proposition}

\begin{example}\label{ex:merkle_tree}
	Consider the sequence of pre-hashed messages \(S = \Tuple{3, 4, 7, 7}\) and the 
	compression function 
	\(\call{C}{x, y}: \Tuple{x, y} \mapsto \Parens*{xy \bmod 13} + 1\) (for ease of exposition, 
	we work over integers instead of bit strings, but the two can be readily converted into one 
	another).
	\Cref{fig:merkle_tree} shows the contents of the associated Merkle Tree.
	Note that the real message is not stored in the Merkle Tree, but only the `first level' of hashes.
	The authentication path of the leaf labelled with \(3\) consists of the tuple \(\Tuple{3, 4, 11}\):
	by computing \(\call{H}{3, 4} = 13\) and then \(\call{H}{13, 11} = 1\) we can verify that the 
	commitment is respected.
\end{example}
\begin{figure}
	\centering
	\begin{tikzpicture}[node distance={32pt}, node/.style = {draw, circle},on grid=true]
		\node[node] (x1) {\(3\)};
		\node[node,draw=none] (n1) [right of=x1] {};
		\node[node] (x2) [right of=n1] {\(4\)};
		\node[node,shape=rectangle] (c1) [above of=n1] {\(C\)};
		\node[node,draw=none] (n2) [right of=x2] {};
		\node[node] (x3) [right of=n2] {\(7\)};
		\node[node,draw=none] (n3) [right of=x3] {};
		\node[node] (x4) [right of=n3] {\(7\)};
		\node[node,shape=rectangle] (c2) [above of=n3] {\(C\)};
		\node[node] (x5) [above of=c1] {\(13\)};
		\node[node,draw=none] (n3) [above of=n2] {};
		\node[node,draw=none] (n4) [above of=n3] {};
		\node[node] (x6) [above of=c2] {\(11\)};
		\node[node,shape=rectangle] (c3) [above of=n4] {\(C\)};
		\node[node] (x7) [above of=c3] {\(1\)};
		\draw[->] (x1) to (c1);
		\draw[->] (x2) to (c1);
		\draw[->] (x3) to (c2);
		\draw[->] (x4) to (c2);
		\draw[->] (c1) to (x5);
		\draw[->] (c2) to (x6);
		\draw[->] (x5) to (c3);
		\draw[->] (x6) to (c3);
		\draw[->] (c3) to (x7);
	\end{tikzpicture}
	\caption{Merkle tree of \Cref{ex:merkle_tree}.}\label{fig:merkle_tree}
\end{figure}

\subsection{Augmented Binary Tree}
The Merkle tree is the de-facto standard for blockchain applications, and basically for any 
scenario for which a `linear' hash function cannot be used.
In~\cite{Stam2008}, it was given a lower bound on the amount of queries necessary to obtain a 
collision for a \(\Parens*{m+s}\)-to-\(s\)-bit CHF \(H\) (the \(m\) is variable) built from a 
\(\Parens*{n+c}\)-to-\(n\)-bit OWCF \(C\): if \(H\) makes \(r\) queries to \(C\), it is possible 
to find a collision by making \(2^{\frac{nr + cr - m}{r + 1}}\) queries to \(H\).
By combining this result with the \(2^{s/2}\) upper bound of the birthday paradox, one can 
immediately obtain a tight bound \(m = \frac{2nr + 2cr -sr - s}{2}\) for the variable length \(m\) 
of the message.

\begin{definition}[Compactness~\cite{AndreevaBR2021}]
	The \emph{compactness} of an \(\Parens*{m+s}\)-to-\(s\)-bit hash function making \(r\) queries to 
	an underlying \(\Parens*{n+c}\)-to-\(n\)-bit one-way compression function is the value
	\(\alpha = \frac{2m}{2nr + 2cr -sr - s}\).
\end{definition}

\begin{example}\label{ex:mtree_compactness}
	Consider a \(2n\)-to-\(n\) bit OWCF and a Merkle Tree of height \(h\): the computation 
	of the tree is a \(\Parens*{2^{h-1}n}\)-to-\(n\)-bit hash function, and makes exactly 
	\(r = 2^{h-1} - 1\) queries to \(C\).
	We have \(s = c = n\) and \(m = 2^{h-1}n - n = nr\), therefore the compactness of the Merkle 
	Tree construction is:
	\[
		\alpha = \frac{2m}{2nr + 2cr -sr - s} = 
		\frac{2nr}{2nr + 2nr - nr - n} =
		\frac{2r}{3r - 1}
	\]
	Which tends to \(2/3\) when \(r\) tends to infinity.
\end{example}

\begin{definition}[Augmented Binary tRee~\cite{AndreevaBR2021}]
	An \emph{Augmented Binary tRee (ABR)} of height \(h \in \mathbb{N}\) over a 
	\(2n\)-to\(n\) compression function \(C\) is a complete binary tree of height \(h\) 
	augmented with \emph{middle} nodes such that, given a sequence of input messages
	\(S = \Tuple{m_1, \dots, m_{2^{h-1} + 2^{h-2}-1} \mid \forall i\colon m_i \in \Set{0, 1}^{*}}\), 
	it produces an output digest \(d \in \Set{0, 1}^n\) in the following way:
	\begin{enumerate}
		\item The leaf nodes \(\nu_{1}, \dots, \nu_{2^{h-1}}\) contain \(\call{C}{m_1}, \dots,
		      \call{C}{m_{2^{h-1}}}\).
		\item There are no middle nodes in the leaf layer.
		\item The middle nodes \(\nu_{2^{h-1}+1}, \dots, \nu_{\abs{S}}\) contain
		      \(\call{C}{m_{2^{h-1}+1}}, \dots, \call{C}{m_{\abs{S}}}\).
		\item Every other node \(\nu \) contains \(\call{C}{\nu_l \oplus \nu_m, \nu_r \oplus
		      \nu_m} \oplus \nu_r \), where \(\nu_l\) is the left child of \(\nu \), \(\nu_r\)
		      is the right child of \(\nu \), and \(\nu_m\) is the middle child of \(\nu \), or \(0\)
		      if \(\nu \) doesen't have a middle child.
	\end{enumerate}
\end{definition}

The authentication path of the ABR is similar to the one of the Merkle tree, but also includes 
the middle nodes encountered during the traversal.

\begin{proposition}[Security of ABR mode of hash~\cite{AndreevaBR2021}]
	Given a one-way \(2n\)-to-\(n\) compression function \(C\), the ABR over \(C\) is a cryptographic 
	hash function.
\end{proposition}

An ABR of height \(h\) can process 50\% more messages than a Merkle Tree of the same height, 
while performing the same number of queries to the underlying compression function, with the 
additional cost introduced by the intermediate \(\oplus \) operations being negligible in most 
scenarios.

\begin{example}
	Consider a \(2n\)-to-\(n\) bit OWCF and an ABR of height \(h\): the computation 
	of the tree is a \(\Parens*{2^{h-1} + 2^{h-2}-1}n\)-to-\(n\)-bit hash function, 
	and makes exactly \(r = 2^{h-1} - 1\) queries to \(C\).
	Like in \Cref{ex:mtree_compactness}, we have \(s = c = n\), but this time 
	\(m = \Parens*{2^{h-1} + 2^{h-2}-1}n - n = nr + {nr}/2 - n\), so the compactness of the ABR 
	construction is:
	\[
		\alpha = \frac{2m}{2nr + 2cr - sr - s} = 
		\frac{2nr + nr - 2n}{2nr + 2nr - nr - n} =
		\frac{3r - 2}{3r - 1}
	\]
	Which approaches \(1\) as \(r\) approaches infinity, meaning that the ABR construction achieves
	optimal compactness.
\end{example}

It is worth of notice that, while the ABR hash mode achieves collision resistance, it does not 
achieve \emph{indifferentiability} (a weaker notion of indistinguishability between Turing 
machines~\cite{MaurerRH2003}), hence a modified construction, called ABR+, was also proposed, 
although it does not achieve perfect compactness.
\begin{figure}
	\centering
	\begin{tikzpicture}[node distance={32pt}, node/.style = {draw, circle},on grid=true]
		\node[node] (x1) {\(3\)};
		\node[node,draw=none] (n1) [right of=x1] {};
		\node[node] (x2) [right of=n1] {\(4\)};
		\node[node,shape=rectangle] (c1) [above of=n1] {\(C\)};
		\node[node,draw=none] (n2) [right of=x2] {};
		\node[node] (x3) [right of=n2] {\(7\)};
		\node[node,draw=none] (n3) [right of=x3] {};
		\node[node] (x4) [right of=n3] {\(7\)};
		\node[node,shape=rectangle] (c2) [above of=n3] {\(C\)};
		\node[node,draw=none] (n4) [above of=n2] {};
		\node[node] (x8) [above of=n4] {\(10\)};
		\node[node] (x5) [above of=c1] {\(13\)};
		\node[node] (x6) [above of=c2] {\(11\)};
		\node[node,draw=none] (n5) [above of=x8] {};
		\node[node,shape=rectangle] (c3) [above of=n5] {\(C\)};
		\node[node,shape=rectangle] (a1) [below left of=c3] {\(\oplus \)};
		\node[node,shape=rectangle] (a2) [below right of=c3] {\(\oplus \)};
		\node[node,shape=rectangle] (a3) [above of=c3] {\(\oplus \)};
		\node[node] (x7) [above of=a3] {\(10\)};
		\draw[->] (x1) to (c1);
		\draw[->] (x2) to (c1);
		\draw[->] (x3) to (c2);
		\draw[->] (x4) to (c2);
		\draw[->] (c1) to (x5);
		\draw[->] (c2) to (x6);
		\draw[->] (x5) to (a1);
		\draw[->] (x6) to (a2);
		\draw[->] (x8) to (a1);
		\draw[->] (x8) to (a2);
		\draw[->] (a1) to (c3);
		\draw[->] (a2) to (c3);
		\draw[->] (c3) to (a3);
		\draw[->] (x5) [bend left] to (a3);
		\draw[->] (a3) to (x7);
	\end{tikzpicture}
	\caption{ABR of \Cref{ex:abr}.}\label{fig:abr}
\end{figure}
\begin{example}\label{ex:abr}
	Consider the same compression function \(C\) of \Cref{ex:merkle_tree}, and consider the 
	sequence of pre-hashed messages \(S' = \Tuple{3, 4, 7, 7, 10}\), in this case we interpret 
	\(x \oplus y \equiv \Parens*{x + y \bmod 13} + 1\).
	\Cref{fig:abr} shows the resulting ABR\@.
	The authentication path of the node labelled with \(3\) consists of the tuple 
	\(\Tuple{3, 4, 10, 11}\): by computing \(\call{H}{3, 4} = 13\) and then 
	\(\call{H}{13 \oplus 10, 11 \oplus 10} \oplus 13 = 10\) we can verify that the commitment is 
	respected.
\end{example}

\section{ZK-SNARK systems}
As we saw in \Cref{subsec:nizk}, researchers were able to construct ZK-NARK systems whose 
verification complexity was linear in the size of the problem instance, which is provided as a 
boolean circuit.
Furthermore, in the CRS model, by using a block cipher, it is also possible to have 
\emph{publicly verifiable} constructions~\cite{LapidotS1991}, meaning that \emph{any} verifier, 
not just the one that engages the protocol, is able to check the proof, which is encrypted with a 
\emph{proving key}, by using a public \emph{verification key}.

\begin{proposition}[Fiat-Shamir heuristic~\cite{FiatS1987}]
  Suppose a probabilistic I/O TM \(\mathcal{P}\) with access to a CHF \(H\) wants to prove its 
  knowledge of the discrete logarithm \(x = \call{\log}{y}\) for some value 
  \(y \in \mathbb{Z}_p\), where \(p\) is a large prime number.
  Then \(\mathcal{P}\) can sample a random value \(v\) from \(\Taperand \), compute the digest 
  \(d = \call{H}{p, y, p^v}\), the result \(r = {v - dx} \bmod \Parens*{p - 1}\), and finally 
  output the quadruple \(\Tuple{p, y, p^v, r}\).
  Any \textnormal{\textsc{PTIME}} TM \(\mathcal{V}\) with access to \(\Tuple{p, y, p^v, r}\) 
  and \(H\) can recompute \(d\) and check whether \(p^v = p^{r}y^{d}\)
  (If \(\mathcal{P}\) is not cheating, then \(p^{r}y^{d} = p^{v - dx}\Parens*{p^{x}}^d = 
  p^{v - dx}p^{dx} = p^{v - dx + dx} = p^v\)).
  Assuming that the discrete logarithm is hard and that true CHF exist, if equality holds 
  \(\mathcal{V}\) is convinced that \(\mathcal{P}\) knows \(x\) but is not able to retrieve it 
  except with negligible probability.
\end{proposition}

\begin{definition}[Succint proof]
  A \emph{succint proof} for a statement \(\sigma \) over a language \(L \subseteq \Set{0, 1}^*\) 
  is a proof \(\pi \) such that \(\abs{\pi} = \BigO{\call{\log}{\abs{\sigma}}}\).
\end{definition}

Similarly, one can define the notion of succint argument of knowledge, and in particular, a 
succint ZK-NARK system is called a ZK-SNARK system.
\begin{definition}[Probabilistically checkable proof system~\cite{BabaiFLS1991,FeigeGLSS1991}]
  A \emph{probabilistically checkable proof system} (PCP system) is an interactive proof system 
  \(\Tuple{\mathcal{P, V}}\) such that for any proof \(\pi \) provided by \(\mathcal{P}\):
  \(\exists k\colon \call{\Time}{\mathcal{V}} = \BigO{\call{\log^k}{\abs{\pi}}}\).
\end{definition}

In a PCP system, the prover \(\mathcal{P}\) constructs a proof \(\pi \) of size polynomial in the 
length of the original statement \(\sigma \); since the verifier \(\mathcal{V}\) is 
polylogarithmically bound to the size of the proof, it can only query a small portion of it, 
however, it is enough to get statistical completeness and soundness.

In~\cite{Kilian1992}, the author uses Merkle trees to have the prover commit to a proof \(\pi \), 
(the bits of \(\pi \) are the leaves and the root, whcih has constant size, is sent to the verifier). 
The verifier then queries a certain number of authentication paths, which have length
\(\BigO{\call{\log}{\abs{\pi}}}\), and decides whether to accept or reject.
In this sense, the protocol is therefore succint.
In~\cite{Micali2000}, the construction was extended and, by applying the Fiat-Shamir heuristic, it 
is possible to make the protocol non-interactive.

One of the first \emph{succint} ZK-NARK (ZK-SNARK) systems that didn't make explicit use of 
PCPs was devised in~\cite{Groth2010}, but had one important drawback: while the size of the proof 
is constant, the size of the CRS, and the computation that the prover has to perform is 
\emph{quadratic} in the size of the input circuit
(this bound wass slightly improved in~\cite{Lipmaa2011}).

However, by first transforming the circuit into \emph{quadratic span programs} (QSPs), 
the boolean equivalent of QAPs (\Cref{subsec:qap}), it was possible to reduce both the size of the 
CRS and the prover's computational complexity to linear, while still having succint proofs.
Since all these constructions make use of encryption based on the hardness of finding the discrete
logarithm of a number over a big finite field, dealing with boolean circuits and QSPs is not 
very efficient; although both polynomially sized boolean and arithmetic circuits are equivalent to 
\textsc{PTIME} Turing machines~\cite{PippengerF1979}, working over arithmetic programs, and hence 
using R1CSs~\cite{CramerD1998} and QAPs over QSPs, can greatly reduce the constant factors involved 
in such constructions, although this depends on the kind of input problem (numeric problems 
can exploit arithmetic circuits much better than, say, propositional problems).

\subsection{Pinocchio}
An important application of ZK-SNARK systems is in \emph{verifiable computation}.
Consider a client (say, a mobile phone) that wants to delegate to a server (say, a cloud provider) 
some computation, for which several inputs are required: some are provided by the client, 
and some by the server:
\begin{itemize}
  \item The client does not trust the server, so we would need a proof system, but since the server 
        is not computationally unbounded, an \emph{argument of knowledge} system will suffice.
  \item The server might have to interact with many clients or, similarly, many different clients
        might require the same computation, the system must be \emph{non-interactive}.
  \item Verifying the computation must be cheaper than performing it, otherwise the client wouldn't 
        have to ask the server in the first place, the system must provide \emph{succint} proofs.
  \item The server has too interests in to the client that the computation was correct, say to 
        avoid legal liability, but it is not willing to share its own inputs, so our system must
        be \emph{zero-knowledge}.
\end{itemize}
Clearly, among the various constructions we saw up to now, ZK-SNARK systems are the only one that 
can reasonably fulfill all these requirements.
However, all the constructions we saw, due to the high overheads involved 
(generating the CRS, building the QSP/QAP, generating the proof, etc.), were not efficient enough 
to make the whole process faster than just letting the client perform the computations by itself.

The first construction that was efficient enough to be practically usable was 
\emph{Pinocchio}~\cite{ParnoGHR2013}.


\subsection{Groth16}
\subsection{PLONK}



\chapter{Cryptographic Primitives from Generalized Triangular Dynamical Systems}\label{chap:arion}
One of the most important applications of zero-knowledge verifiable computation lies in digital 
currency transactions over the blockchain infrastructure.
An example of ZK-SNARK applied in the real world is the ZCash cryptocurrency~\cite{SassonCGGMTV2014}, 
which is inspired by the more famous Bitcoin~\cite{NarayananBFMG2016}, and was devised by the 
authors of \texttt{libsnark} (which frames the zero-knowledge backend of the currency).

As we discussed in \Cref{sec:tree_hash}, the fundamental component of a blockchain is the 
Merkle tree, which uses one-way compression functions in order to produce the binding 
commitment.
In a digital currency scenario, the leaves of the Merkle tree consist of the details of some 
transaction, typical information include the ID of the sender, the ID of the recipient, and the 
amount of currency to be transferred. 
Without a zero-knowledge framework in place, when one wants to verify whether a user did abide to 
their commitment, the only possible solution is to ask the user to disclose his transaction, 
together with the authentication path, and check that the tree commitment is respected. 
When using currencies like Bitcoin or Ethereum\footnote{\url{https://ethereum.org/}}, anyone 
can see the details of every single transaction being performed on the relative 
blockchain, meaning that there is no privacy whatsoever\footnote{For example, on 
\url{https://etherscan.io/} you can see the transactions on the Ethereum blockchain. %It is 
%curious how privacy has often been foisted as a feature of mainstream cryptocurrencies while, 
%on the contrary, any bank offers much more privacy!
}.
However, if we translate the Merkle tree computation in an equivalent circuit, it is possible to 
apply a zero-knowledge scheme that allows a verifier to be sure (with overwhelming probability) 
of the validity of a transaction without actually having to see it!
Since a Merkle tree applies over and over the underlying compression function, the problem of 
creating a circuit for the former immediately reduces to the problem of creating a circuit for the 
latter.

In \Cref{sec:sota} we will review the evolution of the state of the art concerning zero-knowledge 
friendly compression functions.
Then, in \Cref{sec:gtds}, we present a new algebraic framework to represent cryptographic 
primitives, the \emph{Generalized Triangular Dynamic System}, and apply it to construct the 
\Arion{} block cipher and the \Arionhash{} hash function.
Finally, in \Cref{sec:performance}, we compare our new construction to the state of the art using 
the \texttt{libsnark} library, showing extremely competitive results.
\section{State of the art}\label{sec:sota}
The standard compression function used in Merkle trees is usually one of the SHA-2 or SHA-3 
functions~\cite{Dang2015}: this is certainly the most sensible choice in a \emph{native} 
environment, as SHA is specifically designed to be fast in both software and 
hardwaare~\cite{DaddaMO2004,MichailAKTG2012} implementations, and is the most studied hash function 
from a security standpoint (e.g.\ for SHA-2 see~\cite{KhovratovichRS2012,GuoLRW2010,DobraunigEM2016}).

However, when working with arithmetic circuits over a prime field \(\mathbb{F}_p\), SHA has a lot 
of issues: the underlying operations being performed are bitwise XOR, bitwise AND, 
bit shifts/rotations and additions modulo \(2^{32}\).
While shifts and rotations come at no cost, as they basically consist in a renaming of the circuit 
wires/variables, bitwise operations and addition which is not modulo \(p\) have to be simulated 
bit-by-bit, and the overhead introduced in such a translation is huge.
For example, for SHA-256, over a bilinear group like BN254 for which 
\(\abs{\mathbb{F}_p} \approx 2^{256}\), we would need \(256\) input variables each holding a
\(256\)-bit integer to simulate the behaviour of every single bit during the SHA computation; 
clearly, this is decisely suboptimal.

\begin{example}
  Suppose we are given two strings \(a, b \in \Set{0, 1}^{n}\), and we want to compute 
  \(a \bitxor b\).
  By interpreting them as vectors \(\bm{v}, \bm{w} \in \mathbb{F}_{p}^{n}\), we can simulate 
  bitwise XOR by computing, \(\forall i \le n\):
  \[\bm{v}_{i} \bitxor \bm{w}_{i} = \bm{v}_{i} + \bm{w}_{i} - 2\bm{v}_{i}\bm{w}_{i}\]
  that is, every XOR operation requires one multiplication gate.
  Similarly, bitwise AND and non-native addition also require multiplications to be simulated.
  Furthermore, we must guarantee that the values \(\bm{v}_i\) and \(\bm{w}_i\) are boolean, as 
  in principle they could assume any value in \(\mathbb{F}_p\), so we must also add constraints of 
  the kind \(\bm{v}_{i}\Parens*{\bm{v}_i - 1} = 0\).
\end{example}

\subsection{MiMC}
In an effort to find secure cryptographic designs that could be efficient in zero-knowledge 
settings, called \emph{zk-friendly} designs, researchers began to study the properties of 
permutations that make use of a low number of multiplications 
(\emph{multiplicative complexity})~\cite{AlbrechtRSTZ2016}.

One of the first constructions over finite fields was the \emph{Minimal Multiplicative Complexity}
(MiMC) family of cryptographic permutations~\cite{AlbrechtGRRT2016}.
The idea of MiMC, reprising an older proposal~\cite{NybergK1995}, is to use a very simple 
polynomial permutation as its core component, and by repeating it for an adequate number of rounds,
obtain a secure construction.
\begin{definition}[MiMC keyed permutation]
  Given a finite field \(\mathbb{F}_p\), a number of rounds 
  \(r = \Ceil*{\frac{\call{\log}{p}}{\call{\log}{3}}}\), some constants 
  \(c_1, \dots, c_r \in \mathbb{F}_p\) and a set of functions 
  \(f_1, \dots, f_r\colon \mathbb{F}_p \times \mathbb{F}_p \to \mathbb{F}_p\) such that 
  \(\forall i \le r\colon \call{f_i}{x, k} = x^3 + k + c_i\), the \emph{MiMC keyed permutation}
  is defined as:
  \[
    \call{E_{MiMC}}{x, k}\colon \mathbb{F}_p \times \mathbb{F}_p \to \mathbb{F}_p = 
    \call{\Parens*{f_r \compose \dots \compose f_1}}{x, k} + k
  \]
\end{definition}

The MiMC keyed permutation is also called MiMC-\(n/n\). 
By applying the Feistel construction on the MiMC permutation, one obtains the Feistel MiMC function, 
or MiMC-\(2n/n\).
Finally, by applying the sponge construction, one can obtain the MiMC hash function.
In alternative, it is also possible to build an hash function using first the Davies-Meyer 
construction to obtain a one-way compression function, and then the Merkle-Damg\"{a}rd construction
to obtain an hash function.

There are some important observations to be made on the MiMC construction.
First, the round permutation uses a low degree polynomial, but it is repeated for a high number of 
rounds: for example, if the size of the underlying field is \(\approx 2^{256}\), the number of 
rounds will be \(r = 162\). 
Note that \(x^3\) might not actually induce a permutation over \(\mathbb{F}_p\), as in general 
\(3\) is not coprime with \(\call{\totient}{p}\) (in fact, in the underlying fields of both BN254 
and BLS12, \(3\) is a factor of \(p - 1\)).
In such cases, one should modify the definition to consider the smallest prime number \(d\) such 
that \(\call{\gcd}{d, \call{\totient}{p}} = 1\), and reduce the number of rounds to
\(r = \Ceil*{\frac{\call{\log}{p}}{\call{\log}{d}}}\).

A second observation is that \(r\) must be chosen to thwart many different types of cryptanalysis 
techniques: since the MiMC permutation corresponds to the 
polynomial \(p = \Parens*{x^3 + k + c_1}\dots\Parens*{x^3 + k + c_r}\) 
(which has degree \(\call{\deg}{p} = 3^r\)), in addition to the traditional \emph{brute-force}, 
\emph{meet-in-the-middle}~\cite{DiffieH1977}, \emph{differential}~\cite{BihamS1991} and 
\emph{linear}~\cite{Matsui1994} attacks, one must also consider \emph{algebraic attacks}, 
which exploit the inherent nature of this type of constructions.

In fact, traditional attacks don't tend to pose a major threat to these kinds of constructions:
brute force is clearly too expensive and meet-in-the-middle is also infeasible both due to the high 
number of rounds and to the huge degree of the inverse permutation (usually \(1/3 \gg 3\)).
The permutation \(x^3\) is not approximable by a linear function~~\cite{AbdelraheemABL2012}, 
hence linear attacks are not a threat, and since it can be easily shown that any arbitrary input 
difference \(\delta_{in} \) propagates to any arbitrary output difference \(\delta_{out} \) with a 
probability of at most \({2}/{2^n}\), differential attacks are also ineffective~\cite{Nyberg1994}.

On the side of algebraic cryptanalysis, one might attempt an \emph{interpolation attack}, which 
uses Lagrange interpolation to find a polynomial \(\tilde{p}\) which behaves like a keyless version 
of \(p\)~\cite{JakobsenK1997}.
This attack's complexity depends solely on \(\call{\deg}{p}\) (in fact, an interpolation can be 
computed in \(\BigO{n\call{\log}{n}}\), where \(n = \call{\deg}{p}\)~\cite{Stoss1985}), hence we 
must be sure that the degree of \(p\) also grows exponentially round by round (as it is the case).
Another kind of algebraic attack is the \emph{GCD attack}: by using two plaintext/ciphertext pairs,
once can compute their greatest common divisor which will allow to easily retrieve the secret key.
Again, computing the GCD depends almost linearly on the degree of the polynomial, hence one must 
again be sure that the degree grows exponentially.

\subsubsection*{MiMC vs.\ SHA-256}

\subsection{Poseidon}
\subsection{Griffin}
\subsection{Other designs}

\section{\Arion{}: A new ZK-friendly permutation}\label{sec:gtds}
The constructions that we saw in \Cref{sec:sota} are prominent examples of different 
\emph{generations} of \emph{Arithmetization Oriented} (AO) designs.
For example, \Mimc{} is an example of a Gen-I design: its main purpose was mostly to demonstrate 
that it was indeed possible to construct secure and efficient cryptographic primitives by 
stacking simple, low-degree round functions.
On the other hand, the \Hades{} framework, its derivative \Poseidon{} and other similar 
constructions like \Rescue{} are examples of Gen-II designs: by tweaking the SPN and 
Feistel constructions, their purpose was to massively improve the efficiency over Gen-I designs.
Finally, \Griffin{} is an example of a Gen-III design: the underlying \Horst{} scheme is neither 
a ``pure'' Feistel nor SPN design and, by deviating from such standard constructions, its authors 
were able to improve the efficiency even further. 
Furthermore, in third generation designs it was shown that one does not necessarily need to use 
round functions with a low degree, as long as the resulting constraint system is not affected 
negatively\footnote{As it is often the case in research, the separation line is a bit blurry, 
as \Rescue, which we said to be a Gen-II design, already used inverse exponentiations}.

Other from \Griffin, there is another very recent Gen-III construction, called 
\Anemoi~\cite{BouvierBCPSVW2022} and based on the \Flystel{} design, which has some common points 
with the \Horst{} construction.
A very interesting fact used in \Flystel{} was the notion of CCZ-equivalence~\cite{CarletCZ1998},
a concept which generalizes the intuition behind the idea that there is ``no difference'' 
between, say, using \(x^{d}\) and \(x^{\frac{1}{d}}\).
\begin{definition}[Affine function]
  An \emph{affine function} over an \(n\)-dimensional vector space \(\mathbb{F}^n\) is a function
  \(\call{f}{\bm{x}}\colon \mathbb{F}^n \to \mathbb{F}^n = \bm{Mx} + q\), where 
  \(\bm{M} \in \mathbb{F}^{n \times n}\) and \(q \in \mathbb{F}^n\).
\end{definition}
\begin{definition}[Function graph]
  Given a set \(S\), the \emph{function graph} of a function \(f\colon S \to S\) is the pair 
  \(\Gamma = \Tuple{S, E}\) where \(E = \Set{\Tuple{x, \call{f}{x}} \mid x \in S}\).
\end{definition}
\begin{definition}[Induced permutation]
  Given a \emph{function graph} \(\Gamma = \Tuple{S, E}\) and a permutation \(P\colon S^2 \to S^2\), 
  the \emph{induced permutation} of \(\Gamma \) by \(P\) is the function graph 
  \(\call{P}{\Gamma} = \Tuple{S, E'}\) where \(E' = \Set{\call{P}{e} \mid e \in E}\).
\end{definition}
\begin{definition}[CCZ equivalence~\cite{BouvierBCPSVW2022,CarletCZ1998}]
  Given a vector space \(\mathbb{V}\) and two functions 
  \(f,g\colon \mathbb{V} \to \mathbb{V}\), \(f\) and \(g\) are \emph{CCZ-equivalent} if there 
  is an affine permutation \(L\colon \mathbb{V}^2 \to \mathbb{V}^2\) such 
  that \(\Gamma_{f} = \call{L}{\Gamma_{g}}\).
\end{definition}

Clearly, CCZ-equivalence is an equivalence relation over a vector space \(\mathbb{V}\), hence it 
induces a partitioning of \(\mathbb{V}^2\) into equivalence classes.
An interesting fact is that all CCZ-equivalent functions share the same linear and differential 
properties.
Even more importantly for our purposes is that CCZ-equivalent functions are indistinguishable under 
constraint verification: checking the constraint system of any member of the class verifies 
the validity of the computation of any other member.
However, one thing that CCZ-equivalent functions do not share in general is their degree, hence we 
can use the one with the highest degree in the actual computation to provide the most strict security 
guarantees against algebraic attacks, while using the one with the lowest degree when building
the constraint system for the verification in the SNARK framework.

In a high-level, intuitive way we can say that CCZ-equivalence allows us to ``ignore the order'' 
in which the witnesses of a certain computation are obtained: reprising our usual example,
the verifier does not care that the prover first must know \(x\) in order to obtain \(y = x^{1/d}\), 
all it matters is that the prover knows both of them and that their relationship is correct.
In fact, even more generally, there is no way to know in which order someone actually got hold of 
the intermediate values of a computation, hence we might as well exploit this to our advantage in 
order to reduce the complexity of the protocol.

\subsection{The Generalized Triangular Dynamical System}
As we said, the design of third generation AO cryptographic primitives diverts from the plain 
SPN or Feistel constructions.
For this reason, we introduce the \emph{Generalized Triangular Dynamical System}~\cite{RoyS2022}, 
GTDS for short, an algebraic framework which generalizes many previous designs (such as Feistel, 
SPN, \Horst{}, \dots) and their instantiations (\Mimc{}, \Poseidon{}, \Griffin{}, \dots), and enables 
us to provide a systematic security analysis of the constructions derived from it.
In particular:
\begin{itemize}
  \item The input is split in branches like in earlier designs.
  \item The round function offers the strength of all the incorporated designs.
  \item It is secure against classical attacks like differential cryptanalysis.
  \item It is secure against interpolation attacks already at the first round.
  \item Its linear layer mixes all the branches through a circulant matrix with no zero entries.  
  \item It uses inverse exponents, but decouples them from the direct exponent. 
\end{itemize}

\begin{definition}[GTDS of \Arion~\cite{RoyS2022}]\label{def:gtds}
  Given a prime field \(\mathbb{F}_p\), a number of branches \(t \in \mathbb{N}\), the smallest 
  integer \(d_1\) such that \(\call{\gcd}{d_1, p - 1} = 1\), an arbitrary 
  integer \(d_2\) such that \(\call{\gcd}{d_2, p - 1} = 1\), some constants 
  \(\alpha_{1}, \beta_{1}, \gamma_1, \dots, \alpha_{t - 1}, \beta_{t - 1}, \gamma_{t - 1} \in \mathbb{F}_p\) 
  such that \(\forall i < t\colon \alpha_i^2 - 4\beta_i\) is a quadratic non-residue modulo 
  \(p\), let \(e = {1}/{d_2}\) and, for all \(i < t\) let:
  \begin{align*}
    & \call{g_i}{x}\colon \mathbb{F}_p \to \mathbb{F}_p = x^2 + \alpha_{i}x + \beta_{i} \\
    & \call{h_i}{x}\colon \mathbb{F}_p \to \mathbb{F}_p = x^2 + \gamma_{i}x
  \end{align*}
  Then, the GTDS of \Arion{} is the function 
  \(\call{F_{GTDS}}{\bm{x}}\colon \mathbb{F}_p^t \to \mathbb{F}_p^t\) such that:
  \[
    \call{F_{GTDS}}{\bm{x}}_i = \bm{y}_i = 
    \begin{cases}
      \bm{x}_i^{d_1}\call{g_i}{\sigma_{i+1, t}} + \call{h_i}{\sigma_{i+1, t}} & 1 \le i < t \\
      \bm{x}_i^e & i = t
    \end{cases}
  \]
  where \(\sigma_{i, k} = \sum_{j=i}^{k}{\bm{x}_j + \bm{y}_j}\).
\end{definition}

It can be shown~\cite{RoyS2022} that the GTDS is function is invertible.
An interesting detail which came up only in the later phases of the GTDS design is the decoupling 
of the inverse exponentiation from the direct exponentiation, in the sense that, instead of 
using \(x^d\) and \(x^{1/d}\), we use \(x^{d_1}\) and \(x^{1/d_2}\).
The rationale of this choice is that some security considerations about cryptographic constructions
over the GTDS depend on the size of \(d_2\): if it is too small, we would need more rounds to 
achieve the desired level of security, hence increasing the circuit complexity.
If \(d_2\) were to be equal to \(d_1\), the reduction in terms of number of rounds would be 
overcompensated by the increase of the complexity of a single round, but since \(d_2\) is only 
used in the last branch, the trade-off becomes more convenient, especially for bigger branch sizes.

Of course, \(d_2\) should require an optimal number of constraints.
Note that, since we have at our disposal intermediate results, the optimal number of operations 
required to exponentiate a number is not determined by the classical \emph{binary exponentiation} 
algorithm~\cite{Gueron2011}, but rather by \emph{addition chains}~\cite{BosC1990}.

\subsection{\Arion{} and \Arionhash{}}
Depending on our security and efficiency needs, we can instantiate the GTDS in many 
ways.
We design \Arion{} and \Arionhash{}~\textbf{\cite{RoyST2023}} to work over fields of size 
\(\approx 2^{256}\).
In order to achieve a \emph{degree overflow} in the first round, it can be shown that \(4e\), where 
\(e = {1}/{d_2}\), should be greater than \(p\). 
For BN254 and BLS12, this can be achieved by \(d_2 \in \Set{121, 123, 125, 161, 257}\).
For example, the optimal way to compute \(x^{121}\) is:
\begin{align*}
  & y = \Parens*{x^{2}}^{2} && z = \Parens*{y^{2}y}^{2} && x^{121} = \Parens*{z^{2}}^{2}zx
\end{align*}
It is not hard to see that numbers of the type \(2^k + 1\), for \(k \in \mathbb{N}\), are the 
most efficient to compute, and \(257\) is a particularly attractive candidate as it is also a 
prime number, and requires the same number of multiplications to be computed as the other candidates, 
(unfortunately, \(129 = 43 \cdot 3\) is not invertible neither in BN254 nor in BLS12).

To introduce mixing between the various branches, we use an \emph{affine layer} which employs a 
circulant matrix that has no zero entries and that is efficiently computable.
\begin{definition}[Affine layer of \Arion]
  The \emph{affine layer} of \Arion{} over a vector space \(\mathbb{F}_p^t\) is the function:
  \[\call{L}{\bm{x}, \bm{c}}\colon \Parens*{\mathbb{F}_p^n}^2 \to \mathbb{F}_p^n = 
  \call{\circulant}{1, \dots, t}\bm{x} + \bm{c}\]
\end{definition}

It is worth noting that the very simple matrix \(\call{\circulant}{1, \dots, t}\) is an MDS matrix
for any prime field \(\mathbb{F}_p\) such that \(p \ge 2^{39}\) and for values of
\(t \in \Iinterval{2}{12}\).
Furthermore, computing the matrix-vector product using \(\call{\circulant}{1, \dots, t}\) can be 
done in \(\BigO{t}\) time instead of the typical \(\BigO{t^2}\) required by a standard matrix-vector 
multiplication algorithm, by using \Cref{alg:circ_mult}.
\begin{algorithm}
  \begin{algorithmic}
    \Function{circ\_mul}{$\bm{v} \in \mathbb{F}_p^t$} %chktex 46
    \State{\(\bm{w} \gets \bm{0} \in \mathbb{F}_p^t\)}
    \State{\(\sigma \gets \sum_{i=1}^{t}{\bm{v}_i}\)}
    \State{\(\bm{w}_1 \gets \sigma + \sum_{i=1}^{t}{\Parens*{i - 1}\bm{v}_i}\)}
    \For{\(i \in \Iinterval{2}{t}\)}
    \State{\(\bm{w}_i \gets \bm{w}_{i-1} - \sigma + n\bm{v}_{i-1}\)}
    \EndFor{}
    \State{\Return{\(\bm{w}\)}}
    \EndFunction{}
  \end{algorithmic}
  \caption{Efficient evaluation of the matrix-vector product with 
    \(\call{\circulant}{1, \dots, t}\)}\label{alg:circ_mult}
\end{algorithm}

\begin{definition}[\Arion{} keyed permutation~\textbf{\cite{RoyST2023}}]
  Given a prime field \(\mathbb{F}_p\), a number of branches \(t \in \mathbb{N}\), a number 
  of rounds \(r \in \mathbb{N}\), some constants \(\bm{c}_1, \dots, \bm{c}_r \in \mathbb{F}_p^t\), 
  the \emph{\Arion{} keyed permutation} is the function \(\Arion = \Arion_r\), where:
  \[
    \call{\Arion_i}{\bm{x}, \bm{k}_0, \dots, \bm{k}_r}\colon 
      \Parens*{\mathbb{F}_p^t}^{r+2} \to \mathbb{F}_p^t = \bm{y}_i =
      \begin{cases}
        \call{L}{\bm{x}, \bm{0}} + \bm{k}_i & i = 0 \\
        \call{L}{\call{F_{GTDS}}{\bm{x}}, \bm{c}_i} + \bm{k}_i & 1 \le i \le r
      \end{cases}
  \]
\end{definition}

\begin{definition}[\Arionp{} unkeyed permutation]
  The \emph{\Arion{} unkeyed permutation} is the function:
  \[
    \call{\Arionp}{\bm{x}}\colon \mathbb{F}_p^t \to \mathbb{F}_p^t = 
      \call{\Arion}{\bm{x}, \bm{0}, \dots, \bm{0}}
  \]
\end{definition}

We can instantiate the hash function \Arionhash{} by applying the sponge construction to 
the \Arionp{} permutation.
Since it has been shown that, for a pseudorandom permutation \(P\) over a vector space 
\(\mathbb{F}_p^{t}\), a rate \(r\) and a capacity 
\(c\) such that \(t = r + c\), the sponge construction is indifferentiable from a random 
distribution up to \(\call{\min}{p^r, p^{c/2}}\) queries~\cite{BertoniDPV2008}, to provide 
\(\kappa \) bits of security, we must require that \(r \ge {\kappa}/{\call{\log}{p}}\) and 
\(c \ge {2\kappa}/{\call{\log}{p}}\).

As a padding scheme for a message \(m \in \Set{0, 1}^*\) whose length is not a multiple of the rate 
\(r\), we use \(\call{\Pad}{m} = m \concat 0^{\Parens*{-\abs{m} \bmod r}}\), and we replace the 
initial value \(v = 0^{\Parens*{t\abs{\Encode{p}}}}\) (where \(\Encode{x}\) denotes the binary 
encoding of an object \(x\)) with 
\(v' = \Encode{\abs{m}} \concat 0^{\Parens*{t-1}\abs{\Encode{p}}}\).
Note that we assume,  as basically any constructions does, that \(\abs{m} < p\), 
which should not be a problem as typically \(p \approx 2^{256}\), and we don't expect to hash 
messages of length \(\abs{m} > 2^{256}\).

\begin{table}
  \centering
  \caption{Instantiation parameters of \Arion{} and \Aarion{} for \(128\) bits of security and 
    primes \(p \geq 2^{60}\).}\label{tab:arion_instantiation}
  \begin{tabular}[t]{  c  c  c  c  }
      \toprule

      \phantom{ }\(d_1\)\phantom{ } & \phantom{ }\(t\)\phantom{ } & \phantom{ }\Arion{} \phantom{ } & \phantom{ }\Aarion{} \phantom{ } \\
      \midrule
      \multicolumn{2}{  c  }{} & \multicolumn{2}{ c  }{\phantom{ }Rounds\phantom{ }} \\
      \midrule
      \(3\) & \(3\) & \(6\) & \(5\) \\
      \(5\) & \(3\) & \(6\) & \(4\) \\

      \(3\) & \(4\) & \(6\) & \(4\) \\
      \(5\) & \(4\) & \(5\) & \(4\) \\

      \(3\) & \(5\) & \(5\) & \(4\) \\
      \(5\) & \(5\) & \(5\) & \(4\) \\

      \(3\) & \(6\) & \(5\) & \(4\) \\
      \(5\) & \(6\) & \(5\) & \(4\) \\

      \(3\) & \(8\) & \(4\) & \(4\) \\
      \(5\) & \(8\) & \(4\) & \(4\) \\
      \bottomrule
  \end{tabular}
\end{table}

In \Cref{tab:arion_instantiation} you can find the parameters for \Arion{} and 
its aggressive variant \Aarion{}, the parameters for the respective hash functions are respectively 
the same.
The aggressive variants have been parametrized in a way to provide the desired security level 
against all attacks but probabilistic Gr\"{o}bner basis attacks.
The choice to provide them anyway was dictated by the fact that, to the best of our knowledge, none 
of the competitor designs has been proved secure against such attacks, hence the aggressive 
versions provide the same guarantees in terms of security as the current state of the art.
For a detailed security analysis of \Arion, \Arionhash{} and their aggressive variants, refer 
to~\textbf{\cite{RoyST2023}}.

\section{Performance evaluation of \Arion}\label{sec:performance}
As we saw in \Cref{sec:gtds}, the concept of CCZ-equivalence introduced in the \Anemoi{} 
proposal~\cite{BouvierBCPSVW2022} plays an important role to build high degree permutations 
verifiable with low degree constraint systems.
In fact, it is possible to extend the concept of CCZ-equivalence by allowing any permutation.
\begin{definition}[\(\pi \)-equivalence]
  Given a vector space \(\mathbb{V}\) and two functions \(f,g\colon \mathbb{V} \to \mathbb{V}\), 
  \(f\) and \(g\) are \emph{\(\pi \)-equivalent} if there is a permutation 
  \(\pi\colon \mathbb{V}^2 \to \mathbb{V}^2\) such that \(\Gamma_{f} = \call{\pi}{\Gamma_{g}}\).
\end{definition}

Any function \(\pi\colon \mathbb{V}^2 \to \mathbb{V}^2\) can be decomposed into its 
\emph{projections} \(\pi_{\hat{x}}, \pi_{\hat{y}}\colon \mathbb{V}^2 \to \mathbb{V}\) such that:
\[
  \forall \bm{x}, \bm{y} \in \mathbb{V}\colon \call{\pi}{\bm{x}, \bm{y}} = 
    \Tuple{\call{\pi_{\hat{x}}}{\bm{x}, \bm{y}}, \call{\pi_{\hat{y}}}{\bm{x}, \bm{y}}}
\]

Thus, for any function \(f\colon \mathbb{V} \to \mathbb{V}\), we have that 
\(\call{\pi}{\bm{x}, \call{f}{\bm{x}}} = 
  \Tuple{\call{\pi_{\hat{x}}}{\bm{x}, \call{f}{\bm{x}}}, 
  \call{\pi_{\hat{y}}}{\bm{x}, \call{f}{\bm{y}}}}\).
Now, let \(\call{f_{\hat{x}}}{\bm{x}} = \call{\pi_{\hat{x}}}{\bm{x}, \call{f}{\bm{x}}}\) and 
\(\call{f_{\hat{y}}}{\bm{x}} = \call{\pi_{\hat{y}}}{\bm{x}, \call{f}{\bm{x}}}\), then 
\(\call{\pi}{\Gamma_f} = \Gamma_{f'}\) if and only if \(\pi_{\hat{x}}\) is a permutation, and 
in particular \(f' = f_{\hat{y}} \compose f_{\hat{x}}^{-1}\) is \(\pi \)-equivalent to \(f\).

Just like CCZ-equivalence, \(\pi \)-equivalence is also an equivalence relation, denoted 
\(\pieq \), which allows us to identify equiivalence classes of functions over \(\mathbb{V}\).
\begin{lemma}[\(\pi \)-equivalence of permutations]\label{lem:pi_equiv}
  All permutations over a vector space \(\mathbb{V}\) are \(\pi \)-equivalent.
\end{lemma}
\begin{proof}
  We just have to prove that, for every permutation \(f\colon \mathbb{V} \to \mathbb{V}\), it 
  is the case that \(f \pieq \fooid \), and the result will follow.
  Clearly, the function \(\call{\pi}{\bm{x}, \bm{y}}\colon \mathbb{V}^2 \to \mathbb{V}^2 = 
    \Tuple{\bm{x}, \call{f^{-1}}{\bm{y}}}\) is a permutation, and 
  \(\call{\pi}{\Gamma_f} = \Gamma_{\fooid}\).
\end{proof}

\begin{definition}[Alternative GTDS of \Arion]
  Given the same parameters as in \Cref{def:gtds}, the \emph{alternative GTDS of \Arion} is the
  function \(\tilde{F}_{GTDS}\colon \mathbb{F}_p^t \to \mathbb{F}_p^t\) such that:
  \[
    \call{\tilde{F}_{GTDS}}{\bm{x}}_i = \bm{y}_i =
    \begin{cases}
      \bm{x}_i^{d_1}\call{g_i}{\tilde{\sigma}_{i+1,t}} + 
        \call{h_i}{\tilde{\sigma}_{i+1,t}} & 1 \le i < t \\
      \bm{x}_i^{d_2} & i = t
    \end{cases}
  \]
  where \(\tilde{\tau}_{i, k} = \bm{x}_k + \bm{x}_{k}^{e} + \sum_{j=i}^{k-1}{\bm{x}_j + \bm{y}_j}\).

\end{definition}

\begin{proposition}[\(\pi \)-equivalence of GTDS]
  \(F_{GTDS} \pieq \tilde{F}_{GTDS}\).
\end{proposition}
\begin{proof}
  \(F_{GTDS}\) and \(\tilde{F}_{GTDS}\) are both permutations over the vector space \(\mathbb{F}_p^t\),
  therefore by \Cref{lem:pi_equiv} the claim follows.
\end{proof}

As a corollary, we have that verifying the constraint system of \(F_{GTDS}\) is equivalent to 
verifying the constraint system of \(\tilde{F}_{GTDS}\).
Computing the number of multiplicative constraints for \Arionhash{} is quite straightforward:
\begin{lemma}[R1CS constraints for \Arionhash]
  Given the \Arionhash{} function over a prime field \(\mathbb{F}_p\) with branch size 
  \(t\) and rate \(r\), let \(\call{\minmul}{x}\colon \mathbb{F}_p \to \mathbb{N}\) be the minimum
  number of field multiplications required to compute \(y^x\) for any \(y \in \mathbb{F}\).
  The number of R1CS constraints required by \Arionhash{} is:
  \[
    N_{\Arionhash} = 
      r\Parens*{\Parens*{n - 1}\Parens*{\call{\minmul}{d_1} + 2} + \call{\minmul}{d_2}}
  \]
\end{lemma}
\begin{proof}
  Consider the alternative GTDS \(\tilde{F}_{GTDS}\): we need \(\call{\minmul}{d_2}\) 
  multiplicative constraints in the last branch.
  In the remaining \(n - 1\) branches, we need \(\call{\minmul}{d_1}\) multiplications to 
  compute \(\bm{x}_i^{d_1}\), one multiplication for computing \(g_i\), one for computing \(h_i\), 
  and one to multiply \(g_i\) with \(\bm{x}_i^{d_1}\).
\end{proof}

For reference, the number of R1CS constraints required by \Poseidon{} and \Griffin{} 
over their respective parameters (see \Cref{def:poseidon} and \Cref{def:griffin}) are given by:
\begin{align*}
  & N_{\Poseidon} = \call{\minmul}{d}\Parens*{2tr_f + r_P} \\
  & N_{\Griffin} = 2r\Parens{\call{\minmul}{d} + t - 2}
\end{align*}

\begin{table}
  \centering
  \caption{R1CS constraint comparison over \(256\)-bit prime fields and \(128\) bits of security 
  with \(d_2 \in \Set{121, 123, 125, 161, 257}\).}\label{tab:arion_compare_muls}
  \begin{tabular}{  c c c c c c c  }
      \toprule
      \phantom{ }\(d_1\)\phantom{ } & \phantom{ }\(t\)\phantom{ } & \phantom{ }\Arionhash{}\phantom{ } & \phantom{ }\Aarionhash{}\phantom{ } & \phantom{ }\Griffin{}\phantom{ } & \phantom{ }\Anemoi{}\phantom{ } & \phantom{ }\Poseidon{}\phantom{ }             \\
      \midrule
      \multicolumn{2}{  c | }{} & \multicolumn{5}{ c }{Rounds} \\
      \midrule
      \(3\) & \(3\) & \(6\) & \(5\) & \(12\) &      & \phantom{ }\(r_f = 4,\ r_P = 84\)\phantom{ } \\
      \(5\) & \(3\) & \(6\) & \(4\) & \(12\) &      & \phantom{ }\(r_f = 4,\ r_P = 56\)\phantom{ } \\

      \(3\) & \(4\) & \(6\) & \(4\) & \(11\) & \(12\) & \phantom{ }\(r_f = 4,\ r_P = 84\)\phantom{ } \\
      \(5\) & \(4\) & \(5\) & \(4\) & \(11\) & \(12\) & \phantom{ }\(r_f = 4,\ r_P = 56\)\phantom{ } \\

      \(3\) & \(5\) & \(5\) & \(4\) &      &      & \phantom{ }\(r_f = 4,\ r_P = 84\)\phantom{ } \\
      \(5\) & \(5\) & \(5\) & \(4\) &      &      & \phantom{ }\(r_f = 4,\ r_P = 56\)\phantom{ } \\

      \(3\) & \(6\) & \(5\) & \(4\) &      & \(10\) & \phantom{ }\(r_f = 4,\ r_P = 84\)\phantom{ } \\
      \(5\) & \(6\) & \(5\) & \(4\) &      & \(10\) & \phantom{ }\(r_f = 4,\ r_P = 84\)\phantom{ } \\

      \(3\) & \(8\) & \(4\) & \(4\) & \(9\)  & \(10\) & \phantom{ }\(r_f = 4,\ r_P = 84\)\phantom{ } \\
      \(5\) & \(8\) & \(4\) & \(4\) & \(9\)  & \(10\) & \phantom{ }\(r_f = 4,\ r_P = 56\)\phantom{ } \\

      \midrule

      \multicolumn{2}{  c | }{} & \multicolumn{5}{ c  }{R1CS Constraints} \\
      \midrule

      \(3\) & \(3\) & \(102\) & \(85\)  & \(72\)  &       & \(216\) \\
      \(5\) & \(3\) & \(114\) & \(76\)  & \(96\)  &       & \(240\) \\

      \(3\) & \(4\) & \(126\) & \(84\)  & \(88\)  & \(96\)  & \(232\) \\
      \(5\) & \(4\) & \(120\) & \(96\)  & \(110\) & \(120\) & \(264\) \\

      \(3\) & \(5\) & \(120\) & \(100\) &       &       & \(248\) \\
      \(5\) & \(5\) & \(125\) & \(116\) &       &       & \(288\) \\

      \(3\) & \(6\) & \(145\) & \(116\) &       & \(120\) & \(264\) \\
      \(5\) & \(6\) & \(170\) & \(136\) &       & \(150\) & \(312\) \\

      \(3\) & \(8\) & \(148\) & \(148\) & \(144\) & \(160\) & \(296\) \\
      \(5\) & \(8\) & \(176\) & \(176\) & \(162\) & \(200\) & \(360\) \\

      \bottomrule
  \end{tabular}
\end{table}

\Cref{tab:arion_compare_muls} shows the number of constraints required for specific instantiations 
of \Arion, \Aarion, \Anemoi, \Poseidon{} and \Griffin{} over \(\approx 256\)-bit prime fields for a 
target \(128\) bits of security.

\subsection{\texttt{libsnark} implementation and experiments}



\chapter{Conclusions and Future Work}\label{chap:conclusions}
In this work we have studied the history of zero-knowledge, succint and non-interactive 
argument of knowledge protocols, a highly interdisciplinar field of study which involves advanced
concepts of mathematic, computer science and cryptography.
We have seen how such systems have seen an incredible improvement over the last ten years, to the 
point of finally becoming useful in many real-world scenarios, from private verifiable cloud 
computing to anonymous blockchain commitments.

We have seen how the best ZK-SNARK systems known today work over arithmetic models of 
computation, much differerent from the boolean ones algorithms are usually designed for.
Since the most prominent application of such systems is in cryptography, and particularly in 
Merkle tree commitment verification, we revised the very recent history and the state of the art of 
arithmetization-oriented cryptographic primitives.

By distilling the core ideas in this field and introducing new ones, we then formalized them in the 
so-called Generalized Triangular Dynamical System framework, which we used to extract a new family 
of cryptographic functions: \Arion.
We finally compared \Arion{} with the competitor designs, showing that we beaat the state of the 
art when providing the same level of security guarantees, and by matching it when providing 
additional guarantees.

This work has been a very exciting journey, which is however just a glimpse of what could lie ahead.
New proof systems like the circuit-depth oriented transparent ZK-STARKs, or the Lagrange-bases 
oriented \(\Plonk \) offer new challenges, as they divert from the standard metrics used to measure 
the complexity of a good ZK-friendly function. 

Another interesting direction of research is the application of ZK-SNARK systems and the 
related cryptographic functions in a wide range of hardware devices: from high performance 
computing (HPC) machines (e.g.\ verifiable delegated computation), to personal computers 
(e.g.\ anonymous transactions) to emdedded internet-of-things (IoT) chips 
(e.g.\ private household monitoring). 

%% Capitolo
%\part{Zero Knowledge friendly permutations}\label{part:zk-hash}


%% Fine dei capitoli normali, inizio dei capitoli-appendice (opzionali)
%\appendix

%\part{Appendix}

%% Parte conclusiva del documento; tipicamente per riassunto, bibliografia e/o indice analitico.
\backmatter%

%% Riassunto (opzionale)
%\summary
%Maecenas tempor elit sed arcu commodo, dapibus sagittis leo egestas. Praesent at ultrices urna. Integer et nibh in augue mollis facilisis sit amet eget magna. Fusce at porttitor sapien. Phasellus imperdiet, felis et molestie vulputate, mauris sapien tincidunt justo, in lacinia velit nisi nec ipsum. Duis elementum pharetra lorem, ut pellentesque nulla congue et. Sed eu venenatis tellus, pharetra cursus felis. Sed et luctus nunc. Aenean commodo, neque a aliquam bibendum, mauris augue fringilla justo, et scelerisque odio mi sit amet diam. Nulla at placerat nibh, nec rutrum urna. Donec ut egestas magna. Aliquam erat volutpat. Phasellus vestibulum justo sed purus mattis, vitae lacinia magna viverra. Nulla rutrum diam dui, vel semper mi mattis ac. Vestibulum ante ipsum primis in faucibus orci luctus et ultrices posuere cubilia Curae; Donec id vestibulum lectus, eget tristique est.

%% Bibliografia (praticamente obbligatoria)
\bibliographystyle{alpha}%% Carica l'omonimo file .bst, dove \languagename � la lingua attiva.
%% Nel caso in cui si usi un file .bib (consigliato)
{\footnotesize \bibliography{biblio.bib}}
%% Nel caso di bibliografia manuale, usare l'environment thebibliography.

%% Per l'indice analitico, usare il pacchetto makeidx (o analogo).

\end{document}
